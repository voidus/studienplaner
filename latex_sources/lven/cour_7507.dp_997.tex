% Lehrveranstaltungsbeschreibung
% Informationsgrad : extern
% Sprache: de
\begin{course}

\setdoclanguagegerman
\coursedegreeprogramme{Informatik}
\coursemodulename{Virtual Engineering II (S.~\pageref{mod_4269.dp_997})[IN3MACHVE2]}
\courseID{2122378}
\coursename{Virtual Engineering II}
\coursecoordination{}

\documentdate{2011-03-09 17:35:29.483375}

\courselevel{4}
\coursecredits{5}
\courseterm{Sommersemester}
\coursehours{2/1}
\courseinstructionlanguage{de}

\coursehead

% For index (key word@display). Key word is used for sorting - no Umlauts please.
\index{Virtual Engineering II@Virtual Engineering II}

% For later referencing
\label{cour_7507.dp_997}


\begin{styleenv}
\begin{assessment}
Die Erfolgskontrolle erfolgt in Form einer mündlichen Prüfung um Umfang von 40 min über die Inhalte der Veranstaltung \emph{Virtual Engineering I} [21352] und \emph{Virtual Engineering II} [21378].

 

Die mündliche Prüfung kann auch nur über die Inhalte der Veranstaltung \emph{Virtual Engineering II} [21378] erfolgen. In diesem Fall verkürzt sich die Zeit der Prüfung auf 20


\end{assessment}

\begin{conditions}Werden in der Modulbeschreibung erläutert.

\end{conditions}

\begin{recommendations}Es werden Kenntnisse über CAx vorausgesetzt. Daher empfiehlt es sich, die Lehrveranstaltung Virtual Engineering I [2121352] im Vorfeld zu besuchen.

\end{recommendations}
\end{styleenv}

\begin{learningoutcomes}
Die Studenten verstehen was Virtual Reality bedeutet, wie der stereoskopische Effekt zustande kommt und mit welchen Technologien dieser Effekt simuliert werden kann.

 

Desweiteren wissen sie wie eine VR-Szene modelliert sowie intern in einem Rechner abgespeichert wird und wie die Pipeline zur Visualisierung dieser Szene funktioniert. Sie kennen sich mit verschiedenen Systemen zur Interaktion mit dieser VR-Szene aus und können die Vor- und Nachteile verschiedener Manipulations- und Trackinggeräte abschätzen.

 

Desweiteren wissen sie welche Validierungsuntersuchungen mit Hilfe eines Virtual-Mock-Up (VMU) im Produktentstehungsprozess durchgeführt werden können und den Unterschied zwischen VMU, Physical-Mock-Up (PMU) und einem virtuellen Prototypen (VP).

 

Sie wissen wie eine integrierte virtuelle Produktentwicklung in der Zukunft funktionieren sollte und verstehen welche Herausforderungen hierzu zu bewältigen sind.


\end{learningoutcomes}

\begin{content}
Die Vorlesung vermittelt die Informationstechnischen Aspekte und Zusammenhänge der virtuellen Produktentstehung:

 \begin{itemize}\item Virtual Reality-Systeme ermöglichen in Realzeit die hochimmersive und interaktive Visualisierung der entsprechenden Modelle, von den Einzelteilen bis zum vollständigen Zusammenbau.  \item Virtuelle Prototypen vereinigen CAD-Daten sowie Informationen über weitere Eigenschaften der Bauteile und Baugruppen für immersive Visualisierungen, Funktionalitätsuntersuchungen und Simulations- und Validierungstätigkeiten in und mit Unterstützung der VR/AR/MR-Umgebung.   \item Integrierte Virtuelle Produktentstehung verdeutlicht beispielhaft den Produktentstehungsprozess aus der Sicht des Virtual Engineerings.   \end{itemize}

Ziel der Vorlesung ist es, die Verknüpfung von Konstruktions- und Validierungstätigkeiten unter Nutzung Virtueller Prototypen und VR/AR-Visualisierungstechniken in Verbindung mit PDM/PLM-Systemen zu verdeutlichen. Ergänzt wird dies durch Einführungen in die jeweiligen IT-Systeme anhand praxisbezogener Aufgaben.


\end{content}

\begin{media}Skript zur Veranstaltung

\end{media}





\end{course}