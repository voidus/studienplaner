% Lehrveranstaltungsbeschreibung
% Informationsgrad : extern
% Sprache: de
\begin{course}

\setdoclanguagegerman
\coursedegreeprogramme{Informatik}
\coursemodulename{Wirtschaftsprivatrecht (S.~\pageref{mod_2655.dp_997})[IN3INJUR2]}
\courseID{24011}
\coursename{Handels- und Gesellschaftsrecht}
\coursecoordination{P. Sester}

\documentdate{2011-07-29 10:34:26.448401}

\courselevel{3}
\coursecredits{3}
\courseterm{Wintersemester}
\coursehours{2/0}
\courseinstructionlanguage{de}

\coursehead

% For index (key word@display). Key word is used for sorting - no Umlauts please.
\index{Handels- und Gesellschaftsrecht@Handels- und Gesellschaftsrecht}

% For later referencing
\label{cour_4389.dp_997}


\begin{styleenv}
\begin{assessment}
Die Erfolgskontrolle erfolgt in Form schriftlicher Prüfungen (Klausuren) im Rahmen der Veranstaltung „Privatrechtliche Übung” im Umfang von je 90 min. nach § 4, Abs. 2 Nr. 3 SPO.


\end{assessment}

\begin{conditions}Keine.\end{conditions}


\end{styleenv}

\begin{learningoutcomes}
Aufbauend auf den Vorlesungen zum Bürgerlichen Recht wird den Studenten ein Überblick über die Besonderheiten der Handelsgeschäfte, der handelsrechtlichen Stellvertretung und dem Kaufmannsrecht vermittelt. Darüber hinaus erhalten die Studenten einen Überblick über die Organisationsformen, die das deutsche Gesellschaftsrecht für unternehmerische Aktivitäten zur Verfügung stellt.


\end{learningoutcomes}

\begin{content}
Die Vorlesung beginnt mit einer Einführung in die Kaufmannsbegriffe des Handelsgesetzbuches. Danach wird das Firmenrecht, das Handelsregisterrecht und die handelsrechtliche Stellvertretung besprochen. Es folgen die allgemeinen Bestimmungen zu den Handelsgeschäften und die besonderen Handelsgeschäfte. Im Gesellschaftsrecht werden zunächst die Grundlagen der Personengesellschaften erläutert. Danach erfolgt eine Konzentration auf das Kapitalgesellschaftsrecht, welches die Praxis dominiert.


\end{content}

\begin{media}Folien.

\end{media}

\begin{literature}Klunzinger, Eugen

 \begin{itemize}\item Grundzüge des Handelsrechts, Verlag Vahlen, in der neuesten Auflage   \item Grundzüge des Gesellschaftsrechts, Verlag Vahlen, in der neuesten Auflage  \end{itemize}

\textbf{Weiterführende Literatur:}

 

Wird in der Vorlesung bekannt gegeben.

\end{literature}



\end{course}