% Lehrveranstaltungsbeschreibung
% Informationsgrad : extern
% Sprache: de
\begin{course}

\setdoclanguagegerman
\coursedegreeprogramme{Informatik}
\coursemodulename{Stochastische Methoden und Simulation (S.~\pageref{mod_3855.dp_997})[IN3WWOR4]}
\courseID{2550682}
\coursename{Stochastische Entscheidungsmodelle II}
\coursecoordination{K. Waldmann}

\documentdate{2011-10-28 15:13:00.391373}

\courselevel{4}
\coursecredits{4,5}
\courseterm{Sommersemester}
\coursehours{2/1/2}
\courseinstructionlanguage{de}

\coursehead

% For index (key word@display). Key word is used for sorting - no Umlauts please.
\index{Stochastische Entscheidungsmodelle II@Stochastische Entscheidungsmodelle II}

% For later referencing
\label{cour_7911.dp_997}


\begin{styleenv}
\begin{assessment}

\end{assessment}

\begin{conditions}Keine.\end{conditions}


\end{styleenv}

\begin{learningoutcomes}
Die Studierenden erwerben die Fähigkeit, Markovsche Entscheidungsprozesse als Analyseinstrument zur Steuerung und Optimierung zufallsabhängiger dynamischer Systeme einzusetzen und auf konkrete Problemstellungen anzupassen. Hierzu sind sie in der Lage, ein Optimalitätskriterium festzulegen und die daraus resultierende Optimalitätsgleichung im Hinblick auf die Zielgröße und eine optimale Strategie effizient zu lösen.


\end{learningoutcomes}

\begin{content}
Markovsche Entscheidungsprozesse: Theoretische Grundlagen, Optimalitätskriterien, Lösung der Optimalitätsgleichung, Optimalität einfach strukturierter Entscheidungsregeln, Anwendungen.


\end{content}

\begin{media}Tafel, Folien, Flash-Animationen, Simulationssoftware

\end{media}

\begin{literature}Skript

 

\textbf{Weiterführende Literatur:}

 

Waldmann, K.H. , Stocker, U.M. (2004): Stochastische Modelle - eine anwendungsorientierte Einführung; Springer

 

Puterman, M.L. (1994): Markov Decision Processes: Discrete Stochastic Dynamic Programming; John Wiley

\end{literature}

\begin{remarks}Die Lehrveranstaltung wird nicht regelmäßig angeboten. Das für zwei Studienjahre im voraus geplante Lehrangebot kann im Internet nachgelesen werden.

\end{remarks}

\end{course}