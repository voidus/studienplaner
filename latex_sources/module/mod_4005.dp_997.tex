% Modulbeschreibung 
% Informationsgrad : extern
% Sprache: de
\begin{module}

\setdoclanguagegerman
\moduledegreeprogramme{Informatik (B.Sc.)}
\modulesubject{EF E-Technik}
\moduleID{IN3EITBIOM}
\modulename{Biomedizinische Technik I}
\modulecoordination{O. Dössel}

\documentdate{2009-07-03 11:49:44}

\modulecredits{21}
\moduleduration{2}
\modulecycle{Jedes Semester}



\modulehead

% For index (key word@display). Key word is used for sorting - no Umlauts please.
\index{Biomedizinische Technik I@Biomedizinische Technik I (M)}

% For later referencing
\label{mod_4005.dp_997}

\begin{courselist}
23261 & Bildgebende Verfahren in der Medizin I (S.~\pageref{cour_8135.dp_997}) & 2 & W & 3 & O. Dössel\\
23262 & Bildgebende Verfahren in der Medizin II (S.~\pageref{cour_8141.dp_997}) & 2 & S & 3 & O. Dössel\\
23269 & Biomedizinische Messtechnik I (S.~\pageref{cour_8137.dp_997}) & 3 & W & 5 & A. Bolz\\
23270 & Biomedizinische Messtechnik II (S.~\pageref{cour_8149.dp_997}) & 3 & S & 5 & A. Bolz\\
23276 & Praktikum für biomedizinische Messtechnik  (S.~\pageref{cour_8147.dp_997}) & 4 & S & 6 & A. Bolz\\
23281 & Physiologie und Anatomie I (S.~\pageref{cour_8139.dp_997}) & 2 & W & 3 & U. Müschen\\
23282 & Physiologie und Anatomie II (S.~\pageref{cour_8143.dp_997}) & 2 & S & 3 & U. Müschen\\
23264 & Bioelektrische Signale und Felder (S.~\pageref{cour_8145.dp_997}) & 2 & S & 3 & G. Seemann\\
\end{courselist}

\begin{styleenv}
\begin{assessment}
Die Erfolgskontrollen werden in den Lehrveranstaltungbeschreibungen erläutert.

 

Die Gesamtnote des Moduls wird aus den mit LP gewichteten Teilnoten gebildet und nach der ersten Kommastelle abgeschnitten.


\end{assessment}

\begin{conditions}Keine.\end{conditions}


\end{styleenv}

\begin{learningoutcomes}

\end{learningoutcomes}

\begin{content}
Die Inhalte werden in den einzelnen Lehrveranstaltungsbeschreibungen erläutert.


\end{content}



\end{module}

