% Lehrveranstaltungsbeschreibung
% Informationsgrad : extern
% Sprache: de
\begin{course}

\setdoclanguagegerman
\coursedegreeprogramme{Informatik}
\coursemodulename{Einführung in Multimedia (S.~\pageref{mod_4203.dp_997})[IN3INEIM]}
\courseID{24185}
\coursename{Einführung in Multimedia}
\coursecoordination{P. Deussen}

\documentdate{2011-03-17 15:38:12.319361}

\courselevel{3}
\coursecredits{3}
\courseterm{Wintersemester}
\coursehours{2}
\courseinstructionlanguage{de}

\coursehead

% For index (key word@display). Key word is used for sorting - no Umlauts please.
\index{Einfuehrung in Multimedia@Einführung in Multimedia}

% For later referencing
\label{cour_8437.dp_997}


\begin{styleenv}
\begin{assessment}
Die Erfolgskontrolle erfolgt in Form einer mündlichen Prüfung im Umfang von i.d.R. 20 Minuten nach § 4 Abs. 2 Nr. 2 SPO.


\end{assessment}

\begin{conditions}Keine.\end{conditions}


\end{styleenv}

\begin{learningoutcomes}
Den Studierenden wird in dieser Querschnittsvorlesung ein Überblick über einige Informatikfächer vermittelt.

 

Ferner erhalten die Studierenden Kenntnisse in

 \begin{itemize}\item  der Physiologie des Ohres und der Augen,   \item  der notwendigen Physik.  \end{itemize}
\end{learningoutcomes}

\begin{content}
Multimedia ist eine Querschnittstechnologie, die die unterschiedlichsten Gebiete der Informatik zusammenbindet: Datenverwaltung, Telekommunikation, Mensch-Maschine-Kommunikation, aber auch Fragen der Farben, der Sinnesphysiologie und des Designs.\newline
\newline
 Die Einführungsvorlesung will diese Dinge ansprechen, hauptsächlich aber die folgenden Bereiche behandeln:\newline
 Digitale Behandlung von Tönen, von Bildern und Filmen samt den notwendigen Kompressionstechniken. Aber auch das wichtige Kapitel der Farben eben sowie die Fernseh- und Monitortechnik.


\end{content}

\begin{media}Vorlesungsfolien

\end{media}

\begin{literature}\textbf{Weiterführende Literatur:}

 

Hinweise in Vorlesungsfolien

\end{literature}



\end{course}