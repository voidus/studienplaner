% Modulbeschreibung 
% Informationsgrad : extern
% Sprache: de
\begin{module}

\setdoclanguagegerman
\moduledegreeprogramme{Informatik (B.Sc.)}
\modulesubject{Praktische Informatik}
\moduleID{IN2INKD}
\modulename{Kommunikation und Datenhaltung}
\modulecoordination{K. Böhm, M. Zitterbart}

\documentdate{2012-02-01 10:22:06.912691}

\modulecredits{8}
\moduleduration{1}
\modulecycle{Jedes 2. Semester, Sommersemester}



\modulehead

% For index (key word@display). Key word is used for sorting - no Umlauts please.
\index{Kommunikation und Datenhaltung@Kommunikation und Datenhaltung (M)}

% For later referencing
\label{mod_2517.dp_997}

\begin{courselist}
24516 & Datenbanksysteme (S.~\pageref{cour_8649.dp_997}) & 2/1 & S & 4 & K. Böhm\\
24519 & Einführung in Rechnernetze (S.~\pageref{cour_8645.dp_997}) & 2/1 & S & 4 & M. Zitterbart\\
\end{courselist}

\begin{styleenv}
\begin{assessment}
Die Erfolgskontrolle zur Lehrveranstaltung \textbf{\emph{Einführung in Rechnernetze}} erfolgt in Form einer schriftlichen Prüfung nach § 4 Abs. 2 Nr. 1 SPO.

 

Die Erfolgskontrolle zur Lehrveranstaltung \textbf{\emph{Datenbanksysteme}} erfolgt semesterbegleitend als benotete Erfolgskontrolle anderer Art nach § 4 Abs. 2. Nr. 3 SPO durch Bearbeiten von Übungsaufgaben, deren Lösungen benotet werden. Am Ende des Semesters wird eine benotete schriftliche Präsenzübung durchgeführt.

 

Die Gesamtnote des Moduls wird aus den mit Leistungspunkten gewichteten Teilnoten der einzelnen Lehrveranstaltungen gebildet und nach der ersten Kommastelle abgeschnitten.


\end{assessment}

\begin{conditions}Keine.\end{conditions}

\begin{recommendations}Kenntnisse aus den Vorlesungen \emph{Betriebssysteme} und \emph{Softwaretechnik I }werden empfohlen.

\end{recommendations}
\end{styleenv}

\begin{learningoutcomes}
Der/die Studierende

 \begin{itemize}\item kennt die Grundlagen der Datenübertragung sowie den Aufbau von Kommunikationssystemen,  \item ist mit der Zusammensetzung von Protokollen aus einzelnen Protokollmechanismen vertraut und konzipiert einfache Protokolle eigenständig,  \item kennt und versteht das Zusammenspiel einzelner Kommunikationsschichten und Anwendungen,  \item stellt den Nutzen von Datenbank-Technologie dar,  \item definiert die Modelle und Methoden bei der Entwicklung von funktionalen Datenbank-Anwendungen, legt selbstständig einfache Datenbanken an und tätigt Zugriffe auf diese,  \item kennt und versteht die entsprechenden Begrifflichkeiten und die Grundlagen der zugrundeliegenden Theorie.  \end{itemize}
\end{learningoutcomes}

\begin{content}
Verteilte Informationssysteme sind nichts anderes als zu jeder Zeit von jedem Ort durch jedermann zugängliche, weltweite Informationsbestände. Den räumlich verteilten Zugang regelt die Telekommunikation, die Bestandsführung über beliebige Zeiträume und das koordinierte Zusammenführen besorgt die Datenhaltung. Wer global ablaufende Prozesse verstehen will, muss also sowohl die Datenübertragungsechnik als auch die Datenbanktechnik beherrschen, und dies sowohl einzeln als auch in ihrem Zusammenspiel.


\end{content}

\begin{remarks}Zur Lehrveranstaltung Datenbanksysteme [24516] ist es möglich als weitergehende Übung im Wahlfach das Modul \emph{Weitergehende Übung Datenbanksysteme} [IN3INWDS] (dieses Modul wird zurzeit nicht angeboten) zu belegen.

\end{remarks}

\end{module}

