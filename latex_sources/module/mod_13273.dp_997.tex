% Modulbeschreibung 
% Informationsgrad : extern
% Sprache: de
\begin{module}

\setdoclanguagegerman
\moduledegreeprogramme{Informatik (B.Sc.)}
\modulesubject{EF Informationsmanagement im Ingenieurwesen}
\moduleID{IN3MACHVRP}
\modulename{Virtual Reality Praktikum }
\modulecoordination{Jivka Ovtcharova}

\documentdate{2011-07-25 12:19:30.689704}

\modulecredits{4}
\moduleduration{1}
\modulecycle{Jedes 2. Semester, Wintersemester}



\modulehead

% For index (key word@display). Key word is used for sorting - no Umlauts please.
\index{Virtual Reality Praktikum @Virtual Reality Praktikum  (M)}

% For later referencing
\label{mod_13273.dp_997}

\begin{courselist}
2123375 & Virtual Reality Praktikum  (S.~\pageref{cour_13209.dp_997}) & 3 & S & 4 & J. Ovtcharova\\
\end{courselist}

\begin{styleenv}
\begin{assessment}
Die Erfolgskontrolle setzt sich zusammen aus: Präsentation der Projektarbeit (40\%), Individuelles Projektportfolio in der Anwendungsphase für die Arbeit im Team (30\%), Schriftliche Wissensabfrage (20\%) und soziale Kompetenz (10\%).


\end{assessment}

\begin{conditions}Die Module \textbf{\emph{Product Lifecycle Managemet}} [IN3MACHPLM] und \textbf{\emph{Technische Informationssysteme}} [IN3INMACHTI] müssen im Ergänzungsfach Informationsmanagement im Ingenieurwesen geprüft werden.

\end{conditions}


\end{styleenv}

\begin{learningoutcomes}
Der/ die Studierende sind in der Lage die bestehende Infrastruktur (Hardware und Software) für Virtual Reality (VR) Anwendungen bedienen und benutzen zu können um:

 \begin{itemize}\item die Lösung einer komplexen Aufgabenstellung im Team zu konzipieren,  \item unter Berücksichtigung der Schnittstellen in kleineren Gruppen die Teilaufgaben innerhalb eines bestimmten Arbeitspaketes zu lösen und  \item diese anschließend in ein vollständiges Endprodukt zusammenzuführen.  \end{itemize}

Angestrebte Kompetenzen: \newline
 Methodisches Vorgehen mit praxisorientierten Ingenieuraufgaben, Teamfähigkeit, Arbeit in interdisziplinären Gruppen, Zeitmanagement


\end{learningoutcomes}

\begin{content}
\begin{itemize}\item Einführung und Grundlagen in VR (Hardware, Software, Anwendungen)  \item Vorstellung und Nutzung von „3DVIA Virtools“ als Werkzeug und Entwicklungsumgebung  \item Selbständige Entwicklung eines Fahrsimulators in VR in kleinen Gruppen  \end{itemize}
\end{content}



\end{module}

