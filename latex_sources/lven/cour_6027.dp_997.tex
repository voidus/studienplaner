% Lehrveranstaltungsbeschreibung
% Informationsgrad : extern
% Sprache: de
\begin{course}

\setdoclanguagegerman
\coursedegreeprogramme{Informatik}
\coursemodulename{eBusiness und Service Management (S.~\pageref{mod_1611.dp_997})[IN3WWBWL2]}
\courseID{2595466}
\coursename{eServices}
\coursecoordination{C. Weinhardt, H. Fromm, J. Kunze von Bischhoffshausen}

\documentdate{2012-02-03 16:06:51.018264}

\courselevel{3}
\coursecredits{5}
\courseterm{Sommersemester}
\coursehours{2/1}
\courseinstructionlanguage{en}

\coursehead

% For index (key word@display). Key word is used for sorting - no Umlauts please.
\index{eServices@eServices}

% For later referencing
\label{cour_6027.dp_997}


\begin{styleenv}
\begin{assessment}
Die Erfolgskontrolle erfolgt in Form einer 60min. schriftlichen Prüfung (nach § 4, (2), 1 SPO) und durch Ausarbeiten von Übungsaufgaben als Erfolgskontrolle anderer Art (nach §4(2), 3 SPO).


\end{assessment}

\begin{conditions}Keine.\end{conditions}


\end{styleenv}

\begin{learningoutcomes}
Diese Vorlesung vermittelt das grundlegende Wissen um die Bedeutsamkeit von Dienstleistungen in der Wirtschaft sowie den Einfluss von IKT auf bestehende und neue Service-Industrien. Durch die Kombination von theoretischen Modellen, praktischen Fallstudien und verschiedenen Anwendungsszenarien werden Studierende

 \begin{itemize}\item unterschiedliche Service-Perspektiven und das Konzept der „Value Co-Creation“ verstehen,  \item Konzepte, Methoden und Werkzeuge für die Gestaltung, die Entwicklung und das Management von eServices kennen und anwenden können,  \item mit aktuellen Forschungsthemen vertraut sein,  \item Erfahrung in Gruppenarbeit sowie im Lösen von Fallstudien sammeln und gleichzeitig ihre Präsentationsfähigkeiten verbessern,  \item den Umgang mit der englischen Sprache als Vorbereitung auf die Arbeit in einem internationalem Umfeld üben.  \end{itemize}
\end{learningoutcomes}

\begin{content}
Die Weltwirtschaft wird mehr und mehr durch Dienstleistungen bestimmt: in den Industriestaaten sind „Services“ bereits für ca. 70\% der Bruttowertschöpfung verantwortlich. Für die Gestaltung, die Entwicklung und das Management von Dienstleistungen sind jedoch traditionelle, auf Güter fokussierte Konzepte häufig unpassend oder unzureichend. Zudem treibt der rasante Fortschritt der Informations- und Kommunikations-Technologie (IKT) die ökonomische Bedeutung elektronisch erbrachter Dienstleistungen (eServices) noch schneller voran und verändert das Wettbewerbsumfeld: IKT-basierte Interaktion und Individualisierung eröffnen ganz neue Dimensionen der gemeinsamen Wertschöpfung zwischen Anbietern und Kunden, dynamische und skalierbare „service value networks“ verdrängen etablierte Wertschöpfungsketten; digitale Dienstleistungen werden über geographische Grenzen hinweg global erbracht.\newline
\newline
Aufbauend auf der grundsätzlichen Idee der „Value Co-Creation“ und einer systematischen Kategorisierung von (e)Services betrachten wir grundlegende Konzepte für die Entwicklung als auch für das Management von IT-basierten Services als Grundlage zur weiteren Spezialisierung in den Vertiefungsfächern am KSRI. Unter anderem beschäftigen wir uns mit Service-Innovation, Service Economics, Service-Modellierung sowie der Transformation und der Koordination von Service-Netzwerken.\newline
\newline
Zusätzlich wird die Anwendung der Konzepte in Fallstudien, praktischen Übungen und Gastvorträgen trainiert. Der gesamte Kurs wird in englischer Sprache gehalten. Die Studenten sollen so die Gelegenheit bekommen, Erfahrungen im - in Praxis wie Wissenschaft bedeutsamen - internationalen Umfeld zu sammeln.


\end{content}

\begin{media}\begin{itemize}\item Powerpoint-Folien  \end{itemize}\end{media}

\begin{literature}\begin{itemize}\item Anderson, J./ Nirmalya, K. / Narus, J. (2007), Value Merchants.  \item Lovelock, C. / Wirtz, J. (2007) Services Marketing, 6th ed.  \item Meffert, H./Bruhn, M. (2006), Dienstleistungsmarketing, 5. Auflage,  \item Spohrer, J. et al. (2007), Steps towards a science of service systems. In: IEEE Computer, 40 (1), p. 70-77  \item Stauss, B. et al. (Hrsg.) (2007), Service Science – Fundamentals Challenges and Future Developments.  \item Teboul, (2007), Services is Front Stage.  \item Vargo, S./Lusch, R. (2004) Evolving to a New Dominant Logic for Marketing, in: Journal of Marketing 68(1): 1–17.  \item Shapiro, C. / Varian, H. (1998), Information Rules - A Strategic Guide to the Network Economy  \end{itemize}\end{literature}

\begin{remarks}Die Veranstaltung wird ab dem SS2012 nicht mehr in den Masterstudiengängen angeboten. Angefangene Module können aber wie vorgesehen geprüft werden.

\end{remarks}

\end{course}