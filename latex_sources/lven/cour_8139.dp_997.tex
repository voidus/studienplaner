% Lehrveranstaltungsbeschreibung
% Informationsgrad : extern
% Sprache: de
\begin{course}

\setdoclanguagegerman
\coursedegreeprogramme{Informatik}
\coursemodulename{Biomedizinische Technik I (S.~\pageref{mod_4005.dp_997})[IN3EITBIOM]}
\courseID{23281}
\coursename{Physiologie und Anatomie I}
\coursecoordination{U. Müschen}

\documentdate{2010-07-15 12:00:44.882728}

\courselevel{3}
\coursecredits{3}
\courseterm{Wintersemester}
\coursehours{2}
\courseinstructionlanguage{de}

\coursehead

% For index (key word@display). Key word is used for sorting - no Umlauts please.
\index{Physiologie und Anatomie I@Physiologie und Anatomie I}

% For later referencing
\label{cour_8139.dp_997}


\begin{styleenv}
\begin{assessment}
Die Erfolgskontrolle erfolgt in Form einer mündlichen Prüfung im Umfang von i.d.R. 20 Minuten nach § 4 Abs. 2 Nr. 2 SPO.


\end{assessment}

\begin{conditions}Keine.\end{conditions}


\end{styleenv}

\begin{learningoutcomes}

\end{learningoutcomes}

\begin{content}
Inhalt:

 \begin{itemize}\item Einführung in die Stammesgeschichte von Homo sapiens und in seine Individualentwicklung (Embryologie)  \item  Zellaufbau, Zellphysiologie  \item Transportmechanismen  \item vielzellige Organisation (Gewebe)  \item Neurophysiologie I (Nervenzelle, Muskelzelle, biologischer Sensor, das autonome Nervensystem)  \item Herz und Kreislauf  \item Atmung  \item Blut  \item Niere  \end{itemize}
\end{content}







\end{course}