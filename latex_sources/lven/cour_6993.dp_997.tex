% Lehrveranstaltungsbeschreibung
% Informationsgrad : extern
% Sprache: de
\begin{course}

\setdoclanguagegerman
\coursedegreeprogramme{Informatik}
\coursemodulename{Risk and Insurance Management (S.~\pageref{mod_1553.dp_997})[IN3WWBWL6], Insurance Markets and Management (S.~\pageref{mod_1565.dp_997})[IN3WWBWL7]}
\courseID{2550055}
\coursename{Principles of Insurance Management}
\coursecoordination{U. Werner}

\documentdate{2012-01-22 17:22:34.406461}

\courselevel{4}
\coursecredits{4,5}
\courseterm{Sommersemester}
\coursehours{3/0}
\courseinstructionlanguage{de}

\coursehead

% For index (key word@display). Key word is used for sorting - no Umlauts please.
\index{Principles of Insurance Management@Principles of Insurance Management}

% For later referencing
\label{cour_6993.dp_997}


\begin{styleenv}
\begin{assessment}
Die Erfolgskontrolle setzt sich zusammen aus einer mündlichen Prüfung (nach §4(2), 2 SPO) und Vorträgen und Ausarbeitungen im Rahmen der Veranstaltung (nach §4(2), 3 SPO).

 

Die Note setzt sich zu je 50\% aus den Vortragsleistungen (inkl. Ausarbeitungen) und der mündlichen Prüfung zusammen.


\end{assessment}

\begin{conditions}Keine.\end{conditions}


\end{styleenv}

\begin{learningoutcomes}
\begin{itemize}\item Funktion von Versicherungsschutz als risikopolitisches Instrument auf einzel- und gesamtwirtschaftlicher Ebene einschätzen können;  \item rechtliche Rahmenbedingungen und die Technik der Produktion von Versicherungsschutz sowie weiterer Leistungen von Versicherungsunternehmen (Kapitalanlage, Risikoberatung, Schadenmanagement) kennen lernen.  \end{itemize}
\end{learningoutcomes}

\begin{content}
Die Fragen ‚Was ist Versicherung?’ bzw. ‚Wie ist es möglich, dass Versicherer Risiken von anderen übernehmen und dennoch recht sichere und rentable Unternehmen sind, in die Warren Buffett gerne investiert?’ wird auf mehreren Ebenen beantwortet:

 

Zunächst untersuchen wir die Funktion von Versicherungsschutz als risikopolitisches Instrument auf einzel- und gesamtwirtschaftlicher Ebene und lernen die rechtlichen Rahmenbedingungen sowie die Technik der Produktion von Versicherungsschutz kennen. Dann erkunden wir weitere Leistungen von Versicherungsunternehmen wie Risikoberatung, Schadenmanagement und Kapitalanlage.

 

Die zentrale Finanzierungsfunktion (wer finanziert die Versicherer? wen finanzieren die Versicherer? über wie viel Kapital müssen Versicherer mindestens verfügen, um die übernommenen Risiken tragen zu können?) stellt einen weiteren Schwerpunkt dar.

 

Abschließend werden ausgewählte Aspekte wichtiger Versicherungsprodukte vorgestellt.

 

Alle Teilnehmer tragen aktiv zur Veranstaltung bei, indem sie mindestens 1 Vortrag präsentieren und mindestens eine Ausarbeitung anfertigen.


\end{content}



\begin{literature}\begin{itemize}\item D. Farny. \emph{Versicherungsbetriebslehre. Karlsruhe} 2011.  \item P. Koch. \emph{Versicherungswirtschaft - ein einführender Überblick.} 2005.  \item M. Rosenbaum, F. Wagner. Versicherungsbetriebslehre. Grundlegende Qualifikationen. Karlsruhe 2002.  \item U. Werner. Einführung in die Versicherungsbetriebslehre. Skript zur Vorlesung.  \end{itemize}

\textbf{Weiterführende Literatur:}

 

Erweiterte Literaturangaben werden in der Vorlesung bekannt gegeben.

\end{literature}

\begin{remarks}Aus organisatorischen Gründen ist eine Anmeldung erforderlich im Sekretariat des Lehrstuhls: thomas.mueller3@kit.edu.

\end{remarks}

\end{course}