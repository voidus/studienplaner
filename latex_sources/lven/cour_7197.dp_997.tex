% Lehrveranstaltungsbeschreibung
% Informationsgrad : extern
% Sprache: de
\begin{course}

\setdoclanguagegerman
\coursedegreeprogramme{Informatik}
\coursemodulename{Praktische Mathematik (S.~\pageref{mod_2551.dp_997})[IN2MATHPM]}
\courseID{01874}
\coursename{Numerische Mathematik für die Fachrichtungen Informatik und Ingenieurwesen}
\coursecoordination{C. Wieners, Neuß,  Rieder}

\documentdate{2008-09-26 10:54:31}

\courselevel{2}
\coursecredits{4,5}
\courseterm{Sommersemester}
\coursehours{2/1}
\courseinstructionlanguage{de}

\coursehead

% For index (key word@display). Key word is used for sorting - no Umlauts please.
\index{Numerische Mathematik fuer die Fachrichtungen Informatik und Ingenieurwesen@Numerische Mathematik für die Fachrichtungen Informatik und Ingenieurwesen}

% For later referencing
\label{cour_7197.dp_997}


\begin{styleenv}
\begin{assessment}
Die Erfolgskontrolle erfolgt durch eine schriftliche Prüfung nach § 4 Abs. 2 Nr. 1 SPO im Umfang von 120 Minuten. Weiterhin muß ein Übungschein (Erfolgskontrolle anderer Art nach § 4 Abs. 2 Nr. 3 SPO) bestanden werden.

 

Gewichtung: 100 \% Klausurnote


\end{assessment}

\begin{conditions}Empfehlung: Das Modul \emph{Höhere Mathematik} [IN1MATHHM] bzw. \emph{Analysis} [INMATHANA] sollte abgeschlossen sein.

\end{conditions}


\end{styleenv}

\begin{learningoutcomes}
Die Studenten lernen in dieser Vorlesung die Umsetzung des im Mathematik-Modul erarbeiteten Wissens in die zahlenmäßige Lösung praktisch relevanter Fragestellungen. Dies ist ein wichtiger Beitrag zum tieferen Verständnis sowohl der Mathematik als auch der Anwendungsprobleme. \newline
\newline
Im Einzelnen sollen die Studenten\newline
\newline
1. entscheiden lernen, mit welchen numerischen Verfahren sie mathematische Probleme numerisch lösen können,\newline
2. das qualitative und asymptotische Verhalten von numerischen Verfahren beurteilen,\newline
3. die Qualität der numerischen Lösung kontrollieren.


\end{learningoutcomes}

\begin{content}
- Gleitkommarechnung\newline
 - Kondition mathematischer Probleme\newline
 - Vektor- und Matrixnormen\newline
 - Direkte Lösung linearer Gleichungssysteme\newline
 - Iterative Lösung linearer Gleichungssysteme\newline
 - Lineare Ausgleichsprobleme\newline
 - Lineare Eigenwertprobleme\newline
 - Lösung nichtlinearer Probleme: Fixpunktsatz, Newton-Verfahren\newline
 - Polynominterpolation\newline
 - Fouriertransformation (optional)\newline
 - Numerische Quadratur\newline
 - Numerische Lösung gewöhnlicher Differentialgleichungen (optional)


\end{content}

\begin{media}Tafel/Folien/Computerdemos

\end{media}

\begin{literature}\textbf{Weiterführende Literatur:}

- Vorlesungsskript (N. Neuß)\newline
- W. Dahmen/A. Reusken: Numerik für Ingenieure und Naturwissenschaftler

\end{literature}



\end{course}