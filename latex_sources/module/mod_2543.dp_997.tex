% Modulbeschreibung 
% Informationsgrad : extern
% Sprache: de
\begin{module}

\setdoclanguagegerman
\moduledegreeprogramme{Informatik (B.Sc.)}
\modulesubject{}
\moduleID{IN3INDBE}
\modulename{Datenbankeinsatz}
\modulecoordination{K. Böhm}

\documentdate{2012-01-04 12:46:24.549471}

\modulecredits{5}
\moduleduration{1}
\modulecycle{Jedes 2. Semester, Sommersemester}



\modulehead

% For index (key word@display). Key word is used for sorting - no Umlauts please.
\index{Datenbankeinsatz@Datenbankeinsatz (M)}

% For later referencing
\label{mod_2543.dp_997}

\begin{courselist}
dbe & Datenbankeinsatz (S.~\pageref{cour_5111.dp_997}) & 2/1 & S & 5 & K. Böhm\\
\end{courselist}

\begin{styleenv}
\begin{assessment}
Es wird mindestens sechs Wochen im Voraus angekündigt, ob die Erfolgskontrolle in Form einer schriftlichen Prüfung (Klausur) im Umfang von 1h nach § 4 Abs. 2 Nr. 1 SPO oder in Form einer mündlichen Prüfung im Umfang von i.d.R. 20 Minuten nach § 4 Abs. 2 Nr. 2 SPO stattfindet.


\end{assessment}

\begin{conditions}Datenbankkenntnisse, z.B. aus der Vorlesung \emph{Datenbanksysteme} [24516].

\end{conditions}


\end{styleenv}

\begin{learningoutcomes}
Am Ende der Lehrveranstaltung sollen die Teilnehmer Datenbank-Konzepte (insbesondere Datenmodelle, Anfragesprachen) – breiter, als es in einführenden Datenbank-Veranstaltungen vermittelt wurde – erläutern und miteinander vergleichen können. Sie sollten Alternativen bezüglich der Verwaltung komplexer Anwendungsdaten mit Datenbank-Technologie kennen und bewerten können.


\end{learningoutcomes}

\begin{content}
Diese Vorlesung soll Studierende an den Einsatz moderner Datenbanksysteme heranführen, in Breite und Tiefe. ’Breite’ erreichen wir durch die ausführliche Betrachtung und die Gegenüberstellung unterschiedlicher Datenmodelle, insbesondere des relationalen und des semistrukturierten Modells (vulgo XML), und entsprechender Anfragesprachen (SQL, XQuery). ’Tiefe’ erreichen wir durch die Betrachtung mehrerer nichttrivialer Anwendungen. Dazu gehören beispielhaft die Verwaltung von XML-Datenbeständen oder E-Commerce Daten, die Implementierung von Retrieval-Modellen mit relationaler Datenbanktechnologie oder die Verwendung von SQL für den Zugriff auf Sensornetze. Diese Anwendungen sind von allgemeiner Natur und daher auch isoliert betrachtet bereits interessant.


\end{content}

\begin{remarks}\textcolor{red}{Dieses Modul wird ab dem SS 2012 nicht mehr im Bachelor-Studiengang Informatik, sondern nur noch im Master-Studiengang Informatik angeboten.}

 

Im Bachelor-Studiengang kann die Prüfung nur noch nach Antragstellung und Genehmigung durch das Service-Zentrum Studium und Lehre erfolgen.

\end{remarks}

\end{module}

