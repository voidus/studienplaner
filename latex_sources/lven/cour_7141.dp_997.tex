% Lehrveranstaltungsbeschreibung
% Informationsgrad : extern
% Sprache: de
\begin{course}

\setdoclanguagegerman
\coursedegreeprogramme{Informatik}
\coursemodulename{Schlüsselqualifikationen (S.~\pageref{mod_2523.dp_997})[IN1HOCSQ]}
\courseID{PUB}
\coursename{Praxis der Unternehmensberatung}
\coursecoordination{K. Böhm, Dürr}

\documentdate{2010-09-24 11:58:39.366348}

\courselevel{4}
\coursecredits{1}
\courseterm{Winter-/Sommersemester}
\coursehours{2}
\courseinstructionlanguage{de}

\coursehead

% For index (key word@display). Key word is used for sorting - no Umlauts please.
\index{Praxis der Unternehmensberatung@Praxis der Unternehmensberatung}

% For later referencing
\label{cour_7141.dp_997}


\begin{styleenv}
\begin{assessment}
Die Erfolgskontrolle erfolgt in Form einer “Erfolgskontrolle anderer Art” und besteht aus mehreren Teilaufgaben (§ 4 Abs. 2 Nr. 3 SPO). Dazu gehören Vorträge, Marktstudien, Projekte, Fallstudien und Berichte.

 

Die Veranstaltung wird mit “bestanden” oder “nicht bestanden” bewertet (§ 7 Abs. 3 SPO). Zum Bestehen der Veranstaltung müssen alle Teilaufgaben erfolgreich bestanden werden.


\end{assessment}

\begin{conditions}Es müssen weitere Vorlesungen/Seminare im Umfang von insgesamt mindestens 5 LP aus dem Gebiet der Informationssysteme am Lehrstuhl Prof. Böhm gehört worden sein bzw. im selben Semester belegt werden. Dazu zählt nicht die Veranstaltung \emph{Datenbanksysteme} [24516].

\end{conditions}


\end{styleenv}

\begin{learningoutcomes}
Am Ende der Lehrveranstaltung sollen die Teilnehmer

 \begin{itemize}\item Wissen und Verständnis für den Ablauf des Prozesses der Allgemeinen Unternehmensberatung entwickelt haben,  \item Wissen und Verständnis für die Funktions-spezifische DV-Beratung entwickelt haben,  \item einen Überblick über Beratungsunternehmen bekommen haben,  \item konkrete Beispiele der Unternehmensberatung kennen,  \item erfahren haben, wie effektive Arbeit im Team funktioniert, sowie  \item einen Einblick in das berufliche Tätigkkeitsfeld “Beratung” bekommen haben.  \end{itemize}
\end{learningoutcomes}

\begin{content}
Der Markt für Beratungsleistungen wächst jährlich um 20\% und ist damit eine der führenden Wachstumsbranchen und Arbeitsfelder der Zukunft. Dieser Trend wird insbesondere durch die Informatik vorangetrieben. Dort verschiebt die Verbreitung von Standardsoftware den Schwerpunkt des zukünftigen Arbeitsfeldes von der Entwicklung vermehrt in den Bereich der Beratung. Beratungsleistungen sind dabei i.a. sehr breit definiert und reichen von der reinen DV-bezogenen Beratung (z.B. SAP Einführung) bis hin zur strategischen Unternehmensberatung (Strategie, Organisation etc.). Entgegen verbreiteter Vorurteile sind hierfür BWL-Kenntnisse nicht zwingend. Dies eröffnet gerade für Studenten der Informatik den Einstieg in ein abwechslungsreiches und spannendes Arbeitsfeld mit herausragenden Entwicklungsperspektiven.\newline
\newline
In der Vorlesung werden thematisch die Bereiche Allgemeine Unternehmensberatung und Funktions-spezifische Beratung (am Beispiel der DV-Beratung) behandelt. Die Struktur der Vorlesung orientiert sich dabei an den Phasen eines Beratungsprojekts:

 \begin{itemize}\item Diagnose: Der Berater als analytischer Problemlöser.  \item Strategische Neuausrichtung/Neugestaltung der Kernprozesse: Optimierung/Neugestaltung wesentlicher Unternehmensfunktionen zur Lösung des diagnostizierten Problems in gemeinschaftlicher Arbeit mit dem Klienten.  \item Umsetzung: Verankerung der Maßnamen in der Klientenorganisation zur Sicherstellung der Implementierung.  \end{itemize}

Thematische Schwerpunkte der Vorlesung sind:

 \begin{itemize}\item Elementare Problemlösung: Problemdefinition, Strukturierung von Problemen und Fokussierung durch Anwendung von Werkzeugen (z.B. Logik- und Hypothesenbäume), Kreativitätstechniken, Lösungssysteme etc.  \item Effektive Gewinnung von Informationen: Zugriff auf Informationsquellen, Interviewtechniken etc.  \item Effektive Kommunikation von Erkenntnissen/Empfehlungen: Kommunikationsanalyse/-planung (Medien, Zuhörerschaft, Formate), Kommunikationsstile (z.B. Top-down vs. Bottom-up), Sonderthemen (z.B. Darstellung komplexer Informationen) etc.  \item Effizientes Arbeiten im Team: Hilfsmittel zur Optimierung effizienter Arbeit, Zusammenarbeit mit Klienten, intellektuelle und Prozess-Führerschaft im Team etc.  \end{itemize}
\end{content}

\begin{media}Folien, Fallstudien.

\end{media}



\begin{remarks}Die Plätze sind begrenzt und die Anmeldung findet durch das Sekretariat Prof. Böhm statt.

 

Die Veranstaltung findet planmäßig alle drei Semester statt. Das nächste mal im Wintersemester 2009/2010.

\end{remarks}

\end{course}