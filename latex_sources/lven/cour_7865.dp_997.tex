% Lehrveranstaltungsbeschreibung
% Informationsgrad : extern
% Sprache: de
\begin{course}

\setdoclanguagegerman
\coursedegreeprogramme{Informatik}
\coursemodulename{Einführung in Algebra und Zahlentheorie (S.~\pageref{mod_3095.dp_997})[IN3MATHAG02]}
\courseID{1524}
\coursename{Einführung in Algebra und Zahlentheorie}
\coursecoordination{F. Herrlich, S. Kühnlein, C. Schmidt}

\documentdate{2011-10-06 17:44:29.907874}

\courselevel{}
\coursecredits{9}
\courseterm{Sommersemester}
\coursehours{6}
\courseinstructionlanguage{}

\coursehead

% For index (key word@display). Key word is used for sorting - no Umlauts please.
\index{Einfuehrung in Algebra und Zahlentheorie@Einführung in Algebra und Zahlentheorie}

% For later referencing
\label{cour_7865.dp_997}


\begin{styleenv}
\begin{assessment}
Die Erfolgskontrolle wird in der Modulbeschreibung erläutert.


\end{assessment}

\begin{conditions}Keine.\end{conditions}

\begin{recommendations}Folgende Module sollten bereits belegt worden sein (Empfehlung):\newline
Lineare Algebra 1+2\newline
Analysis 1+2

\end{recommendations}
\end{styleenv}

\begin{learningoutcomes}
\begin{itemize}\item Beherrschung der grundlegenden algebraischen und zahlentheoretischen Strukturen  \item Einführung in die Denkweise der modernen Algebra  \item Grundlage für Seminare und weiterführende Vorlesungen im Bereich Algebra  \end{itemize}
\end{learningoutcomes}

\begin{content}
\begin{itemize}\item Gruppentheorie  \item Ringtheorie  \item Primzahlen  \item Modulares Rechnen  \end{itemize}
\end{content}







\end{course}