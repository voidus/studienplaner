% Lehrveranstaltungsbeschreibung
% Informationsgrad : extern
% Sprache: de
\begin{course}

\setdoclanguagegerman
\coursedegreeprogramme{Informatik}
\coursemodulename{Energiebewusste Systeme (S.~\pageref{mod_10583.dp_997})[IN3INEBS]}
\courseID{LPD}
\coursename{Praktikum Low Power Design}
\coursecoordination{J. Henkel}

\documentdate{2011-02-24 11:50:15.487997}

\courselevel{4}
\coursecredits{3}
\courseterm{Sommersemester}
\coursehours{2}
\courseinstructionlanguage{de}

\coursehead

% For index (key word@display). Key word is used for sorting - no Umlauts please.
\index{Praktikum Low Power Design@Praktikum Low Power Design}

% For later referencing
\label{cour_7993.dp_997}


\begin{styleenv}
\begin{assessment}
Die Erfolgskontrolle erfolgt nach § 4 Abs. 2 Nr. 3 SPO als Erfolgskontrolle anderer Art. Die Leistungskontrolle erfolgt dabei kontinuierlich für die einzelnen Projekte sowie durch eine Abschlusspräsentation. Die Bewertung ist “bestanden” / “nicht bestanden”.


\end{assessment}

\begin{conditions}Keine.\end{conditions}


\end{styleenv}

\begin{learningoutcomes}
Ein eingebettetes System auf den Leistungsverbrauch hin zu optimieren zu können.


\end{learningoutcomes}

\begin{content}
Low Power Design gehört zu den wichtigsten Entwurfskriterien eingebetteter Systeme, da dadurch speziell die Effizienz mobiler eingebetteter Systeme erhöht wird und eine höhere Verlässlichkeit erzielt werden kann. In dem Praktikum werden Techniken zur Analyse und Optimierung erlernt und angewandt, die zu energieeffizienten eingebetteten Systemen führen.


\end{content}







\end{course}