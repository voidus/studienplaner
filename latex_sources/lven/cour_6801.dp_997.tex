% Lehrveranstaltungsbeschreibung
% Informationsgrad : extern
% Sprache: de
\begin{course}

\setdoclanguagegerman
\coursedegreeprogramme{Informatik}
\coursemodulename{Topics in Finance I (S.~\pageref{mod_1575.dp_997})[IN3WWBWL13]}
\courseID{2530210}
\coursename{Interne Unternehmensrechnung (Rechnungswesen II)}
\coursecoordination{T. Lüdecke}

\documentdate{2012-01-09 19:05:26.325892}

\courselevel{3}
\coursecredits{4,5}
\courseterm{Sommersemester}
\coursehours{2/1}
\courseinstructionlanguage{de}

\coursehead

% For index (key word@display). Key word is used for sorting - no Umlauts please.
\index{Interne Unternehmensrechnung (Rechnungswesen II)@Interne Unternehmensrechnung (Rechnungswesen II)}

% For later referencing
\label{cour_6801.dp_997}


\begin{styleenv}
\begin{assessment}
Die Erfolgskontrolle erfolgt in Form einer schriftlichen Prüfung im Umfang von 60min (nach §4(2), 1 SPO).

 

Die Prüfung wird in jedem Semester angeboten und kann zu jedem ordentlichen Prüfungstermin wiederholt werden.


\end{assessment}

\begin{conditions}Keine.\end{conditions}


\end{styleenv}

\begin{learningoutcomes}
Die Studierenden erlernen den Zweck verschiedener Kostenrechnungssysteme,\newline
die Verwendung von Kosteninformationen für typische Entscheidungs- und\newline
Kontrollrechnungen im Unternehmen sowie den Nutzen gängiger Instrumente des\newline
Kostenmanagements.


\end{learningoutcomes}

\begin{content}
\begin{itemize}\item Einleitung und Überblick  \item Systeme der Kostenrechnung  \item Entscheidungsrechnungen  \item Kontrollrechnungen  \end{itemize}
\end{content}



\begin{literature}\textbf{Weiterführende Literatur:}

 \begin{itemize}\item Coenenberg, A.G. Kostenrechnung und Kostenanalyse, 6. Aufl. 2007.  \item Ewert, R. und Wagenhofer, A. Interne Unternehmensrechnung, 7. Aufl. 2008.  \item Götze, U. Kostenrechnung und Kostenmanagement. 3. Aufl. 2007.  \item Kilger, W., Pampel, J., Vikas, K. Flexible Plankostenrechnung und Deckungsbeitragsrechnung , 11. Aufl. 2002.  \end{itemize}\end{literature}



\end{course}