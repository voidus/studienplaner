% Modulbeschreibung 
% Informationsgrad : extern
% Sprache: de
\begin{module}

\setdoclanguagegerman
\moduledegreeprogramme{Informatik (B.Sc.)}
\modulesubject{}
\moduleID{IN3INEIM}
\modulename{Einführung in Multimedia}
\modulecoordination{P. Deussen}

\documentdate{2011-03-17 15:38:04.615870}

\modulecredits{3}
\moduleduration{1}
\modulecycle{Jedes 2. Semester, Wintersemester}



\modulehead

% For index (key word@display). Key word is used for sorting - no Umlauts please.
\index{Einfuehrung in Multimedia@Einführung in Multimedia (M)}

% For later referencing
\label{mod_4203.dp_997}

\begin{courselist}
24185 & Einführung in Multimedia (S.~\pageref{cour_8437.dp_997}) & 2 & W & 3 & P. Deussen\\
\end{courselist}

\begin{styleenv}
\begin{assessment}
Die Erfolgskontrolle erfolgt in Form einer mündlichen Prüfung im Umfang von i.d.R. 20 Minuten nach § 4 Abs. 2 Nr. 2 SPO.\newline
\newline
Die Modulnote ist die Note der mündlichen Prüfung.


\end{assessment}

\begin{conditions}Keine.\end{conditions}


\end{styleenv}

\begin{learningoutcomes}
Den Studierenden wird in dieser Querschnittsvorlesung ein Überblick über einige Informatikfächer vermittelt.

 

Ferner erhalten die Studierenden Kenntnisse in

 \begin{itemize}\item  der Physiologie des Ohres und der Augen,   \item  der notwendigen Physik.  \end{itemize}
\end{learningoutcomes}

\begin{content}
Multimedia ist eine Querschnittstechnologie, die die unterschiedlichsten Gebiete der Informatik zusammenbindet: Datenverwaltung, Telekommunikation, Mensch-Maschine-Kommunikation, aber auch Fragen der Farben, der Sinnesphysiologie und des Designs.\newline
\newline
 Die Einführungsvorlesung will diese Dinge ansprechen, hauptsächlich aber die folgenden Bereiche behandeln:\newline
 Digitale Behandlung von Tönen, von Bildern und Filmen samt den notwendigen Kompressionstechniken. Aber auch das wichtige Kapitel der Farben eben sowie die Fernseh- und Monitortechnik.


\end{content}



\end{module}

