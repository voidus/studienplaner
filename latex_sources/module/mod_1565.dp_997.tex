% Modulbeschreibung 
% Informationsgrad : extern
% Sprache: de
\begin{module}

\setdoclanguagegerman
\moduledegreeprogramme{Informatik (B.Sc.)}
\modulesubject{EF Betriebswirtschaftslehre}
\moduleID{IN3WWBWL7}
\modulename{Insurance Markets and Management}
\modulecoordination{U. Werner}

\documentdate{2012-01-22 16:46:41.524494}

\modulecredits{9}
\moduleduration{2}
\modulecycle{Jedes Semester}



\modulehead

% For index (key word@display). Key word is used for sorting - no Umlauts please.
\index{Insurance Markets and Management@Insurance Markets and Management (M)}

% For later referencing
\label{mod_1565.dp_997}

\begin{courselist}
2550055 & Principles of Insurance Management (S.~\pageref{cour_6993.dp_997}) & 3/0 & S & 4,5 & U. Werner\\
2530323 & Insurance Marketing (S.~\pageref{cour_6747.dp_997}) & 3/0 & S & 4,5 & E. Schwake\\
2530050 & Private and Social Insurance (S.~\pageref{cour_6369.dp_997}) & 2/0 & W & 2,5 & W. Heilmann, K. Besserer\\
2530350 & Current Issues in the Insurance Industry (S.~\pageref{cour_6373.dp_997}) & 2/0 & S & 2,5 & W. Heilmann\\
2530353 & International Risk Transfer (S.~\pageref{cour_5121.dp_997}) & 2/0 & S & 2,5 & W. Schwehr\\
 INSGAME & Unternehmensplanspiel Versicherungen – INSGAME (S.~\pageref{cour_14347.dp_997}) & 0/2 & W & 3 & U. Werner\\
\end{courselist}

\begin{styleenv}
\begin{assessment}
Die Modulprüfung erfolgt in Form von Teilprüfungen (nach §4(2), 1-3 SPO) über die Kernveranstaltung und weitere Lehrveranstaltungen des Moduls im Umfang von insgesamt mindestens 9 LP. Die Erfolgskontrolle wird bei jeder Lehrveranstaltung dieses Moduls beschrieben.

 

Die Gesamtnote des Moduls wird aus den mit LP gewichteten Noten der Teilprüfungen gebildet und nach der ersten Nachkommastelle abgeschnitten.


\end{assessment}

\begin{conditions}Nur prüfbar in Kombination mit dem Modul \emph{Grundlagen der BWL}.

 

Das Modul ist nur zusammen mit dem Pflichtmodul \emph{Grundlagen der BWL} [IN3WWBWL] prüfbar.

\end{conditions}


\end{styleenv}

\begin{learningoutcomes}
Der/die Studierende

 \begin{itemize}\item kennt und versteht die wirtschaftlichen, rechtlichen und sozialen Rahmenbedingungen des Wirtschaftszweigs Versicherung,  \item kennt und versteht die Grundlagen der Leistungserstellung und des Marketings einer komplexen Dienstleistung.  \end{itemize}
\end{learningoutcomes}

\begin{content}
Das Modul vermittelt Kenntnisse über wirtschaftliche, rechtliche und soziale Rahmenbedingungen des Wirtschaftszweigs Versicherung sowie Grundlagen der Leistungserstellung und des Marketings einer komplexen Dienstleistung.


\end{content}

\begin{remarks}Das Modul wird nicht mehr angeboten. Studierende, die Teile des Moduls bereits absolviert haben, können die restlichen Prüfungsleistungen noch bis incl. WS 2012/13 erbringen.

\end{remarks}

\end{module}

