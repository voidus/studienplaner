% Lehrveranstaltungsbeschreibung
% Informationsgrad : extern
% Sprache: de
\begin{course}

\setdoclanguagegerman
\coursedegreeprogramme{Informatik}
\coursemodulename{Mikroökonomische Theorie (S.~\pageref{mod_2461.dp_997})[IN3WWVWL6]}
\courseID{26240}
\coursename{Wettbewerb in Netzen}
\coursecoordination{K. Mitusch}

\documentdate{2011-12-21 08:41:25.782635}

\courselevel{3}
\coursecredits{4,5}
\courseterm{Wintersemester}
\coursehours{2/1}
\courseinstructionlanguage{de}

\coursehead

% For index (key word@display). Key word is used for sorting - no Umlauts please.
\index{Wettbewerb in Netzen@Wettbewerb in Netzen}

% For later referencing
\label{cour_4959.dp_997}


\begin{styleenv}
\begin{assessment}

\end{assessment}

\begin{conditions}Keine.\end{conditions}

\begin{recommendations}Grundkenntnisse und Fertigkeiten der Mikroökonomie aus einem Bachelorstudium der Ökonomie werden vorausgesetzt. Besonders hilfreich, aber nicht notwendig: Industrieökonomie und Principal-Agent- oder Vertragstheorie.

\end{recommendations}
\end{styleenv}

\begin{learningoutcomes}
Die Vorlesung vermittelt den Studenten das grundlegende ökonomische Verständnis für Netzwerkindustrien wie Telekom-, Versorgungs-, IT- und Verkehrssektoren. Sie bereitet die Studenten auch auf einen möglichen Berufseinstieg in Netzwerkindustrien vor. Der Student soll eine plastische Vorstellung der besonderen Charakteristika von Netzwerkindustrien hinsichtlich Planung, Wettbewerb, Wettbewerbsverzerrung und staatlichem Eingriff bekommen. Er soll in der Lage sein, abstrakte Konzepte und formale Methoden auf diese Anwendungsfelder übertragen zu können.


\end{learningoutcomes}

\begin{content}
Netzwerkindustrien bilden das Rückgrat moderner Volkswirtschaften. Hierzu zählen u.a. Verkehrs-, Versorgungs- oder Kommunikationsnetzwerke. Die Vorlesung stellt die ökonomischen Grundlagen der Netzwerkindustrien dar. Die Planung von Netzwerken unterliegt höheren Komplexitätsanforderungen. Komplexe Interdependenzen zeichnen zudem auch die Wettbewerbsformen auf bzw. mit Netzwerken aus: Netzwerkeffekte, Skaleneffekte, Effekte vertikaler Integration, Wechselkosten, Standardisierung, Kompatibilität usw. treten in diesen Sektoren verstärkt und in Kombination auf. Hinzu kommen staatliche Eingriffe, die teils wettbewerbspolitisch, teils industriepolitisch intendiert sind. Alle diese Themen werden in der Vorlesung angesprochen, analysiert und durch zahlreiche praktische Beispiele illustriert und abgerundet.


\end{content}



\begin{literature}Literatur und Skripte werden in der Veranstaltung angegeben.

\end{literature}



\end{course}