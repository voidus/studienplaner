% Lehrveranstaltungsbeschreibung
% Informationsgrad : extern
% Sprache: de
\begin{course}

\setdoclanguagegerman
\coursedegreeprogramme{Informatik}
\coursemodulename{eFinance (S.~\pageref{mod_2729.dp_997})[IN3WWBWL15]}
\courseID{2511402}
\coursename{Intelligente Systeme im Finance}
\coursecoordination{D. Seese}

\documentdate{2011-09-22 10:52:52.805313}

\courselevel{4}
\coursecredits{5}
\courseterm{Sommersemester}
\coursehours{2/1}
\courseinstructionlanguage{de}

\coursehead

% For index (key word@display). Key word is used for sorting - no Umlauts please.
\index{Intelligente Systeme im Finance@Intelligente Systeme im Finance}

% For later referencing
\label{cour_4637.dp_997}


\begin{styleenv}
\begin{assessment}
Die Erfolgskontrolle erfolgt in Form einer schriftlichen Prüfung (Klausur) nach §4, Abs. 2, 1 der Prüfungsordnung für Informationswirtschaft in der ersten Woche nach Ende der Vorlesungszeit des Semesters.

 

Bei einer zu geringen Zahl von Anmeldungen für die Klausur ist eine mündliche Prüfung möglich.

 

\textbf{Voraussetzungen} für die \textbf{Zulassung} zur Prüfung:

 \begin{itemize}\item Bearbeitung und Abgabe von 2 Sonderübungsblättern zu den veröffentlichten Fristen. Die Sonderübungen werden bewertet und anschließend in der zugehörigen Übung besprochen. Pro Übung können 10 Punkte erreicht werden, für die Zulassung zur Prüfung sind mindestens 12 Punkte erforderlich. Die Punkte der Übung können nicht als Bonuspunkte für die Klausur angerechnet werden.  \item Anwesenheitspflicht in der Sonderübung und Bereitschaft des Vorstellens seiner Ergebnisse in der Übung  \end{itemize}

Die Prüfung wird in jedem Semester angeboten und kann zu jedem ordentlichen Prüfungstermin wiederholt werden.


\end{assessment}

\begin{conditions}Keine.\end{conditions}


\end{styleenv}

\begin{learningoutcomes}
\begin{itemize}\item Die Studierenden erwerben Fähigkeiten und Kenntnisse von Methoden und Systemen aus dem Bereich Maschinelle Lernverfahren und lernen deren Einsatzmöglichkeiten im Kernanwendungsbereich Finance kennen.  \item Es wird die Fähigkeit vermittelt diese Methoden und Systeme situationsangemessen auszuwählen, zu gestalten und zur Problemlösung im Bereich Finance einzusetzen.  \item Die Studierenden erhalten die Befähigung zum Finden strategischer und kreativer Antworten bei der Suche nach Lösungen für genau definierte, konkrete und abstrakte Probleme.  \item Dabei zielt diese Vorlesung auf die Vermittlung von Grundlagen und Methoden im Kontext ihrer Anwendungsmöglichkeiten in der Praxis ab. Auf der Basis eines grundlegenden Verständnisses der Konzepte und Methoden der Informatik sollten die Studierenden in der Lage sein, die heute im Berufsleben auf sie zukommenden, rasanten Entwicklungen im Bereich der Informatik schnell zu erfassen und richtig einzusetzen.  \end{itemize}
\end{learningoutcomes}

\begin{content}
Gegenwärtig wird eine neue Generation von Berechnungsmethoden, allgemein bezeichnet als „Intelligente Systeme”, bei verschiedenen wirtschaftlichen und finanziellen Modellierungsaufgaben eingesetzt. Dabei erzielen diese Methoden oftmals bessere Ergebnisse als klassische statistische Ansätze. Die Vorlesung setzt sich zum Ziel, eine fundierte Einführung in die Grundlagen dieser Techniken und deren Anwendungen zu geben. Vorgestellt werden intelligente Softwareagenten, Genetische Algorithmen, Neuronale Netze, Support Vector Machines, Fuzzy-Logik, Expertensysteme und intelligente Hybridsysteme. Der Anwendungsschwerpunkt wird auf dem Bereich Finance liegen. Speziell behandelt werden dabei Risk Management (Credit Risk und Operational Risk), Aktienkursanalyse und Aktienhandel, Portfoliomanagement und ökonomische Modellierung. Zur Sicherung eines starken Anwendungsbezugs wird die Vorlesung in Kooperation mit der Firma msgGILLARDON vorbereitet. Die Vorlesung startet mit einer Einführung in Kernfragestellungen des Bereichs, z.B. Entscheidungsunterstützung für Investoren, Portfolioselektion unter Nebenbedingungen, Aufbereitung von Fundamentaldaten aus Geschäftsberichten, Entdeckung profitabler Handelsregeln in Kapitalmarktdaten, Modellbildung für nicht rational erklärbare Kursverläufe an Kapitalmärkten, Erklärung beobachtbarer Phänomene am Kapitalmarkt erklären, Entscheidungsunterstützung im Risikomanagement (Kreditrisiko, operationelles Risiko). Danach werden Grundlagen intelligenter Systeme besprochen. Es schließen sich die Grundideen und Kernresultate zu verschiedenen stochastischen heuristischen Ansätzen zur lokalen Suche an, insbesondere Hill Climbing, Simulated Annealing, Threshold Accepting und Tabu Search. Danach werden verschiedene populationsbasierte Ansätze evolutionärer Verfahren, speziell Genetische Algorithmen, Evolutionäre Strategien und Programmierung, Genetische Programmierung, Memetische Algorithmen und Ameisenalgorithmen. Danach werden grundlegende Konzepte und Methoden aus den Bereichen Neuronalse Netze, Support Vector Machines und Fuzzylogik besprochen. Es folgen Ausführungen zu Softwareagenten und agentenbasierten Finanzmarktmodellen. Die Vorlesung schließt mit einem Überblick über die Komplexität algorithmischer Probleme im Bereich Finance und motiviert dadurch die Notwerndigkeit zur Benutzung intelligenter Methoden und Heuristiken.


\end{content}

\begin{media}Folien.

\end{media}

\begin{literature}Es existiert kein Lehrbuch, welches den Vorlesungsinhalt vollständig abdeckt.

 \begin{itemize}\item Z. Michalewicz, D. B. Fogel. How to Solve It: Modern Heuristics. Springer 2000.  \item J. Hromkovic. Algorithms for Hard Problems. Springer-Verlag, Berlin 2001.  \item P. Winker. Optimization Heuristics in Econometrics. John Wiley \& Sons, Chichester 2001.  \item Christopher M. Bishop: Pattern Recognition and Machine Learning,Springer 2006  \item A. Brabazon, M. O’Neill. Biologically Inspired Algorithms for Financial Modelling. Springer, 2006.  \item A. Zell. Simulation Neuronaler Netze. Addison-Wesley 1994.  \item R. Rojas. Theorie Neuronaler Netze. Springer 1993.  \item N. Cristianini, J. Shawe-Taylor. An Introduction to Support Vector Machines and other kernal-based learning methods. Cambridge University Press 2003.  \item G. Klir, B. Yuan. Fuzzy Sets and Fuzzy Logic: Theory and Applications. Prentice-Hall, 1995.  \item F. Schlottmann, D. Seese. Modern Heuristics for Fiance Problems: A Survey of Selected Methods and Applications. In S. T. Rachev (Ed.) Handbook of Computational and Numerical Mrthods in Finance, Birkhäuser, Boston 2004, pp. 331 - 359.  \end{itemize}

Weitere Literatur wird in den jeweiligen Vorlesungsabschnitten angegeben.

 

\textbf{Weiterführende Literatur:}

 \begin{itemize}\item S. Goonatilake, Ph. Treleaven (Eds.). Intelligent Systems for Finance and Business. John Wiley \& Sons, Chichester 1995.  \item F. Schlottmann, D. Seese. Financial applications of multi-objective evolutionary algorithms, recent developments and future directions. Chapter 26 of C. A. Coello Coello, G. B.Lamont (Eds.) Applications of Multi-Objective Evolutionary Algorithms, World Scientific, New Jersey 2004, pp. 627 - 652.  \item D. Seese, F. Schlottmann. Large grids and local information flow as reasons for high complexity. In: G. Frizelle, H. Richards (eds.), Tackling industrial complexity: the ideas that make a difference, Proceedings of the 2002 conference of the Manufacturing Complexity Network, University of Cambridge, Institute of Manufacturing, 2002, pp. 193-207. (ISBN 1-902546-24-5).  \item R. Almeida Ribeiro, H.-J. Zimmermann, R. R. Yager, J. Kacprzyk (Eds.). Soft Computing in Financial Engineering. Physica-Verlag, 1999.  \item S. Russel, P. Norvig. Künstliche Intelligenz Ein moderner Ansatz. 2. Auflage, Pearson Studium, München 2004.  \item M. A. Arbib (Ed.). The Handbook of Brain Theory and neural Networks (second edition). The MIT Press 2004.  \item J.E. Gentle, W. Härdle, Y. Mori (Eds.). Handbook of Computational Statistics. Springer 2004.  \item F. Schweitzer. Brownian Agents and Active Particles. Collective Dynamics in the Natural and Social Sciences, Springer 2003.  \item D. Seese, C. Weinhardt, F. Schlottmann (Eds.) Handbook on Information Technology in Finance, Springer 2008.  \item Weitere Referenzen werden in der Vorlesung angegeben.  \end{itemize}\end{literature}

\begin{remarks}Der Inhalt der Vorlesung wird ständig an neue Entwicklungen angepasst. Dadurch können sich Veränderungen zum oben beschriebenen Stoff und Ablauf ergeben.

 

Bitte beachten Sie, dass die Lehrveranstaltung “Intelligente Systeme im Finance” im SS 2016 NICHT mehr angeboten wird! Die Prüfung wird noch bis mindestens Sommersemester 2015 angeboten. Eine letztmalige Wiederholungsprüfung wird es im Sommersemester 2015 geben (nur für Nachschreiber)!

\end{remarks}

\end{course}