% Lehrveranstaltungsbeschreibung
% Informationsgrad : extern
% Sprache: de
\begin{course}

\setdoclanguagegerman
\coursedegreeprogramme{Informatik}
\coursemodulename{Topics in Finance I (S.~\pageref{mod_1575.dp_997})[IN3WWBWL13], eBusiness und Service Management (S.~\pageref{mod_1611.dp_997})[IN3WWBWL2], eFinance (S.~\pageref{mod_2729.dp_997})[IN3WWBWL15]}
\courseID{2540454}
\coursename{eFinance: Informationswirtschaft für den Wertpapierhandel}
\coursecoordination{R. Riordan}

\documentdate{2011-12-20 16:01:44.485931}

\courselevel{4}
\coursecredits{4,5}
\courseterm{Wintersemester}
\coursehours{2/1}
\courseinstructionlanguage{en}

\coursehead

% For index (key word@display). Key word is used for sorting - no Umlauts please.
\index{eFinance: Informationswirtschaft fuer den Wertpapierhandel@eFinance: Informationswirtschaft für den Wertpapierhandel}

% For later referencing
\label{cour_4851.dp_997}


\begin{styleenv}
\begin{assessment}
Die Erfolgskontrolle erfolgt in Form einer schriftlichen Prüfung (Klausur) nach § 4 Abs. 2 Nr. 1 SPO und durch Ausarbeiten von Übungsaufgaben als Erfolgskontrolle anderer Art nach § 4 Abs. 2 Nr. 3 SPO. In die Benotung geht die Klausur zu 70\% und die Übung zu 30\% ein.


\end{assessment}

\begin{conditions}Keine.\end{conditions}


\end{styleenv}

\begin{learningoutcomes}
Die Studierenden

 \begin{itemize}\item können die theoretischen und praktischen Aspekte im Wertpapierhandel verstehen  \item können relevanten elektronischen Werkzeugen für die Auswertung von Finanzdaten bedienen  \item können die Anreize der Händler zur Teilnahme an verschiedenen Marktplattformen identifizieren,  \item können Finanzmarktplätze hinsichtlich ihrer Effizienz und ihrer Schwächen und ihrer technischen Ausgestaltung analysieren  \item können theoretische Methoden aus dem Ökonometrie anwenden,  \item können finanzwissenschaftliche Artikel verstehen, kritisieren und wissenschaftlich präsentieren,  \item lernen die Erarbeitung von Lösungen in Teams  \end{itemize}
\end{learningoutcomes}

\begin{content}
Der theoretische Teil der Vorlesung beginnt mit der Neuen Institutionenökonomik, die unter anderem eine theoretisch fundierte Begründung für die Existenz von Finanzintermediären und Märkten liefert. Hierauf aufbauend werden auf der Grundlage der Marktmikrostruktur die einzelnen Einflussgrößen und Erfolgsfaktoren des elektronischen Wertpapierhandels untersucht. Diese entlang des Wertpapierhandelsprozesses erarbeiteten Erkenntnisse werden durch die Analyse von am Lehrstuhl entstandenen prototypischen Handelssystemen und ausgewählten - aktuell im Börsenumfeld zum Einsatz kommenden - Systemen vertieft und verifiziert. Im Rahmen dieses praxisnahen Teils der Vorlesung werden ausgewählte Referenten aus der Praxis die theoretisch vermittelten Inhalte aufgreifen und die Verbindung zu aktuell im Wertpapierhandel eingesetzten Systemen herstellen.


\end{content}

\begin{media}\begin{itemize}\item Folien  \item Aufzeichnung der Vorlesung im Internet  \end{itemize}\end{media}

\begin{literature}\begin{itemize}\item Picot, Arnold, Christine Bortenlänger, Heiner Röhrl (1996): “Börsen im Wandel”. Knapp, Frankfurt  \item Harris, Larry (2003): “Trading and Exchanges - Market Microstructure for Practitioners””. Oxford University Press, New York  \end{itemize}

\textbf{Weiterführende Literatur:}

 \begin{itemize}\item Gomber, Peter (2000): “Elektronische Handelssysteme - Innovative Konzepte und Technologien”. Physika Verlag, Heidelberg  \item Schwartz, Robert A., Reto Francioni (2004): “Equity Markets in Action - The Fundamentals of Liquidity, Market Structure and Trading”. Wiley, Hoboken, NJ  \end{itemize}\end{literature}



\end{course}