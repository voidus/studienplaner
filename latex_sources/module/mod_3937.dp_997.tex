% Modulbeschreibung 
% Informationsgrad : extern
% Sprache: de
\begin{module}

\setdoclanguagegerman
\moduledegreeprogramme{Informatik (B.Sc.)}
\modulesubject{EF E-Technik}
\moduleID{IN3EITST}
\modulename{Systemtheorie}
\modulecoordination{F. Puente León }

\documentdate{2010-08-02 12:35:32.447010}

\modulecredits{15}
\moduleduration{2}
\modulecycle{Jedes Semester}



\modulehead

% For index (key word@display). Key word is used for sorting - no Umlauts please.
\index{Systemtheorie@Systemtheorie (M)}

% For later referencing
\label{mod_3937.dp_997}

\begin{courselist}
23109 & Signale und Systeme (S.~\pageref{cour_7991.dp_997}) & 2/1 & W & 5 & F. Puente León\\
23105 & Messtechnik (S.~\pageref{cour_7997.dp_997}) & 2/1 & S & 5 & F. Puente León\\
23155 & Systemdynamik und Regelungstechnik (S.~\pageref{cour_6891.dp_997}) & 2/1 & S & 5 & M. Kluwe\\
\end{courselist}

\begin{styleenv}
\begin{assessment}
Die Erfolgskontrolle wird in den Lehrveranstaltungsbeschreibungen erläutert.

 

Die Gesamtnote des Moduls wird aus den mit LP gewichteten Noten der Teilprüfungen gebildet und nach der ersten Kommastelle abgeschnitten.


\end{assessment}

\begin{conditions}Dieses Modul muss in Kombination mit dem Modul “Praktikum Automation und Information” [IN3EITPAI] absolvieren werden.

\end{conditions}

\begin{recommendations}Die Studierenden sollten mit den Grundlagen von Integraltransformation vertraut sein.

\end{recommendations}
\end{styleenv}

\begin{learningoutcomes}
Ziel ist die Vermittlung fundamentaler Kenntnisse auf dem Gebiet der Systemtheorie. So werden die Studierenden zum einen mit den Grundlagen der Signal- und Systemtheorie vertraut gemacht und erlernen die elementaren Methoden zur Analyse und den Entwurf von Regelungen und Steuerungen. Zum anderen erfolgt eine Einführung in die Verfahren der Messtechnik.


\end{learningoutcomes}

\begin{content}
Der Inhalt ergibt sich aus den Inhalten der einzelnen Lehrveranstaltungen.


\end{content}



\end{module}

