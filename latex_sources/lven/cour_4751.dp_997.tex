% Lehrveranstaltungsbeschreibung
% Informationsgrad : extern
% Sprache: de
\begin{course}

\setdoclanguagegerman
\coursedegreeprogramme{Informatik}
\coursemodulename{Grundlagen des Marketing (S.~\pageref{mod_1621.dp_997})[IN3WWBWL9]}
\courseID{2572177}
\coursename{Markenmanagement}
\coursecoordination{B. Neibecker}

\documentdate{2011-12-21 15:40:15.851530}

\courselevel{3}
\coursecredits{4,5}
\courseterm{Wintersemester}
\coursehours{2/1}
\courseinstructionlanguage{de}

\coursehead

% For index (key word@display). Key word is used for sorting - no Umlauts please.
\index{Markenmanagement@Markenmanagement}

% For later referencing
\label{cour_4751.dp_997}


\begin{styleenv}
\begin{assessment}
Die Erfolgskontrolle erfolgt in Form einer schriftlichen Prüfung (60 min.) (nach §4(2), 1 SPO).


\end{assessment}

\begin{conditions}Keine.\end{conditions}


\end{styleenv}

\begin{learningoutcomes}
Die Studierenden erwerben folgende Fähigkeiten:\newline
• Auflisten der Schlüsselbegriffe im Markenmanagement\newline
• Erkennen und definieren von betriebswirtschaftlichen Konstrukten zur Steuerung von Marken\newline
• Identifizieren wichtiger Forschungstrends\newline
• Analysieren und interpretieren von wissenschaftlichen Journalbeiträgen\newline
• Entwickeln von Teamfähigkeit (”weiche” Kompetenz) und Planungskompetenz (”harte” Faktoren)\newline
• Beurteilung von methodisch fundierten Forschungsergebnissen und vorbereiten praktischer Handlungsanweisungen und Empfehlungen


\end{learningoutcomes}

\begin{content}
Die Studierenden sollen grundlegende wissenschaftliche und praktische Ansätze des Marketing am konkreten Managementproblem der Markenführung erlernen. Es wird vermittelt, wie der Aufbau von Marken der Identifizierung von Waren und Dienstleistungen eines Unternehmens dient und die Differenzierung von den Wettbewerbern fördert. Konzepte wie: Markenpositionierung, Wertschätzung, Markenloyalität und Markenwert werden als zentrale Ziele eines erfolgreichen Markenmanagement vermittelt. Hierbei steht nicht nur die kurzfristige Gewinnerzielung im Fokus, sondern auch die langfristige Strategie der Markenführung mit einer kontinuierlichen Kommunikation gegenüber Konsumenten und weiteren Anspruchsgruppen wie z.B. Kapitalgebern und dem Staat. Die Strategien und Techniken der Markenführung werden durch Auszüge aus verschiedenen Fallstudien vertieft. Hierbei wird auch Englisch als internationale Fachsprache im Marketing durch entsprechende Folien und wissenschaftliche Fachartikel vermittelt. Zum Inhalt:

 

Zunächst wird ein Zielsystem der Markenführung entwickelt und managementorientierte Kriterien zur Markendefinition diskutiert. Aufbauend auf den psychologischen und sozialen Grundlagen des Konsumentenverhaltens werden wichtige Aspekte einer integrierten Marketing-Kommunikation vermittelt. In einem Stragieteil werden grundlegende Markenstrategien verglichen. Das Konzept der Markenpersönlichkeit wird sowohl von praktischer Seite, als auch aus wissenschaftlicher Sicht diskutiert. Methoden zur Messung des kundenorientierten Markenwertes werden den finanzorientierten Verfahren gegenüber gestellt und anlassspezifisch integriert. Eine Analyse der “Brand Equity Driverrundet zusammen mit Auszügen aus Fallstudien das inhaltliche Angebot ab. An einem wissensbasierten System zur Werbewirkungsanalyse wird gezeigt, wie das vermittelte Wissen systematisch gebündelt und angewendet werden kann.


\end{content}

\begin{media}Folien, Powerpoint Präsentationen, Website mit Online-Vorlesungsunterlagen

\end{media}

\begin{literature}\begin{itemize}\item Aaker, J. L.: Dimensions of Brand Personality. In: Journal of Marketing Research 34, 1997, 347-356.  \item BBDO-Düsseldorf (Hrsg.): Brand Equity Excellence. 2002.  \item BBDO-Düsseldorf (Hrsg.): Brand Equity Drivers Modell. 2004.  \item Bruhn, M. und GEM: Was ist eine Marke? Gräfelfing: Albrecht (voraussichtlich 2003).  \item Esch, F.-R.: Strategie und Technik der Markenführung. München: Vahlen 2003.  \item Keller, K. L.: Kundenorientierte Messung des Markenwerts. In: Esch, F.-R. (Hrsg.): Moderne Markenführung. 3. Aufl. 2001.  \item Kotler, P.; V. Wong; J. Saunders und G. Armstrong: Principles of Marketing (European Edition). Harlow: Pearson 2005.  \item Krishnan, H. S.: Characteristics of memory associations: A consumer-based brand equity perspective. In: Internat. Journal of Research in Marketing 13, 1996, 389-405.  \item Leesch, C.: Stabilität oder Fragilität des Effekts des regulatorischen Fits? Marketing ZFP 33, 2011, 19-31.  \item Meffert, H.; C. Burmann und M. Koers (Hrsg.): Markenmanagement. Grundfragen der identitätsorientierten Markenführung. Wiesbaden: Gabler 2002.  \item Neibecker, B.: Tachometer-ESWA: Ein werbewissenschaftliches Expertensystem in der Beratungspraxis. In: Computer Based Marketing, H. Hippner, M. Meyer und K. D. Wilde (Hrsg.), Vieweg: 1998, 149-157.  \item Riesenbeck, H. und J. Perrey: Mega-Macht Marke. McKinsey\&Company, Frankfurt/Wien: Redline 2004.  \item Solomon, M., G. Bamossy, S. Askegaard und M. K. Hogg: Consumer Behavior, 4rd ed., Harlow: Pearson 2010.  \end{itemize}\end{literature}



\end{course}