% Modulbeschreibung 
% Informationsgrad : extern
% Sprache: de
\begin{module}

\setdoclanguagegerman
\moduledegreeprogramme{Informatik (B.Sc.)}
\modulesubject{EF Betriebswirtschaftslehre}
\moduleID{IN3WWBWL9}
\modulename{Grundlagen des Marketing}
\modulecoordination{B. Neibecker}

\documentdate{2011-09-23 10:31:49.667155}

\modulecredits{9}
\moduleduration{1}
\modulecycle{Jedes Semester}



\modulehead

% For index (key word@display). Key word is used for sorting - no Umlauts please.
\index{Grundlagen des Marketing@Grundlagen des Marketing (M)}

% For later referencing
\label{mod_1621.dp_997}

\begin{courselist}
2572177 & Markenmanagement (S.~\pageref{cour_4751.dp_997}) & 2/1 & W & 4,5 & B. Neibecker\\
2571152 & Marketing Mix (S.~\pageref{cour_15379.dp_997}) &  &  & 4,5 & M. Klarmann\\
 & Dienstleistungs- und B2B Marketing (S.~\pageref{cour_15381.dp_997}) &  &  & 3 & M. Klarmann\\
 & International Marketing (S.~\pageref{cour_15383.dp_997}) &  &  & 1,5 & M. Klarmann\\
\end{courselist}

\begin{styleenv}
\begin{assessment}
  
\end{assessment}

\begin{conditions}Nur prüfbar in Kombination mit dem Modul \emph{Grundlagen der BWL}.

  

Das Modul ist nur zusammen mit dem Pflichtmodul \emph{Grundlagen der BWL} [IN3WWBWL] prüfbar.

\end{conditions}


\end{styleenv}

\begin{learningoutcomes}
Der/die Studierende

 \begin{itemize}\item soll grundlegende, fundierte Kenntnisse des Marketing und der Marktforschung erlangen,   \item soll in die Lage versetzt werden, Marktdaten zu interpretieren und die Auswirkungen von Marketingentscheidungen zu beurteilen,   \item kennt und versteht die typischen Marketingprobleme,  \item ist in der Lage, Standard-Marketing Fragestellungen im beruflichen Umfeld bearbeiten zu können.   \end{itemize}

Die im Modul vermittelten Kenntnisse bieten eine gute Grundlage für weitergehende Studien mit Marketingbezug im Masterstudiengang.


\end{learningoutcomes}

\begin{content}
Zu den Grundlagen des Marketing gehören u.a.: Ansätze und Theorien zum Konsumenten- und Kaufverhalten: Prinzip und Bedeutung der Aktivierung, Umweltspezifische Aspekte des Konsumentenverhaltens, Aspekte der Informationsaufnahme, -verarbeitung und -speicherung, Bedeutung von Emotionen, Motiven und Einstellungen, Denken und Lernen bei der Kaufentscheidung, Einzelhandel und Kaufverhalten, Methoden der empirischen Konsumentenverhaltensforschung, Marketingpolitische Instrumente, Produktpolitische Maßnahmen, Produktpositionierung im Wettbewerbsumfeld, produktspezifische Marktsegmentierung, Distributionspolitische Entscheidungen und Marketing-Logistik, Entgeltpolitische Instrumente und Preisoptimierung, Kommunikationspolitische Instrumente und Werbewirkungskontrolle, Entscheidungsverhalten und Reiz-Reaktions-Schema, Beeinflussungsmöglichkeiten durch Werbung, Steuerungstechniken der Werbung.\newline
Ausgehend vom Internet als Kommunikationsplattform werden Beziehungen zwischen Web Mining und Problemstellungen der Marktforschung aufgezeigt. Zusätzlich vorgestellt und diskutiert werden multivariate Analyseverfahren in der Marktforschung wie z.B. Clusteranalyse, Multidimensionale Skalierung, Conjoint-Analyse, Faktorenanalyse, Diskriminanzanalyse.\newline
Beim Markenmanagement werden u.a. Ziele der Markenführung und Markenstrategien, Markenpersönlichkeit, Markenwert und Markenwertmessung durch Assoziationstechniken (kundenorientierter Ansatz) angesprochen.

 

Dem Institut ist es ein Anliegen, dass Studierende möglichst viele Lehrangebote selbst zu einem Modul zusammenstellen können. Deshalb erfolgt eine Einteilung in Kern- und Ergänzungsveranstaltungen. Kernveranstaltungen gehören zum Pflichtprogramm der angebotenen Module, Ergänzungsveranstaltungen können nach eigenem Ermessen, im Rahmen der angegebenen Bedingungen, hinzugewählt werden.


\end{content}



\end{module}

