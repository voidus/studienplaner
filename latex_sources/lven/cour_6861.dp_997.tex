% Lehrveranstaltungsbeschreibung
% Informationsgrad : extern
% Sprache: de
\begin{course}

\setdoclanguagegerman
\coursedegreeprogramme{Informatik}
\coursemodulename{Real Estate Management (S.~\pageref{mod_1641.dp_997})[IN3WWBWL17]}
\courseID{2585400/2586400}
\coursename{Real Estate Management II}
\coursecoordination{T. Lützkendorf}

\documentdate{2011-06-27 15:17:05.719382}

\courselevel{3}
\coursecredits{4,5}
\courseterm{Sommersemester}
\coursehours{2/2}
\courseinstructionlanguage{de}

\coursehead

% For index (key word@display). Key word is used for sorting - no Umlauts please.
\index{Real Estate Management II@Real Estate Management II}

% For later referencing
\label{cour_6861.dp_997}


\begin{styleenv}
\begin{assessment}
Die jeweiligen Prüfungen zu den Lehrveranstaltungen erfolgen i.d.R durch eine 60-minütige Klausur. Eine 20-minütige mündliche Prüfung wird i.d.R. nur nach der zweiten nicht erfolgreich absolvierten Prüfung zugelassen. Die jeweilige Teilprüfung (REM I bzw. REM II) erfolgt nur in dem Semester, in dem die entsprechende Vorlesung angeboten wird. Derzeit wird damit REM I nur im Wintersemester und REM II nur im Sommersemester geprüft. Die Prüfung wird in jedem Semester zweimal angeboten und kann zu jedem ordentlichen Prüfungstermin wiederholt werden.


\end{assessment}

\begin{conditions}Es wird eine Kombination mit dem Modul \emph{Bauökologie} \emph{I} [IN3WWBWL16] empfohlen. Weiterhin empfehlenswert ist die Kombination mit Lehrveranstaltungen aus den Bereichen

 \begin{itemize}\item Finanzwirtschaft und Banken  \item Versicherungen  \item Bauingenieurwesen und Architektur (Bauphysik, Baukonstruktion, Facility Management)  \end{itemize}\end{conditions}


\end{styleenv}

\begin{learningoutcomes}
Anwendung betriebswirtschaftlicher Methoden auf die Gebiete Immobilienökonomie und nachhaltiges Bauen


\end{learningoutcomes}

\begin{content}
Die Vorlesungsreihe Real Estate Management II greift Fragestellungen im Zusammenhang mit dem Management umfangreicher Immobilienportfolios in der Wohnungs- und Immobilienwirtschaft auf. Themen sind u.a. Wertermittlung, Markt- und Objektrating, Instandhaltungs- und Modernisierungmanagement, Immobilien-Portfoliomanagement und Risikomanagement.\newline
Die Übung dient der Vertiefung und praktischen Anwendung der in der Vorlesung erworbenen Kenntnisse an Beispielen aus der Immobilienwirtschaft.


\end{content}

\begin{media}Die Vorlesungsfolien und ergänzende Unterlagen werden teils als Ausdruck, teils online zur Verfügung gestellt.

\end{media}

\begin{literature}\textbf{Weiterführende Literatur:}

 \begin{itemize}\item Gondring (Hrsg.): „Immobilienwirtschaft: Handbuch für Studium und Praxis“. ISBN 3-8006-2989-5. Vahlen 2004  \item Kühne-Büning (Hrsg.): „Grundlagen der Wohnungs- und Immobilienwirtschaft“. ISBN 3-8314-0706-1. Knapp \& Hammonia-Verlag 2005  \item Schulte (Hrsg.): „Immobilienökonomie Bd. I“. ISBN 3-486-25430-8. Oldenbourg 2000  \end{itemize}\end{literature}

\begin{remarks}Das Angebot wird durch Vorträge von Gästen aus verschiedenen Bereichen der Wohnungswirtschaft und durch Exkursionen ergänzt.

\end{remarks}

\end{course}