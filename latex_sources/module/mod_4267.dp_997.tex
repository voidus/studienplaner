% Modulbeschreibung 
% Informationsgrad : extern
% Sprache: de
\begin{module}

\setdoclanguagegerman
\moduledegreeprogramme{Informatik (B.Sc.)}
\modulesubject{EF Informationsmanagement im Ingenieurwesen}
\moduleID{IN3MACHVE1}
\modulename{Virtual Engineering I}
\modulecoordination{Maier}

\documentdate{2011-07-25 10:56:28.129278}

\modulecredits{6}
\moduleduration{1}
\modulecycle{Unregelmäßig}



\modulehead

% For index (key word@display). Key word is used for sorting - no Umlauts please.
\index{Virtual Engineering I@Virtual Engineering I (M)}

% For later referencing
\label{mod_4267.dp_997}

\begin{courselist}
2121352 & Virtual Engineering I (S.~\pageref{cour_7503.dp_997}) & 2/3 & W & 6 & J. Ovtcharova\\
\end{courselist}

\begin{styleenv}
\begin{assessment}
Die Erfolgskontrolle wird in der Lehrveranstaltungsbeschreibung erläutert.


\end{assessment}

\begin{conditions}Die Module \emph{Virtual Engineering I}\emph{I} [IN3INMACHVE2] und \emph{Product Lifecycle Management} [IN3MACHPMI] müssen geprüft werden.

\end{conditions}


\end{styleenv}

\begin{learningoutcomes}
Der/ die Studierende

 \begin{itemize}\item Versteht das Konzept des Virtual Engineering im Kontext der Virtuellen Produktentstehung,   \item Besitzt grundlegende Kenntnisse in den Bereichen Product Lifecycle Management, Computer Aided Design, Computer Aided Engineering, Computer Aided Manufacturing,  \item ist in der Lage, gängige CAx- und PLM-Systeme im Produktentstehungsprozess einzusetzen.   \end{itemize}
\end{learningoutcomes}

\begin{content}
Die Vorlesung vermittelt die Informationstechnischen Zusammenhänge der virtuellen Produktentstehung. Dabei stehen die in der industriellen Praxis verwendeten IT-Systeme zur Unterstützung der Prozesskette des Virtual Engineerings im Mittelpunkt:

 \begin{itemize}\item \textbf{Product Lifecycle Management} befasst sich mit der Datenverwaltung und -integration über den gesamten Lebenszyklus eines Produktes, angefangen mit der Konzeptphase bis zu Demontage und Recycling;   \item \textbf{CAx-Systeme} für die virtuelle Produktentstehung ermöglichen die erweiterte geometrische und funktionale Modellierung des digitalen Produktes im Hinblick auf die Planung, Konstruktion, Fertigung, Montage und Wartung;   \item \textbf{Validierungssysteme }ermöglichen die Überprüfung des Produktes im Hinblick auf Statik, Dynamik, Sicherheit und Baubarkeit;   \end{itemize}
\end{content}

\begin{remarks}\textcolor{red}{Das Modul wird im Bachelor-Studiengang nicht mehr angeboten, Prüfungen sind möglich bis Wintersemester 2012/13.}

\end{remarks}

\end{module}

