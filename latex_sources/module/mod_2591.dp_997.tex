% Modulbeschreibung 
% Informationsgrad : extern
% Sprache: de
\begin{module}

\setdoclanguagegerman
\moduledegreeprogramme{Informatik (B.Sc.)}
\modulesubject{}
\moduleID{IN3INES2}
\modulename{Entwurf und Architekturen für Eingebettete Systeme (ES2)}
\modulecoordination{J. Henkel}

\documentdate{2011-03-22 09:43:44.969629}

\modulecredits{3}
\moduleduration{1}
\modulecycle{Jedes 2. Semester, Wintersemester}



\modulehead

% For index (key word@display). Key word is used for sorting - no Umlauts please.
\index{Entwurf und Architekturen fuer Eingebettete Systeme (ES2)@Entwurf und Architekturen für Eingebettete Systeme (ES2) (M)}

% For later referencing
\label{mod_2591.dp_997}

\begin{courselist}
24106 & Entwurf und Architekturen für Eingebettete Systeme (ES2) (S.~\pageref{cour_7229.dp_997}) & 2 & W & 3 & J. Henkel\\
\end{courselist}

\begin{styleenv}
\begin{assessment}
Die Erfolgskontrolle erfolgt in Form einer mündlichen Prüfung nach § 4 Abs. 2 Nr. 2 SPO im Umfang von i.d.R. 20 Minuten.

 

Die Modulnote ist die Note der mündlichen Prüfung.


\end{assessment}

\begin{conditions}Der erfolgreiche Abschluss der Module \emph{Technische Informatik }[IN1INTI] und \emph{Rechnerstrukturen} [IN3INRS] sind Voraussetzung.

\end{conditions}


\end{styleenv}

\begin{learningoutcomes}
Erlernen von Methoden zur Beherrschung von Komplexität. \newline
Anwendung dieser Methoden auf den Entwurf eingebetteter Systeme.\newline
Beurteilung und Auswahl spezifischer Architekturen für Eingebettete Systeme.\newline
Zugang zu aktuellen Forschungsthemen erschließen.


\end{learningoutcomes}

\begin{content}
Heutzutage ist es möglich, mehrere Milliarden Transistoren auf einem einzigen Chip zu integrieren und damit komplette SoCs (Systems-On-Chip) zu realisieren. Der Trend, mehr und mehr Transistoren verwenden zu können, hält ungebremst an, so dass die Komplexität solcher Systeme ebenfalls immer weiter zulegen wird. Computer werden vermehrt ubiquitär sein, das heißt, sie werden in die Umgebung integriert sein und nicht mehr als Computer vom Menschen wahrgenommen werden. Beispiele sind Sensornetzwerke, “Electronic Textiles” und viele mehr. Die physikalisch mögliche Komplexität wird allerdings praktisch nicht ohne weiteres erreichbar sein, da zur Zeit leistungsfähige Entwurfsverfahren fehlen, die in der Lage wären, diese hohe Komplexität zu handhaben. Es werden leistungsfähige ESL Werkzeuge (”Electronic System Level Design Tools”), sowie neuartige Architekturen benötigt werden. Der Schwerpunkt dieser Vorlesung liegt deshalb auf high-level Entwurfsmethoden und Architekturen für Eingebettete Systeme. Da der Leistungsverbrauch der (meist mobilen) Eingebetteten Systeme von entscheidender Bedeutung ist, wird ein Schwerpunkt der Entwurfsverfahren auf dem Entwurf mit Hinblick auf geringem Leistungsverbrauch liegen.


\end{content}



\end{module}

