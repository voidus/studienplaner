% Modulbeschreibung 
% Informationsgrad : extern
% Sprache: de
\begin{module}

\setdoclanguagegerman
\moduledegreeprogramme{Informatik (B.Sc.)}
\modulesubject{EF Recht}
\moduleID{IN3INJUR1}
\modulename{Einführung in das Privatrecht}
\modulecoordination{T. Dreier}

\documentdate{2008-10-22 10:29:07}

\modulecredits{4}
\moduleduration{1}
\modulecycle{Jedes 2. Semester, Wintersemester}



\modulehead

% For index (key word@display). Key word is used for sorting - no Umlauts please.
\index{Einfuehrung in das Privatrecht@Einführung in das Privatrecht (M)}

% For later referencing
\label{mod_2653.dp_997}

\begin{courselist}
24012 & BGB für Anfänger (S.~\pageref{cour_4383.dp_997}) & 4/0 & W & 4 & T. Dreier, P. Sester\\
\end{courselist}

\begin{styleenv}
\begin{assessment}
Die Erfolgskontrolle des Moduls erfolgt in Form einer schriftlichen Prüfung nach § 4(2), 1 SPO im Umfang von 90 Minuten.

 

Die Modulnote entspricht der Note der schriftlichen Prüfung.


\end{assessment}

\begin{conditions}Keine.\end{conditions}


\end{styleenv}

\begin{learningoutcomes}
Der/die Studierende

 \begin{itemize}\item erkennt rechtliche Problemlagen und Fragestellungen und ist in der Lage, einfach gelagerte rechtlich relevante Sacherhalte auf dem Gebiet des Zivilrechts zu verstehen,   \item kennt und versteht die Unterschiede von Privatrecht, öffentlichem Recht und Strafrecht,  \item analysiert das Zusammenwirken der Grundbegriffe des Bürgerlichen Rechts und wendet deren Ausformung im deutschen Bürgerlichen Gesetzbuch (BGB) an (Rechtssubjekte, Rechtsobjekte, Willenserklärung, Vertragsschluß,allgemeine Geschäftsbedingungen, Verbraucherschutz, Leistungstörungen usw.),   \item entwickelt zivilrechtliche Lösungsmuster in Bezug auf konkrete Streitfälle wie auch in rechtspolitischer Hinsicht   \item bewertet rechtlich relevante Sachverhalte zutreffend und kann einfache Fälle eigenständig lösen.  \end{itemize}
\end{learningoutcomes}

\begin{content}
Das Modul gibt eine allgemeine Einführung ins Recht. Was ist Recht, warum gilt Recht und was will Recht im Zusammenspiel mit Sozialverhalten, Technikentwicklung und Markt? Welche Beziehung besteht zwischen Recht und Gerechtigkeit? Ebenfalls einführend wird die Unterscheidung von Privatrecht, öffentlichem Recht und Strafrecht vorgestellt sowie die Grundzüge der gerichtlichen und außergerichtlichen einschließlich der internationalen Rechtsdurchsetzung erläutert. Anschließend werden die Grundbegriffe des Rechts in ihrer konkreten Ausformung im deutschen Bürgerlichen Gesetzbuch (BGB) besprochen. Das betrifft insbesondere Rechtssubjekte, Rechtsobjekte, Willenserklärung, die Einschaltung Dritter (insbes. Stellvertretetung), Vertragsschluß (einschließlich Trennungs- und Abstraktionsprinzip), allgemeine Geschäftsbedingungen, Verbraucherschutz, Leistungsstörungen. Abschließend erfolgt ein Ausblick auf das Schuld- und das Sachenrecht. Schließlich wird eine Einführung in die Subsumtionstechnik gegeben.


\end{content}



\end{module}

