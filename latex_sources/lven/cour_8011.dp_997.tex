% Lehrveranstaltungsbeschreibung
% Informationsgrad : extern
% Sprache: de
\begin{course}

\setdoclanguagegerman
\coursedegreeprogramme{Informatik}
\coursemodulename{Funktionentheorie (S.~\pageref{mod_3305.dp_997})[IN3MATHAN04]}
\courseID{1560}
\coursename{Funktionentheorie}
\coursecoordination{G. Herzog, M. Plum, W. Reichel, C. Schmoeger, R. Schnaubelt, L. Weis}

\documentdate{2011-10-06 18:22:56.836843}

\courselevel{}
\coursecredits{8}
\courseterm{Sommersemester}
\coursehours{4/2}
\courseinstructionlanguage{}

\coursehead

% For index (key word@display). Key word is used for sorting - no Umlauts please.
\index{Funktionentheorie@Funktionentheorie}

% For later referencing
\label{cour_8011.dp_997}


\begin{styleenv}
\begin{assessment}
Prüfung: schriftliche oder mündliche Prüfung\newline
Notenbildung: Note der Prüfung


\end{assessment}

\begin{conditions}Keine.\end{conditions}

\begin{recommendations}Folgende Module sollten bereits belegt worden sein (Empfehlung):\newline
Analysis 1-3

\end{recommendations}
\end{styleenv}

\begin{learningoutcomes}
Einführung in die Hauptsätze der komplexen Analysis


\end{learningoutcomes}

\begin{content}
\begin{itemize}\item Holomorphie  \item Elementare Funktionen  \item Integralsatz und -formel von Cauchy  \item Potenzreihen  \item Satz von Liouville  \item Maximumsprinzip  \item Satz von der Gebietstreue  \item Pole  \item Laurentreihen  \item Residuensatz und reelle Integrale  \item Harmonische Funktionen  \end{itemize}
\end{content}







\end{course}