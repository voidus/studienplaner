% Lehrveranstaltungsbeschreibung
% Informationsgrad : extern
% Sprache: de
\begin{course}

\setdoclanguagegerman
\coursedegreeprogramme{Informatik}
\coursemodulename{Softwaretechnik II (S.~\pageref{mod_3881.dp_997})[IN3INSWT2]}
\courseID{24076}
\coursename{Softwaretechnik II}
\coursecoordination{R. Reussner, W. Tichy}

\documentdate{2011-11-14 11:27:55.016540}

\courselevel{4}
\coursecredits{6}
\courseterm{Wintersemester}
\coursehours{3/1}
\courseinstructionlanguage{de}

\coursehead

% For index (key word@display). Key word is used for sorting - no Umlauts please.
\index{Softwaretechnik II@Softwaretechnik II}

% For later referencing
\label{cour_7361.dp_997}


\begin{styleenv}
\begin{assessment}
Die Erfolgskontrolle wird in der Modulbeschreibung erläutert.


\end{assessment}

\begin{conditions}Keine.\end{conditions}

\begin{recommendations}Die Lehrveranstaltung \emph{Softwaretechnik I} sollte bereits gehört worden sein.

\end{recommendations}
\end{styleenv}

\begin{learningoutcomes}
Die Studierenden erlernen Vorgehensweisen und Techniken für systematische Softwareentwicklung, indem fortgeschrittene Themen der Softwaretechnik behandelt werden.


\end{learningoutcomes}

\begin{content}
Requirements Engineering, Softwareprozesse, Software-Qualität, Software-Architekturen, MDD, Enterprise Software Patterns

 

Software-Wartbarkeit, Sicherheit, Verläßlichkeit (Dependability), eingebettete Software, Middleware, statistisches Testen


\end{content}

\begin{media}Vorlesungsfolien, Sekundärliteratur

\end{media}

\begin{literature}Wird in der Vorlesung bekanntgegeben.

\end{literature}



\end{course}