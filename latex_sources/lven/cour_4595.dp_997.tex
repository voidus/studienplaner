% Lehrveranstaltungsbeschreibung
% Informationsgrad : extern
% Sprache: de
\begin{course}

\setdoclanguagegerman
\coursedegreeprogramme{Informatik}
\coursemodulename{Industrielle Produktion I (S.~\pageref{mod_1595.dp_997})[IN3WWBWL10]}
\courseID{2581950}
\coursename{Grundlagen der Produktionswirtschaft}
\coursecoordination{F. Schultmann}

\documentdate{2012-01-02 11:05:36.903673}

\courselevel{3}
\coursecredits{5,5}
\courseterm{Sommersemester}
\coursehours{2/2}
\courseinstructionlanguage{de}

\coursehead

% For index (key word@display). Key word is used for sorting - no Umlauts please.
\index{Grundlagen der Produktionswirtschaft@Grundlagen der Produktionswirtschaft}

% For later referencing
\label{cour_4595.dp_997}


\begin{styleenv}
\begin{assessment}
Die Erfolgskontrolle wird in der Modulbeschreibung erläutert.


\end{assessment}

\begin{conditions}Keine.\end{conditions}


\end{styleenv}

\begin{learningoutcomes}
\begin{itemize}\item Die Studierenden benennen Problemstellungen aus dem Bereich der strategischen Unternehmensplanung .  \item Die Studierenden kennen Lösungsansätze für die benannten Probleme und wenden diese an.  \end{itemize}
\end{learningoutcomes}

\begin{content}
Im Mittelpunkt stehen Fragestellungen des strategischen Produktionsmanagements, die auch unter ökologischen Aspekten betrachtet werden. Die Aufgaben der industriellen Produktionswirtschaft werden mittels interdisziplinärer Ansätze der Systemtheorie beschrieben. Bei der strategischen Unternehmensplanung zur langfristigen Existenzsicherung hat die Forschung und Entwicklung (F\&E) eine besondere Bedeutung. Bei der betrieblichen Standortplanung für einzelne Unternehmen und Betriebe sind bereits bestehende bzw. geplante Produktionsstätten, Zentral-, Beschaffungs- oder Auslieferungslager zu berücksichtigen. Unter produktionswirtschaftlicher Sichtweise werden bei der Logistik die inner- und außerbetrieblichen Transport- und Lagerprobleme betrachtet. Dabei werden auch Fragen der Entsorgungslogistik und des Supply Chain Managements behandelt.


\end{content}

\begin{media}Medien werden über die Lernplattform bereit gestellt.

\end{media}

\begin{literature}Wird in der Veranstaltung bekannt gegeben.

\end{literature}



\end{course}