% Lehrveranstaltungsbeschreibung
% Informationsgrad : extern
% Sprache: de
\begin{course}

\setdoclanguagegerman
\coursedegreeprogramme{Informatik}
\coursemodulename{Multilinguale Mensch-Maschine-Kommunikation (S.~\pageref{mod_2623.dp_997})[IN3INMMMK]}
\courseID{24600}
\coursename{Multilinguale Mensch-Maschine-Kommunikation}
\coursecoordination{T. Schultz, F. Putze}

\documentdate{2011-04-07 15:20:13.941319}

\courselevel{4}
\coursecredits{6}
\courseterm{Sommersemester}
\coursehours{4}
\courseinstructionlanguage{de}

\coursehead

% For index (key word@display). Key word is used for sorting - no Umlauts please.
\index{Multilinguale Mensch-Maschine-Kommunikation@Multilinguale Mensch-Maschine-Kommunikation}

% For later referencing
\label{cour_7267.dp_997}


\begin{styleenv}
\begin{assessment}
Die Erfolgskontrolle wird in der Modulbeschreibung erläutert.


\end{assessment}

\begin{conditions}Keine.\end{conditions}


\end{styleenv}

\begin{learningoutcomes}
Die Studierenden werden in die Grundlagen der automatischen Spracherkennung und –verarbeitung eingeführt.

 

Dazu werden zunächst die theoretischen Grundlagen der Signalverarbeitung und der Modellierung von Sprache vorgestellt. Besonderes Augenmerk wird hier auf statistische Modellierungsmethoden gelegt. Der gegenwärtige Stand der Forschung und Entwicklung wird anhand zahlreicher Anwendungsbeispiele veranschaulicht. Nach dem Besuch der Veranstaltung sollten die Studierenden in der Lage sein, das Potential sowie die Herausforderungen und Grenzen moderner Sprachtechnologien und Anwendungen einzuschätzen.

 

Das mit der Vorlesung verbundene Praktikum „Multilingual Speech Processing“ [24280] und das Seminar „Aktuelle Themen der Sprachverarbeitung“ [SemAKTSV] bietet den Studierenden die Möglichkeit, die in der Vorlesung erworbenen Kenntnisse in die Praxis umzusetzen bzw. anhand aktueller Forschungsarbeiten zu vertiefen.


\end{learningoutcomes}

\begin{content}
Die Vorlesung \emph{Multilinguale Mensch-Maschine-Kommunikation} bietet eine Einführung in die automatische Spracherkennung und Sprachverarbeitung. Dazu werden zunächst die theoretischen Grundlagen der Signalverarbeitung und der Modellierung von Sprache vorgestellt. Besonderes Augenmerk wird hier auf statistische Modellierungsmethoden gelegt. Anschließend werden die wesentlichen praktischen Ansätze und Methoden behandelt, die für eine erfolgreiche Umsetzung der Theorie in die Praxis der sprachlichen Mensch-Maschine Kommunikation relevant sind. Die modernen Anforderungen der Spracherkennung und Sprachverarbeitung im Zuge der Globalisierung werden in der Vorlesung anhand zahlreicher Beispiele von state-of-the-art Systemen illustriert und im Kontext der Multilingualität beleuchtet.

 

Weitere Informationen unter http://csl.anthropomatik.kit.edu.


\end{content}

\begin{media}Vorlesungsfolien (verfügbar als pdf von http://csl.anthropomatik.kit.edu)

\end{media}

\begin{literature}\textbf{Weiterführende Literatur:}

 

Xuedong Huang, Alex Acero und Hsiao-wuen Hon, Spoken Language Processing, Prentice Hall PTR, NJ, 2001\newline
Tanja Schultz und Katrin Kirchhoff (Hrsg.), Multilingual Speech Processing, Elsevier, Academic Press, 2006

\end{literature}

\begin{remarks}Sprache der Lehrveranstaltung: Deutsch (auf Wunsch auch Englisch)

\end{remarks}

\end{course}