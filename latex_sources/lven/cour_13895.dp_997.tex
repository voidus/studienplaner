% Lehrveranstaltungsbeschreibung
% Informationsgrad : extern
% Sprache: de
\begin{course}

\setdoclanguagegerman
\coursedegreeprogramme{Informatik}
\coursemodulename{Proseminar (S.~\pageref{mod_2385.dp_997})[IN2INPROSEM]}
\courseID{24530}
\coursename{Proseminar Zellularautomaten und diskrete komplexe Systeme}
\coursecoordination{R. Vollmar, T. Worsch}

\documentdate{2011-04-14 10:27:01.681190}

\courselevel{1}
\coursecredits{3}
\courseterm{Sommersemester}
\coursehours{2}
\courseinstructionlanguage{de}

\coursehead

% For index (key word@display). Key word is used for sorting - no Umlauts please.
\index{Proseminar Zellularautomaten und diskrete komplexe Systeme@Proseminar Zellularautomaten und diskrete komplexe Systeme}

% For later referencing
\label{cour_13895.dp_997}


\begin{styleenv}
\begin{assessment}
Die Erfolgskontrolle erfolgt durch Ausarbeiten einer schriftlichen Proseminararbeit sowie der Präsentation derselbigen als Erfolgskontrolle anderer Art nach § 4 Abs. 2 Nr. 3 der SPO. Die Gesamtnote setzt sich aus den benoteten und gewichteten Erfolgskontrollen (i.d.R. Seminararbeit 50\%, Präsentation 50\%) zusammen.


\end{assessment}

\begin{conditions}Keine.\end{conditions}


\end{styleenv}

\begin{learningoutcomes}
\begin{itemize}\item Die Studierenden erhalten eine erste Einführung in das wissenschaftliche Arbeiten auf einem speziellen Fachgebiet.  \item Die Bearbeitung der Proseminararbeit bereitet zudem auf die Abfassung der Bachelorarbeit vor.  \item  Mit dem Besuch der Proseminarveranstaltungen werden neben Techniken des wissenschaftlichen Arbeitens auch Schlüsselqualifikationen integrativ vermittelt.  \end{itemize}
\end{learningoutcomes}

\begin{content}
Es werden ausgewählte Themen aus dem Bereich Zellularautomaten (ZA) und diskrete komplexe Systeme behandelt. Dazu gehören zum Beispiel ZA als paralleles Modell, reversible ZA, Simulation realer Phänomene mit ZA, unendliche Parkettierungen, asynchrone Logik und vieles mehr.


\end{content}



\begin{literature}Wissenschaftliche Aufsätze

\end{literature}



\end{course}