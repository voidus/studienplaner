% Modulbeschreibung 
% Informationsgrad : extern
% Sprache: de
\begin{module}

\setdoclanguagegerman
\moduledegreeprogramme{Informatik (B.Sc.)}
\modulesubject{}
\moduleID{IN3INMP1}
\modulename{Mikroprozessoren I}
\modulecoordination{W. Karl}

\documentdate{2011-02-17 10:45:34.629131}

\modulecredits{3}
\moduleduration{1}
\modulecycle{Jedes 2. Semester, Sommersemester}



\modulehead

% For index (key word@display). Key word is used for sorting - no Umlauts please.
\index{Mikroprozessoren I@Mikroprozessoren I (M)}

% For later referencing
\label{mod_4045.dp_997}

\begin{courselist}
24688 & Mikroprozessoren I (S.~\pageref{cour_7169.dp_997}) & 2 & S & 3 & W. Karl\\
\end{courselist}

\begin{styleenv}
\begin{assessment}
Die Erfolgskontrolle erfolgt in Form einer mündlichen Prüfung im Umfang von i.d.R. etwa 30 Minuten gemäß § 4 Abs. 2 Nr. 2 SPO. \newline
Die Modulnote ist die Note der mündlichen Prüfung.


\end{assessment}

\begin{conditions}Keine.\end{conditions}


\end{styleenv}

\begin{learningoutcomes}
\begin{itemize}\item Die Studierenden sollen detaillierte Kenntnisse über den Aufbau und die Organisation von Mikroprozessorsystemen in den verschiedenen Einsatzgebieten erwerben.   \item Die Studierenden sollen die Fähigkeit erwerben, Mikroprozessoren für verschiedene Einsatzgebiete bewerten und auswählen zu können.   \item Die Studierenden sollen die Fähigkeit erwerben, systemnahe Funktionen programmieren zu können.   \item Die Studierenden sollen Architekturmerkmale von Mikroprozessoren zur Beschleunigung von Anwendungen und Systemfunktionen ableiten, bewerten und entwerfen können.   \item Die Studierenden sollen die Fähigkeiten erwerben, Mikroprozessorsysteme in strukturierter und systematischer Weise entwerfen zu können.   \end{itemize}
\end{learningoutcomes}

\begin{content}
Das Modul befasst sich im ersten Teil mit Mikroprozessoren, die in Desktops und Ser vern eingesetzt werden. Ausgehend von den grundlegenden Eigenschaften dieser Rechner und dem Systemaufbau werden die Architekturmerkmale von Allzweck- und Hochleistungs-Mikroprozessoren vermittelt. Insbesondere sollen die Techniken und Mechanismen zur Unterstützung von Betriebssystemfunktionen, zur Beschleunigung durch Ausnützen des Parallelismus auf Maschinenbefehlsebene und Aspekte der Speicherhierarchie vermittelt werden. \newline
Der zweite Teil behandelt Mikroprozessoren, die in eingebetteten Systemen eingesetzt werden. Es werden die grundlegenden Eigenschaften von Microcontrollern vermittelt. Eigenschaften von Mikroprozessoren, die auf spezielle Einsatzgebiete zugeschnitten sind, werden ausführlich behandelt.


\end{content}



\end{module}

