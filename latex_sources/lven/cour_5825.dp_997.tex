% Lehrveranstaltungsbeschreibung
% Informationsgrad : extern
% Sprache: de
\begin{course}

\setdoclanguagegerman
\coursedegreeprogramme{Informatik}
\coursemodulename{Algorithmen zur Visualisierung von Graphen (S.~\pageref{mod_2939.dp_997})[IN3INALGVG]}
\courseID{24118}
\coursename{Algorithmen zur Visualisierung von Graphen}
\coursecoordination{D. Wagner, R. Görke}

\documentdate{2011-07-22 15:24:09.361388}

\courselevel{4}
\coursecredits{5}
\courseterm{Winter-/Sommersemester}
\coursehours{2/1}
\courseinstructionlanguage{de}

\coursehead

% For index (key word@display). Key word is used for sorting - no Umlauts please.
\index{Algorithmen zur Visualisierung von Graphen@Algorithmen zur Visualisierung von Graphen}

% For later referencing
\label{cour_5825.dp_997}


\begin{styleenv}
\begin{assessment}

\end{assessment}

\begin{conditions}Keine.\end{conditions}

\begin{recommendations}Kenntnisse zu Grundlagen aus der Graphentheorie und Algorithmentechnik sind hilfreich.

\end{recommendations}
\end{styleenv}

\begin{learningoutcomes}
Die Studierenden erwerben ein systematisches Verständnis algorithmischer Fragestellungen und Lösungsansätze im Bereich der Visualisierung von Graphen, das auf dem bestehenden Wissen in den Themenbereichen Graphentheorie und Algorithmik aufbaut. Die auftretenden Fragestellungen werden auf ihren algorithmischen Kern reduziert und anschließend, soweit aus komplexitätstheoretischer Sicht möglich, effizient gelöst. Studierende lernen die vorgestellten Methoden und Techniken autonom auf verwandte Fragestellungen anzuwenden und können mit dem erworbenen Wissen an aktuellen Forschungsthemen der Visualisierung von Graphen arbeiten.


\end{learningoutcomes}

\begin{content}
Netzwerke sind relational strukturierte Daten, die in zunehmendem Maße und in den unterschiedlichsten Anwendungsbereichen auftreten. Die Beispiele reichen von physischen Netzwerken, wie z.B. Transport- und Versorgungsnetzen, hin zu abstrakten Netzwerken, z.B. sozialen Netzwerken. Für die Untersuchung und das Verständnis von Netzwerken ist die Netzwerkvisualisierung ein grundlegendes Werkzeug.\newline
\newline
Mathematisch lassen sich Netzwerke als Graphen modellieren und das Visualisierungsproblem lässt sich auf das algorithmische Kernproblem reduzieren, ein Layout des Graphen, d.h. geeignete Knoten- und Kantenpositionen in der Ebene, zu bestimmen. Dabei werden je nach Anwendung und Graphenklasse unterschiedliche Anforderungen an die Art der Zeichnung und die zu optimierenden Gütekriterien gestellt. Das Forschungsgebiet des Graphenzeichnens greift dabei auf Ansätze aus der klassischen Algorithmik, der Graphentheorie und der algorithmischen Geometrie zurück. \newline
\newline
Im Laufe der Veranstaltung wird eine repräsentative Auswahl an Visualisierungsalgorithmen vorgestellt und vertieft.


\end{content}

\begin{media}Tafel, Vorlesungsfolien, Skript

\end{media}

\begin{literature}\textbf{Weiterführende Literatur:}

 \begin{itemize}\item Di Battista, Eades, Tamassia, Tollis: Graph Drawing, Prentice Hall 1999  \item Kaufmann, Wagner: Drawing Graphs, Springer-Verlag, 2001  \end{itemize}\end{literature}



\end{course}