% Lehrveranstaltungsbeschreibung
% Informationsgrad : extern
% Sprache: de
\begin{course}

\setdoclanguagegerman
\coursedegreeprogramme{Informatik}
\coursemodulename{Design und Evaluation innovativer Benutzerschnittstellen (S.~\pageref{mod_10771.dp_997})[IN3INDEB]}
\courseID{24103}
\coursename{Design und Evaluation innovativer Benutzerschnittstellen}
\coursecoordination{T. Schultz, F. Putze}

\documentdate{2011-04-07 15:12:41.798967}

\courselevel{3}
\coursecredits{3}
\courseterm{Wintersemester}
\coursehours{2}
\courseinstructionlanguage{de}

\coursehead

% For index (key word@display). Key word is used for sorting - no Umlauts please.
\index{Design und Evaluation innovativer Benutzerschnittstellen@Design und Evaluation innovativer Benutzerschnittstellen}

% For later referencing
\label{cour_10777.dp_997}


\begin{styleenv}
\begin{assessment}
Die Erfolgskontrolle wird in der Modulbeschreibung erläutert.


\end{assessment}

\begin{conditions}Keine.\end{conditions}

\begin{recommendations}Kenntnisse im Bereich der Kognitiven Systeme, Sprachverarbeitung oder Biosignale sind hilfreich.

\end{recommendations}
\end{styleenv}

\begin{learningoutcomes}
Die Studierenden haben einen breiten Überblick über die Methoden zum Entwurf und zur Evaluierung von Benutzerschnittstellen, die Gebrauch machen von Techniken zur natürlichen Eingabe oder impliziten Steuerung. Die Studierenden können entsprechende Systeme bezüglich des aktuellen Stands der Wissenschaft einordnen, deren Fähigkeiten und Einschränkungen einschätzen und besitzen die Grundlagen zum eigenen Entwurf neuer Schnittstellen.


\end{learningoutcomes}

\begin{content}
Die Vorlesung beschäftigt sich mit innovativen Benutzerschnittstellen, die Techniken der Biosignal- oder Sprachverarbeitung einsetzen. Dazu gehören einerseits Schnittstellen zur natürlichen expliziten Eingabe wie beispielsweise Sprachdialogsysteme oder Systeme mit Gesteneingabe. Andererseits behandelt die Vorlesung Schnittstellen zur impliziten Steuerung, beispielsweise mittels biosignalbasierter Erkennung von Emotionen oder mentaler Auslastung. Die Vorlesung beginnt mit einer Einführung der notwendigen Techniken sowie den theoretischen Grundlagen. Weitere Vorlesungen beschäftigen sich mit dem Entwurf und der Implementierung kompletter Schnittstellen auf Basis dieser Einzeltechniken. Zentral ist dabei, welche Vorteile aber auch welche neuen Herausforderungen, zum Beispiel auf dem Gebiet der Multimodalität, dadurch entstehen. Außerdem wird behandelt, wie Benutzer mit solchen neuartigen Schnittstellen umgehen und mit welchen Methoden die Stärken und Schwächen solcher Systeme systematisch untersucht werden können.


\end{content}

\begin{media}Vorlesungsfolien.

\end{media}

\begin{literature}Wird in der Vorlesung bekannt gegeben.

\end{literature}



\end{course}