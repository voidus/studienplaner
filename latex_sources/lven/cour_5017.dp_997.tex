% Lehrveranstaltungsbeschreibung
% Informationsgrad : extern
% Sprache: de
\begin{course}

\setdoclanguagegerman
\coursedegreeprogramme{Informatik}
\coursemodulename{Strategie und Organisation (S.~\pageref{mod_1659.dp_997})[IN3WWBWL11]}
\courseID{2577902}
\coursename{Organisationsmanagement}
\coursecoordination{H. Lindstädt}

\documentdate{2011-12-15 15:48:25.157892}

\courselevel{4}
\coursecredits{4}
\courseterm{Wintersemester}
\coursehours{2/0}
\courseinstructionlanguage{de}

\coursehead

% For index (key word@display). Key word is used for sorting - no Umlauts please.
\index{Organisationsmanagement@Organisationsmanagement}

% For later referencing
\label{cour_5017.dp_997}


\begin{styleenv}
\begin{assessment}
Die Erfolgskontrolle erfolgt in Form einer schriftlichen Prüfung (60min.) (nach §4(2), 1 SPO) zu Beginn der vorlesungsfreien Zeit des Semesters.

 

Die Prüfung wird in jedem Semester angeboten und kann zu jedem ordentlichen Prüfungstermin wiederholt werden.


\end{assessment}

\begin{conditions}Keine.\end{conditions}


\end{styleenv}

\begin{learningoutcomes}
Die Teilnehmer sollen durch den Kurs in die Lage versetzt werden, Stärken und Schwächen existierender organisationaler Strukturen und Regelungen anhand systematischer Kriterien zu beurteilen. Dabei werden Konzepte und Modelle für die Gestaltung organisationaler Strukturen, die Regulierung organisationaler Prozesse und die Steuerung organisationaler Veränderungen vorgestellt und anhand von Fallstudien diskutiert. Der Kurs ist handlungsorientiert aufgebaut und soll den Studierenden ein realistisches Bild von Möglichkeiten und Grenzen rationaler Gestaltungsansätze vermitteln.


\end{learningoutcomes}

\begin{content}
\begin{itemize}\item Grundlagen des Organisationsmanagements  \item Management organisationaler Strukturen und Prozesse: Die Wahl der Gestaltungsparameter  \item Idealtypische Organisationsstrukturen: Wahl und Wirkung der Parameterkombination  \item Management organisationaler Veränderungen  \end{itemize}
\end{content}

\begin{media}Folien.

\end{media}

\begin{literature}\begin{itemize}\item Laux, H.; Liermann, F.: \emph{Grundlagen der Organisation}, Springer. 6. Aufl. Berlin 2005.  \item Lindstädt, H.: \emph{Organisation}, in Scholz, C. (Hrsg.): Vahlens Großes Personallexikon, Verlag Franz Vahlen. 1. Aufl. München, 2009.  \item Schreyögg, G.: \emph{Organisation. Grundlagen moderner Organisationsgestaltung}, Gabler. 4. Aufl. Wiesbaden 2003.  \end{itemize}

Die relevanten Auszüge und zusätzlichen Quellen werden in der Veranstaltung bekannt gegeben.

\end{literature}



\end{course}