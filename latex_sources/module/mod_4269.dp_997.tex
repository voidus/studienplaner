% Modulbeschreibung 
% Informationsgrad : extern
% Sprache: de
\begin{module}

\setdoclanguagegerman
\moduledegreeprogramme{Informatik (B.Sc.)}
\modulesubject{EF Informationsmanagement im Ingenieurwesen}
\moduleID{IN3MACHVE2}
\modulename{Virtual Engineering II}
\modulecoordination{Maier}

\documentdate{2011-08-04 10:24:05.538243}

\modulecredits{5}
\moduleduration{1}
\modulecycle{Jedes 2. Semester, Sommersemester}



\modulehead

% For index (key word@display). Key word is used for sorting - no Umlauts please.
\index{Virtual Engineering II@Virtual Engineering II (M)}

% For later referencing
\label{mod_4269.dp_997}

\begin{courselist}
2122378 & Virtual Engineering II (S.~\pageref{cour_7507.dp_997}) & 2/1 & S & 5 & \\
\end{courselist}

\begin{styleenv}
\begin{assessment}
Die Erfolgskontrolle wird in der Lehrveranstaltungsbeschreibung erläutert.


\end{assessment}

\begin{conditions}Die Module \emph{Virtual Engineering I} [IN3INMACHVE1] und \emph{Product Lifecycle Management} [IN3MACHPMI] müssen geprüft werden.

\end{conditions}


\end{styleenv}

\begin{learningoutcomes}
Der/ die Studierende

 \begin{itemize}\item besitzt grundlegende Kenntnisse über die Funktionsweise von Virtual, Augmented und Mixed Reality Systemen sowie über deren Einsatzmöglichkeiten in der Virtuellen Produktentstehung,   \item versteht die Problematik des Virtual Mock-Ups als Grundlage für die Prozesse der Virtuellen Produktentstehung,   \item versteht die Verknüpfung von Konstruktions- und Validierungstätigkeiten unter Nutzung virtueller Prototypen und VR/AR/MR-Visualisierungstechniken in Verbindung mit PLM-Systemen  \end{itemize}
\end{learningoutcomes}

\begin{content}
Die Vorlesung vermittelt die Informationstechnischen Zusammenhänge der virtuellen Produktentstehung. Dabei stehen die in der industriellen Praxis verwendeten IT-Systeme zur Unterstützung der Prozesskette des Virtual Engineerings im Mittelpunkt:

 \begin{itemize}\item \textbf{Virtual Reality-Systeme} erlauben die immersive Visualisierung der entsprechenden Produktmodelle, vom Einzelteil bis zum vollständigen Zusammenbau;   \item \textbf{Virtuelle Prototypen} vereinigen erweiterte CAD-Daten mit technischen Informationen für immersive Visualisierung, Funktionalitätsuntersuchungen und -validierungen im Kontext des gesamten Produktes mit Unterstützung von VR/AR/MR-Umgebungen.  \item \textbf{Integrierte Virtuelle Produktentstehung} verdeutlicht beispielhaft den virtuellen Produktentstehungsprozess aus der Sicht des Virtual Engineerings.  \end{itemize}
\end{content}

\begin{remarks}\textcolor{red}{Das Modul wird im Bachelor-Studiengang nicht mehr angeboten, Prüfungen sind möglich bis Wintersemester 2012/13.}

\end{remarks}

\end{module}

