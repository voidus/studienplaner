% Lehrveranstaltungsbeschreibung
% Informationsgrad : extern
% Sprache: de
\begin{course}

\setdoclanguagegerman
\coursedegreeprogramme{Informatik}
\coursemodulename{IN3INFON (S.~\pageref{mod_15643.dp_997})[IN4INFON]}
\courseID{24665}
\coursename{Fortgeschrittene Objektorientierung}
\coursecoordination{G. Snelting}

\documentdate{2012-01-17 14:44:55.146728}

\courselevel{4}
\coursecredits{5}
\courseterm{Sommersemester}
\coursehours{2/2}
\courseinstructionlanguage{de}

\coursehead

% For index (key word@display). Key word is used for sorting - no Umlauts please.
\index{Fortgeschrittene Objektorientierung@Fortgeschrittene Objektorientierung}

% For later referencing
\label{cour_6273.dp_997}


\begin{styleenv}
\begin{assessment}
Die Erfolgskontrolle wird in der Modulbeschreibung erläutert.


\end{assessment}

\begin{conditions}Vorangegangene erfolgreiche Teilnahme an den Pflichtveranstaltungen der ersten 3 Semester des Bachelor-Studiums Informatik.

\end{conditions}

\begin{recommendations}Gute Java-Kenntnisse

\end{recommendations}
\end{styleenv}

\begin{learningoutcomes}
Die Teilnehmer kennen Grundlagen verschiedener objektorientierter Sprachen (z.B. Java, C\#, Smalltalk, Scala) Die Teilnehmer kennen Verhalten, Implementierung, Semantik und softwaretechnische Nutzung von Vererbung und dynamischer Bindung. Die Teilnehmer kennen innovative objektorientierte Sprachkonzepte (z.B. Generizität, Aspekte, Traits). Die Teilnehmer kennen theoretische Grundlagen (z.B. Typsysteme), softwaretechnische Werkzeuge (z.B. Refaktorisierung) und Verfahren zur Analyse von objektorientierten Programmen (z.B. Points-to Analyse). Die Teilnehmer haben einen Überblick über aktuelle Forschung im Bereich objektorientierter Programmierung.


\end{learningoutcomes}

\begin{content}
\begin{itemize}\item Verhalten und Semantik von dynamischer Bindung  \item Implementierung von Einfach- und Mehrfachvererbung  \item Generizität, Refaktorisierung  \item Traits und Mixins, Virtuelle Klassen  \item Cardelli-Typsystem  \item Palsberg-Schwartzbach Typinferenz  \item Call-Graph Analysen, Points-to Analysen  \item operationale Semantik, Typsicherheit  \item Bytecode, JVM, Bytecode Verifier, dynamische Compilierung  \end{itemize}
\end{content}





\begin{remarks}\textcolor{red}{Der Umfang der Leistungspunkte reduziert sich ab dem SS 2012 auf 5 (2/2 SWS).}

\end{remarks}

\end{course}