% Modulbeschreibung 
% Informationsgrad : extern
% Sprache: de
\begin{module}

\setdoclanguagegerman
\moduledegreeprogramme{Informatik (B.Sc.)}
\modulesubject{}
\moduleID{IN3INEBS}
\modulename{Energiebewusste Systeme}
\modulecoordination{F. Bellosa, J. Henkel}

\documentdate{2011-05-09 14:13:14.566929}

\modulecredits{6}
\moduleduration{1}
\modulecycle{Jedes Semester}



\modulehead

% For index (key word@display). Key word is used for sorting - no Umlauts please.
\index{Energiebewusste Systeme@Energiebewusste Systeme (M)}

% For later referencing
\label{mod_10583.dp_997}

\begin{courselist}
24127 & Power Management (S.~\pageref{cour_6229.dp_997}) & 2 & W & 3 & F. Bellosa\\
24181 & Power Management Praktikum (S.~\pageref{cour_6237.dp_997}) & 2 & W & 3 & F. Bellosa,  Merkel\\
24672 & Low Power Design (S.~\pageref{cour_6269.dp_997}) & 2 & S & 3 & J. Henkel\\
LPD & Praktikum Low Power Design (S.~\pageref{cour_7993.dp_997}) & 2 & S & 3 & J. Henkel\\
\end{courselist}

\begin{styleenv}
\begin{assessment}
Die Erfolgskontrolle erfolgt in Form einer mündlichen Gesamtprüfung über Vorlesung und Praktikum im Umfang von i.d.R. 30 Minuten nach § 4 Abs. 2 Nr. 2 SPO.

 

Praktika: Zusätzlich muss ein unbenoteter Übungsschein als Erfolgskontrolle anderer Art nach § 4 Abs. 2 Nr. 3 SPO erbracht werden.

 

Die Modulnote ist die Note der mündlichen Prüfung.


\end{assessment}

\begin{conditions}Folgende Kombinationen können gewählt werden:\newline
 - Vorlesung \emph{Low Power Design} und \emph{Power Managemen}t\newline
 - Vorlesung \emph{Low Power Design} und \emph{Praktikum Low Power Design}\newline
 - Vorlesung \emph{Power Management} und \emph{Power Management Praktikum}

\end{conditions}

\begin{recommendations}Ein erfolgreicher Abschluss der Module Betriebssysteme [IN2INBS] und Technische Informatik [IN1INTI] wird empfohlen.

\end{recommendations}
\end{styleenv}

\begin{learningoutcomes}
Der Student soll energergiegewahre Systeme von der Hardware bis zur Systemsoftware entwerfen, implementieren und analysieren können. Er kennt die Möglichkeiten, welche die Hardware bietet, um ihren Energieverbrauch zu beeinflussen, sowie die Auswirkungen einer Verbrauchsreduzierung auf die Performanz.


\end{learningoutcomes}

\begin{content}
Inhalte:

 \begin{itemize}\item Entwurfsverfahren  \item Syntheseverfahren  \item Schätzverfahren  \item  Betriebssystemstrategien   \end{itemize}
\end{content}



\end{module}

