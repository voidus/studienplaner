% Lehrveranstaltungsbeschreibung
% Informationsgrad : extern
% Sprache: de
\begin{course}

\setdoclanguagegerman
\coursedegreeprogramme{Informatik}
\coursemodulename{Praxis der Software-Entwicklung (S.~\pageref{mod_2611.dp_997})[IN2INSWP]}
\courseID{PSE}
\coursename{Software-Entwicklung }
\coursecoordination{G. Snelting}

\documentdate{2011-11-14 13:19:16.492790}

\courselevel{2}
\coursecredits{6}
\courseterm{Wintersemester}
\coursehours{4}
\courseinstructionlanguage{de}

\coursehead

% For index (key word@display). Key word is used for sorting - no Umlauts please.
\index{Software-Entwicklung @Software-Entwicklung }

% For later referencing
\label{cour_8211.dp_997}


\begin{styleenv}
\begin{assessment}
Die Erfolgskontrolle wird in der Modulbeschreibung erläutert.


\end{assessment}

\begin{conditions}Die Voraussetzungen werden in der Modulbeschreibung erläutert.

\end{conditions}


\end{styleenv}

\begin{learningoutcomes}
Die Teilnehmer lernen, ein vollständiges Softwareprojekt nach dem Stand der Softwaretechnik in einem Team mit ca. 5-7 Teilnehmern durchzuführen. Ziel ist es inbesondere, Verfahren des Software-Entwurfs und der Qualitätssicherung praktisch einzusetzen, Implementierungskompetenz umzusetzen, und arbeitsteilig im Team zu kooperieren.


\end{learningoutcomes}

\begin{content}
Erstellung des Pflichtenheftes incl. Verwendungsszenarien – Objektorientierter Entwurf nebst Feinspezifikation – Implementierung in einer objektorientierten Sprache – Funktionale Tests und Überdeckungstests – Einsatz von Werkzeugen (zB Eclipse, UML, Java, Junit, Jcov) – Präsentation des fertigen Systems


\end{content}

\begin{media}Unterlagen zu den vorangegangenen Lehrveranstaltungen des 1. und 2. Semesters, insbesondere zu \emph{Programmieren} und \emph{Softwaretechnik I}.

\end{media}





\end{course}