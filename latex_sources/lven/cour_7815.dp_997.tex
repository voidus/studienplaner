% Lehrveranstaltungsbeschreibung
% Informationsgrad : extern
% Sprache: de
\begin{course}

\setdoclanguagegerman
\coursedegreeprogramme{Informatik}
\coursemodulename{Supply Chain Management (S.~\pageref{mod_2721.dp_997})[IN3WWBWL14], Anwendungen des Operations Research (S.~\pageref{mod_3831.dp_997})[IN3WWOR2], Stochastische Methoden und Simulation (S.~\pageref{mod_3855.dp_997})[IN3WWOR4]}
\courseID{2550488}
\coursename{Taktisches und operatives Supply Chain Management}
\coursecoordination{S. Nickel}

\documentdate{2011-12-01 13:10:59.395993}

\courselevel{4}
\coursecredits{4,5}
\courseterm{Wintersemester}
\coursehours{2/1}
\courseinstructionlanguage{de}

\coursehead

% For index (key word@display). Key word is used for sorting - no Umlauts please.
\index{Taktisches und operatives Supply Chain Management@Taktisches und operatives Supply Chain Management}

% For later referencing
\label{cour_7815.dp_997}


\begin{styleenv}
\begin{assessment}
Die Erfolgskontrolle erfolgt in Form einer 120-minütigen schriftlichen Prüfung (nach §4(2), 1 SPO).

 

Die Prüfung wird jedes Semester angeboten.

 

Zulassungsvoraussetzung zur Klausur ist die erfolgreiche Teilnahme an den Online-Übungen.


\end{assessment}

\begin{conditions}Zulassungsvoraussetzung zur Klausur ist die erfolgreiche Teilnahme an den Online-Übungen.

\end{conditions}


\end{styleenv}

\begin{learningoutcomes}
Hauptziel der Vorlesung ist die Vermittlung grundlegender Verfahren aus den Bereichen der Beschaffungs- und Distributionslogistik, sowie Methoden der Lagerbestands- und Losgrößenplanung. Die Studierenden erwerben hiermit die Fähigkeit, quantitative Modelle in der Transportplanung (Langstreckenplanung und Auslieferungsplanung), dem Lagerhaltungsmanagement und der Losgrößenplanung in der Produktion einzusetzen. Die erlernten Verfahren werden in der parallel zur Vorlesung angebotenen Übung vertieft und anhand von Fallstudien praxisnah illustriert.


\end{learningoutcomes}

\begin{content}
Die Planung des Materialtransports ist wichtiger Bestandteil des Supply Chain Management. Durch eine Aneinanderreihung von Transportverbindungen und Zwischenstationen wird die Lieferstelle (Produzent) mit der Empfangsstelle (Kunde) verbunden.

 

Die allgemeine Belieferungsaufgabe lässt sich folgendermaßen formulieren (siehe Gudehus): Für vorgegebene Warenströme oder Sendungen ist aus den möglichen Logistikketten die optimale Liefer- und Transportkette auszuwählen, die bei Einhaltung der geforderten Lieferzeiten und Randbedingungen mit den geringsten Kosten verbunden ist. Ziel der Bestandsplanung im Warenlager ist die optimale Bestimmung der zu bestellenden Warenmengen, so dass die fixen und variablen Bestellkosten minimiert und etwaige Ressourcenbeschräkungen oder Vorgaben an die Lieferfähigkeit und den Servicegrad eingehalten werden. Ähnlich gelagert ist das Problem der Losgrößenplanung in der Produktion, das sich mit der optimale Bestimmung der an einem Stück zu produzierenden Produktmengen beschäftigt.

 

Gegenstand der Vorlesung ist eine Einführung in die Begriffe des Supply Chain Managements und die Vorstellung der wichtigsten quantitativen Planungsmodelle zur Distributions-, Touren-, Bestands-, und Losgrößenplanung. Darüber hinaus werden Fallstudien besprochen.


\end{content}



\begin{literature}\textbf{Weiterführende Literatur:}

 \begin{itemize}\item Domschke: Logistik: Transporte, 5. Auflage, Oldenbourg, 2005  \item Domschke: Logistik: Rundreisen und Touren, 4. Auflage, Oldenbourg, 1997  \item Ghiani, Laporte, Musmanno: Introduction to Logistics Systems Planning and Control, Wiley, 2004  \item Gudehus: Logistik, 3. Auflage, Springer, 2005  \item Simchi-Levi, Kaminsky, Simchi-Levi: Designing and Managing the Supply Chain, 3rd edition, McGraw-Hill, 2008  \item Silver, Pyke, Peterson: Inventory management and production planning and scheduling, 3rd edition, Wiley, 1998  \end{itemize}\end{literature}

\begin{remarks}Das für drei Studienjahre im Voraus geplante Lehrangebot kann im Internet nachgelesen werden.

\end{remarks}

\end{course}