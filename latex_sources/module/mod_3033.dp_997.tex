% Modulbeschreibung 
% Informationsgrad : extern
% Sprache: de
\begin{module}

\setdoclanguagegerman
\moduledegreeprogramme{Informatik (B.Sc.)}
\modulesubject{EF Betriebswirtschaftslehre}
\moduleID{IN3WWBWL}
\modulename{Grundlagen der BWL}
\modulecoordination{R. Hilser}

\documentdate{2010-04-09 09:45:59.937071}

\modulecredits{12}
\moduleduration{1}
\modulecycle{Jedes Semester}



\modulehead

% For index (key word@display). Key word is used for sorting - no Umlauts please.
\index{Grundlagen der BWL@Grundlagen der BWL (M)}

% For later referencing
\label{mod_3033.dp_997}

\begin{courselist}
2600002 & Rechnungswesen (S.~\pageref{cour_4301.dp_997}) & 2/2 & W & 4 & T. Lüdecke\\
2600024 & Allgemeine Betriebswirtschaftslehre B (S.~\pageref{cour_6081.dp_997}) & 2/0/2 & S & 4 & M. Ruckes, W. Fichtner, M. Klarmann, Th. Lützkendorf, F. Schultmann\\
2600026 & Allgemeine Betriebswirtschaftslehre C (S.~\pageref{cour_6109.dp_997}) & 2/0/2 & W & 4 & M. Ruckes, M. Uhrig-Homburg\\
\end{courselist}

\begin{styleenv}
\begin{assessment}
Die Erfolgskontrolle wird in den Lehrveranstaltungsbeschreibungen erläutert.


\end{assessment}

\begin{conditions}Dieses Modul ist Pflicht, wenn das Ergänzungsfach Wirtschaftswissenschaften, Fach BWL, abgelegt werden soll. Um das Fach abzuschliessen, muss ein weiteres Modul aus dem Fach BWL (Modulcode IN3WWBWL...) oder das Modul Entrepreneurship [IN3INEPS] mit 9 LP geprüft werden.

\end{conditions}


\end{styleenv}

\begin{learningoutcomes}
Der/die Studierende

 \begin{itemize}\item hat fundierte Kenntnisse in den zentralen Fragestellungen der Allgemeinen Betriebswirtschaftslehre insbesondere mit Blick auf entscheidungsorientiertes Handeln und die modellhafte Betrachtung der Unternehmung,  \item beherrscht die Grundlagen des betriebswirtschaftlichen Rechnungswesens und Grundlagen der Allgemeinen Betriebswirtschaftslehre,   \item ist in der Lage, die zentralen Tätigkeitsbereiche, Funktionen und Entscheidungen in einer marktwirtschaftlichen Unternehmung zu analysieren und zu bewerten.  \end{itemize}

Mit dem Basiswissen sind im Bereich BWL die Voraussetzungen geschaffen, dieses Wissen im Vertiefungsprogramm zu erweitern.


\end{learningoutcomes}

\begin{content}
Es werden die Grundlagen des internen und externen Rechnungswesen und der Allgemeinen Betriebswirtschaftslehre als die Lehre vom Wirtschaften im Betrieb vermittelt. Darauf aufbauend werden schwerpunktartig die Bereiche Marketing, Produktionswirtschaft, Informationswirtschaft, Unternehmensführung und Organisation, Investition und Finanzierung sowie Controlling erörtert.


\end{content}



\end{module}

