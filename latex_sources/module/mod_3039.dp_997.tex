% Modulbeschreibung 
% Informationsgrad : extern
% Sprache: de
\begin{module}

\setdoclanguagegerman
\moduledegreeprogramme{Informatik (B.Sc.)}
\modulesubject{EF Operations Research}
\moduleID{IN3WWOR}
\modulename{Grundlagen des OR}
\modulecoordination{R. Hilser}

\documentdate{2010-05-10 10:51:54.839634}

\modulecredits{12}
\moduleduration{1}
\modulecycle{Jedes Semester}



\modulehead

% For index (key word@display). Key word is used for sorting - no Umlauts please.
\index{Grundlagen des OR@Grundlagen des OR (M)}

% For later referencing
\label{mod_3039.dp_997}

\begin{courselist}
2550040 & Einführung in das Operations Research I (S.~\pageref{cour_4401.dp_997}) & 2/2/2 & S & 6 & S. Nickel, O. Stein, K. Waldmann\\
2530043 & Einführung in das Operations Research II (S.~\pageref{cour_4403.dp_997}) & 2/2/2 & W & 6 & S. Nickel, O. Stein, K. Waldmann\\
\end{courselist}

\begin{styleenv}
\begin{assessment}
Die Modulprüfung erfolgt in Form einer schriftlichen Gesamtklausur (120 min.) (nach §4(2), 1 SPO).

 

Die Klausur wird in jedem Semester (in der Regel im März und Juli) angeboten und kann zu jedem ordentlichen Prüfungstermin wiederholt werden.

 

Die Modulnote entspricht der Klausurnote.


\end{assessment}

\begin{conditions}Dieses Modul ist Pflicht im Ergänzungsfach Wirtschaftswissenschaften, Fach OR. Es ist ein weiteres Modul im Umfang von 9 LP aus dem Fach OR (Modulcode IN3WWOR...) zu prüfen.

\end{conditions}


\end{styleenv}

\begin{learningoutcomes}
Der/die Studierende

 \begin{itemize}\item benennt und beschreibt die Grundbegriffe der entscheidenden Teilbereiche im Fach Operations Research (Lineare Optimierung, Graphen und Netzwerke, Ganzzahlige und kombinatorische Optimierung, Nichtlineare Optimierung, Dynamische Optimierung und stochastische Modelle),  \item kennt die für eine quantitative Analyse unverzichtbaren Methoden und Modelle,  \item modelliert und klassifiziert Optimierungsprobleme und wählt geeignete Lösungsverfahren aus, um einfache Optimierungsprobleme selbständig zu lösen,  \item validiert, illustriert und interpretiert erhaltene Lösungen.  \end{itemize}
\end{learningoutcomes}

\begin{content}
Beispiel für typische OR-Probleme.

 

Lineare Optimierung: Grundbegriffe, Simplexmethode, Dualität, Sonderformen des Simplexverfahrens (duale Simplexmethode, Dreiphasenmethode), Sensitivitätsanalyse, Parametrische Optimierung, Spieltheorie

 

Graphen und Netzwerke: Grundbegriffe der Graphentheorie, kürzeste Wege in Netzwerken, Terminplanung von Projekten, maximale und kostenminimale Flüsse in Netzwerken.


\end{content}



\end{module}

