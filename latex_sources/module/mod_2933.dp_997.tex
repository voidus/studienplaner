% Modulbeschreibung 
% Informationsgrad : extern
% Sprache: de
\begin{module}

\setdoclanguagegerman
\moduledegreeprogramme{Informatik (B.Sc.)}
\modulesubject{}
\moduleID{IN3INALGAHS}
\modulename{Algorithmen für Ad-hoc- und Sensornetze}
\modulecoordination{D. Wagner}

\documentdate{2012-01-24 13:55:19.053313}

\modulecredits{3}
\moduleduration{1}
\modulecycle{Jedes 2. Semester, Sommersemester}



\modulehead

% For index (key word@display). Key word is used for sorting - no Umlauts please.
\index{Algorithmen fuer Ad-hoc- und Sensornetze@Algorithmen für Ad-hoc- und Sensornetze (M)}

% For later referencing
\label{mod_2933.dp_997}



\begin{styleenv}
\begin{assessment}
Die Erfolgskontrolle erfolgt in Form einer mündlichen Prüfung im Umfang von i.d.R. 20 Minuten nach § 4 Abs. 2 Nr. 2 SPO.\newline
Die Modulnote ist die Note der mündlichen Prüfung.


\end{assessment}

\begin{conditions}Keine.\end{conditions}

\begin{recommendations}Kenntnisse zu Grundlagen der Graphentheorie und Algorithmentechnik sind hilfreich.

\end{recommendations}
\end{styleenv}

\begin{learningoutcomes}
Die Studierenden erwerben ein systematisches Verständnis algorithmischer Fragestellungen in geometrisch verteilten Systemen und relevanter Techniken. Sie lernen am Beispiel von Problemen der Kommunikation und Selbstorganisation die Modellierung als geometrische und graphentheoretische Probleme kennen, sowie die Entwicklung und Analyse zentraler und verteilter Algorithmen zu deren Lösung. Sie sind fähig, diese Erkenntnisse auf andere Probleme zu übertragen und können mit dem erworbenen Wissen an aktuellen Forschungsthemen des akademischen Faches arbeiten.


\end{learningoutcomes}

\begin{content}
Sensornetze bestehen aus einer Vielzahl kleiner Sensorknoten, vollwertiger, wenngleich leistungsarmer Kleinstrechner, \newline
die drahtlos miteinander kommunizieren und ihre Umwelt mit Hilfe zumeist einfacher Sensorik beobachten. Die Entwicklung solcher Sensorknoten ist die Konsequenz immer kleiner und leistungsfähiger werdender Komponenten: Hochintegrierte Mikrocontroller, Speicher und Funkchips, Sensoren für Druck, Licht, Wärme, Chemikalien usw.\newline
Die technische Realisierbarkeit solcher Sensornetze hat in den letzten Jahren für ein großes Forschungsinteresse gesorgt. Es stellen sich interessante algorithmische Probleme durch den engen Zusammenhang von Geometrie und der Vernetzung der Knoten. Dazu gehören z.B. das Routing oder die Topologiekontrolle.\newline
Diese Vorlesung beschäftigt sich mit algorithmischen Fragestellungen unterschiedlicher Teilgebiete der Forschung in Sensor- und Ad-Hoc-Netzen, insbesondere mit unterschiedlichen Modellierungen als graphentheoretische oder geometrische Probleme sowie dem Entwurf verteilter Algorithmen.


\end{content}

\begin{remarks}\textcolor{red}{Dieses Modul wird nicht mehr angeboten, Prüfungen werden noch bis SS 2013 angeboten.}

\end{remarks}

\end{module}

