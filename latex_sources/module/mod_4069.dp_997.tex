% Modulbeschreibung 
% Informationsgrad : extern
% Sprache: de
\begin{module}

\setdoclanguagegerman
\moduledegreeprogramme{Informatik (B.Sc.)}
\modulesubject{}
\moduleID{IN2INBPHS}
\modulename{Basispraktikum TI: Hardwarenaher Systementwurf}
\modulecoordination{W. Karl}

\documentdate{2011-02-17 10:47:13.593500}

\modulecredits{4}
\moduleduration{1}
\modulecycle{Jedes Semester}



\modulehead

% For index (key word@display). Key word is used for sorting - no Umlauts please.
\index{Basispraktikum TI: Hardwarenaher Systementwurf@Basispraktikum TI: Hardwarenaher Systementwurf (M)}

% For later referencing
\label{mod_4069.dp_997}

\begin{courselist}
24309/24901 & Basispraktikum TI: Hardwarenaher Systementwurf (S.~\pageref{cour_8241.dp_997}) & 4 & W/S & 4 & W. Karl\\
\end{courselist}

\begin{styleenv}
\begin{assessment}
Die Erfolgskontrolle erfolgt benotet nach § 4 Abs. 2 Nr. 3 der SPO als Erfolgskontrolle anderer Art und besteht aus mehreren Teilaufgaben.

 

Die Modulnote entspricht dieser Note.


\end{assessment}

\begin{conditions}Keine.\end{conditions}

\begin{recommendations}Es wird der Besuch der LV \emph{Digitaltechnik und Entwurfsverfahren} empfohlen.

\end{recommendations}
\end{styleenv}

\begin{learningoutcomes}

\end{learningoutcomes}

\begin{content}
Der Entwurf von Schaltungen und integrierten Schaltkreisen erfolgt heute durch hochsprachlichen Entwurf mit Hilfe von Hardware¬Beschreibungs¬sprachen. \newline
Im Rahmen dieses Basispraktikums werden in Form von Übungsaufgaben Schaltungen mit Hilfe von Hardware-Beschreibungssprachen entworfen und mit Hilfe von Entwurfswerkzeugen implementiert und getestet. \newline
\newline
Das Praktikum umfasst

 \begin{itemize}\item die schrittweise Einführung in die Hardware-Beschreibungssprache VHDL  \item die schrittweise Einführung in Hardware-Entwurfswerkzeuge   \item die Einführung in programmierbare Logik-Bausteine und  \item den Schaltungsentwurf, die Implementierung und den Test von einfachen Schaltungen   \end{itemize}
\end{content}



\end{module}

