% Lehrveranstaltungsbeschreibung
% Informationsgrad : extern
% Sprache: de
\begin{course}

\setdoclanguagegerman
\coursedegreeprogramme{Informatik}
\coursemodulename{Wirtschaftsprivatrecht (S.~\pageref{mod_2655.dp_997})[IN3INJUR2]}
\courseID{24504}
\coursename{BGB für Fortgeschrittene}
\coursecoordination{T. Dreier, P. Sester}

\documentdate{2005-12-21 12:10:28}

\courselevel{1}
\coursecredits{3}
\courseterm{Sommersemester}
\coursehours{2/0}
\courseinstructionlanguage{de}

\coursehead

% For index (key word@display). Key word is used for sorting - no Umlauts please.
\index{BGB fuer Fortgeschrittene@BGB für Fortgeschrittene}

% For later referencing
\label{cour_4387.dp_997}


\begin{styleenv}
\begin{assessment}
Die Erfolgskontrolle erfolgt in Form schriftlicher Prüfungen (Klausuren) im Rahmen der Veranstaltung \emph{Privatrechtliche Übung} im Umfang von je 90 min. nach § 4, Abs. 2 Nr. 3 der SPO.


\end{assessment}

\begin{conditions}Es wird die Lehrveranstaltung \emph{BGB für Anfänger} [24012] vorausgesetzt.

\end{conditions}


\end{styleenv}

\begin{learningoutcomes}
Aufbauend auf den in der Vorlesung \emph{BGB für Anfänger} erworbenen Grundkenntnissen des Zivilrechts und insbesondere des allgemeinen Teils des Bürgerlichen Gesetzbuches (BGB) werden den Studenten in dieser Vorlesung Kenntnisse des allgemeinen und des besonderen Schuldrechts sowie des Sachenrechts vermittelt. Die Studenten wiederholen und vertiefen die gesetzlichen Grundregelungen von Leistungsort und Leistungszeit einschließlich der Modalitäten der Leistungsabwicklung sowie die gesetzliche Regelung des Rechts der Leistungsstörungen (Unmöglichkeit, Nichtleistung, verspätete Leistung, Schlechtleistung). Im Weiteren werden die Studenten mit den Grundzügen der gesetzlichen Vertragstypen und der Verschuldens- wie auch der Gefährdungshaftung vertraut gemacht. Aus dem Sachenrecht sollen die Studenten die unterschiedlichen Arten der Übereignung unterscheiden können und einen Überblick über die dinglichen Sicherungsrechte gewinnen.


\end{learningoutcomes}

\begin{content}
Aufbauend auf den in der Vorlesung BGB für Anfänger
erworbenen Grundkenntnissen des Zivilrechts und insbesondere des
allgemeinen Teils des Bürgerlichen Gesetzbuches (BGB)
behandelt die Vorlesung die gesetzlichen Regelungen des allgemeinen
und des besonderen Schuldrechts, also zum einen die gesetzlichen
Grundregelungen von Leistungsort und Leistungszeit
einschließlich der Modalitäten der Leistungsabwicklung
und des Rechts der Leistungsstörungen (Unmöglichkeit,
Nichtleistung, verspätete Leistung, Schlechtleistung). Zum
anderen werden die gesetzlichen Vertragstypen (insbesondere Kauf,
Miete, Werk- und Dienstvertrag, Leihe, Darlehen), vorgestellt und
Mischtypen besprochen (Leasing, Factoring, neuere
Computerverträge). Darüber hinaus wird das Haftungsrecht
in den Formen der Verschuldens- und der Gefährdungshaftung
besprochen. Im Sachenrecht geht es um Besitz und Eigentum, um die
verschiedenen Übereignungstatbestände sowie um die
wichtigsten dinglichen Sicherungsrechte.
\end{content}

\begin{media}Folien\end{media}

\begin{literature}Wird in der Vorlesung bekannt gegeben.

\textbf{Weiterführende Literatur:}

Wird in der Vorlesung bekannt gegeben.

\end{literature}



\end{course}