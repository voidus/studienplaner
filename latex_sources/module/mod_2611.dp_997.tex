% Modulbeschreibung 
% Informationsgrad : extern
% Sprache: de
\begin{module}

\setdoclanguagegerman
\moduledegreeprogramme{Informatik (B.Sc.)}
\modulesubject{Praktische Informatik}
\moduleID{IN2INSWP}
\modulename{Praxis der Software-Entwicklung}
\modulecoordination{G. Snelting}

\documentdate{2011-12-29 11:46:03.954056}

\modulecredits{6}
\moduleduration{1}
\modulecycle{Jedes Semester}



\modulehead

% For index (key word@display). Key word is used for sorting - no Umlauts please.
\index{Praxis der Software-Entwicklung@Praxis der Software-Entwicklung (M)}

% For later referencing
\label{mod_2611.dp_997}

\begin{courselist}
PSE & Software-Entwicklung  (S.~\pageref{cour_8211.dp_997}) & 4 & W & 6 & G. Snelting\\
\end{courselist}

\begin{styleenv}
\begin{assessment}
Die Erfolgskontrolle erfolgt nach § 4 Abs. 2 Nr. 3 SPO als benotete Erfolgskontrolle anderer Art.

 

Die in den Anmerkungen genannten Artefakte werden separat benotet und gehen mit folgendem Prozentsatz in die Gesamtnote ein:\newline
\newline
Pflichtenheft 10\%\newline
Entwurf 30\%\newline
Implementierung 30\%\newline
Qualitätssicherung 20\%\newline
Abschlusspräsentation 10\%


\end{assessment}

\begin{conditions}Das Modul muss zusammen mit dem Modul \emph{Teamarbeit in der Software-Entwicklung} [IN2INSWPS] belegt werden.

 

Der erfolgreiche Abschluss der Module \emph{Grundbegriffe der Informatik} [IN1INGI],\emph{ Programmieren} [IN1INPROG] wird vorausgesetzt.

\end{conditions}

\begin{recommendations}Die Veranstaltung sollte erst belegt werden, wenn alle Module aus den ersten beiden Semestern abgeschlossen sind.

\end{recommendations}
\end{styleenv}

\begin{learningoutcomes}
Die Teilnehmer lernen, ein vollständiges Softwareprojekt nach dem Stand der Softwaretechnik in einem Team mit ca. 5-7 Teilnehmern durchzuführen. Ziel ist es insbesondere, Verfahren des Software-Entwurfs und der Qualitätssicherung praktisch einzusetzen, Implementierungskompetenz umzusetzen, und arbeitsteilig im Team zu kooperieren.


\end{learningoutcomes}

\begin{content}
Erstellung des Pflichtenheftes incl. Verwendungsszenarien – Objektorientierter Entwurf nebst Feinspezifikation – Implementierung in einer objektorientierten Sprache – Funktionale Tests und Überdeckungstests – Einsatz von Werkzeugen (z.B. Eclipse, UML, Java, Junit, Jcov) – Präsentation des fertigen Systems


\end{content}

\begin{remarks}Zur Struktur: Das Praktikum gliedert sich in die Phasen Pflichtenheft, Entwurf und Feinspezifikation, Implementierung, Qualitätssicherung, Abschlusspräsentation. Alle Phasen werden nach dem Stand der Softwaretechnik objektorientiert und werkzeugunterstützt durchgeführt. Zu jeder Phase muss das entsprechende Artefakt (Pflichtenheft, UML-Diagramme mit Erläuterungen, vollständiger Java-Quellcode, Testprotokolle, laufendes System) in einem Kolloquium präsentiert werden. Das vollständige System wird von den Betreuern auf Funktionalität, Bedienbarkeit und Robustheit geprüft.

 

PSE kann im 3. oder 4. Semester besucht werden. Falls die Fakultät im 3. Sem nicht genug Plätze anbieten kann, werden die Anmeldungen bevorzugt, die die o.g. Empfehlung (erfolgreicher Abschluss der Module des 1. Studienjahres) erfüllen. Alle anderen Anmeldungen erhalten einen Platz im 4. Sem.

\end{remarks}

\end{module}

