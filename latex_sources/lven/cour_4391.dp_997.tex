% Lehrveranstaltungsbeschreibung
% Informationsgrad : extern
% Sprache: de
\begin{course}

\setdoclanguagegerman
\coursedegreeprogramme{Informatik}
\coursemodulename{Verfassungs- und Verwaltungsrecht (S.~\pageref{mod_2661.dp_997})[IN3INJUR3]}
\courseID{24016}
\coursename{Öffentliches Recht I - Grundlagen}
\coursecoordination{I. Spiecker genannt Döhmann}

\documentdate{2012-01-19 11:20:12.199826}

\courselevel{2}
\coursecredits{3}
\courseterm{Wintersemester}
\coursehours{2/0}
\courseinstructionlanguage{de}

\coursehead

% For index (key word@display). Key word is used for sorting - no Umlauts please.
\index{OEffentliches Recht I - Grundlagen@Öffentliches Recht I - Grundlagen}

% For later referencing
\label{cour_4391.dp_997}


\begin{styleenv}
\begin{assessment}
Die Erfolgskontrolle erfolgt in Form einer schriftlichen Prüfung im Umfang von i.d.R. 60 Minuten nach § 4 Abs. 2 Nr. 1 SPO.


\end{assessment}

\begin{conditions}Keine.\end{conditions}

\begin{recommendations}Parallel zu den Veranstaltungen werden begleitende Tutorien angeboten, die insbesondere der Vertiefung der juristischen Arbeitsweise dienen. Ihr Besuch wird nachdrücklich empfohlen.\newline
Während des Semesters wird eine Probeklausur zu jeder Vorlesung mit ausführlicher Besprechung gestellt. Außerdem wird eine Vorbereitungsstunde auf die Klausuren in der vorlesungsfreien Zeit angeboten.\newline
Details dazu auf der Homepage des ZAR (www.kit.edu/zar).

\end{recommendations}
\end{styleenv}

\begin{learningoutcomes}
Die Vorlesung vermittelt die Grundlagen des öffentlichen Rechts. Die Studierenden sollen die staatsorganisationsrechtlichen\newline
Grundlagen, die Grundrechte, die das staatliche Handeln und das gesamte Rechtssystem steuern, sowie die Handlungsmöglichkeiten und -formen (insb. Gesetz, Verwaltungsakt, Öff.-rechtl. Vertrag) der öffentlichen Hand kennen\newline
lernen. Ferner wird der Unterschied zwischen dem Privatrecht und dem öffentlichem Recht verdeutlicht. Darüber sollen\newline
die Rechtsschutzmöglichkeiten mit Blick auf das behördliche Handeln erarbeitet werden. Die Studierenden sollen\newline
Probleme im öffentlichen Recht einordnen lernen und einfache Fälle mit Bezug zum öffentlichen Recht lösen können.


\end{learningoutcomes}

\begin{content}
Die Vorlesung umfasst Kernaspekte des Verfassungsrechts (Staatsrecht und Grundrechte) und des Verwaltungsrechts. In einem ersten Schritt wird der Unterschied zwischen dem Privatrecht und dem öffentlichem Recht verdeutlicht. Im verfassungsrechtlichen Teil werden schwerpunktmässig das Rechtsstaatsprinzip des Grundgesetzes und die Grundrechte besprochen (v.a. die Kommunikations- und Wirtschaftsgrundrechte). Im verwaltungsrechtlichen Teil werden die verschiedenen Formen des behördlichen Handelns (Verwaltungsakt; Öffentlichrechtlicher Vertrag; Rechtsverordnungen etc.) behandelt und ihre Voraussetzungen besprochen. Ferner werden die Rechtsschutzmöglichkeiten in Bezug auf behördliches Handeln erarbeitet. Die Studenten werden an die Falllösungstechnik im Öffentlichen Recht herangeführt.


\end{content}

\begin{media}Ausführliches Skript mit Fällen, Gliederungsübersichten, Unterlagen in den Veranstaltungen.

\end{media}

\begin{literature}Wird in der Vorlesung bekannt gegeben.

 

\textbf{Weiterführende Literatur:}

 

Wird in der Vorlesung bekannt gegeben.

\end{literature}

\begin{remarks}Zum WS 08/09 wurde der Vorlesungsturnus der Veranstaltung Öffentliches Recht I+II von SS/WS auf WS/SS umgestellt.\newline
D.h.:

 \begin{enumerate}\item Im Wintersemester 08/09 fand die Vorlesung ÖRecht I statt.  \item Im Sommersemester 09 findet die Vorlesung ÖRecht II statt.  \end{enumerate}\end{remarks}

\end{course}