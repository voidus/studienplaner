% Lehrveranstaltungsbeschreibung
% Informationsgrad : extern
% Sprache: de
\begin{course}

\setdoclanguagegerman
\coursedegreeprogramme{Informatik}
\coursemodulename{Web-Anwendungen und Web-Technologien (S.~\pageref{mod_2925.dp_997})[IN3INWAWT]}
\courseID{24304/24873}
\coursename{Praktikum Web-Technologien}
\coursecoordination{S. Abeck, Gebhart, Hoyer, Link, Pansa}

\documentdate{2011-11-14 11:33:03.503731}

\courselevel{4}
\coursecredits{5}
\courseterm{Winter-/Sommersemester}
\coursehours{2/0}
\courseinstructionlanguage{de}

\coursehead

% For index (key word@display). Key word is used for sorting - no Umlauts please.
\index{Praktikum Web-Technologien@Praktikum Web-Technologien}

% For later referencing
\label{cour_5681.dp_997}


\begin{styleenv}
\begin{assessment}
Die Erfolgskontrolle wird in der Modulbeschreibung erläutert.


\end{assessment}

\begin{conditions}Keine.\end{conditions}


\end{styleenv}

\begin{learningoutcomes}
Die in einer realen Projektumgebung eingesetzten Web-Technologien werden durchdrungen.

 

Die Aufgabenstellung des Praktikums wird verstanden und kann in eigenen Worten formuliert werden.

 

Die Web-Technologien können zur Lösung der Aufgabe angewendet werden.

 

Die erzielten Ergebnisse können klar und verständlich dokumentiert und präsentiert werden.


\end{learningoutcomes}

\begin{content}
Der Praktikant wird in eines der in der Forschungsgruppe laufenden Projektteams integriert und erhält eine klar umgrenzte Aufgabe, in der er/sie einen Teil einer fortgeschrittenen Web-Anwendung mittels aktueller Web-Technologien zu erstellen hat.

 

Beispiele für solche Aufgabenstellungen sind:

 \begin{itemize}\item Einsatz von Portaltechnologien zur Erstellung der Benutzerschnittstelle einer Web-Anwendung  \item Entwurf und Implementierung von Webservices unter Nutzung des Java-Rahmenwerks  \item Erweiterung einer Zugriffskontrolle auf eine dienstorientierte Web-Anwendung unter Nutzung einer bestehenden Identitätsmanagementlösung  \end{itemize}
\end{content}

\begin{media}Vorlagen zur effizienten Ergebnisdokumentation (z.B. Projektdokumente, Präsentationsmaterial)

\end{media}

\begin{literature}\begin{itemize}\item Anleitung der Forschungsgruppe zur Durchführung von Arbeiten im Projektteam  \item Vorlesungsskript „Advanced Web Applications”  \end{itemize}

\textbf{Weiterführende Literatur:}

 

Literaturbestand des jeweiligen Projektteams

\end{literature}

\begin{remarks}\textcolor{red}{Diese Lehrveranstaltung wurde SS 2011 letztmalig angeboten. Prüfungen sind für Wiederholer bis Wintersemester 2012/13 möglich.}

\end{remarks}

\end{course}