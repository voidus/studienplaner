% Lehrveranstaltungsbeschreibung
% Informationsgrad : extern
% Sprache: de
\begin{course}

\setdoclanguagegerman
\coursedegreeprogramme{Informatik}
\coursemodulename{Grundlagen der Nachrichtentechnik (S.~\pageref{mod_3965.dp_997})[IN3EITGNT], Systemtheorie (S.~\pageref{mod_3937.dp_997})[IN3EITST]}
\courseID{23155}
\coursename{Systemdynamik und Regelungstechnik}
\coursecoordination{M. Kluwe}

\documentdate{2012-01-09 11:15:19.291642}

\courselevel{3}
\coursecredits{5}
\courseterm{Sommersemester}
\coursehours{2/1}
\courseinstructionlanguage{de}

\coursehead

% For index (key word@display). Key word is used for sorting - no Umlauts please.
\index{Systemdynamik und Regelungstechnik@Systemdynamik und Regelungstechnik}

% For later referencing
\label{cour_6891.dp_997}


\begin{styleenv}
\begin{assessment}
Die Erfolgskontrolle erfolgt in Form einer schriftlichen Prüfung im Umfang von 120 Minuten (nach § 4 Abs. 2 Nr. 1 SPO) in der vorlesungsfreien Zeit des Semesters.

 

Die Prüfung wird in jedem Semester angeboten und kann zu jedem ordentlichen Prüfungstermin wiederholt werden.

 

Die Note der Lehrveranstaltung ist die Note der Klausur.


\end{assessment}

\begin{conditions}Kenntnisse in der höheren Mathematik, “Wahrscheinlichkeitstheorie” (1305) und “Signale und Systeme” (23109) werden vorausgesetzt.

\end{conditions}

\begin{recommendations}Es werden Kenntnisse über Integraltransformationen vorausgesetzt. Daher empfiehlt es sich, die Lehrveranstaltungen \emph{Komplexe Analysis und Integraltransformationen} im Vorfeld zu besuchen.

\end{recommendations}
\end{styleenv}

\begin{learningoutcomes}
Der/die Studierende

 \begin{itemize}\item kennt die grundlegende Begriffe der Regelungstechnik,  \item kennt und versteht die Elemente sowie die Struktur und das Verhalten dynamischer Systeme,  \item besitzt grundlegende Kenntnisse der Aufgabenstellungen beim Reglerentwurf und entsprechende Lösungsmethoden im Frequenz- und Zeitbereich  \end{itemize}
\end{learningoutcomes}

\begin{content}
\textbf{Einführung: }Übersicht und Begriffsbildung, Steuerung und Regelung, Entwicklungsablauf für Regelungssysteme

 

\textbf{Klassifizierung und Beschreibung von linearen Regelkreisen: }Einführung und Grundbegriffe, Signalflussbild, Verhalten elementarer zeitkontinuierlicher Regelkreisglieder, Standardregelkreis und Signalflussbildumformungen, Aufbau digitaler Regelkreise, Beschreibung digitaler Regelkreise, Simulation zeitkontinuierlicher Regelkreise

 

\textbf{Analyse von linearen zeitkontinuierlichen Regelkreisen: }Stationäres Verhalten und charakteristische Größen, Frequenzgang und Ortskurve, Frequenzkennlinien, Grundlagen zur Stabilität, Algebraische Stabilitätskriterien, Graphische Stabilitätskriterien

 

\textbf{Analyse von linearen zeitdiskreten Regelkreisen: }Stationäres Verhalten, Frequenzgang, Ortskurve und Frequenzkennlinie, Grundlagen zur Stabilität, Algebraische Stabilitätskriterien, Graphische Stabilitätskriterien

 

\textbf{Synthese von linearen zeitkontinuierlichen Regelkreisen: }Forderungen an den Regelkreis, Direkte Verfahren, Entwurf mit dem Frequenzkennlinienverfahren, Entwurf mit dem Wurzelortskurvenverfahren, Heuristische Verfahren, Vermaschte Regelkreise

 

\textbf{Synthese von linearen zeitdiskreten Regelkreisen: }Fast Sampling Design, Direkte Verfahren, Frequenzkennlinienverfahren und Wurzelortskurvenverfahren

 

\textbf{Das \textbf{Berufsbild des Automatisierungstechnikers}}


\end{content}

\begin{media}Die Unterlagen zur Lehrveranstaltung finden sich online unterwww.irs.kit.eduunter „Studium und Lehre“ und können dort mit einem Passwort heruntergeladen werden.

\end{media}

\begin{literature}\begin{itemize}\item O. Föllinger unter Mitwirkung von F. Dörrscheidt und M. Klittich:\newline
 Regelungstechnik, Einführung in die Methoden und ihre Anwendung\newline
 10. Auflage, Hüthig-Verlag, 2008  \item J. Lunze:\newline
 Regelungstechnik I\newline
 7. Auflage, Springer-Verlag, 2008  \item R. Dorf - R. Bishop:\newline
 Modern Control Systems \newline
 11th edition, Addison-Wesley, 2007  \item C. Phillips - R. Harbor: \newline
 Feedback Control Systems \newline
 4th edition, Prentice-Hall, 2000  \item O. Föllinger:\newline
 Lineare Abtastsysteme\newline
 5. Auflage, R. Oldenbourg Verlag, 1993  \item K. Ogata:\newline
 Discrete-Time control systems\newline
 Prentice Hall Verlag, 1987  \end{itemize}\begin{itemize}\item G.C. Goodwin:\newline
 Control System Design\newline
 Prentice Hall Verlag,  \end{itemize}\end{literature}



\end{course}