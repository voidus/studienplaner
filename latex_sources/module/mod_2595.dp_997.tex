% Modulbeschreibung 
% Informationsgrad : extern
% Sprache: de
\begin{module}

\setdoclanguagegerman
\moduledegreeprogramme{Informatik (B.Sc.)}
\modulesubject{}
\moduleID{IN3INMK}
\modulename{Mobilkommunikation}
\modulecoordination{M. Zitterbart}

\documentdate{2010-06-07 09:55:48.862067}

\modulecredits{4}
\moduleduration{1}
\modulecycle{Jedes 2. Semester, Sommersemester}



\modulehead

% For index (key word@display). Key word is used for sorting - no Umlauts please.
\index{Mobilkommunikation@Mobilkommunikation (M)}

% For later referencing
\label{mod_2595.dp_997}

\begin{courselist}
24643 & Mobilkommunikation (S.~\pageref{cour_5385.dp_997}) & 2/0 & S & 4 & O. Waldhorst\\
\end{courselist}

\begin{styleenv}
\begin{assessment}
Die Erfolgskontrolle erfolgt in Form einer mündlichen Prüfung im Umfang von I.d.R. 20 Minuten gemäß § 4 Abs. 2 Nr. 2 SPO.

 

Die Modulnote ist die Note der mündlichen Prüfung.


\end{assessment}

\begin{conditions}Inhalte der Vorlesung \emph{Einführung in Rechnernetze} [24519] (Teil des Pflichtmoduls \emph{Kommunikation und Datenhaltung} [IN3INKD]) und des Stammmoduls \emph{Telematik }[IN3INTM] werden vorausgesetzt.

 

Das Stammmodul Telematik muss belegt und geprüft werden.

\end{conditions}


\end{styleenv}

\begin{learningoutcomes}
Ziel der Vorlesung ist es, die technischen Grundlagen der Mobilkommunikation (Signalausbreitung, Medienzugriff, etc.) zu vermitteln. Zusätzlich werden aktuelle Entwicklungen in der Forschung (Mobile IP, Ad-hoc Netze, Mobile TCP, etc.) betrachtet.


\end{learningoutcomes}

\begin{content}
Die Vorlesung “Mobilkommunikation” beginnt mit einer Diskussion der historischen Entwicklung mobiler Kommunikationssysteme sowie deren Einfluss auf unser Leben. Als Grundlagen für das Verständnis der später behandelten Systeme werden Frequenzbereiche, Signale, Modulation und Multiplextechniken besprochen. Anhand von Beispielen werden verschiedene Architekturen für Mobilfunknetze erläutert, insbesondere zellulare Kommunikationsnetze (z.B. GSM, UMTS), drahtlose LANs (Local Area Networks, z.B. IEEE 802.11), drahtlose MANs (Metropolitan Area Networks, z.B. IEEE 802.16) und drahtlose PANs (Personal Area Networks, z.B. Bluetooth, ZigBee). Die Realisierung von IP-basierter Kommunikation über diese Netze mit Hilfe von Mobile IP ist ein weiteres Thema. Kapitel zu selbstorganisierenden Netzen (Mobile Ad-hoc Netze) und zur Positionsbestimmung mit Hilfe von mobilen Geräten schließen die Vorlesung ab.


\end{content}



\end{module}

