% Lehrveranstaltungsbeschreibung
% Informationsgrad : extern
% Sprache: de
\begin{course}

\setdoclanguagegerman
\coursedegreeprogramme{Informatik}
\coursemodulename{Anwendungen des Operations Research (S.~\pageref{mod_3831.dp_997})[IN3WWOR2]}
\courseID{2550490}
\coursename{Software-Praktikum: OR-Modelle I}
\coursecoordination{S. Nickel}

\documentdate{2011-12-01 13:05:40.339121}

\courselevel{4}
\coursecredits{4,5}
\courseterm{Wintersemester}
\coursehours{1/2}
\courseinstructionlanguage{de}

\coursehead

% For index (key word@display). Key word is used for sorting - no Umlauts please.
\index{Software-Praktikum: OR-Modelle I@Software-Praktikum: OR-Modelle I}

% For later referencing
\label{cour_7845.dp_997}


\begin{styleenv}
\begin{assessment}
Die Erfolgskontrolle erfolgt in Form einer Prüfung mit schriftlichem und praktischem Teil (nach §4(2), 1 SPO).

 

Die Prüfung wird im Semester des Software-Praktikums und dem darauf folgenden Semester angeboten.


\end{assessment}

\begin{conditions}Sichere Kenntnisse des Stoffs aus der Vorlesung \emph{Einführung in das Operations Research I} [2550040] im Modul \emph{Operations Research} [WI1OR].

\end{conditions}


\end{styleenv}

\begin{learningoutcomes}
Die Veranstaltung hat das Ziel, die Studierenden mit den Einsatzmöglichkeiten des Computers in der praktischen Anwendung von Methoden des Operations Research vertraut zu machen. Ein großer Nutzen liegt in der erworbenen Fähigkeit, die grundlegenden Möglichkeiten und Verwendungszwecke von Modellierungssoftware und Implementierungssprachen für OR Modelle einzuordnen und abzuschätzen. Da die Software in vielen Unternehmen eingesetzt wird, ist die Veranstaltung für praktische Tätigkeiten im Planungsbereich von großem Nutzen.


\end{learningoutcomes}

\begin{content}
Nach einer Einführung in die allgemeinen Konzepte von Modellierungstools (Implementierung, Datenhandling, Ergebnisinterpretation, ...) wird konkret das Programm Xpress-MP IVE und dessen Modellierungssprache Mosel vorgestellt.

 

Im Anschluss daran werden Übungsaufgaben ausführlich behandelt. Ziele der aus Lehrbuch- und Praxisbeispielen bestehenden Aufgaben liegen in der Modellierung linearer und gemischt-ganzzahliger Programme, dem sicheren Umgang mit den vorgestellten Tools zur Lösung dieser Optimierungsprobleme, sowie der Implementierung heuristischer Lösungsverfahren für gemischt-ganzzahlige Probleme.


\end{content}





\begin{remarks}Aufgrund der begrenzten Teilnehmerzahl wird um eine Voranmeldung gebeten. Weitere Informationen entnehmen Sie der Internetseite des Software-Praktikums.

 

Das für drei Studienjahre im Voraus geplante Lehrangebot kann im Internet nachgelesen werden.

\end{remarks}

\end{course}