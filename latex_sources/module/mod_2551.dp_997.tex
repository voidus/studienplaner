% Modulbeschreibung 
% Informationsgrad : extern
% Sprache: de
\begin{module}

\setdoclanguagegerman
\moduledegreeprogramme{Informatik (B.Sc.)}
\modulesubject{}
\moduleID{IN2MATHPM}
\modulename{Praktische Mathematik}
\modulecoordination{C. Wieners, N. Henze}

\documentdate{2008-09-24 12:28:33}

\modulecredits{9}
\moduleduration{}
\modulecycle{}



\modulehead

% For index (key word@display). Key word is used for sorting - no Umlauts please.
\index{Praktische Mathematik@Praktische Mathematik (M)}

% For later referencing
\label{mod_2551.dp_997}

\begin{courselist}
01874 & Numerische Mathematik für die Fachrichtungen Informatik und Ingenieurwesen (S.~\pageref{cour_7197.dp_997}) & 2/1 & S & 4,5 & C. Wieners, Neuß,  Rieder\\
01335 & Grundlagen der Wahrscheinlichkeitstheorie und Statistik für Studierende der Informatik (S.~\pageref{cour_7235.dp_997}) & 2/1 & W & 4,5 & D. Kadelka\\
\end{courselist}

\begin{styleenv}
\begin{assessment}
Die Erfolgskontrolle wird in den jeweiligen Lehrveranstaltungsbeschreibungen erläutert.

 

Die Gesamtnote des Moduls wird aus den mit LP gewichteten Noten der Teilprüfungen gebildet und nach der ersten Kommastelle abgeschnitten.


\end{assessment}

\begin{conditions}Keine.

\end{conditions}

\begin{recommendations}Für die Teilnahme an der Prüfung zu \emph{Numerische Mathematik für die Fachrichtungen Informatik und Ingenieurwesen} [1874] sollte das Modul \emph{Höhere Mathematik} [IN1MATHHM] bzw. \emph{Analysis} [IN1MATHANA] abgeschlossen sein.

\end{recommendations}
\end{styleenv}

\begin{learningoutcomes}
Die Lernziele werden in der Lehrveranstaltungsbeschreibung näher erläutert.


\end{learningoutcomes}

\begin{content}
Die Inhalte werden in den Lehrveranstaltungsbeschreibungen erläutert.


\end{content}

\begin{remarks}Moduldauer: 2 Semester

 

Das Modul kann erst ab dem WS 09/10 belegt werden.

\end{remarks}

\end{module}

