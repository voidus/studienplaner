% Modulbeschreibung 
% Informationsgrad : extern
% Sprache: de
\begin{module}

\setdoclanguagegerman
\moduledegreeprogramme{Informatik (B.Sc.)}
\modulesubject{EF Informationsmanagement im Ingenieurwesen}
\moduleID{IN3MACHVEMP}
\modulename{Virtual Engineering für mechatronische Produkte}
\modulecoordination{Maier}

\documentdate{2011-07-19 13:04:13.231849}

\modulecredits{4}
\moduleduration{1}
\modulecycle{Jedes 2. Semester, Wintersemester}



\modulehead

% For index (key word@display). Key word is used for sorting - no Umlauts please.
\index{Virtual Engineering fuer mechatronische Produkte@Virtual Engineering für mechatronische Produkte (M)}

% For later referencing
\label{mod_4293.dp_997}

\begin{courselist}
2121370 & Virtual Engineering für mechatronische Produkte (S.~\pageref{cour_7511.dp_997}) & 3/0 & W & 4 & J. Ovtcharova, S. Rude\\
\end{courselist}

\begin{styleenv}
\begin{assessment}
Die Erfolgskontrolle erfolgt in Form einer mündlichen Prüfung im Umfang von 30 Minuten (nach § 4 (2), 2 SPO).

 

Die Note entspricht der Note der mündlichen Prüfung.


\end{assessment}

\begin{conditions}Die Module \emph{Virtual Engineering I} und \emph{Virtual Engineering II} und \emph{Product Lifecycle Management} müssen belegt werden.

\end{conditions}


\end{styleenv}

\begin{learningoutcomes}
Der/ die Studierende

 \begin{itemize}\item versteht die Vorgehensweise zur Integration mechatronischer Komponenten in Produkte,  \item  versteht die besonderen Anforderungen funktional vernetzter Systeme.  \end{itemize}
\end{learningoutcomes}

\begin{content}
Der Einzug mechatronischer Komponenten in alle Produkte verändert geometrieorientierte Konstruktionsabläufe in funktionsorientierte Abläufe. Damit verbunden ist die Anwendung von IT-Systemen neu auszurichten. Die Vorlesung behandelt hierzu:

 \begin{itemize}\item Herausforderungen an den Konstruktionsprozess aus der Sicht der Integration mechatronischer Komponenten in Produkte,  \item Unterstützung der Aufgabenklärung durch Anforderungsmanagement,  \item Lösungsfindung auf Basis funktional vernetzter Systeme,  \item Realisierung von Lösungen auf Basis von Elektronik (Sensoren, Aktuatoren, vernetzte Steuergeräte),  \item Beherrschung verteilter Software-Systeme durch Software-Engineering und  \item Herausforderungen an Test und Absicherung aus der Sicht zu erreichender Systemqualität.   \end{itemize}

Anwendungsfelder und Systembeispiele stammen aus der Automobilindustrie.


\end{content}

\begin{remarks}\textcolor{red}{Das Modul wird nicht mehr angeboten, Prüfungen sind möglich bis Wintersemester 2012/13.}

\end{remarks}

\end{module}

