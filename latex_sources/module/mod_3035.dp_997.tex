% Modulbeschreibung 
% Informationsgrad : extern
% Sprache: de
\begin{module}

\setdoclanguagegerman
\moduledegreeprogramme{Informatik (B.Sc.)}
\modulesubject{EF Volkswirtschaftslehre}
\moduleID{IN3WWVWL}
\modulename{Grundlagen der VWL}
\modulecoordination{R. Hilser}

\documentdate{2010-04-09 09:47:01.590285}

\modulecredits{12}
\moduleduration{1}
\modulecycle{Jedes Semester}



\modulehead

% For index (key word@display). Key word is used for sorting - no Umlauts please.
\index{Grundlagen der VWL@Grundlagen der VWL (M)}

% For later referencing
\label{mod_3035.dp_997}

\begin{courselist}
2600014 & Volkswirtschaftslehre II: Makroökonomie (S.~\pageref{cour_6845.dp_997}) & 3/0/2 & S & 6 & B. Wigger\\
2600012 & Volkswirtschaftslehre I: Mikroökonomie (S.~\pageref{cour_4329.dp_997}) & 3/0/2 & W & 6 & G. Liedtke\\
\end{courselist}

\begin{styleenv}
\begin{assessment}
Die Modulprüfung erfolgt in Form von schriftlichen Teilprüfungen (nach §4(2), 1 SPO) über die einzelnen Lehrveranstaltungen des Moduls.

 

Die Gesamtnote des Moduls wird aus den mit LP gewichteten Noten der Teilprüfungen gebildet und nach der ersten Nachkommastelle abgeschnitten.

 

Die Erfolgskontrolle wird bei jeder Lehrveranstaltung dieses Moduls beschrieben.


\end{assessment}

\begin{conditions}Das Modul ist Pflicht für das Ergänzungsfach Wirtschaftswissenschaften, Fach VWL. Es muss ein weiteres Modul aus der VWL mit 9 LP geprüft werden (Modulcode IN3WWVWL...).

\end{conditions}


\end{styleenv}

\begin{learningoutcomes}
Der/die Studierende

 \begin{itemize}\item kennt und versteht die grundsätzlichen volkswirtschaftlichen Fragestellungen,  \item versteht die aktuellen wirtschaftspolitischen Probleme der globalisierten Welt,   \item ist in der Lage, elementare Lösungsstrategien zu entwickeln.   \end{itemize}

Dabei ist der Fokus der beiden Lehrveranstaltungen des Moduls unterschiedlich. Während in der Vorlesung \emph{VWL I} die ökonomischen Probleme hauptsächlich als Entscheidungsprobleme aufgefasst und gelöst werden, soll in \emph{VWL II} das Verständnis des Studenten für die Dynamik wirtschaftlicher Prozesse gefördert werden.


\end{learningoutcomes}

\begin{content}
Das Modul vermittelt fundierte Grundlagenkenntnisse in Mikro- und Makroökonomischer Theorie. Neben Haushalts- und Firmenentscheidungen werden auch Probleme des Allgemeinen Gleichgewichts auf Güter- und Arbeitsmärkten behandelt. Der Hörer der Vorlesung soll schließlich auch in die Lage versetzt werden, grundlegende spieltheoretische Argumentationsweisen, wie sie sich in der modernen VWL durchgesetzt haben, zu verstehen.

 

In der \emph{VWL I }werden Fragen der mikroökonomischen Entscheidungstheorie (Haushalts- und Firmenentscheidungen) sowie Fragen der Markttheorie (Gleichgewichte und Effizienz auf Konkurrenz-Märkten) behandelt. Im letzten Teil der Vorlesung werden Probleme des unvollständigen Wettbewerbs (Oligopolmärkte) sowie Grundzüge der Spieltheorie vermittelt.

 

Die \emph{VWL II }vermittelt Volkswirtschaftliches Denken, Kenntnisse über Ordnungsmodelle in der Volkswirtschaft, Deutschland im Zeitalter der Globalisierung, Volkswirtschaftliche Gesamtrechnung, Außenhandel und Zahlungsbilanz, Geld und Kredit, Gesamtwirtschaftliches Gleichgewicht, Unterbeschäftigungstheorien, Wachstum und Konjunktur und Erwartungen, Spekulationen und Krisen.


\end{content}



\end{module}

