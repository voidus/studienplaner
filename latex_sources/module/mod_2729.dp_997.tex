% Modulbeschreibung 
% Informationsgrad : extern
% Sprache: de
\begin{module}

\setdoclanguagegerman
\moduledegreeprogramme{Informatik (B.Sc.)}
\modulesubject{EF Betriebswirtschaftslehre}
\moduleID{IN3WWBWL15}
\modulename{eFinance}
\modulecoordination{C. Weinhardt}

\documentdate{2011-12-20 16:37:34.455728}

\modulecredits{9}
\moduleduration{2}
\modulecycle{Jedes Semester}



\modulehead

% For index (key word@display). Key word is used for sorting - no Umlauts please.
\index{eFinance@eFinance (M)}

% For later referencing
\label{mod_2729.dp_997}

\begin{courselist}
2540454 & eFinance: Informationswirtschaft für den Wertpapierhandel (S.~\pageref{cour_4851.dp_997}) & 2/1 & W & 4,5 & R. Riordan\\
2511402 & Intelligente Systeme im Finance (S.~\pageref{cour_4637.dp_997}) & 2/1 & S & 5 & D. Seese\\
2530550 & Derivate (S.~\pageref{cour_4501.dp_997}) & 2/1 & S & 4,5 & M. Uhrig-Homburg\\
2530296 & Börsen (S.~\pageref{cour_6289.dp_997}) & 1 & S & 1,5 & J. Franke\\
2530570 & Internationale Finanzierung (S.~\pageref{cour_6447.dp_997}) & 2 & S & 3 & M. Uhrig-Homburg, Walter\\
\end{courselist}

\begin{styleenv}
\begin{assessment}
Die Erfolgskontrolle erfolgt in Form von Teilprüfungen (nach §4(2), 1-3 SPO) über die Kernveranstaltung und weitere Lehrveranstaltungen des Moduls im Umfang von insgesamt 9 LP. Die Erfolgskontrolle wird bei jeder Lehrveranstaltung dieses Moduls beschrieben.

 

Die Gesamtnote des Moduls wird aus den mit LP gewichteten Noten der Teilprüfungen gebildet und nach der ersten Nachkommastelle abgeschnitten.


\end{assessment}

\begin{conditions}Nur prüfbar in Kombination mit dem Modul \emph{Grundlagen der BWL}.

 \end{conditions}


\end{styleenv}

\begin{learningoutcomes}
 

Die Studierenden

 \begin{itemize}\item verstehen und analysieren die Wertschöpfungskette im Wertpapierhandel,  \item bestimmen und gestalten Methoden und Systeme situationsangemessen und wenden diese zur Problemlösung im Bereich Finance an,  \item beurteilen und kritisieren die Investitionsentscheidungen von Händler,  \item wenden theoretische Methoden aus dem Ökonometrie an,  \item lernen die Erarbeitung von Lösungen in Teams.  \end{itemize}
\end{learningoutcomes}

\begin{content}
Das Modul “eFinance: Informationswirtschaft in der Finanzindustrie” adressiert aktuelle Probleme der Finanzwirtschaft und untersucht, welche Rolle dabei Information und Wissen spielen und wie Informationssysteme diese Probleme lösen bzw. mildern können. Dabei werden die Veranstaltungen von erfahrenen Vertretern aus der Praxis ergänzt. Das Modul ist unterteilt in eine Veranstaltung zum Umfeld von Banken und Versicherungen sowie eine weitere zum Bereich des elektronischen Handels von Finanztiteln in globalen Finanzmärkten. Zur Wahl steht auch die Vorlesung Derivate, welche sich mit Produkten auf Finanzmärkten, und insbesondere mit Future- und Forwardkontrakten sowie der Bewertung von Optionen befasst. Als Ergänzung können zudem die Veranstaltungen Börsen und Internationale Finanzierung gewählt werden, um ein besseres Verständnis für Kapitalmärkte zu entwickeln.

 

In der Veranstaltung “eFinance: Informationssysteme für den Wertpapierhandel” stehen Themen der Informationswirtschaft, zum Bereich Wertpapierhandel, im Mittelpunkt. Für das Funktionieren der internationalen Finanzmärkte spielt der effiziente Informationsfluss eine ebenso entscheidende Rolle wie die regulatorischen Rahmenbedingungen. In diesem Kontext werden die Rolle und das Funktionieren von (elektronischen) Börsen, Online-Brokern und anderen Finanzintermediären und ihrer Plattformen näher vorgestellt. Dabei werden nicht nur IT-Konzepte deutscher Finanzintermediäre, sondern auch internationale Systemansätze verglichen. Die Vorlesung wird durch Praxisbeiträge (und ggf. Exkursionen) aus dem Hause der Deutschen und der Stuttgarter Börse ergänzt.


\end{content}

\begin{remarks}Das aktuelle Angebot an Seminaren passend zu diesem Modul ist auf der folgenden Webseite aufgelistet: http://www.iism.kit.edu/im/lehre

\end{remarks}

\end{module}

