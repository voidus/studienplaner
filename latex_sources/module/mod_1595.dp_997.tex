% Modulbeschreibung 
% Informationsgrad : extern
% Sprache: de
\begin{module}

\setdoclanguagegerman
\moduledegreeprogramme{Informatik (B.Sc.)}
\modulesubject{EF Betriebswirtschaftslehre}
\moduleID{IN3WWBWL10}
\modulename{Industrielle Produktion I}
\modulecoordination{F. Schultmann}

\documentdate{2012-01-02 10:55:12.183072}

\modulecredits{9}
\moduleduration{2}
\modulecycle{Jedes Semester}



\modulehead

% For index (key word@display). Key word is used for sorting - no Umlauts please.
\index{Industrielle Produktion I@Industrielle Produktion I (M)}

% For later referencing
\label{mod_1595.dp_997}

\begin{courselist}
2581950 & Grundlagen der Produktionswirtschaft (S.~\pageref{cour_4595.dp_997}) & 2/2 & S & 5,5 & F. Schultmann\\
2581960 & Stoffstromorientierte Produktionswirtschaft (S.~\pageref{cour_6615.dp_997}) & 2/0 & W & 3,5 & F. Schultmann, M. Fröhling\\
2581996 & Logistik und Supply Chain Management (S.~\pageref{cour_8227.dp_997}) & 2/0 & W & 3,5 & F. Schultmann\\
\end{courselist}

\begin{styleenv}
\begin{assessment}
Die Modulprüfung erfolgt in Form von schriftlichen Teilprüfungen (nach §4(2), 1 SPO) über die Vorlesung \emph{Grundlagen der Produktionswirtschaft }[2581950] und eine Ergänzungsveranstaltung\emph{. }Die Prüfungen werden in jedem Semester angeboten und können zu jedem ordentlichen Prüfungstermin wiederholt werden.

 

Die Gesamtnote des Moduls wird aus den mit LP gewichteten Noten der Teilprüfungen gebildet und nach der ersten Nachkommastelle abgeschnitten. Zusätzliche Studienleistungen können auf Antrag eingerechnet werden.

 

Die Erfolgskontrolle wird bei jeder Lehrveranstaltung dieses Moduls beschrieben.

 
\end{assessment}

\begin{conditions}Nur prüfbar in Kombination mit dem Modul \emph{Grundlagen der BWL}.

 

Das Modul ist nur zusammen mit dem Pflichtmodul \emph{Grundlagen der BWL} [IN3WWBWL] prüfbar.

\end{conditions}


\end{styleenv}

\begin{learningoutcomes}
\begin{itemize}\item Die Studierenden beschreiben das Gebiet der industriellen Produktion und Logistik und erkennen deren Bedeutung für Industriebetriebe und die darin tätigen Wirtschaftsingenieure/Informationswirtschaftler und Volkswirtschaftler.  \item Die Studierenden verwenden wesentliche Begriffe aus der Produktionswirtschaft und Logistik korrekt.  \item Die Studierenden geben produktionswirtschaftlich relevante Entscheidungen im Unternehmen und dafür wesentliche Rahmenbedingungen wieder.  \item Die Studierenden kennen die wesentlichen Planungsaufgaben, -probleme und Lösungsstrategien des strategischen Produktionsmanagements sowie der Logistik.  \item Die Studierenden kennen wesentliche Ansätze zur Modellierung von Produktions- und Logistiksystemen.  \item Die Studierenden kennen die Bedeutung von Stoff- und Energieflüssen in der Produktion.  \item Die Studierenden wenden exemplarische Methoden zur Lösung ausgewählter Problemstellungen an.  \end{itemize}
\end{learningoutcomes}

\begin{content}
Das Modul gibt eine Einführung in das Gebiet der Industriellen Produktion und Logistik. Im Mittelpunkt stehen Fragestellungen des strategischen Produktionsmanagements, die auch unter nachhaltig zeitrelevanten Aspekten betrachtet werden. Die Aufgaben der industriellen Produktionswirtschaft und Logistik werden mittels interdisziplinärer Ansätze der Systemtheorie beschrieben. Die behandelten Fragestellungen umfassen strategische Unternehmensplanung, die Forschung und Entwicklung (F\&E) sowie die betriebliche Standortplanung. Unter produktionswirtschaftlicher Sichtweise werden zudem inner- und außerbetrieblichen Transport- und Lagerprobleme betrachtet. Dabei werden auch Fragen der Entsorgungslogistik und des Supply Chain Managements behandelt.


\end{content}



\end{module}

