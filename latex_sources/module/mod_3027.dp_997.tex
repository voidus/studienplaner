% Modulbeschreibung 
% Informationsgrad : extern
% Sprache: de
\begin{module}

\setdoclanguagegerman
\moduledegreeprogramme{Informatik (B.Sc.)}
\modulesubject{}
\moduleID{IN1MATHLAAG}
\modulename{Lineare Algebra und Analytische Geometrie}
\modulecoordination{K. Spitzmüller, S. Kühnlein}

\documentdate{2009-02-16 10:40:43}

\modulecredits{18}
\moduleduration{2}
\modulecycle{Jedes 2. Semester, Wintersemester}



\modulehead

% For index (key word@display). Key word is used for sorting - no Umlauts please.
\index{Lineare Algebra und Analytische Geometrie@Lineare Algebra und Analytische Geometrie (M)}

% For later referencing
\label{mod_3027.dp_997}

\begin{courselist}
01007 & Lineare Algebra und Analytische Geometrie 1 (S.~\pageref{cour_7565.dp_997}) & 4/2/2 & W & 9 & F. Herrlich, E. Leuzinger, C. Schmidt, W. Tuschmann\\
01505 & Lineare Algebra und Analytische Geometrie 2 (S.~\pageref{cour_7567.dp_997}) & 4/2/2 & S & 9 & F. Herrlich, E. Leuzinger, C. Schmidt, W. Tuschmann\\
\end{courselist}

\begin{styleenv}
\begin{assessment}
Die Erfolgskontrolle erfolgt in Form einer schriftlichen Gesamtprüfung im Umfang von i.d.R. 210 Minuten nach § 4 Abs. 2 Nr. 1 SPO sowie einem bestandenen Leistungsnachweis aus den Übungsbetrieben zu \emph{Lineare Algebra und Analytische Geometrie I} [1007] oder \emph{Lineare Algebra und Analytische Geometrie II }[1505].

 

Die Modulnote ist die Note der schriftlichen Prüfung.

 

\textbf{Achtung:} Diese Prüfung oder die Prüfung zum Modul \emph{Höhere Mathematik} [IN1MATHHM] oder zum Modul \emph{Analysis} [IN1MATHANA] oder zum Modul \emph{Lineare Algebra }[IN1MATHLA] ist bis zum Ende des 2. Fachsemesters anzutreten und bis zum Ende des 3. Fachsemesters zu bestehen, da sie Bestandteil der Orientierungsprüfung nach § 8 Abs. 1 SPO ist.


\end{assessment}

\begin{conditions}Keine.\end{conditions}


\end{styleenv}

\begin{learningoutcomes}
Die Studierenden sollen am Ende des Moduls

 \begin{itemize}\item den Übergang von der Schule zur Universität bewältigt haben,  \item mit logischem Denken und strengen Beweisen vertraut sein,   \item die Methoden und grundlegenden Strukturen der Linearen Algebra und Analytischen Geometrie beherrschen.  \end{itemize}
\end{learningoutcomes}

\begin{content}
\begin{itemize}\item Grundbegriffe (Mengen, Abbildungen, Relationen, Gruppen, Ringe, Körper, Matrizen, Polynome)   \item Lineare Gleichungssysteme (Gauß´sches Eliminationsverfahren, Lösungstheorie)   \item Vektorräume (Beispiele, Unterräume, Quotientenräume, Basis und Dimension)   \item Lineare Abbildungen (Kern, Bild, Rang, Homomorphiesatz, Vektorräume von Abbildungen, Dualraum, Darstellungsmatrizen, Basiswechsel, Endomorphismenalgebra, Automorphismengruppe)   \item Determinanten   \item Eigenwerttheorie (Eigenwerte, Eigenvektoren, charakteristisches Polynom, Normalformen)  \item Vektorräume mit Skalarprodukt (bilineare Abbildungen, Skalarprodukt, Norm, Orthogonalität, adjungierte Abbildung, normale und selbstadjungierte Endomorphismen, Spektralsatz, Isometrien und Normalformen)  \item Affine Geometrie (Affine Räume, Unterräume, Affine Abbildungen, affine Gruppe, Fixelemente)  \item Euklidische Räume (Unterräume, Bewegungen, Klassifikation, Ähnlichkeitsabbildungen)  \item Quadriken (Affine Klassifikation, Euklidische Klassifikation)  \end{itemize}
\end{content}

\begin{remarks}Moduldauer: 2 Semester

 

Dieses Modul kann anstatt dem Pflichtmodul\emph{ Lineare Algebra} [IN1MATHLA] gewählt werden (z.B. wenn Mathematik als Parallelstudium absolviert wird).

\end{remarks}

\end{module}

