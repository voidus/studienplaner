% Lehrveranstaltungsbeschreibung
% Informationsgrad : extern
% Sprache: de
\begin{course}

\setdoclanguagegerman
\coursedegreeprogramme{Informatik}
\coursemodulename{Topics in Finance I (S.~\pageref{mod_1575.dp_997})[IN3WWBWL13], eFinance (S.~\pageref{mod_2729.dp_997})[IN3WWBWL15]}
\courseID{2530550}
\coursename{Derivate}
\coursecoordination{M. Uhrig-Homburg}

\documentdate{2011-12-23 18:38:02.655411}

\courselevel{4}
\coursecredits{4,5}
\courseterm{Sommersemester}
\coursehours{2/1}
\courseinstructionlanguage{de}

\coursehead

% For index (key word@display). Key word is used for sorting - no Umlauts please.
\index{Derivate@Derivate}

% For later referencing
\label{cour_4501.dp_997}


\begin{styleenv}
\begin{assessment}
Die Erfolgskontrolle wird in der Modulbeschreibung erläutert.


\end{assessment}

\begin{conditions}Keine.\end{conditions}


\end{styleenv}

\begin{learningoutcomes}
Ziel der Vorlesung Derivate ist es, mit den Finanz- und Derivatemärkten vertraut zu werden. Dabei werden gehandelte Instrumente und häufig verwendete Handelsstrategien vorgestellt, die Bewertung von Derivaten abgeleitet und deren Einsatz im Risikomanagement besprochen.


\end{learningoutcomes}

\begin{content}
Die Vorlesung Derivate beschäftigt sich mit den Einsatzmöglichkeiten und Bewertungsproblemen von derivativen Finanzinstrumenten. Nach einer Übersicht über die wichtigsten Derivate und deren Bedeutung werden zunächst Forwards und Futures analysiert. Daran schließt sich eine Einführung in die Optionspreistheorie an. Der Schwerpunkt liegt auf der Bewertung von Optionen in zeitdiskreten und zeitstetigen Modellen. Schließlich werden Konstruktions- und Einsatzmöglichkeiten von Derivaten etwa im Rahmen des Risikomanagement diskutiert.


\end{content}

\begin{media}Folien, Übungsblätter.

\end{media}

\begin{literature}\begin{itemize}\item Hull (2005): Options, Futures, \& Other Derivatives, Prentice Hall, 6th Edition  \end{itemize}

\textbf{Weiterführende Literatur:}

 

Cox/Rubinstein (1985): Option Markets, Prentice Hall

\end{literature}



\end{course}