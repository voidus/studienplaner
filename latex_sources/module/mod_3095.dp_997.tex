% Modulbeschreibung 
% Informationsgrad : extern
% Sprache: de
\begin{module}

\setdoclanguagegerman
\moduledegreeprogramme{Informatik (B.Sc.)}
\modulesubject{EF Mathematik}
\moduleID{IN3MATHAG02}
\modulename{Einführung in Algebra und Zahlentheorie}
\modulecoordination{S. Kühnlein}

\documentdate{2011-10-06 17:44:53.303497}

\modulecredits{9}
\moduleduration{1}
\modulecycle{Jedes 2. Semester, Sommersemester}



\modulehead

% For index (key word@display). Key word is used for sorting - no Umlauts please.
\index{Einfuehrung in Algebra und Zahlentheorie@Einführung in Algebra und Zahlentheorie (M)}

% For later referencing
\label{mod_3095.dp_997}

\begin{courselist}
1524 & Einführung in Algebra und Zahlentheorie (S.~\pageref{cour_7865.dp_997}) & 6 & S & 9 & F. Herrlich, S. Kühnlein, C. Schmidt\\
\end{courselist}

\begin{styleenv}
\begin{assessment}
Prüfung: schriftliche oder mündliche Prüfung\newline
Notenbildung: Note der Prüfung


\end{assessment}

\begin{conditions}Das Modul \emph{Proseminar Mathematik} [IN3MATHPS] muss geprüft werden.

 \end{conditions}

\begin{recommendations}Folgende Module sollten bereits belegt worden sein (Empfehlung):\newline
Lineare Algebra 1+2\newline
Analysis 1+2

\end{recommendations}
\end{styleenv}

\begin{learningoutcomes}
\begin{itemize}\item Beherrschung der grundlegenden algebraischen und zahlentheoretischen Strukturen  \item Einführung in die Denkweise der modernen Algebra  \item Grundlage für Seminare und weiterführende Vorlesungen im Bereich Algebra  \end{itemize}
\end{learningoutcomes}

\begin{content}
\begin{itemize}\item Gruppentheorie  \item Ringtheorie  \item Primzahlen  \item Modulares Rechnen  \end{itemize}
\end{content}



\end{module}

