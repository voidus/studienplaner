% Modulbeschreibung 
% Informationsgrad : extern
% Sprache: de
\begin{module}

\setdoclanguagegerman
\moduledegreeprogramme{Informatik (B.Sc.)}
\modulesubject{EF Mathematik}
\moduleID{IN3MATHPS}
\modulename{Proseminar Mathematik}
\modulecoordination{S. Kühnlein, Stefan Kühnlein}

\documentdate{2011-02-17 13:23:19.492905}

\modulecredits{3}
\moduleduration{1}
\modulecycle{Jedes Semester}



\modulehead

% For index (key word@display). Key word is used for sorting - no Umlauts please.
\index{Proseminar Mathematik@Proseminar Mathematik (M)}

% For later referencing
\label{mod_4165.dp_997}

\begin{courselist}
ProsemMath & Proseminar Mathematik (S.~\pageref{cour_8397.dp_997}) & 2 & W/S & 3 & Dozenten der Fakultät für Mathematik\\
SemMath & Seminar Mathematik (S.~\pageref{cour_8399.dp_997}) & 2 & W/S & 3 & Dozenten der Fakultät für Mathematik\\
\end{courselist}

\begin{styleenv}
\begin{assessment}
Die Erfolgskontrolle erfolgt als Erfolgskontrolle anderer Art nach § 4 Abs. 2 Nr. 3 SPO.

 

Die Modulnote entspricht der Bewertung dieser Erfolgskontrolle.


\end{assessment}

\begin{conditions}Keine.\end{conditions}


\end{styleenv}

\begin{learningoutcomes}
\begin{itemize}\item Die Studierenden erhalten eine erste Einführung in das wissenschaftliche Arbeiten auf einem speziellen Fachgebiet.  \item Die Bearbeitung der Proseminar-/Seminararbeit bereitet zudem auf die Abfassung der Bachelorarbeit vor.  \item Mit dem Besuch der Proseminar-/Seminarveranstaltungen werden neben Techniken des wissenschaftlichen Arbeitens auch Schlüsselqualifikationen integrativ vermittelt.  \end{itemize}
\end{learningoutcomes}

\begin{content}
Das Modul behandelt in den angebotenen Proseminaren/Seminaren spezifische Themen, die teilweise in entsprechenden Vorlesungen angesprochen wurden und vertieft diese. In der Regel ist die Voraussetzung für das Bestehen des Moduls die Anfertigung einer schriftlichen Ausarbeitung von max. 15 Seiten sowie eine mündliche Präsentation von 20 - 45 Minuten. Dabei ist auf ein ausgewogenes Verhältnis zu achten.


\end{content}



\end{module}

