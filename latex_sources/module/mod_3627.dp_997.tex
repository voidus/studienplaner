% Modulbeschreibung 
% Informationsgrad : extern
% Sprache: de
\begin{module}

\setdoclanguagegerman
\moduledegreeprogramme{Informatik (B.Sc.)}
\modulesubject{EF Mathematik}
\moduleID{IN3MATHST01}
\modulename{Einführung in die Stochastik}
\modulecoordination{N. Henze}

\documentdate{2011-10-06 17:57:30.218176}

\modulecredits{6}
\moduleduration{1}
\modulecycle{Jedes 2. Semester, Wintersemester}



\modulehead

% For index (key word@display). Key word is used for sorting - no Umlauts please.
\index{Einfuehrung in die Stochastik@Einführung in die Stochastik (M)}

% For later referencing
\label{mod_3627.dp_997}

\begin{courselist}
1071 & Einführung in die Stochastik (S.~\pageref{cour_8037.dp_997}) & 3/1/2 & W & 6 & N. Bäuerle, N. Henze, B. Klar, G. Last\\
\end{courselist}

\begin{styleenv}
\begin{assessment}
Prüfung: schriftliche oder mündliche Prüfung\newline
Notenbildung: Note der Prüfung


\end{assessment}

\begin{conditions}Die Module \emph{Wahrscheinlichkeitstheorie} [IN3MATHST02] und \emph{Markovsche Ketten} [IN3MATHST03] müssen geprüft werden.

 

Das Modul \emph{Proseminar Mathematik} [IN3MATHPS] muss geprüft werden.

\end{conditions}

\begin{recommendations}Folgende Module sollten bereits belegt worden sein (Empfehlung):\newline
Lineare Algebra 1+2\newline
Analysis 1+2

\end{recommendations}
\end{styleenv}

\begin{learningoutcomes}
In der Stochastik werden Vorgänge und Strukturen, die vom Zufall abhängen, mathematisch beschrieben. Die Studierenden sollen

 \begin{itemize}\item den Begriff der Wahrscheinlichkeit und die mathematische Umsetzung kennen und verstehen lernen,  \item die Modellierung in einfachen, diskreten und stetigen stochastischen Modellen verstehen und anwenden können,  \item Techniken erlernen, die zur Analyse stochastischer Modelle grundlegend sind.  \end{itemize}
\end{learningoutcomes}

\begin{content}
\begin{itemize}\item Deskriptive Statistik  \item Ereignisse  \item Zufallsvariablen  \item Diskrete Wahrscheinlichkeitsräume  \item Laplace-Modell  \item Elementare Kombinatorik  \item Verteilung und Erwartungswert einer Zufallsvariablen  \item Wichtige diskrete Verteilungen  \item Mehrstufige Experimente  \item Bedingte Wahrscheinlichkeiten  \item Stochastische Unabhängigkeit  \item Gemeinsame Verteilung und Unabhängigkeit von Zufallsvariablen  \item Varianz, Kovarianz und Korrelation  \item Gesetz großer Zahlen  \item Zentraler Grenzwertsatz  \item Schätzprobleme und statistische Tests am Beispiel der Binomialverteilung  \item Allgemeine Wahrscheinlichkeitsräume  \item Stetige Verteilungen  \end{itemize}
\end{content}



\end{module}

