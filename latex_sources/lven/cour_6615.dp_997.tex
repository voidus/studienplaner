% Lehrveranstaltungsbeschreibung
% Informationsgrad : extern
% Sprache: de
\begin{course}

\setdoclanguagegerman
\coursedegreeprogramme{Informatik}
\coursemodulename{Industrielle Produktion I (S.~\pageref{mod_1595.dp_997})[IN3WWBWL10]}
\courseID{2581960}
\coursename{Stoffstromorientierte Produktionswirtschaft}
\coursecoordination{F. Schultmann, M. Fröhling}

\documentdate{2012-01-02 11:09:13.145007}

\courselevel{3}
\coursecredits{3,5}
\courseterm{Wintersemester}
\coursehours{2/0}
\courseinstructionlanguage{de}

\coursehead

% For index (key word@display). Key word is used for sorting - no Umlauts please.
\index{Stoffstromorientierte Produktionswirtschaft@Stoffstromorientierte Produktionswirtschaft}

% For later referencing
\label{cour_6615.dp_997}


\begin{styleenv}
\begin{assessment}
Die Erfolgskontrolle wird in der Modulbeschreibung erläutert.


\end{assessment}

\begin{conditions}Keine.\end{conditions}


\end{styleenv}

\begin{learningoutcomes}
\begin{itemize}\item Der Studierende benennt Problemstellungen aus dem Bereich der Stoff- und Energieflüsse in der Ökonomie.  \item Der Studierende kennt Lösungsansätze für die benannten Probleme und wendet diese an.  \end{itemize}
\end{learningoutcomes}

\begin{content}
Kern der Veranstaltung sind die Analyse von Stoffströmen und das betriebliche und überbetriebliche Stoffstrommanagement. Dabei liegt der Schwerpunkt auf der kosten- und ökologisch effizienten Ausgestaltung von Maßnahmen zur Vermeidung, Verminderung und Verwertung von Emissionen, Reststoffen und Altprodukten und der Erhöhung der Ressourceneffizienz. Als Methoden werden u.a. die Stoffstromanalyse (MFA), Ökobilanzierung (LCA) sowie OR-Methoden, z. B. zur Entscheidungsunterstützung, vorgestellt.

 

Themen:\newline
- Stoffrecht\newline
- Rohstoffe, Reserven und deren Verfügbarkeit\newline
- Stoffstromanalysen (MFA/SFA)\newline
- Stoffstromorientierte Kennzahlen/Ökoprofile, u.a. Carbon Footprint\newline
- Ökobilanzierung (LCA)\newline
- Ressourceneffizienz\newline
- Emissionsminderung\newline
- Abfall- und Kreislaufwirtschaft\newline
- Rohstoffnahe Produktionssysteme\newline
\newline
- Umweltmanagement (EMAS, ISO 14001, Ökoprofit) und Ökocontrolling


\end{content}

\begin{media}Medien zur Vorlesung werden über die Lernplattform bereit gestellt.

\end{media}

\begin{literature}wird in der Veranstaltung bekannt gegeben

\end{literature}



\end{course}