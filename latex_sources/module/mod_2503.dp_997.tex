% Modulbeschreibung 
% Informationsgrad : extern
% Sprache: de
\begin{module}

\setdoclanguagegerman
\moduledegreeprogramme{Informatik (B.Sc.)}
\modulesubject{}
\moduleID{IN3INTM}
\modulename{Telematik}
\modulecoordination{M. Zitterbart}

\documentdate{2010-09-13 12:44:58.039231}

\modulecredits{6}
\moduleduration{1}
\modulecycle{Jedes 2. Semester, Wintersemester}



\modulehead

% For index (key word@display). Key word is used for sorting - no Umlauts please.
\index{Telematik@Telematik (M)}

% For later referencing
\label{mod_2503.dp_997}

\begin{courselist}
24128 & Telematik (S.~\pageref{cour_6141.dp_997}) & 2 & W & 4 & M. Zitterbart\\
24443 & Praxis der Telematik (S.~\pageref{cour_8163.dp_997}) & 1 & W & 2 & M. Zitterbart\\
\end{courselist}

\begin{styleenv}
\begin{assessment}
Die Erfolgskontrolle erfolgt

 \begin{itemize}\item zur Lehrveranstaltung \emph{Telematik} [24128] in Form einer mündlichen Prüfung im Umfang von i.d.R. 20 Minuten nach § 4 Abs. 2 Nr. 2 SPO. Nach § 6 Abs. 3 SPO wird bei unvertretbar hohem Prüfungsaufwand eine schriftliche Prüfung im Umfang von ca. 60 Minuten anstatt einer mündlichen Prüfung angeboten. Dies wird mindestens 6 Wochen vor der Prüfung bekannt gegeben.  \item zur Lehrveranstaltung \emph{Praxis der Telematik} [24443] als Erfolgskontrolle anderer Art nach § 4 Abs. 2 Nr. 3 in Form eines unbenoteteten Leistungsnachweises entweder für die Übung (Scheinklausur) oder die erfolgreiche Teilnahme an dem semesterbegleitenden Projekt.  \end{itemize}

Die Modulnote entspricht der Prüfung zur Lehrveranstaltung \emph{Telematik} [24128].


\end{assessment}

\begin{conditions}Es gelten die Voraussetzungen der Lehrveranstaltung \emph{Telematik} [24128].

\end{conditions}


\end{styleenv}

\begin{learningoutcomes}
In dieser Veranstaltung sollen die Teilnehmer ausgewählte Protokolle, Architekturen, sowie Verfahren und Algorithmen, welche bereits in der Vorlesung \emph{Einführung in Rechnernetze} erlernt wurden, im Detail kennenlernen. Den Teilnehmern soll dabei ein Systemverständnis sowie das Verständnis der in einem weltumspannenden, dynamischen Netz auftretenden Probleme und der zur Abhilfe eingesetzten Protokollmechanismen vermittelt werden.


\end{learningoutcomes}

\begin{content}
Die Vorlesung behandelt Protokolle, Architekturen, sowie Verfahren und Algorithmen, die u.a. im Internet für die Wegewahl und für das Zustandekommen einer zuverlässigen Ende-zu-Ende-Verbindung zum Einsatz kommen. Neben verschiedenen Medienzuteilungsverfahren in lokalen Netzen werden auch weitere Kommunikationssysteme, wie z.B. das leitungsvermittelte ISDN behandelt. Die Teilnehmer sollten ebenfalls verstanden haben, welche Möglichkeiten zur Verwaltung und Administration von Netzen zur Verfügung stehen.


\end{content}

\begin{remarks}Das Modul \emph{Telematik} ist ein Stammmodul.

\end{remarks}

\end{module}

