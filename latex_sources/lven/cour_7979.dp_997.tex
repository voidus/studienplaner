% Lehrveranstaltungsbeschreibung
% Informationsgrad : extern
% Sprache: de
\begin{course}

\setdoclanguagegerman
\coursedegreeprogramme{Informatik}
\coursemodulename{Riemannsche Geometrie (S.~\pageref{mod_3109.dp_997})[IN3MATHAG04]}
\courseID{1036}
\coursename{Riemannsche Geometrie}
\coursecoordination{E. Leuzinger}

\documentdate{2011-10-06 18:13:20.021764}

\courselevel{}
\coursecredits{9}
\courseterm{Wintersemester}
\coursehours{4/2}
\courseinstructionlanguage{}

\coursehead

% For index (key word@display). Key word is used for sorting - no Umlauts please.
\index{Riemannsche Geometrie@Riemannsche Geometrie}

% For later referencing
\label{cour_7979.dp_997}


\begin{styleenv}
\begin{assessment}
Die Erfolgskontrolle wird in der Modulbeschreibung erläutert.


\end{assessment}

\begin{conditions}Keine.\end{conditions}

\begin{recommendations}Folgende Module sollten bereits belegt worden sein (Empfehlung):\newline
Lineare Algebra 1+2\newline
Analysis 1+2\newline
Einführung in Geometrie und Topologie

\end{recommendations}
\end{styleenv}

\begin{learningoutcomes}
Einführung in die Konzepte der Riemannschen Geometrie


\end{learningoutcomes}

\begin{content}
\begin{itemize}\item Mannigfaltigkeiten  \item Riemannsche Metriken  \item Affine Zusammenhänge  \item Geodätische  \item Krümmung  \item Jacobi-Felder  \item Längen-Metrik  \item Krümmung und Topologie  \end{itemize}
\end{content}







\end{course}