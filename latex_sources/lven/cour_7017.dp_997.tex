% Lehrveranstaltungsbeschreibung
% Informationsgrad : extern
% Sprache: de
\begin{course}

\setdoclanguagegerman
\coursedegreeprogramme{Informatik}
\coursemodulename{Algorithmen I (S.~\pageref{mod_2413.dp_997})[IN1INALG1]}
\courseID{24500}
\coursename{Algorithmen I}
\coursecoordination{M. Zitterbart}

\documentdate{2011-05-05 14:46:38.714393}

\courselevel{1}
\coursecredits{6}
\courseterm{Sommersemester}
\coursehours{3/1/2}
\courseinstructionlanguage{de}

\coursehead

% For index (key word@display). Key word is used for sorting - no Umlauts please.
\index{Algorithmen I@Algorithmen I}

% For later referencing
\label{cour_7017.dp_997}


\begin{styleenv}
\begin{assessment}
Die Erfolgskontrolle wird in der Modulbeschreibung erläutert.


\end{assessment}

\begin{conditions}Keine.\end{conditions}


\end{styleenv}

\begin{learningoutcomes}
Der/die Studierende

 \begin{itemize}\item kennt und versteht grundlegende, häufig benötigte Algorithmen, ihren Entwurf, Korrektheits- und Effizienzanalyse,  \item Implementierung, Dokumentierung und Anwendung,  \item kann mit diesem Verständnis auch neue algorithmische Fragestellungen bearbeiten,  \item wendet die im Modul Grundlagen der Informatik (Bachelor Informationswirtschaft) erworbenen Programmierkenntnisse  \item auf nichttriviale Algorithmen an,  \item wendet die in Grundbegriffe der Informatik (Bachelor Informatik) bzw. Grundlagen der Informatik (Bachelor Informationswirtschaft) und den Mathematikvorlesungen erworbenen mathematischen Herangehensweise an die Lösung von Problemen an. Schwerpunkte sind hier formale Korrektheitsargumente und eine mathematische Effizienzanalyse.  \end{itemize}
\end{learningoutcomes}

\begin{content}
Dieses Modul soll Studierenden grundlegende Algorithmen und Datenstrukturen vermitteln.

 

Die Vorlesung behandelt unter anderem:

 \begin{itemize}\item Grundbegriffe des Algorithm Engineering  \item Asymptotische Algorithmenanalyse (worst case, average case, probabilistisch, amortisiert)  \item Datenstrukturen z.B. Arrays, Stapel, Warteschlangen und Verkettete Listen  \item Hashtabellen  \item Sortieren: vergleichsbasierte Algorithmen (z.B. quicksort, insertionsort), untere Schranken, Linearzeitalgorithmen (z.B. radixsort)  \item Prioritätslisten  \item Sortierte Folgen,Suchbäume und Selektion  \item Graphen (Repräsentation, Breiten-/Tiefensuche, Kürzeste Wege,Minimale Spannbäume)  \item Generische Optimierungsalgorithmen (Greedy, Dynamische Programmierung, systematische Suche, Lokale Suche)  \item Geometrische Algorithmen  \end{itemize}
\end{content}

\begin{media}Vorlesungsfolien, Tafelanschrieb

\end{media}

\begin{literature}Algorithmen - Eine Einführung\newline
T. H. Cormen, C. E. Leiserson, R. L. Rivest, und C. Stein\newline
Oldenbourg, 2007

 

\textbf{Weiterführende Literatur:}

 

Algorithms and Data Structures -- The Basic Toolbox\newline
K. Mehlhorn und P. Sanders\newline
Springer 2008

 

Algorithmen und Datenstrukturen\newline
T. Ottmann und P. Widmayer\newline
Spektrum Akademischer Verlag, 2002\newline
\newline
Algorithmen in Java. Teil 1-4: Grundlagen, Datenstrukturen, Sortieren, Suchen \newline
R. Sedgewick\newline
Pearson Studium 2003

 

Algorithm Design\newline
J. Kleinberg and É. Tardos\newline
Addison Wesley, 2005

 

Vöcking et al.\newline
Taschenbuch der Algorithmen\newline
Springer, 2008

\end{literature}



\end{course}