% Modulbeschreibung 
% Informationsgrad : extern
% Sprache: de
\begin{module}

\setdoclanguagegerman
\moduledegreeprogramme{Informatik (B.Sc.)}
\modulesubject{}
\moduleID{IN3INKFC}
\modulename{Kurven und Flächen im CAD}
\modulecoordination{H. Prautzsch}

\documentdate{2012-01-30 12:06:07.188078}

\modulecredits{9}
\moduleduration{1}
\modulecycle{Jedes 2. Semester, Sommersemester}



\modulehead

% For index (key word@display). Key word is used for sorting - no Umlauts please.
\index{Kurven und Flaechen im CAD@Kurven und Flächen im CAD (M)}

% For later referencing
\label{mod_15887.dp_997}

\begin{courselist}
24626 & Kurven und Flächen im CAD (S.~\pageref{cour_15889.dp_997}) & 4/2 & W/S & 9 & H. Prautzsch\\
\end{courselist}

\begin{styleenv}
\begin{assessment}
Die Erfolgskontrolle erfolgt in Form einer mündlichen Prüfung im Umfang von i.d.R. 30 Minuten und durch einen benoteten Übungsschein nach § 4 Abs. 2 Nr. 2 und 3 SPO.

 

Modulnote = 0.8 x Note der mündlichen Prüfung + 0.2 x Note des Übungsscheins, wobei nur die erste Nachkommastelle ohne Rundung berücksichtigt wird.


\end{assessment}

\begin{conditions}Keine.\end{conditions}


\end{styleenv}

\begin{learningoutcomes}
Ziel der Vorlesung ist es, zu vermitteln, wie glatte Freiformkurven und Freiformflächen in CAD-Systemen und in der Computergraphik dargestellt und eingesetzt werden. Die Hörer und Hörerinnen der Vorlesung sollen insbesondere die Darstellung mit Kontrollpunkten und die geometrischen Eigenschaften der Bézier- und B-Spline-Darstellung kennenlernen.


\end{learningoutcomes}

\begin{content}
Bézier\_ und B-Spline-Techniken, Polarformen, Algorithmen von de Casteljau, de Boor und Boehm, Oslo-Algorithmus, Stärks Anschlusskonstruktion, Unterteilung, Übergang zu anderen Darstellungen, Algorithmen zur Erzeugung und Schneiden von Kurven und Flächen, Interpolationssplines, Tensorprodukt- und Dreiecksflächen, konvexe Flächen, Konstruktionen von Powell-Sabin, Clough-Tocher und Piper, Konstruktion glatter Freiformlächen, Punktumschließungsproblem, Boxsplines.


\end{content}



\end{module}

