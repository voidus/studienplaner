% Lehrveranstaltungsbeschreibung
% Informationsgrad : extern
% Sprache: de
\begin{course}

\setdoclanguagegerman
\coursedegreeprogramme{Informatik}
\coursemodulename{Proseminar (S.~\pageref{mod_2385.dp_997})[IN2INPROSEM]}
\courseID{24050}
\coursename{Proseminar Algorithmentechnik}
\coursecoordination{P. Sanders, Veit Batz, Timo Bingmann, Christian Schulz}

\documentdate{2011-11-24 11:35:03.107262}

\courselevel{2}
\coursecredits{3}
\courseterm{Wintersemester}
\coursehours{2}
\courseinstructionlanguage{de}

\coursehead

% For index (key word@display). Key word is used for sorting - no Umlauts please.
\index{Proseminar Algorithmentechnik@Proseminar Algorithmentechnik}

% For later referencing
\label{cour_14479.dp_997}


\begin{styleenv}
\begin{assessment}
Die Erfolgskontrolle erfolgt durch Ausarbeiten einer schriftlichen Proseminararbeit sowie der Präsentation derselbigen als Erfolgskontrolle anderer Art nach § 4 Abs. 2 Nr. 3 der SPO. Die Gesamtnote setzt sich aus den benoteten und gewichteten Erfolgskontrollen (i.d.R. 80 \% Proseminararbeit, 20 \% Präsentation) zusammen.


\end{assessment}

\begin{conditions}Keine.\end{conditions}

\begin{recommendations}Gutes Verständnis des Stoffes aus Algorithmen I wird vorrausgesetzt.

\end{recommendations}
\end{styleenv}

\begin{learningoutcomes}
Ziel der Lehrveranstaltung ist es, die Studierenden mit aktuellen Konzepten der Algorithmik und/oder des Algorithm Engineering vertraut zu machen und damit die in der Vorlesung „Algorithmen I“ erworbenen Kenntnisse zu vertiefen. Die Studierenden sollen für die Problemstellungen in diesen Gebieten sensibilisiert werden und neue Lösungsansätze kennenlernen.

 

Die Studierenden erschließen sich im Rahmen des Seminars ein komplexes Thema in selbständiger Arbeit. Dazu gehört die Erarbeitung und Präsentation eines anschaulichen Vortrags sowie eine Zusammenfassung der erworbenen Kenntnisse im Rahmen einer Ausarbeitung.


\end{learningoutcomes}

\begin{content}
Das Proseminar vertieft im Anschluss an die Vorlesung „Algorithmen I“ das in der Veranstaltung erworbene Wissen um weiterführende Konzepte und Lösungen anhand wissenschaftlicher Publikationen und/oder Lehrbüchern aus den jeweiligen Bereichen.

 

Der Fokus auf einer Vertiefung der algorithmischen Kenntnisse der Studierenden.


\end{content}

\begin{media}Folien, Tafel

\end{media}





\end{course}