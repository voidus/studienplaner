% Modulbeschreibung 
% Informationsgrad : extern
% Sprache: de
\begin{module}

\setdoclanguagegerman
\moduledegreeprogramme{Informatik (B.Sc.)}
\modulesubject{EF Operations Research}
\moduleID{IN3WWOR4}
\modulename{Stochastische Methoden und Simulation}
\modulecoordination{K. Waldmann}

\documentdate{2011-02-01 10:58:41.574111}

\modulecredits{9}
\moduleduration{1}
\modulecycle{Jedes Semester}



\modulehead

% For index (key word@display). Key word is used for sorting - no Umlauts please.
\index{Stochastische Methoden und Simulation@Stochastische Methoden und Simulation (M)}

% For later referencing
\label{mod_3855.dp_997}

\begin{courselist}
2550679 & Stochastische Entscheidungsmodelle I (S.~\pageref{cour_5703.dp_997}) & 2/1/2 & W & 5 & K. Waldmann\\
2550682 & Stochastische Entscheidungsmodelle II (S.~\pageref{cour_7911.dp_997}) & 2/1/2 & S & 4,5 & K. Waldmann\\
2550662 & Simulation I (S.~\pageref{cour_4641.dp_997}) & 2/1/2 & W & 4,5 & K. Waldmann\\
2550665 & Simulation II (S.~\pageref{cour_7579.dp_997}) & 2/1/2 & S & 4,5 & K. Waldmann\\
2550111 & Nichtlineare Optimierung I (S.~\pageref{cour_7885.dp_997}) & 2/1 & S & 4,5 & O. Stein\\
2550488 & Taktisches und operatives Supply Chain Management (S.~\pageref{cour_7815.dp_997}) & 2/1 & W & 4,5 & S. Nickel\\
\end{courselist}

\begin{styleenv}
\begin{assessment}
Die Modulprüfung erfolgt in Form von schriftlichen Teilprüfungen(nach § 4(2), 1 SPO) über die gewählten Lehrveranstaltungen des Moduls, mit denen in Summe die Mindestanforderungen an Leistungspunkten erfüllt ist. Die Erfolgskontrolle wird bei jeder Lehrveranstaltung beschrieben.

 

Die Gesamtnote des Moduls wird aus den mit Leistungspunkten gewichteten Noten der Teilprüfungen gebildet und nach der ersten Nachkommastelle abgeschnitten.


\end{assessment}

\begin{conditions}Nur prüfbar in Kombination mit dem Modul \emph{Grundlagen des OR}.

 \end{conditions}


\end{styleenv}

\begin{learningoutcomes}
Der/die Studierende

 \begin{itemize}\item kennt und versteht stochastische Zusammenhänge,   \item hat vertiefte Kenntnisse in der Modellierung, Analyse und Optimierung stochastischer Systeme in Ökonomie und Technik.  \end{itemize}
\end{learningoutcomes}

\begin{content}
Überblick über den Inhalt:

 

Stochastische Entscheidungsmodelle I: Markov Ketten, Poisson Prozesse.

 

Simulation I: Erzeugung von Zufallszahlen, Monte Carlo Integration, Diskrete Simulation, Zufallszahlen diskreter und stetiger Zufallsvariablen, statistische Analyse simulierter Daten.

 

Simulation II: Varianzreduzierende Verfahren, Simulation stochastischer Prozesse, Fallstudien.


\end{content}

\begin{remarks}Das für drei Studienjahre im Voraus geplante Lehrangebot kann im Internet unter http://www.ior.kit.edu/ nachgelesen werden.

\end{remarks}

\end{module}

