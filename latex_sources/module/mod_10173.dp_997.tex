% Modulbeschreibung 
% Informationsgrad : extern
% Sprache: de
\begin{module}

\setdoclanguagegerman
\moduledegreeprogramme{Informatik (B.Sc.)}
\modulesubject{}
\moduleID{IN3INKAW}
\modulename{Konzepte und Anwendungen von Workflowsystemen}
\modulecoordination{K. Böhm}

\documentdate{2012-01-04 12:48:15.373336}

\modulecredits{5}
\moduleduration{1}
\modulecycle{Jedes 2. Semester, Wintersemester}



\modulehead

% For index (key word@display). Key word is used for sorting - no Umlauts please.
\index{Konzepte und Anwendungen von Workflowsystemen@Konzepte und Anwendungen von Workflowsystemen (M)}

% For later referencing
\label{mod_10173.dp_997}

\begin{courselist}
24111 & Konzepte und Anwendungen von Workflowsystemen (S.~\pageref{cour_10171.dp_997}) & 3 & W & 5 & J. Mülle, Silvia von Stackelberg\\
\end{courselist}

\begin{styleenv}
\begin{assessment}
Es wird im Voraus angekündigt, ob die Erfolgskontrolle in Form einer schriftlichen Prüfung (Klausur) im Umfang von 1h nach § 4, Abs. 2 Nr. 1 SPO oder in Form einer mündlichen Prüfung im Umfang von 20 min. nach § 4 Abs. 2 Nr. 2 SPO stattfindet.


\end{assessment}

\begin{conditions}Wenn das Modul \emph{Workflow-Management-Systeme} [IN4INWMS] bereits geprüft wurde, kann dieses Modul nicht geprüft werden.

\end{conditions}

\begin{recommendations}Datenbankkenntnisse, z.B. aus der Vorlesung \emph{Datenbanksysteme} [24516].

\end{recommendations}
\end{styleenv}

\begin{learningoutcomes}
Am Ende des Kurses sollen die Teilnehmer in der Lage sein, Workflows zu modellieren, die Modellierungsaspekte und ihr Zusammenspiel zu erläutern, Modellierungsmethoden miteinander zu vergleichen und ihre Anwendbarkeit in unterschiedlichen Anwendungsbereichen einzuschätzen. Sie sollten den technischen Aufbau eines Workflow-Management-Systems mit den wichtigsten Komponenten kennen und verschiedene Architekturen und Implementierungsalternativen bewerten können. Schließlich sollten die Teilnehmer einen Einblick in die aktuellen Standards bezüglich der Einsatzmöglichkeiten und in den Stand der Forschung durch aktuelle Forschungsthemen gewonnen haben.


\end{learningoutcomes}

\begin{content}
Workflow-Management-Systeme (WFMS) unterstützen die Abwicklung von Geschäftsprozessen entsprechend vorgegebener Arbeitsabläufe. Immer wichtiger wird die Unterstützung flexibler Abläufe, die Abweichungen, etwa zur Behandlung von Ausnahmen, zur Anpassungen an modifizierte Prozessumgebungen oder für Ad-Hoc-Workflows erlauben.

 

Die Vorlesung beginnt mit der Einordnung von WFMS in betriebliche Informationssysteme und stellt den Zusammenhang mit der Geschäftsprozessmodellierung her. Es werden formale Grundlagen für WFMS eingeführt (Petri-Netze, Pi-Kalkül). Modellierungsmethoden für Workflows und der Entwicklungsprozess von Workflow-Management-Anwendungen werden vorgestellt und in Übungen vertieft.

 

Weiterführende Aspekte betreffen neuere Entwicklungen im Bereich der WFMS. Insbesondere der Einsatz von Internettechniken speziell von Web Services und Standardisierungen für Prozessmodellierung, Orchestrierung und Choreographie in diesem Kontext werden vorgestellt.

 

Im Teil Realisierung von Workflow-Management-Systemen werden verschiedene Implementierungstechniken und Architekturfragen sowie Systemtypen und konkrete Systeme behandelt.

 

Abschließend wird auf anwendungsgetriebene Vorgehensweisen zur Änderung von Workflows, speziell Geschäftsprozess-Reengineering und kontinuierliche Prozessverbesserung, sowie Methoden und Konzepte zur Unterstütz


\end{content}

\begin{remarks}\textcolor{red}{Dieses Modul wird ab dem SS 2012 nicht mehr im Bachelor-Studiengang Informatik, sondern nur noch im Master-Studiengang Informatik angeboten.}

 

Im Bachelor-Studiengang kann die Prüfung nur noch nach Antragstellung und Genehmigung durch das Service-Zentrum Studium und Lehre erfolgen.

\end{remarks}

\end{module}

