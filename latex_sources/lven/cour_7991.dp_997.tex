% Lehrveranstaltungsbeschreibung
% Informationsgrad : extern
% Sprache: de
\begin{course}

\setdoclanguagegerman
\coursedegreeprogramme{Informatik}
\coursemodulename{Systemtheorie (S.~\pageref{mod_3937.dp_997})[IN3EITST], Grundlagen der Nachrichtentechnik (S.~\pageref{mod_3965.dp_997})[IN3EITGNT]}
\courseID{23109}
\coursename{Signale und Systeme}
\coursecoordination{F. Puente León}

\documentdate{2010-08-02 12:24:31.767877}

\courselevel{3}
\coursecredits{5}
\courseterm{Wintersemester}
\coursehours{2/1}
\courseinstructionlanguage{de}

\coursehead

% For index (key word@display). Key word is used for sorting - no Umlauts please.
\index{Signale und Systeme@Signale und Systeme}

% For later referencing
\label{cour_7991.dp_997}


\begin{styleenv}
\begin{assessment}
Die Erfolgskontrolle erfolgt in Form einer schriftlichen Prüfung im Umfang von ca. 120 Minuten nach § 4 Abs. 2 Nr. 1 SPO.

 

Die LV-Note ist die Note der Kausur.


\end{assessment}

\begin{conditions}Es werden Kenntnisse der höheren Mathematik und der “Wahrscheinlichkeitstheorie” (1305) vorausgesetzt.

\end{conditions}


\end{styleenv}

\begin{learningoutcomes}
Grundlagenvorlesung Signalverarbeitung. Schwerpunkte der Vorlesung sind die Betrachtung und Beschreibung von Signalen (zeitlicher Verlauf einer beobachteten Größe) und Systemen. Für den zeitkontinuierlichen und den zeitdiskreten Fall werden die unterschiedlichen Eigenschaften und Beschreibungsformen hergeleitet und analysiert. \newline
Diese Vorlesung vermittelt den Studenten somit einen grundlegenden Überblick über Methoden zur Beschreibung von Signalen und Systemen. Neben den theoretischen Grundlagen werden jedoch auch auf anwendungsspezifische Themen, wie der Filterentwurf im zeitkontinuierlichen oder zeitdiskreten Fall betrachtet.


\end{learningoutcomes}

\begin{content}
Diese Vorlesung stellt eine Einführung in wichtige theoretische Grundlagen der Signalverarbeitung dar, die für Studierende des 3. Semesters Elektrotechnik vorgesehen ist. Nach einer Einführung in die Funktionalanalysis werden zuerst Untersuchungsmethoden von Signalen und dann Eigenschaften, Darstellung, Untersuchung und Entwurf von Systemen sowohl für kontinuierliche als auch für diskrete Zeitänderungen vorgestellt.\newline
Zu Beginn wird ein allgemeiner Überblick über das gesamte Themengebiet gegeben.\newline
Aufbauend auf den Vorlesungen der Höheren Mathematik werden im zweiten Kapitel weitere Begriffe der Funktionalanalysis eingeführt. Ausgehend von linearen Vektorräumen werden die für die Signalverarbeitung wichtigen Hilberträume eingeführt und die linearen Operatoren behandelt. Von diesem Punkt aus ergibt sich eine gute Übersicht über die verwendeten mathematischen Methoden. \newline
Das nächste Kapitel beinhaltet die Betrachtung und Beschreibung von zeitkontinuierlichen Signalen, deren Eigenschaften und ihre unterschiedlichen Beschreibungsformen. Hierzu werden die aus der Funktionalanalysis vorgestellten Hilfsmittel in konkrete mathematische Anweisungen überführt. Dabei wird insbesondere auf die Möglichkeiten der Spektralanalyse mit Hilfe der Fourier-Reihe und der Fourier-Transformation eingegangen.\newline
Im vierten Kapitel werden zuerst allgemeine Eigenschaften von Systemen mit Hilfe von Operatoren formuliert. Anschließend wird die Beschreibung des Systemverhaltens durch Differenzialgleichungen eingeführt. Zur deren Lösung ist die Laplace-Transformation hilfreich. Diese wird mitsamt ihrer Eigenschaften dargestellt. Nach der Filterung mit Fensterfunktionen folgt die Beschreibung für den Entwurf zeitkontinuierlicher Filter im Frequenzbereich. Das Kapitel schließt mit der Behandlung der Hilbert-Transformation.\newline
Anschließend werden zeitdiskrete Signale betrachtet. Der Übergang ist notwendig, da in der Digitaltechnik nur diskrete Werte verarbeitet werden können. Zu Beginn des Kapitels wird auf grundlegende Details und Bedingungen eingegangen, die bei der Abtastung und Rekonstruktion analoger Signale berücksichtigt werden müssen. Im Anschluss wird auf Verfahren zur Spektralanalyse im zeitdiskreten Bereich eingegangen. Dabei steht insbesondere die Diskrete Fourier-Transformation im Fokus der Betrachtungen.\newline
Im letzten Kapitel werden die zeitdiskreten Systeme betrachtet. Zuerst werden die allgemeinen Eigenschaften zeitkontinuierlicher Systeme auf zeitdiskrete Systeme übertragen. Auf Besonderheiten der Zeitdiskretisierung wird explizit eingegangen und elementare Blöcke werden eingeführt. Anschließend wird die mathematische Beschreibung mittels Differenzengleichungen bzw. mit Hilfe der z-Transformation dargestellt. Nach der zeitdiskreten Darstellung zeitkontinuierlicher Systeme behandelt das Kapitel die frequenzselektiven Filter und die Filterung mit Fensterfunktionen, wie sie schon bei den zeitkontinuierlichen Systemen beschrieben wurden. Schließlich werden die eingeführten Begriffe und Definitionen anhand praktischer Beispiele veranschaulicht.\newline
Übungen\newline
Begleitend zur Vorlesung werden Übungsaufgaben zum Vorlesungsstoff gestellt. Diese werden in einer großen Saalübung besprochen und die zugehörigen Lösungen detailliert vorgestellt. Zudem gibt es die Möglichkeit, einen Teil des Stoffes mit Hilfe des Weblearnings zu vertiefen.


\end{content}

\begin{media}Vorlesungsfolien\newline
Übungsblätter

\end{media}

\begin{literature}Prof. Dr.-Ing. Kiencke: Signale und Systeme; Oldenbourg Verlag, 2008

 

\textbf{Weiterführende Literatur:}

 

Wird in der Vorlesung bekanntgegeben.

\end{literature}



\end{course}