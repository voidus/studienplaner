% Lehrveranstaltungsbeschreibung
% Informationsgrad : extern
% Sprache: de
\begin{course}

\setdoclanguagegerman
\coursedegreeprogramme{Informatik}
\coursemodulename{Schlüsselqualifikationen (S.~\pageref{mod_2523.dp_997})[IN1HOCSQ]}
\courseID{SQHoC}
\coursename{Schlüsselqualifikationen HoC}
\coursecoordination{M. Stolle}

\documentdate{2011-07-26 14:00:29.243144}

\courselevel{4}
\coursecredits{4}
\courseterm{}
\coursehours{2}
\courseinstructionlanguage{}

\coursehead

% For index (key word@display). Key word is used for sorting - no Umlauts please.
\index{Schluesselqualifikationen HoC@Schlüsselqualifikationen HoC}

% For later referencing
\label{cour_14497.dp_997}


\begin{styleenv}
\begin{assessment}
In den Veranstaltungen des House of Competence (HoC) sind kompetenzbasierte Prüfungsverfahren integriert. Je nach Veranstaltung kommen verschiede Prüfungsformen zum Einsatz, genaue Angaben finden sich in den Veranstaltungsbeschreibungen des House of Competence (HoC). Hat der Studierende die Leistungsstandards erfüllt, bekommt er eine erfolgreiche Teilnahme von der anbietenden Einrichtung bescheinigt und nach Rücksprache mit dem Dozenten wird eine Prüfungsnote ausgewiesen.


\end{assessment}

\begin{conditions}Keine.\end{conditions}


\end{styleenv}

\begin{learningoutcomes}

\end{learningoutcomes}

\begin{content}
Das House of Competence bietet mit den Veranstaltungen Schlüsselqalifikationen eine breite Auswahl aus fünf Wahlbereichen, in denen Veranstaltungen zur besseren Orientierung thematisch zusammengefasst werden. Die Inhalte werden in den Beschreibungen der Veranstaltungen auf den Internetseiten des HoC (http://www.hoc.kit.edu/) detailliert erläutert.\newline
\newline
Wahlbereiche des HoC:

 \begin{itemize}\item „Kultur – Politik – Wissenschaft – Technik“, 2-3 LP  \item „Kompetenz- und Kreativitätswerkstatt“, 2-3 LP   \item „Fremdsprachen“, 2-3 LP  \item „Persönliche Fitness \& Emotionale Kompetenz“, 2-3 LP  \item „Tutorenprogramme“, 3 LP  \item „Mikrobausteine“, 1 LP  \end{itemize}
\end{content}







\end{course}