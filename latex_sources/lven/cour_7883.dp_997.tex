% Lehrveranstaltungsbeschreibung
% Informationsgrad : extern
% Sprache: de
\begin{course}

\setdoclanguagegerman
\coursedegreeprogramme{Informatik}
\coursemodulename{Methodische Grundlagen des OR (S.~\pageref{mod_3833.dp_997})[IN3WWOR3]}
\courseID{2550113}
\coursename{Nichtlineare Optimierung II}
\coursecoordination{O. Stein}

\documentdate{2011-12-15 17:13:39.313173}

\courselevel{4}
\coursecredits{4,5}
\courseterm{Sommersemester}
\coursehours{2/1}
\courseinstructionlanguage{de}

\coursehead

% For index (key word@display). Key word is used for sorting - no Umlauts please.
\index{Nichtlineare Optimierung II@Nichtlineare Optimierung II}

% For later referencing
\label{cour_7883.dp_997}


\begin{styleenv}
\begin{assessment}
Die Erfolgskontrolle erfolgt in Form einer schriftlichen Prüfung (120min.) (nach §4(2), 1 SPO).

 

Die Prüfung wird im Vorlesungssemester und dem darauf folgenden Semester angeboten.

 

Zulassungsvoraussetzung zur schriftlichen Prüfung ist der Erwerb von mindestens 30\% der Übungspunkte. Die Prüfungsanmeldung über das Online-Portal für die schriftliche Prüfung gilt somit vorbehaltlich der Erfüllung der Zulassungsvoraussetzung.

 

Die Erfolgskontrolle kann auch zusammen mit der Erfolgskontrolle zu \emph{Nichtlineare Optimierung I} [2550111] erfolgen. In diesem Fall beträgt die Dauer der schriftlichen Prüfung 120 min.

 

Bei gemeinsamer Erfolgskontrolle über die Vorlesungen \emph{Nichtlineare Optimierung I} [2550111] und \emph{Nichtlineare Optimierung II} [2550113] wird bei Erwerb von mindestens 60\% der Übungspunkte die Note der bestandenen Klausur um ein Drittel eines Notenschrittes angehoben.

 

Bei gemeinsamer Erfolgskontrolle über die Vorlesungen \emph{Nichtlineare Optimierung I} [2550111] und \emph{Nichtlineare Optimierung II} [2550113] wird bei Erwerb von mindestens 60\% der Rechnerübungspunkte die Note der bestandenen Klausur um ein Drittel eines Notenschrittes angehoben.


\end{assessment}

\begin{conditions}Keine.\end{conditions}


\end{styleenv}

\begin{learningoutcomes}
Der/die Studierende soll

 \begin{itemize}\item mit Grundlagen der nichtlinearen Optimierung vertraut gemacht werden  \item in die Lage versetzt werden, moderne Techniken der nichtlinearen Optimierung in der Praxis auswählen, gestalten und einsetzen zu können.  \end{itemize}
\end{learningoutcomes}

\begin{content}
Die Vorlesung behandelt die Minimierung glatter nichtlinearer Funktionen unter nichtlinearen Restriktionen. Für solche Probleme, die in Wirtschafts-, Ingenieur- und Naturwissenschaften sehr häufig auftreten, werden Optimalitätsbedingungen hergeleitet und darauf basierende numerische Lösungsverfahren angegeben. Teil I der Vorlesung behandelt unrestringierte Optimierungsprobleme. Teil II der Vorlesung ist wie folgt aufgebaut:

 \begin{itemize}\item Topologie und Approximationen erster Ordnung der zulässigen Menge  \item Alternativsätze, Optimalitätsbedingungen erster und zweiter Ordnung für restringierte Probleme  \item Optimalitätsbedingungen für restringierte konvexe Probleme  \item Numerische Verfahren für restringierte Probleme (Strafterm-Verfahren, Multiplikatoren-Verfahren, Barriere-Verfahren, Innere-Punkte-Verfahren, SQP-Verfahren, Quadratische Optimierung)  \end{itemize}

In der parallel zur Vorlesung angebotenen Rechnerübung haben Sie Gelegenheit, die Programmiersprache MATLAB zu erlernen und einige dieser Verfahren zu implementieren und an praxisnahen Beispielen zu testen.


\end{content}



\begin{literature}\textbf{Weiterführende Literatur:}

 \begin{itemize}\item W. Alt, Nichtlineare Optimierung, Vieweg, 2002  \item M.S. Bazaraa, H.D. Sherali, C.M. Shetty, Nonlinear Programming, Wiley, 1993  \item O. Güler, Foundations of Optimization, Springer, 2010  \item H.Th. Jongen, K. Meer, E. Triesch, Optimization Theory, Kluwer, 2004  \item J. Nocedal, S. Wright, Numerical Optimization, Springer, 2000  \end{itemize}\end{literature}

\begin{remarks}Teil I und II der Vorlesung werden nacheinander im \emph{selben }Semester gelesen.

\end{remarks}

\end{course}