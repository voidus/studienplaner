% Lehrveranstaltungsbeschreibung
% Informationsgrad : extern
% Sprache: de
\begin{course}

\setdoclanguagegerman
\coursedegreeprogramme{Informatik}
\coursemodulename{Praktische Mathematik (S.~\pageref{mod_2551.dp_997})[IN2MATHPM]}
\courseID{01335}
\coursename{Grundlagen der Wahrscheinlichkeitstheorie und Statistik für Studierende der Informatik}
\coursecoordination{D. Kadelka}

\documentdate{2008-10-07 10:30:09}

\courselevel{2}
\coursecredits{4,5}
\courseterm{Wintersemester}
\coursehours{2/1}
\courseinstructionlanguage{de}

\coursehead

% For index (key word@display). Key word is used for sorting - no Umlauts please.
\index{Grundlagen der Wahrscheinlichkeitstheorie und Statistik fuer Studierende der Informatik@Grundlagen der Wahrscheinlichkeitstheorie und Statistik für Studierende der Informatik}

% For later referencing
\label{cour_7235.dp_997}


\begin{styleenv}
\begin{assessment}
Die Erfolgskontrolle erfolgt in Form einer schriftlichen Prüfung im Umfang von i.d.R. 90 Minuten gemäß § 4 Abs. 2 Nr. 1 SPO.


\end{assessment}

\begin{conditions}Keine.\end{conditions}


\end{styleenv}

\begin{learningoutcomes}
Das Hauptziel der Vorlesung besteht darin, die Anwender stochastischer Methoden in der Informatik für die vielfältigen Probleme zu sensibilisieren, welche mit der Modellierung zufälliger Phänomene verbunden sind. Mit dieser Sensibilisierung soll ein notwendiger und wünschenswerter Dialog zwischen Anwender und Stochastiker erleichtert werden.


\end{learningoutcomes}

\begin{content}
Dieses Modul soll Studierende in die grundlegenden Methoden der beschreibenden und (rudimentär) schließenden Statistik und in die Wahrscheinlichkeitstheorie einführen.

 

Behandelt werden:\newline
1. Deskriptive Statistik\newline
2. Merkmalräume und Ereignisse\newline
3. Wahrscheinlichkeitsräume\newline
4. Kombinatorik\newline
5. Zufallsvariablen\newline
6. Verteilungen diskreter Zufallsvariablen\newline
7. Wichtige diskrete Verteilungen\newline
8. Verteilungsfunktionen und Dichten\newline
9. Wichtige stetige Verteilungen\newline
10. Übergangswahrscheinlichkeiten und bedingte Wahrscheinlichkeiten\newline
11. Stochastische Unabhängigkeit\newline
12. Maßzahlen von Verteilungen\newline
13. Pseudozufallszahlen und Simulation\newline
14. Grundprobleme der Statistik\newline
15. Punkt-Schätzung\newline
16. Konfidenzbereiche (Bereichs-Schätzer)


\end{content}



\begin{literature}\textbf{Weiterführende Literatur:}

Henze/Kadelka: Skript zur Vorlesung „Wahrscheinlichkeitstheorie und Statistik für Studierende der Informatik“

\end{literature}



\end{course}