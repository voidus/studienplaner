% Lehrveranstaltungsbeschreibung
% Informationsgrad : extern
% Sprache: de
\begin{course}

\setdoclanguagegerman
\coursedegreeprogramme{Informatik}
\coursemodulename{Web-Anwendungen und Web-Technologien (S.~\pageref{mod_2925.dp_997})[IN3INWAWT], Advanced Web Applications (S.~\pageref{mod_4067.dp_997})[IN3INAWA]}
\courseID{24604/24153}
\coursename{Advanced Web Applications}
\coursecoordination{S. Abeck}

\documentdate{2011-11-14 11:32:49.142466}

\courselevel{4}
\coursecredits{4}
\courseterm{Winter-/Sommersemester}
\coursehours{2/0}
\courseinstructionlanguage{de}

\coursehead

% For index (key word@display). Key word is used for sorting - no Umlauts please.
\index{Advanced Web Applications@Advanced Web Applications}

% For later referencing
\label{cour_5663.dp_997}


\begin{styleenv}
\begin{assessment}
Die Erfolgskontrolle wird in der Modulbeschreibung erläutert.


\end{assessment}

\begin{conditions}Keine.\end{conditions}

\begin{recommendations}1. Telematik-Kenntnisse, insbes. zu Schichentenarchitekturen, Anwenungsprotokollen und zur\newline
 Extensible Markup Language.\newline
 2. Softwaretechnik-Kenntnisse, insbes. zu Softwarearchitekturen und deren Modellierung mittels Unified Modeling Language.

\end{recommendations}
\end{styleenv}

\begin{learningoutcomes}
Die Architektur von mehrschichtigen und dienstorientierten Anwendungssystemen ist verstanden.\newline
 Die Softwarearchitektur einer Web-Anwendung kann modelliert werden.\newline
 Die wichtigsten Prinzipien traditioneller Softwareentwicklung und des entsprechenden Entwicklungsprozesses sind bekannt.\newline
Die Verfeinerung höherstufiger Prozessmodelle sowie deren Abbildung auf eine dienstorientierte Architektur sind verstanden.


\end{learningoutcomes}

\begin{content}
Der Kurs setzt sich aus den folgenden Kurseinheiten zusammen:\newline
\newline
 • GRUNDLAGEN FORTGESCHRITTENER WEBANWENDUNGEN: Mehrschichtige Anwendungsarchitekturen, insbesondere die dienstorientierte Architektur (Service-Oriented Architecture, SOA) basierend auf Webservice-Standards wie XML (Extensible Markup Language) und WSDL (Web Services Description Language) werden beschrieben. \newline
 • DIENSTENTWURF: Der Entwicklungsprozess wird um Ansätze zur Abbildung von Geschäftsprozessen auf dienstorientierte Web-Anwendungen und zum Entwurf der dabei notwendigen Dienste erweitert. \newline
 • BENUTZERINTERAKTION: Diese Kursseinheit behandelt die modellgetriebene Sofwareentwicklung von fortgeschrittenen, benutzerzentrierten Web-Anwendungen basierend auf UML (Unified Modeling Language) und MDA (Model-driven Architecture).\newline
 • IDENTITÄTSMANAGEMENT: Die wichtigsten Funktionsbausteine eines Identitätsmanagements werden eingeführt und die spezifischen Anforderungen an eine dienstorientierte Lösung werden abgeleitet.\newline
 • IT-MANAGEMENT: Die Kurseinheit betrachtet prozessorientierte Managementstandards, die durch standardisierte Managementkomponenten umgesetzt werden können.


\end{content}

\begin{media}(1) Lernmaterial: Zu jeder Kurseinheit besteht ein strukturiertes Kursdokument (mit Kurzbeschreibung, Lernzielen, Index, Glossar, Literaturverzeichnis)

 

(2) Lehrmaterial: Folien (integraler Bestandteil der Kursdokumente)

\end{media}

\begin{literature}Thomas Erl: Service-Oriented Architecture –Principles of Service Design, Prentice Hall, 2007.

 

\textbf{Weiterführende Literatur:}

 

(1) Ali Arsanjani: Service-Oriented Modeling and Architecture, IBM developer works, 2004.

 

(2) Thomas Stahl, Markus Völter: Modellgetriebene Softwareentwicklung, dpunkt Verlag, 2005.

 

(3) Eric Yuan, Jin Tong: Attribute Based Access Control (ABAC) for Web Services, IEEE International Conference on Web Services (ICWS 2005), Orlando Florida, July 2005.

\end{literature}

\begin{remarks}\textcolor{red}{Diese Lehrveranstaltung wurde im SS 2011 letztmalig angeboten. Prüfungen sind für Wiederholer bis Wintersemester 2012/13 möglich.}

\end{remarks}

\end{course}