% Lehrveranstaltungsbeschreibung
% Informationsgrad : extern
% Sprache: de
\begin{course}

\setdoclanguagegerman
\coursedegreeprogramme{Informatik}
\coursemodulename{IT-Sicherheitsmanagement für vernetzte Systeme (S.~\pageref{mod_15853.dp_997})[IN3INITSS], Netzwerk- und IT-Sicherheitsmanagement (S.~\pageref{mod_2577.dp_997})[IN3INNITS]}
\courseID{24149}
\coursename{IT-Sicherheitsmanagement für vernetzte Systeme}
\coursecoordination{H. Hartenstein}

\documentdate{2012-02-01 10:25:45.203376}

\courselevel{4}
\coursecredits{5}
\courseterm{Wintersemester}
\coursehours{2/1}
\courseinstructionlanguage{de}

\coursehead

% For index (key word@display). Key word is used for sorting - no Umlauts please.
\index{IT-Sicherheitsmanagement fuer vernetzte Systeme@IT-Sicherheitsmanagement für vernetzte Systeme}

% For later referencing
\label{cour_5399.dp_997}


\begin{styleenv}
\begin{assessment}
Die Erfolgskontrolle wird in der Modulbeschreibung erläutert.


\end{assessment}

\begin{conditions}Grundkenntnisse im Bereich Rechnernetze, entsprechend den Vorlesungen \emph{Datenbanksysteme }[24516] und \emph{Einführung in Rechnernetze} [24519], sind notwendig.

\end{conditions}


\end{styleenv}

\begin{learningoutcomes}
Ziel der Vorlesung ist es, den Studenten die Grundlagen des IT-Sicherheitsmanagements für vernetzte Systeme zu vermitteln. Es sollen sowohl technische als auch zugrunde liegende Management-Aspekte verdeutlicht werden.


\end{learningoutcomes}

\begin{content}
Die Vorlesung dieses Moduls behandelt das Management moderner, verteilter IT-Systeme und - Dienste. Hierfür werden tragende Konzepte und Modelle in den Bereichen IT-Sicherheitsmanagement, Netzwerkmanagement, Identitätsmanagement und IT-Servicemanagement vorgestellt und diskutiert. Aufbauend werden konkrete technische Architekturen, Protokolle und Werkzeuge innerhalb der genannten Bereiche betrachtet.

 

Unter anderem werden die Konzepte von IT-Sicherheitsprozessen anhand des BSI Grundschutzes verdeutlicht, die Steuerung und Überwachung von hochverteilten Rechnernetzen erörtert und die öffentliche IP-Netzverwaltung betrachtet. Weitere Schwerpunkte bilden das Zugangs- und Identitätsmanagement sowie Firewalls, Intrusion Detection und Prevention. Die Themen werden ferner anhand zahlreicher Fallbeispiele aus dem operativen Betrieb des Steinbuch Centre for Computing (SCC) vertieft, wie zum Beispiel im Kontext des glasfasergebundenen Backbones KITnet. Anhand aktueller Forschungsaktivitäten aus den Bereichen Peer-to-Peer-Netze (z.B. BitTorrent) und soziale Netzwerke (z.B. Facebook) werden die vermittelten Managementansätze in einen globalen Kontext gesetzt.


\end{content}

\begin{media}Folien

\end{media}

\begin{literature} 

Jochen Dinger, Hannes Hartenstein, Netzwerk- und IT-Sicherheitsmanagement : Eine Einführung, Universitätsverlag Karlsruhe, 2008.

 

\textbf{Weiterführende Literatur:}

  

Heinz-Gerd Hegering, Sebastian Abeck, Bernhard Neumair, Integriertes Management vernetzter Systeme - Konzepte, Architekturen und deren betrieblicher Einsatz, dpunkt-Verlag, Heidelberg, 1999.

  

James F. Kurose, Keith W. Ross, Computer Networking. A Top-Down Approach Featuring the Internet, 3rd ed., Addison-Wesley Longman, Amsterdam, 2004.

  

Larry L. Peterson, Bruce S. Davie, Computer Networks - A Systems Approach, 3rd ed., Morgan Kaufmann Publishers, 2003.

  

William Stallings, SNMP, SNMPv2, SNMPv3 and RMON 1 and 2, 3rd ed., Addison-Wesley Professional, 1998.

  

Claudia Eckert, IT-Sicherheit. Konzepte - Verfahren - Protokolle, 4. Auflage, Oldenbourg, 2006.

  

Michael E. Whitman, Herbert J. Mattord, Management of Information Security, Course Technology, 2004.

 \end{literature}

\begin{remarks}Die Lehrveranstaltung wurde bis zum WS 2011/12 unter dem Titel \emph{Netzwerk- und IT-Sicherheitsmanagement} angeboten.

\end{remarks}

\end{course}