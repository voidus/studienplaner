% Lehrveranstaltungsbeschreibung
% Informationsgrad : extern
% Sprache: de
\begin{course}

\setdoclanguagegerman
\coursedegreeprogramme{Informatik}
\coursemodulename{Geometrische Optimierung (S.~\pageref{mod_13909.dp_997})[IN3INGO]}
\courseID{24657}
\coursename{Geometrische Optimierung}
\coursecoordination{H. Prautzsch}

\documentdate{2011-03-11 10:53:51.250489}

\courselevel{4}
\coursecredits{3}
\courseterm{Sommersemester}
\coursehours{2}
\courseinstructionlanguage{de}

\coursehead

% For index (key word@display). Key word is used for sorting - no Umlauts please.
\index{Geometrische Optimierung@Geometrische Optimierung}

% For later referencing
\label{cour_13907.dp_997}


\begin{styleenv}
\begin{assessment}
Die Erfolgskontrolle wird in der Modulbeschreibung erläutert.


\end{assessment}

\begin{conditions}Keine.\end{conditions}


\end{styleenv}

\begin{learningoutcomes}
Die Hörer und Hörerinnen der Vorlesung sollen Grundlagen der Optimierung bei geometrischen Anwendungsaufgaben kennenlernen.


\end{learningoutcomes}

\begin{content}
Grundlegende Methoden zur Optimierung wie die Methode der kleinsten Quadrate, Levenber-Marquardt-Algorithmus, Berechnung von Ausgleichsebenen, iterative Ist- und Sollwertanpassung von Punktwolken (iterated closest point), finite Element-Methoden.

 

Optimierung bei Anwendungsaufgaben wie beim Bewegungstransfer zur Anmation, Übertragung von Alterungs- und mimischen Prozessen auf Gesichter, Approximation mit abwickelbaren Flächen zur besseren Fertigung von Objekten, automatische Glättung von Flächen, verzerrungsarme Abbildungen auf gekrümmte Flächen zur Aufbringung planarer Muster und Texturen.

 

Fragen zur numerischen Stabilität und Algorithmen zur exakten Berechung einfacher geometrischer Operationen.

 

Verfahren der algorithmischen Geometrie etwa zur Bestimmung kleinster umhüllender Kugeln (Welzl-Algorithmus)


\end{content}

\begin{media}Tafel, Folien.

\end{media}

\begin{literature}Verschiedene Fachartikel und Buchkapitel. Wird in der Vorlesung bekannt gegeben.

\end{literature}



\end{course}