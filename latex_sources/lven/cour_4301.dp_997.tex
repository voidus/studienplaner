% Lehrveranstaltungsbeschreibung
% Informationsgrad : extern
% Sprache: de
\begin{course}

\setdoclanguagegerman
\coursedegreeprogramme{Informatik}
\coursemodulename{Grundlagen der BWL (S.~\pageref{mod_3033.dp_997})[IN3WWBWL]}
\courseID{2600002}
\coursename{Rechnungswesen}
\coursecoordination{T. Lüdecke}

\documentdate{2012-01-09 19:05:05.893368}

\courselevel{1}
\coursecredits{4}
\courseterm{Wintersemester}
\coursehours{2/2}
\courseinstructionlanguage{de}

\coursehead

% For index (key word@display). Key word is used for sorting - no Umlauts please.
\index{Rechnungswesen@Rechnungswesen}

% For later referencing
\label{cour_4301.dp_997}


\begin{styleenv}
\begin{assessment}
Die Erfolgskontrolle erfolgt in Form einer schriftlichen Klausur (120 min.) nach § 4 Abs. 2 Nr. 1 SPO.

 

Die Prüfung wird in jedem Semester angeboten und kann zu jedem ordentlichen Prüfungstermin wiederholt werden.


\end{assessment}

\begin{conditions}Keine.\end{conditions}


\end{styleenv}

\begin{learningoutcomes}
Die Abbildung des ökonomischen Geschehens in der Unternehmung findet statt im Rechnungswesen, sowohl in Form des externen als auch des internen Rechnungswesen. Ohne Kenntnisse dieser zentralen Bausteine ist der Ablauf und die Analyse einer Unternehmung nicht vorstellbar. Demzufolge bildet die Vermittlung fundierten Wissens des Financial Accounting und Management Accounting eine notwendige Voraussetzung für das Verständnis des gesamten weiteren Studiums mit betriebswirtschaftlichem Bezug. Der Studierende sollte Sicherheit erlangen in Bezug auf den Jahresabschluss sowie das Instrument der Kostenrechnung in Grundzügen beherrschen.


\end{learningoutcomes}

\begin{content}
Nach einer Einführung in die Aufgaben und Grundbegriffe des Rechnungswesen wird das System der Doppik vorgestellt. Typische Buchungsfälle in Handels- und Industrieunternehmen werden abgerundet durch spezielle Probleme der Finanzbuchhaltung. Der Jahresabschluss nach HGB mit Bilanz, Gewinn- und Verlustrechnung sowie Anhang und Lagebericht steht im Zentrum des ersten Teils der Vorlesung. Grundsätze ordnungsmäßiger Bilanzierung in Verbindung mit Bewertungsproblemen schliessen sich an. Der zweite Teil der Vorlesung umfaßt die Kosten- und Leistungsrechnung. Das Instrumentarium der Kostenrechnung in Form von Kostenarten, - stellen und - trägerrechnung wird systematisch dargestellt. Den Abschluss stellen Aspekte moderner entscheidungsorientierter Verfahren und Systeme der KLR dar.


\end{content}

\begin{media}Folien

\end{media}

\begin{literature}\begin{itemize}\item R. Buchner, Buchführung und Jahresabschluss, Vahlen Verlag  \item A. Coenenberg, Jahresabschluss und Jahresabschlussanalyse, Verlag Moderne Industrie  \item A. Coenenberg, Kostenrechnung und Kostenanalyse, Verlag Moderne Industrie  \item R. Ewert, A. Wagenhofer, Interne Unternehmensrechnung, Springer Verlag  \item J. Schöttler, R. Spulak, Technik des betrieblichen Rechnungswesen, Oldenbourg Verlag  \end{itemize}\end{literature}



\end{course}