% Lehrveranstaltungsbeschreibung
% Informationsgrad : extern
% Sprache: de
\begin{course}

\setdoclanguagegerman
\coursedegreeprogramme{Informatik}
\coursemodulename{Theoretische Grundlagen der Informatik (S.~\pageref{mod_2957.dp_997})[IN2INTHEOG]}
\courseID{24005}
\coursename{Theoretische  Grundlagen der Informatik}
\coursecoordination{D. Wagner}

\documentdate{2011-03-31 13:08:00.041982}

\courselevel{2}
\coursecredits{6}
\courseterm{Wintersemester}
\coursehours{3/1}
\courseinstructionlanguage{de}

\coursehead

% For index (key word@display). Key word is used for sorting - no Umlauts please.
\index{Theoretische  Grundlagen der Informatik@Theoretische  Grundlagen der Informatik}

% For later referencing
\label{cour_7481.dp_997}


\begin{styleenv}
\begin{assessment}
Die Erfolgskontrolle wird in der Modulbeschreibung erläutert.


\end{assessment}

\begin{conditions}Keine.\end{conditions}


\end{styleenv}

\begin{learningoutcomes}
Der/die Studierende

 \begin{itemize}\item besitzt einen vertieften Einblick in die Grundlagen der Theoretischen Informatik und beherrscht deren Berechnungsmodelle und Beweistechniken,  \item versteht die Grenzen und Möglichkeiten der Informatik in Bezug auf die Lösung von definierbaren aber nur bedingt berechenbaren Problemen,  \item abstrahiert grundlegende Aspekte der Informatik von konkreten Gegebenheiten wie konkreten Rechnern oder Programmiersprachen und formuliert darüber allgemeingültige Aussagen über die Lösbarkeit von Problemen,  \item ist in der Lage, die erlernten Beweistechniken bei der Spezifikation von Systemen der Informatik und für den systematischen Entwurf von Programmen und Algorithmen anzuwenden.  \end{itemize}
\end{learningoutcomes}

\begin{content}
Es gibt wichtige Probleme, deren Lösung sich zwar klar definieren läßt aber die man niemals wird systematisch berechnen können. Andere Probleme lassen sich “vermutlich” nur durch systematisches Ausprobieren lösen. Andere Themen dieser Vorlesungen legen die Grundlagen für Schaltkreisentwurf, Compilerbau, uvam. Die meisten Ergebnisse dieser Vorlesung werden rigoros bewiesen. Die dabei erlernten Beweistechniken sind wichtig für die Spezifikation von Systemen der Informatik und für den systematischen Entwurf von Programmen und Algorithmen. \newline
Das Modul gibt einen vertieften Einblick in die Grundlagen und Methoden der Theoretischen Informatik. Insbesondere wird dabei eingegangen auf grundlegende Eigenschaften Formaler Sprachen als Grundlagen von Programmiersprachen und Kommunikationsprotokollen (regulär, kontextfrei, Chomsky-Hierarchie), Maschinenmodelle (endliche Automaten, Kellerautomaten, Turingmaschinen, Nichtdeterminismus, Bezug zu Familien formaler Sprachen), Äquivalenz aller hinreichend mächtigen Berechnungsmodelle (Churchsche These), Nichtberechenbarkeit wichtiger Funktionen (Halteproblem,...), Gödels Unvollständigkeitssatz und Einführung in die Komplexitätstheorie (NP-vollständige Probleme und polynomiale Reduktionen).


\end{content}

\begin{media}Folien (pdf), Aufgabenblätter, Skript.

\end{media}

\begin{literature}\textbf{Weiterführende Literatur:}

 \begin{itemize}\item Uwe Schöning: Theoretische Informatik - kurz gefasst. Sprektrum (2001).  \item Ingo Wegener: Theoretische Informatik. Teubner (1999)  \item Ingo Wegener: Kompedium theoretische Informatik. Teubner (1996).  \end{itemize}\end{literature}



\end{course}