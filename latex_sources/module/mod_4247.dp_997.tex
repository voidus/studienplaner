% Modulbeschreibung 
% Informationsgrad : extern
% Sprache: de
\begin{module}

\setdoclanguagegerman
\moduledegreeprogramme{Informatik (B.Sc.)}
\modulesubject{EF Physik}
\moduleID{IN2PHY2}
\modulename{Moderne Physik für Informatiker}
\modulecoordination{Quast}

\documentdate{2011-04-08 12:16:22.044771}

\modulecredits{9}
\moduleduration{1}
\modulecycle{Jedes 2. Semester, Sommersemester}



\modulehead

% For index (key word@display). Key word is used for sorting - no Umlauts please.
\index{Moderne Physik fuer Informatiker@Moderne Physik für Informatiker (M)}

% For later referencing
\label{mod_4247.dp_997}

\begin{courselist}
2400451 & Moderne Physik für Informatiker (S.~\pageref{cour_8651.dp_997}) & 4/2 & S & 9 & Evers\\
\end{courselist}

\begin{styleenv}
\begin{assessment}
Die Erfolgskontrolle besteht aus einer schriftlichen Klausur im Umfang von 90 Minuten nach § 4 Abs. 2 Nr. 1 SPO sowie eiem Leistungsnachweis über die Übungen (nach § 4 Abs. 2 Nr. 3 SPO).

 

Die Modulnote ist die Note der schriftlichen Prüfung.


\end{assessment}

\begin{conditions}\textcolor{red}{Dieses Modul muss zusammen mit dem Modul \emph{Grundlagen der Physik} geprüft werden.}

\end{conditions}


\end{styleenv}

\begin{learningoutcomes}
Der/Die Studierende soll ein grundlegendes Verständnis für die experimentellen und mathematischen Methoden der Quantenphysik (Atome, Moleküle, Festkörper, Kerne und Elementarteilchen) erlangen.


\end{learningoutcomes}

\begin{content}
Einführung in den Mikrokosmos, spezielle Relativitätstheorie, Wellen- und Teilchencharakter des Lichts quantisierte Größen, Welleneigenschaften von Teilchen, deBroglie-Beziehung, Heisenberg'sche Unbestimmtheitsrelation, Schrödingergleichung Quantenmechanische Beschreibung von Atomen, Elektronen-Spin, Pauli-Prinzip, Periodensystem der Elemente, Wechselwirkung von Licht mit Atomen, Laser, Chemische Bindung und Moleküle, Grundprinzipien der Festkörperphysik, Elektronengas, Wärmekapazität, Stromleitung, Bändermodell, Atomkerne, Radioaktivität und Kernkräfte, Kernfusion, Kernmodelle, Kernzerfälle, Kernspaltung, Physik der Sonne, Elementarteilchen als Bausteine der Welt, Astrophysik und Kosmologie


\end{content}



\end{module}

