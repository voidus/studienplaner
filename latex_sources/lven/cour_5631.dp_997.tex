% Lehrveranstaltungsbeschreibung
% Informationsgrad : extern
% Sprache: de
\begin{course}

\setdoclanguagegerman
\coursedegreeprogramme{Informatik}
\coursemodulename{Supply Chain Management (S.~\pageref{mod_2721.dp_997})[IN3WWBWL14]}
\courseID{2118078}
\coursename{Logistik - Aufbau, Gestaltung und Steuerung von Logistiksystemen}
\coursecoordination{K. Furmans}

\documentdate{2011-12-14 13:17:56.332874}

\courselevel{3}
\coursecredits{6}
\courseterm{Sommersemester}
\coursehours{3/1}
\courseinstructionlanguage{de}

\coursehead

% For index (key word@display). Key word is used for sorting - no Umlauts please.
\index{Logistik - Aufbau, Gestaltung und Steuerung von Logistiksystemen@Logistik - Aufbau, Gestaltung und Steuerung von Logistiksystemen}

% For later referencing
\label{cour_5631.dp_997}


\begin{styleenv}
\begin{assessment}
Die Erfolgskontrolle erfolgt in Form einer schriftlichen Prüfung (nach §4(2), 1 SPO). Durch die Abgabe von Fallstudien kann ein Bonus für die schriftliche Prüfung erworben werden.


\end{assessment}

\begin{conditions}Keine.\end{conditions}

\begin{recommendations}Der Besuch der Vorlesungen „Lineare Algebra” und „Stochastik” wird vorausgesetzt.

\end{recommendations}
\end{styleenv}

\begin{learningoutcomes}
Der Student kann grundlegende Fragestellungen aus den Bereichen der Planung und des Betriebs von Materialfluss- und Logistiksystemen einordnen und kann mit geeigneten Verfahren Planungen durchführen. Er kennt die wesentlichen Elemente von Materialfluss-und Logistiksystemen und kann eine Abschätzung der Leistungsfähigkeit durchführen.


\end{learningoutcomes}

\begin{content}
Einführung

 \begin{itemize}\item Historischer Überblick  \item Entwicklungslinien  \item Struktur  \end{itemize}

Aufbau von Logistiksystemen

 

Distributionslogistik

 \begin{itemize}\item Standortplanung  \item Touren- und Routenplanung  \item Distributionszentren  \end{itemize}

Bestandsmanagement

 \begin{itemize}\item Bedarfsplanung  \item Lagerhaltungspolitiken  \item Bullwhip-Effekt  \end{itemize}

Produktionslogistik

 \begin{itemize}\item Layoutplanung  \item Materialfluß  \item Steuerungsverfahren  \end{itemize}

Beschaffungslogistik

 \begin{itemize}\item Informationsfluss  \item Transportorganisation  \item Steuerung und Entwicklung eines Logistiksystems  \item Kooperationsmechanismen  \item Lean SCM  \item SCOR-Modell  \end{itemize}

Identifikationstechniken


\end{content}

\begin{media}Tafel, Datenprojektor. In Übungen ergänzend Nutzung von PCs.

\end{media}

\begin{literature}\textbf{Weiterführende Literatur:}

 \begin{itemize}\item Arnold/Isermann/Kuhn/Tempelmeier. Handbuch Logistik, Springer Verlag, 2002 (Neuauflage in Arbeit)  \item Domschke. Logistik, Rundreisen und Touren, Oldenbourg Verlag, 1982  \item Domschke/Drexl. Logistik, Standorte, Oldenbourg Verlag, 1996  \item Gudehus. Logistik, Springer Verlag, 2007  \item Neumann-Morlock. Operations-Research, Hanser-Verlag, 1993  \item Tempelmeier. Bestandsmanagement in Supply Chains, Books on Demand 2006  \item Schönsleben. Integrales Logistikmanagement, Springer, 1998  \end{itemize}\end{literature}

\begin{remarks}Die Vorlesung trug vorher den Titel \emph{Logistik}.

\end{remarks}

\end{course}