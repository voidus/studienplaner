% Lehrveranstaltungsbeschreibung
% Informationsgrad : extern
% Sprache: de
\begin{course}

\setdoclanguagegerman
\coursedegreeprogramme{Informatik}
\coursemodulename{Einführung in Geometrie und Topologie (S.~\pageref{mod_3101.dp_997})[IN3MATHAG03]}
\courseID{1026}
\coursename{Einführung in Geometrie und Topologie}
\coursecoordination{S. Kühnlein, E. Leuzinger, W. Tuschmann}

\documentdate{2011-10-06 18:08:28.299053}

\courselevel{}
\coursecredits{9}
\courseterm{Wintersemester}
\coursehours{6}
\courseinstructionlanguage{}

\coursehead

% For index (key word@display). Key word is used for sorting - no Umlauts please.
\index{Einfuehrung in Geometrie und Topologie@Einführung in Geometrie und Topologie}

% For later referencing
\label{cour_7867.dp_997}


\begin{styleenv}
\begin{assessment}
Die Erfolgskontrolle wird in der Modulbeschreibung erläutert.


\end{assessment}

\begin{conditions}Keine.\end{conditions}

\begin{recommendations}Folgende Module sollten bereits belegt worden sein (Empfehlung):\newline
Lineare Algebra 1+2\newline
Analysis 1+2

\end{recommendations}
\end{styleenv}

\begin{learningoutcomes}
\begin{itemize}\item Einführung in exemplarische Gegenstände und Denkweisen der modernen Geometrie  \item Vorbereitung auf Seminare und weiterführende Vorlesungen im Bereich Geometrie  \end{itemize}
\end{learningoutcomes}

\begin{content}
\begin{itemize}\item Topologische und metrische Räume  \item Klassifikation von Flächen  \item Differentialgeometrie von Flächen  \item Optional: Raumformen  \end{itemize}
\end{content}







\end{course}