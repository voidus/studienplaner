% Modulbeschreibung 
% Informationsgrad : extern
% Sprache: de
\begin{module}

\setdoclanguagegerman
\moduledegreeprogramme{Informatik (B.Sc.)}
\modulesubject{EF Mathematik}
\moduleID{IN3MATHAN04}
\modulename{Funktionentheorie}
\modulecoordination{L. Weis}

\documentdate{2011-10-06 18:22:59.313842}

\modulecredits{9}
\moduleduration{1}
\modulecycle{Jedes 2. Semester, Sommersemester}



\modulehead

% For index (key word@display). Key word is used for sorting - no Umlauts please.
\index{Funktionentheorie@Funktionentheorie (M)}

% For later referencing
\label{mod_3305.dp_997}

\begin{courselist}
1560 & Funktionentheorie (S.~\pageref{cour_8011.dp_997}) & 4/2 & S & 8 & G. Herzog, M. Plum, W. Reichel, C. Schmoeger, R. Schnaubelt, L. Weis\\
\end{courselist}

\begin{styleenv}
\begin{assessment}
Prüfung: schriftliche oder mündliche Prüfung\newline
Notenbildung: Note der Prüfung


\end{assessment}

\begin{conditions}Das Modul \emph{Einfürhung in Geometrie und Topologie }[IN3MATHAG03] muss geprüft werden.

 

Das Modul \emph{Proseminar Mathematik} [IN3MATHPS] muss geprüft werden.

\end{conditions}

\begin{recommendations}Folgende Module sollten bereits belegt worden sein (Empfehlung):\newline
Analysis 1-3

\end{recommendations}
\end{styleenv}

\begin{learningoutcomes}
Einführung in die Hauptsätze der komplexen Analysis


\end{learningoutcomes}

\begin{content}
\begin{itemize}\item Holomorphie  \item Elementare Funktionen  \item Integralsatz und -formel von Cauchy  \item Potenzreihen  \item Satz von Liouville  \item Maximumsprinzip  \item Satz von der Gebietstreue  \item Pole  \item Laurentreihen  \item Residuensatz und reelle Integrale  \item Harmonische Funktionen  \end{itemize}
\end{content}



\end{module}

