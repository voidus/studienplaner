% Lehrveranstaltungsbeschreibung
% Informationsgrad : extern
% Sprache: de
\begin{course}

\setdoclanguagegerman
\coursedegreeprogramme{Informatik}
\coursemodulename{Topics in Finance I (S.~\pageref{mod_1575.dp_997})[IN3WWBWL13]}
\courseID{2530299}
\coursename{Geschäftspolitik der Kreditinstitute}
\coursecoordination{W. Müller}

\documentdate{2012-01-09 19:08:23.224274}

\courselevel{3}
\coursecredits{3}
\courseterm{Wintersemester}
\coursehours{2}
\courseinstructionlanguage{de}

\coursehead

% For index (key word@display). Key word is used for sorting - no Umlauts please.
\index{Geschaeftspolitik der Kreditinstitute@Geschäftspolitik der Kreditinstitute}

% For later referencing
\label{cour_6423.dp_997}


\begin{styleenv}
\begin{assessment}
Die Erfolgskontrolle erfolgt in Form einer schriftlichen Prüfung (60min.) (nach §4(2), 1 SPO)

 

Die Prüfung wird in jedem Semester angeboten und kann zu jedem ordentlichen Prüfungstermin wiederholt werden.


\end{assessment}

\begin{conditions}Keine.\end{conditions}


\end{styleenv}

\begin{learningoutcomes}
Den Studierenden werden grundlegende Kenntnisse des Bankbetriebs vermittelt.


\end{learningoutcomes}

\begin{content}
Der Geschäftsleitung eines Kreditinstituts obliegt es, unter Berücksichtigung aller maßgeblichen endogenen und exogenen Einflussfaktoren, eine Geschäftspolitik festzulegen und zu begleiten, die langfristig den Erfolg der Bankunternehmung sicherstellt. Dabei wird sie zunehmend durch wissenschaftlich fundierte Modelle und Theorien bei der Beschreibung vom Erfolg und Risiko eines Bankbetriebes unterstützt. Die Vorlesung „Geschäftspolitik der Kreditinstitute“ setzt an dieser Stelle an und stellt den Brückenschlag zwischen der bankwirtschaftlichen Theorie und der praktischen Umsetzung her. Dabei nehmen die Vorlesungsteilnehmer die Sichtweise der Unternehmensleitung ein und setzen sich im ersten Kapitel mit der Entwicklung des Bankensektors auseinander. Mit Hilfe geeigneter Annahmen wird dann im zweiten Abschnitt ein Strategiekonzept entwickelt, das in den folgenden Vorlesungsteilen durch die Gestaltung der Bankleistungen (Kap. 3) und des Marketingplans (Kap. 4) weiter untermauert wird. Im operativen Geschäft muss die Unternehmensstrategie durch eine adäquate Ertrags- und Risikosteuerung (Kap. 5 und 6) begleitet werden, die Teile der Gesamtbanksteuerung (Kap. 7) darstellen. Um die Ordnungsmäßigkeit der Geschäftsführung einer Bank sicherzustellen, sind eine Reihe von bankenaufsichtsrechtlichen Anforderungen (Kap. 8) zu beachten, die maßgeblichen Einfluss auf die Gestaltung der Geschäftspolitik haben.


\end{content}



\begin{literature}\textbf{Weiterführende Literatur:}

 \begin{itemize}\item Ein Skript wird im Verlauf der Veranstaltung kapitelweise ausgeteilt.  \item Hartmann-Wendels, Thomas; Pfingsten, Andreas; Weber, Martin; 2000, Bankbetriebslehre, 2. Auflage, Springer  \end{itemize}\end{literature}



\end{course}