% Lehrveranstaltungsbeschreibung
% Informationsgrad : extern
% Sprache: de
\begin{course}

\setdoclanguagegerman
\coursedegreeprogramme{Informatik}
\coursemodulename{Telematik (S.~\pageref{mod_2503.dp_997})[IN3INTM]}
\courseID{24443}
\coursename{Praxis der Telematik}
\coursecoordination{M. Zitterbart}

\documentdate{2011-11-14 11:27:03.563477}

\courselevel{4}
\coursecredits{2}
\courseterm{Wintersemester}
\coursehours{1}
\courseinstructionlanguage{de}

\coursehead

% For index (key word@display). Key word is used for sorting - no Umlauts please.
\index{Praxis der Telematik@Praxis der Telematik}

% For later referencing
\label{cour_8163.dp_997}


\begin{styleenv}
\begin{assessment}
Die Erfolgskontrolle wird in der Modulbeschreibung erläutert.


\end{assessment}

\begin{conditions}Die LV \emph{Praxis der Telematik} [24443] muss im gleichen Semester besucht werden wie die zugehörige Vorlesung \emph{Telematik} [24128].

\end{conditions}


\end{styleenv}

\begin{learningoutcomes}
In dieser Veranstaltung sollen die Teilnehmer ausgewählte Protokolle, Architekturen, sowie Verfahren und Algorithmen, welche in der Vorlesung Telematik behandelt werden, in der Praxis kennenlernen. Ziel ist es, die dort erlernten Konzepte durch ihre Anwendung in der Übung oder im semesterbegleitenden Projekt zu verinnerlichen.


\end{learningoutcomes}

\begin{content}
Die Veranstaltung behandelt Protokolle, Architekturen, sowie Verfahren und Algorithmen, die u.a. im Internet für die Wegewahl und für das Zustandekommen einer zuverlässigen Ende-zu-Ende-Verbindung zum Einsatz kommen. Neben verschiedenen Medienzuteilungsverfahren in lokalen Netzen werden auch weitere Kommunikationssysteme, wie z.B. das leitungsvermittelte ISDN behandelt. Die Teilnehmer sollten ebenfalls verstanden haben, welche Möglichkeiten zur Verwaltung und Administration von Netzen zur Verfügung stehen.


\end{content}

\begin{media}Übungsblätter

\end{media}

\begin{literature}S. Keshav. \emph{An Engineering Approach to Computer Networking}. Addison-Wesley, 1997\newline
J.F. Kurose, K.W. Ross. \emph{Computer Networking: A Top-Down Approach Featuring the Internet}. 4rd Edition, Addison-Wesley, 2007\newline
W. Stallings. \emph{Data and Computer Communications}. 8th Edition, Prentice Hall, 2006

 

\textbf{Weiterführende Literatur:}

 \begin{itemize}\item D. Bertsekas, R. Gallager. \emph{Data Networks}. 2nd Edition, Prentice-Hall, 1991  \item F. Halsall. \emph{Data Communications, Computer Networks and Open Systems}. 4th Edition, Addison-Wesley Publishing Company, 1996  \item W. Haaß. \emph{Handbuch der Kommunikationsnetze}. Springer, 1997  \item A.S. Tanenbaum. \emph{Computer-Networks}. 4th Edition, Prentice-Hall, 2004  \item Internet-Standards  \item Artikel in Fachzeitschriften  \end{itemize}\end{literature}



\end{course}