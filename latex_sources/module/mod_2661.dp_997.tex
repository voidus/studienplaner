% Modulbeschreibung 
% Informationsgrad : extern
% Sprache: de
\begin{module}

\setdoclanguagegerman
\moduledegreeprogramme{Informatik (B.Sc.)}
\modulesubject{EF Recht}
\moduleID{IN3INJUR3}
\modulename{Verfassungs- und Verwaltungsrecht}
\modulecoordination{I. Spiecker genannt Döhmann}

\documentdate{2012-01-19 11:30:12.171088}

\modulecredits{6}
\moduleduration{2}
\modulecycle{Jedes 2. Semester, Wintersemester}



\modulehead

% For index (key word@display). Key word is used for sorting - no Umlauts please.
\index{Verfassungs- und Verwaltungsrecht@Verfassungs- und Verwaltungsrecht (M)}

% For later referencing
\label{mod_2661.dp_997}

\begin{courselist}
24016 & Öffentliches Recht I - Grundlagen (S.~\pageref{cour_4391.dp_997}) & 2/0 & W & 3 & I. Spiecker genannt Döhmann\\
24520 & Öffentliches Recht II - Öffentliches Wirtschaftsrecht (S.~\pageref{cour_4395.dp_997}) & 2/0 & S & 3 & I. Spiecker genannt Döhmann\\
\end{courselist}

\begin{styleenv}
\begin{assessment}
Die Erfolgskontrolle erfolgt in Form von schriftlichen Prüfungen im Umfang von i.d.R. je 60 Minuten nach § 4 Abs. 2 Nr. 1 SPO zu jeder Lehrveranstaltung.

 

Die Gesamtnote des Moduls wird aus den mit Leistungspunkten gewichteten Teilnoten der einzelnen Lehrveranstaltungen gebildet und nach der ersten Kommastelle abgeschnitten.

 

Es bestehlt die Möglichkeit beide Klausuren an einem Termin zu schreiben.


\end{assessment}

\begin{conditions}Keine.\end{conditions}

\begin{recommendations}\begin{itemize}\item Parallel zu den Veranstaltungen werden begleitende Tutorien angeboten, die insbesondere der Vertiefung der juristischen Arbeitsweise dienen. Ihr Besuch wird nachdrücklich empfohlen.  \item Während des Semesters wird eine Probeklausur zu jeder Vorlesung mit ausführlicher Besprechung gestellt. Außerdem wird eine Vorbereitungsstunde auf die Klausuren in der vorlesungsfreien Zeit angeboten.  \item Details dazu auf der Homepage des ZAR (www.kit.edu/zar).  \item Die Lehrveranstaltung \emph{Öffentliches Recht I} [24016]sollte vor der Lehrveranstaltung \emph{Öffentliches Recht II} [24520] besucht werden.  \end{itemize}\end{recommendations}
\end{styleenv}

\begin{learningoutcomes}
Der/die Studierende

 \begin{itemize}\item ordnet Probleme im öffentlichen Recht ein und löst einfache Fälle mit Bezug zum öffentlichen Recht,  \item bearbeitet einen aktuellen Fall aufbautechnisch,  \item zieht Vergleiche zwischen verschiedenen Rechtsproblemen im Öffentlichen Recht,  \item kennt die methodischen Grundlagen des Öffentlichen Rechts,  \item kennt den Unterschied zwischen Privatrecht und dem öffentlichem Recht,  \item kennt die Rechtsschutzmöglichkeiten mit Blick auf das behördliche Handeln,  \item kann mit verfassungsrechtlichen und spezialgesetzlichen Rechtsnormen umgehen.  \end{itemize}
\end{learningoutcomes}

\begin{content}
Das Modul umfasst die Kernaspekte des Verfassungsrechts (Staatsorganisationsrecht und Grundrechte), des Verwaltungsrechts und des öffentlichen Wirtschaftsrechts. Die Vorlesungen vermitteln die Grundlagen des öffentlichen Rechts. Die Studierenden sollen die staatsorganisationsrechtlichen Grundlagen, die Grundrechte, die das staatliche Handeln und das gesamte Rechtssystem steuern, sowie die Handlungsmöglichkeiten und -formen (insb. Gesetz, Verwaltungsakt, Öff.-rechtl. Vertrag) der öffentlichen Hand kennen lernen. Besonderer Wert wird dabei auf eine systematische Erarbeitung des Stoffs sowie eine Vernetzung der einzelnen Aspekte zu einem systemstringenten Ganzen gelegt. Studenten sollen daher auch methodisch sicher das öffentliche Recht bearbeiten lernen. Daher steht neben der Vermittlung materiell-rechtlicher Inhalte (wie z.B. Inhalte von Staatsprinzipien wie Demokratie- und Rechtsstaatsprinzip, Schutzgehalt der einzelnen Grundrechte, Bedingungen der Rechtmäßigkeit von Verwaltungsakten) immer wieder auch die Einübung von Aufbau, Auslegung, und allgemeiner Herangehensweise an Fälle im Öffentlichen Recht.


\end{content}



\end{module}

