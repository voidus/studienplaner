% Lehrveranstaltungsbeschreibung
% Informationsgrad : extern
% Sprache: de
\begin{course}

\setdoclanguagegerman
\coursedegreeprogramme{Informatik}
\coursemodulename{Web-Anwendungen und Praxis (S.~\pageref{mod_14535.dp_997})[IN3INWAP]}
\courseID{24312}
\coursename{Praktikum Web-Anwendungen und Serviceorientierte Architekturen (I)}
\coursecoordination{S. Abeck}

\documentdate{2011-08-01 12:32:42.055840}

\courselevel{3}
\coursecredits{5}
\courseterm{Wintersemester}
\coursehours{2/0}
\courseinstructionlanguage{de}

\coursehead

% For index (key word@display). Key word is used for sorting - no Umlauts please.
\index{Praktikum Web-Anwendungen und Serviceorientierte Architekturen (I)@Praktikum Web-Anwendungen und Serviceorientierte Architekturen (I)}

% For later referencing
\label{cour_14541.dp_997}


\begin{styleenv}
\begin{assessment}
Die Erfolgskontrolle erfolgt durch Ausarbeiten einer schriftlichen Ergebnisdokumentation sowie der Präsentation derselbigen als Erfolgskontrolle anderer Art nach § 4 Abs. 2 Nr. 3 SPO.


\end{assessment}

\begin{conditions}Teilnahme an der Vorlesung \textbf{\emph{Web-Anwendungen und Serviceorientierte Architekturen (I)}}.

\end{conditions}


\end{styleenv}

\begin{learningoutcomes}
\begin{itemize}\item Die wichtigsten den Stand der Technik repräsentierenden Technologien und Standards zur Entwicklung von traditionellen Web-Anwendungen können genutzt werden.  \item Die Technologien und Werkzeuge können zur Entwicklung von Beispielszenarien angewendet werden.  \end{itemize}
\end{learningoutcomes}

\begin{content}
Die Grundlage des Praktikums bilden die Praktischen Aufgaben, die begleitend zu den in der Vorlesung “Web-Anwendungen und Serviceorientierte Architekturen (I)” behandelten Konzepten gestellt werden. Neben der Lösung der Praktischen Aufgaben ist von dem Studierenden eine individuelle Aufgabe zu bearbeiten.


\end{content}

\begin{media}Vorlagen zur effizienten Ergebnisdokumentation (z.B. Projektdokumente, Präsentationsmaterial).

\end{media}

\begin{literature}\begin{itemize}\item Anleitung der Forschungsgruppe zur Durchführung von Arbeiten im Projektteam  \item Vorlesungsskript \textbf{\emph{Web-Anwendungen und Serviceorientierte Architekturen}\emph{ (I)}}  \end{itemize}\end{literature}



\end{course}