% Lehrveranstaltungsbeschreibung
% Informationsgrad : extern
% Sprache: de
\begin{course}

\setdoclanguagegerman
\coursedegreeprogramme{Informatik}
\coursemodulename{Grundlagen der VWL (S.~\pageref{mod_3035.dp_997})[IN3WWVWL]}
\courseID{2600012}
\coursename{Volkswirtschaftslehre I: Mikroökonomie}
\coursecoordination{G. Liedtke}

\documentdate{2011-12-21 08:43:11.021930}

\courselevel{1}
\coursecredits{6}
\courseterm{Wintersemester}
\coursehours{3/0/2}
\courseinstructionlanguage{de}

\coursehead

% For index (key word@display). Key word is used for sorting - no Umlauts please.
\index{Volkswirtschaftslehre I: Mikrooekonomie@Volkswirtschaftslehre I: Mikroökonomie}

% For later referencing
\label{cour_4329.dp_997}


\begin{styleenv}
\begin{assessment}
Die Erfolgskontrolle erfolgt in Form einer schriftlichen Prüfung (120 min) nach § 4 Abs. 2 Nr. 1 SPO.


\end{assessment}

\begin{conditions}Keine.\end{conditions}


\end{styleenv}

\begin{learningoutcomes}
Hauptziel der Veranstaltung ist die Vermittlung der Grundlagen des Denkens in ökonomischen Modellen. Speziell soll der Hörer dieser Veranstaltung in die Lage versetzt werden, Güter-Märkte und die Determinanten von Markt-Ergebnissen zu analysieren. Im einzelnen sollen die Studenten lernen,

 \begin{itemize}\item einfache mikroökonomische Begriffe anzuwenden,  \item die ökonomische Struktur von realen Phänomenen zu erkennen und  \item die Wirkungen von wirtschaftspolitischen Massnahmen auf das Verhalten von Marktteilnehmern (in einfachen ökonomischen Entscheidungssituationen) zu beurteilen und  \item evtl. Alternativmassnahmen vorzuschlagen,  \item als Besucher eines Tutoriums einfache ökonomische Zusammenhänge anhand der Bearbeitung von Übungsaufgaben zu erläutern und durch eigene Diskussionsbeiträge zum Lernerfolg der Tutoriums-Gruppe beizutragen,  \item terminliche Verpflichtungen durch Abgabe von Übungsaufgaben wahrzunehmen,  \item mit der mikroökonomischen Basisliteratur umzugehen.  \end{itemize}

Damit soll der Student Grundlagenwissen erwerben, um in der Praxis

 \begin{itemize}\item die Struktur ökonomischer Probleme auf mikroökonomischer Ebene zu erkennen und Lösungsvorschläge dafür zu präsentieren,  \item aktive Entscheidungsunterstützung für einfache ökonomische Entscheidungsprobleme zu leisten.  \end{itemize}
\end{learningoutcomes}

\begin{content}
Dieser Kurs vermittelt fundierte Grundlagenkenntnisse in Mikroökonomischer Theorie. Neben Haushalts- und Firmenentscheidungen werden auch Probleme des Allgemeinen Gleichgewichts auf Güter- und Arbeitsmärkten behandelt. Der Hörer der Vorlesung soll schließlich auch in die Lage versetzt werden, grundlegende spieltheoretische Argumentationsweisen, wie sie sich in der modernen VWL durchgesetzt haben, zu verstehen.

 

In den beiden Hauptteilen der Vorlesung werden Fragen der mikroökonomischen Entscheidungstheorie (Haushalts- und Firmenentscheidungen) sowie Fragen der Markttheorie (Gleichgewichte und Effizienz auf Konkurrenz-Märkten) behandelt. Im letzten Teil der Vorlesung werden Probleme des unvollständigen Wettbewerbs (Oligopolmärkte) sowie Grundzüge der Spieltheorie vermittelt.


\end{content}

\begin{media}Vorlesungsunterlagen können vom Webserver heruntergeladen werden.

\end{media}

\begin{literature}\begin{itemize}\item H. Varian, Grundzüge der Mikroökonomik, 5. Auflage (2001), Oldenburg Verlag  \item Pindyck, Robert S./Rubinfeld, Daniel L., Mikroökonomie, 6. Aufl., Pearson. Münschen, 2005  \item Frank, Robert H., Microeconomics and Behavior, 5. Aufl., McGraw-Hill, New York, 2005  \end{itemize}

\textbf{Weiterführende Literatur:}

 \begin{itemize}\item Erweiterte Literaturangaben für Interessierte: Detaillierte Artikel mit Beweisen, Algorithmen ..., Übersichtswerke zum State-of-the-Art, Fachzeitschriften (Praxis) und wissenschaftliche Zeitschriften zu aktuellen Entwicklungen.  \item Tutorien/einfachere Einführungsbücher um etwa fehlende Voraussetzungen nachholen zu können.  \end{itemize}\end{literature}



\end{course}