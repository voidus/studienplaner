% Lehrveranstaltungsbeschreibung
% Informationsgrad : extern
% Sprache: de
\begin{course}

\setdoclanguagegerman
\coursedegreeprogramme{Informatik}
\coursemodulename{Kognitive Systeme (S.~\pageref{mod_2511.dp_997})[IN3INKS]}
\courseID{24572}
\coursename{Kognitive Systeme}
\coursecoordination{R. Dillmann, A. Waibel, Christian Mohr, Markus Przybylski, Kai Welke}

\documentdate{2012-01-18 08:49:56.142563}

\courselevel{4}
\coursecredits{6}
\courseterm{Sommersemester}
\coursehours{3/1}
\courseinstructionlanguage{de}

\coursehead

% For index (key word@display). Key word is used for sorting - no Umlauts please.
\index{Kognitive Systeme@Kognitive Systeme}

% For later referencing
\label{cour_7113.dp_997}


\begin{styleenv}
\begin{assessment}
Die Erfolgskontrolle wird in der Modulbeschreibung erläutert.


\end{assessment}

\begin{conditions}Keine.\end{conditions}

\begin{recommendations}Grundwissen in Informatik ist hilfreich.

\end{recommendations}
\end{styleenv}

\begin{learningoutcomes}
\begin{itemize}\item Die relevanten Elemente des technischen kognitiven Systems können benannt und deren Aufgaben beschrieben werden.  \item Die Problemstellungen dieser verschiedenen Bereiche können erkannt und bearbeitet werden.  \item Weiterführende Verfahren können selbständig erschlossen und erfolgreich bearbeitet werden.  \item Variationen der Problemstellung können erfolgreich gelöst werden.  \item Die Lernziele sollen mit dem Besuch der zugehörigen Übung erreicht sein.  \end{itemize}
\end{learningoutcomes}

\begin{content}
Kognitive Systeme handeln aus der Erkenntnis heraus. Nach der Reizaufnahme durch Perzeptoren werden die Signale verarbeitet und aufgrund einer hinterlegten Wissensbasis gehandelt. In der Vorlesung werden die einzelnen Module eines kognitiven Systems vorgestellt. Hierzu gehören neben der Aufnahme und Verarbeitung von Umweltinformationen (z. B. Bilder, Sprache), die Repräsentation des Wissens sowie die Zuordnung einzelner Merkmale mit Hilfe von Klassifikatoren. Weitere Schwerpunkte der Vorlesung sind Lern- und Planungsmethoden und deren Umsetzung. In den Übungen werden die vorgestellten Methoden durch Aufgaben vertieft.


\end{content}

\begin{media}Vorlesungsfolien, Skriptum (wird zum Download angeboten)

\end{media}

\begin{literature}„Artificial Intelligence – A Modern Approach“, Russel, S.; Norvig, P.; Prentice Hall. ISBN 3895761656.

 

\textbf{Weiterführende Literatur:}

 

„Computer Vision – Das Praxisbuch“, Azad, P.; Gockel, T.; Dillmann, R.; Elektor-Verlag. ISBN 0131038052.

 

“Discrete-Time Signal Processing”, Oppenheim, Alan V.; Schafer, Roland W.; Buck, John R.; Pearson US Imports \& PHIPEs. ISBN 0130834432.\newline
“Signale und Systeme”, Kiencke, Uwe; Jäkel, Holger; Oldenbourg, ISBN 3486578111.

\end{literature}



\end{course}