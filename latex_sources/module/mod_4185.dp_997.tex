% Modulbeschreibung 
% Informationsgrad : extern
% Sprache: de
\begin{module}

\setdoclanguagegerman
\moduledegreeprogramme{Informatik (B.Sc.)}
\modulesubject{EF Recht}
\moduleID{IN3JURASEM}
\modulename{Seminarmodul Recht}
\modulecoordination{T. Dreier}

\documentdate{2010-10-06 10:49:29.847851}

\modulecredits{2}
\moduleduration{1}
\modulecycle{Jedes Semester}



\modulehead

% For index (key word@display). Key word is used for sorting - no Umlauts please.
\index{Seminarmodul Recht@Seminarmodul Recht (M)}

% For later referencing
\label{mod_4185.dp_997}

\begin{courselist}
rechtsem & Seminar aus Rechtswissenschaften (S.~\pageref{cour_4421.dp_997}) & 2 & W/S & 2 & T. Dreier, P. Sester, I. Spiecker genannt Döhmann\\
AFDsem & Seminar: Aktuelle Fragen des Datenschutzrechts (S.~\pageref{cour_8501.dp_997}) & 2 & S & 2 & I. Spiecker genannt Döhmann\\
\end{courselist}

\begin{styleenv}
\begin{assessment}
Die Erfolgskontrolle wird in der LV-Beschreibung erläutert.


\end{assessment}

\begin{conditions}Keine.\end{conditions}


\end{styleenv}

\begin{learningoutcomes}
Der/die Studierende

 \begin{itemize}\item setzt sich mit einem abgegrenzten Problem im Bereich der Rechtswissenschaften auseinander,   \item analysiert und diskutiert Problemstellungen im Rahmen der Veranstaltungen und in den abschließenden Seminararbeiten,  \item erörtert, präsentiert und verteidigt fachspezifische Argumente innerhalb einer vorgegebenen Aufgabenstellung,  \item organisiert die Erarbeitung der abschließenden Seminararbeiten weitestgehend selbstständig.   \end{itemize}

Die im Rahmen des Seminarmoduls erworben Kompetenzen dienen im Besonderen der Vorbereitung auf die Bachelorarbeit. Begleitet durch die entsprechenden Prüfer übt sich der Studierende beim Verfassen der abschließenden Seminararbeiten und bei der Präsentation derselben im selbstständigen wissenschaftlichen Arbeiten.


\end{learningoutcomes}

\begin{content}
Das Modul besteht aus einem Seminar, das thematisch den Rechtswissenschaften zuzuordnen ist. Eine Liste der zugelassenen Lehrveranstaltungen wird im Internet bekannt gegeben.


\end{content}

\begin{remarks}Die im Modulhandbuch aufgeführten Seminartitel sind als Platzhalter zu verstehen. Die für jedes Semester aktuell angebotenen Seminare werden jeweils im Vorlesungsverzeichnis und auf den Internetseiten der Institute bekannt gegeben. In der Regel werden die aktuellen Seminarthemen eines jeden Semesters bereits zum Ende des vorangehenden Semesters bekannt gegeben. Bei der Planung des Seminarmoduls sollte darauf geachtet werden, dass für manche Seminare eine Anmeldung bereits zum Ende des vorangehenden Semesters erorderlich ist.

\end{remarks}

\end{module}

