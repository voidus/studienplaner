% Modulbeschreibung 
% Informationsgrad : extern
% Sprache: de
\begin{module}

\setdoclanguagegerman
\moduledegreeprogramme{Informatik (B.Sc.)}
\modulesubject{}
\moduleID{IN3INBATHESIS}
\modulename{Bachelorarbeit}
\modulecoordination{B. Beckert}

\documentdate{2011-05-23 13:54:28.610844}

\modulecredits{15}
\moduleduration{1}
\modulecycle{Jedes Semester}



\modulehead

% For index (key word@display). Key word is used for sorting - no Umlauts please.
\index{Bachelorarbeit@Bachelorarbeit (M)}

% For later referencing
\label{mod_2361.dp_997}



\begin{styleenv}
\begin{assessment}
Die Bachelorarbeit ist in § 11 der SPO geregelt. Die Bewertung der Bachelorarbeit erfolgt nach § 11 Abs. 7 SPO von einem Betreuer sowie in der Regel von einem weiteren Prüfer.


\end{assessment}

\begin{conditions}Voraussetzung für die Zulassung zur Bachelorarbeit ist, dass die Studierenden sich in der Regel im 3. Studienjahr befinden und nicht mehr als eines der Pflichtmodule, welche der Studienplan für die ersten beiden Studienjahre vorsieht, noch nicht bestanden wurde. Der Antrag auf Zulassung zur Bachelorarbeit ist spätestens drei Monate nach Ablegung der letzten Modulprüfung zu stellen.

\end{conditions}


\end{styleenv}

\begin{learningoutcomes}
\begin{itemize}\item In der Bachelorarbeit bearbeiten die Studierenden selbständig ein Thema der Informatik wissenschaftlich.  \item Für ihr Problem führen sie eine Literaturrecherche nach wissenschaftlichen Quellen durch.  \item Die Studierenden wählen dazu geeignete wissenschaftliche Verfahren und Methoden aus und setzen sie ein. Wenn notwenig, passen sie sie an bzw. entwickeln sie.  \item Die Studierenden vergleichen ihre Ergebnisse kritisch mit dem Stand der Forschung und evaluieren sie.  \item Die Studierenden kommunizieren ihre Ergebnisse klar und in akademisch angemessener Form in ihrer Arbeit.  \end{itemize}
\end{learningoutcomes}

\begin{content}
\begin{itemize}\item Die Bachelorarbeit ist eine schriftliche Arbeit, die zeigt, dass die Studierenden selbständig in der Lage sind, ein Problem aus ihrem Fach wissenschaftlich zu bearbeiten.  \item Die Bachelorarbeit soll in höchstens 450 Stunden bearbeitet werden. Die empfohlene Bearbeitungsdauer beträgt 4 Monate, die maximale Bearbeitungsdauer, einschließlich einer Verlängerung, beträgt 5 Monate. Die Arbeit kann im Einvernehmen mit dem Betreuer auch auf Englisch geschrieben werden.  \item Soll die Bachelorarbeit außerhalb der Fakultät angefertigt werden, bedarf dies der Zustimmung des Prüfungsausschusses.  \item Die Bachelorarbeit kann auch in Form einer Gruppenarbeit zugelassen werden, wenn der als Prüfungsleistung zu bewertende Beitrag des einzelnen Studierenden deutlich unterscheidbar ist.  \item Bei Abgabe der Bachelorarbeit haben die Studierenden schriftlich zu versichern, dass sie die Arbeit selbständig verfasst haben und keine anderen, als die von ihnen angegebenen Quellen und Hilfsmittel benutzt haben. Die wörtlich oder inhaltlich übernommenen Stellen als solche kenntlich gemacht und die Satzung des Karlsruher Institut für Technologie (KIT) zur Sicherung guter wissenschaftlicher Praxis in der jeweils gültigen Fassung beachtet haben.  \item Zeitpunkt der Ausgabe des Themas der Bachelorarbeit und der Zeitpunkt der Abgabe der Bachelorarbeit sind aktenkundig zu machen.  \end{itemize}
\end{content}



\end{module}

