% Lehrveranstaltungsbeschreibung
% Informationsgrad : extern
% Sprache: de
\begin{course}

\setdoclanguagegerman
\coursedegreeprogramme{Informatik}
\coursemodulename{Heterogene parallele Rechensysteme (S.~\pageref{mod_4075.dp_997})[IN3INHPRS]}
\courseID{24117}
\coursename{Heterogene parallele Rechensysteme}
\coursecoordination{W. Karl}

\documentdate{2011-02-17 10:45:05.899841}

\courselevel{4}
\coursecredits{3}
\courseterm{Wintersemester}
\coursehours{2}
\courseinstructionlanguage{de}

\coursehead

% For index (key word@display). Key word is used for sorting - no Umlauts please.
\index{Heterogene parallele Rechensysteme@Heterogene parallele Rechensysteme}

% For later referencing
\label{cour_7173.dp_997}


\begin{styleenv}
\begin{assessment}
Die Erfolgskontrolle wird in der Modulbeschreibung näher erläutert.


\end{assessment}

\begin{conditions}Der erfolgreiche Abschluss des Stammmoduls \emph{Rechnerstrukturen} wird vorausgesetzt.

\end{conditions}


\end{styleenv}

\begin{learningoutcomes}
- Die Studierenden sollen vertiefende Kenntnisse über die Architektur und die Operationsprinzipien von parallelen, heterogenen und verteilten Rechnerstrukturen erwerben.\newline
- Sie sollen die Fähigkeit erwerben, parallele Programmierkonzepte und Werkzeuge zur Analyse paralleler Programme anzuwenden.\newline
- Sie sollen die Fähigkeit erwerben, anwendungsspezifische und rekonfigurierbare Komponenten einzusetzen.\newline
- Sie sollen in die Lage versetzt werden, weitergehende Architekturkonzepte und Werkzeuge für parallele Rechnerstrukturen entwerfen zu können.


\end{learningoutcomes}

\begin{content}
Moderne Rechnerstrukturen nützen den Parallelismus in Programmen auf allen Systemebenen aus. Darüber hinaus werden anwendungsspezifische Koprozessoren und rekonfigurierbare Bausteine zur Anwendungsbeschleunigung eingesetzt. Aufbauend auf den in der Lehrveranstaltung Rechnerstrukturen vermittelten Grundlagen, werden die Architektur und Operationsprinzipien paralleler und heterogener Rechnerstrukturen vertiefend behandelt. Es werden die parallelen Programmierkonzepte sowie die Werkzeuge zur Erstellung effizienter paralleler Programme vermittelt. Es werden die Konzepte und der Einsatz anwendungsspezifischer Komponenten (Koprozessorkonzepte) und rekonfigurierbarer Komponenten vermittelt. Ein weiteres Themengebiet ist Grid-Computing und Konzepte zur Virtualisierung.


\end{content}

\begin{media}Vorlesungsfolien

\end{media}





\end{course}