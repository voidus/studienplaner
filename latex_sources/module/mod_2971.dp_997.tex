% Modulbeschreibung 
% Informationsgrad : extern
% Sprache: de
\begin{module}

\setdoclanguagegerman
\moduledegreeprogramme{Informatik (B.Sc.)}
\modulesubject{Praktische Informatik}
\moduleID{IN3INPROGP}
\modulename{Programmierparadigmen}
\modulecoordination{G. Snelting}

\documentdate{2011-12-29 11:41:47.795296}

\modulecredits{6}
\moduleduration{1}
\modulecycle{Jedes 2. Semester, Wintersemester}



\modulehead

% For index (key word@display). Key word is used for sorting - no Umlauts please.
\index{Programmierparadigmen@Programmierparadigmen (M)}

% For later referencing
\label{mod_2971.dp_997}

\begin{courselist}
24030 & Programmierparadigmen (S.~\pageref{cour_7363.dp_997}) & 3/1 & W & 6 & G. Snelting, R. Reussner\\
\end{courselist}

\begin{styleenv}
\begin{assessment}
Die Erfolgskontrolle erfolgt in Form einer schriftlichen Prüfung im Umfang von 120 Minuten nach § 4 Abs. 2 Nr. 1 der SPO.

 

Modulnote ist die Note der schriftlichen Prüfung.


\end{assessment}

\begin{conditions}Erfolgreicher Abschluss des Moduls \emph{Softwaretechnik I} [IN1INSWT1].

\end{conditions}


\end{styleenv}

\begin{learningoutcomes}
Der/Die Studierenden erlernen

 \begin{itemize}\item Grundlagen und Anwendung von funktionaler Programmierung, Logischer Programmierung, Parallelprogrammierung;  \item elementare Grundlagen des Übersetzerbaus.  \end{itemize}
\end{learningoutcomes}

\begin{content}
Die Teilnehmer sollen nichtimperative Programmierung und ihre Anwendungsgebiete kennenlernen. Im einzelnen werden behandelt:

 \begin{enumerate}\item Funktionale Programmierung - rekursive Funktionen und Datentypen, Funktionen höherer Ordnung, Kombinatoren, lazy Evaluation, lambda-Kalkül, Typsysteme, Anwendungsbeispiele.  \item Logische Programmierung - Terme, Hornklauseln, Unifikation, Resolution, regelbasierte Programmierung, constraint logic programming, Anwendungen.  \item Parallelprogrammierung - message passing, verteilte Software, Aktorkonzept, Anwendungsbeispiele.  \item Elementare Grundlagen des Compilerbaus.  \end{enumerate}

Es werden folgende Programmiersprachen (teils nur kurz) vorgestellt: Haskell, Scala, Prolog, CLP, C++, X10, Java Byte Code.


\end{content}



\end{module}

