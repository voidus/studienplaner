% Lehrveranstaltungsbeschreibung
% Informationsgrad : extern
% Sprache: de
\begin{course}

\setdoclanguagegerman
\coursedegreeprogramme{Informatik}
\coursemodulename{Funktionalanalysis (S.~\pageref{mod_3343.dp_997})[IN3MATHAN05]}
\courseID{01048}
\coursename{Funktionalanalysis}
\coursecoordination{G. Herzog, M. Plum, W. Reichel, C. Schmoeger, R. Schnaubelt, L. Weis}

\documentdate{2011-10-06 18:24:40.004449}

\courselevel{}
\coursecredits{8}
\courseterm{Wintersemester}
\coursehours{4/2}
\courseinstructionlanguage{}

\coursehead

% For index (key word@display). Key word is used for sorting - no Umlauts please.
\index{Funktionalanalysis@Funktionalanalysis}

% For later referencing
\label{cour_7763.dp_997}


\begin{styleenv}
\begin{assessment}
Prüfung: schriftliche oder mündliche Prüfung\newline
Notenbildung: Note der Prüfung


\end{assessment}

\begin{conditions}Keine.\end{conditions}

\begin{recommendations}Folgende Module sollten bereits belegt worden sein (Empfehlung):\newline
Lineare Algebra 1+2\newline
Analysis 1-3

\end{recommendations}
\end{styleenv}

\begin{learningoutcomes}
Einführung in funktionalanalytische Konzepte und Denkweisen


\end{learningoutcomes}

\begin{content}
\begin{itemize}\item Metrische Räume (topologische Grundbegriffe, Kompaktheit)  \item Stetige lineare Operatoren auf Banachräumen (Prinzip der gleichmäßigen Beschränktheit, Homomorphiesatz)  \item Dualräume mit Darstellungssätzen, Satz von Hahn-Banach, schwache Konvergenz, Reflexivität  \item Distributionen, schwache Ableitung, Fouriertransformation, Satz von Plancherel, Sobolevräume in L\textsuperscript{2}, partielle Differentialgleichungen mit konstanten Koeffizienten  \end{itemize}
\end{content}







\end{course}