% Lehrveranstaltungsbeschreibung
% Informationsgrad : extern
% Sprache: de
\begin{course}

\setdoclanguagegerman
\coursedegreeprogramme{Informatik}
\coursemodulename{Grundlagen des OR (S.~\pageref{mod_3039.dp_997})[IN3WWOR]}
\courseID{2550040}
\coursename{Einführung in das Operations Research I}
\coursecoordination{S. Nickel, O. Stein, K. Waldmann}

\documentdate{2011-12-01 13:13:26.550751}

\courselevel{2}
\coursecredits{6}
\courseterm{Sommersemester}
\coursehours{2/2/2}
\courseinstructionlanguage{de}

\coursehead

% For index (key word@display). Key word is used for sorting - no Umlauts please.
\index{Einfuehrung in das Operations Research I@Einführung in das Operations Research I}

% For later referencing
\label{cour_4401.dp_997}


\begin{styleenv}
\begin{assessment}
Die Erfolgskontrolle wird in der Modulbeschreibung erläutert.


\end{assessment}

\begin{conditions}Siehe Modulbeschreibung.

\end{conditions}


\end{styleenv}

\begin{learningoutcomes}
Siehe Modulbeschreibung.


\end{learningoutcomes}

\begin{content}
Beispiel für typische OR-Probleme.

 

Lineare Optimierung: Grundbegriffe, Simplexmethode, Dualität, Sonderformen des Simplexverfahrens (duale Simplexmethode, Dreiphasenmethode), Sensitivitätsanalyse, Parametrische Optimierung, Multikriterielle Optimierung.

 

Graphen und Netzwerke: Grundbegriffe der Graphentheorie, kürzeste Wege in Netzwerken, Terminplanung von Projekten, maximale Flüsse in Netzwerken.


\end{content}

\begin{media}Tafel, Folien, Beamer-Präsentationen, Skript, OR-Software

\end{media}

\begin{literature}\begin{itemize}\item Nickel, Stein, Waldmann: Operations Research, Springer, 2011  \item Hillier, Lieberman: Introduction to Operations Research, 8th edition. McGraw-Hill, 2005  \item Murty: Operations Research. Prentice-Hall, 1995  \item Neumann, Morlock: Operations Research, 2. Auflage. Hanser, 2006  \item Winston: Operations Research - Applications and Algorithms, 4th edition. PWS-Kent, 2004  \end{itemize}\end{literature}



\end{course}