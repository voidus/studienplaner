% Modulbeschreibung 
% Informationsgrad : extern
% Sprache: de
\begin{module}

\setdoclanguagegerman
\moduledegreeprogramme{Informatik (B.Sc.)}
\modulesubject{Praktische Informatik}
\moduleID{IN2INBS}
\modulename{Betriebssysteme}
\modulecoordination{F. Bellosa}

\documentdate{2011-02-17 14:34:28.235840}

\modulecredits{6}
\moduleduration{1}
\modulecycle{Jedes 2. Semester, Wintersemester}



\modulehead

% For index (key word@display). Key word is used for sorting - no Umlauts please.
\index{Betriebssysteme@Betriebssysteme (M)}

% For later referencing
\label{mod_2369.dp_997}

\begin{courselist}
24009 & Betriebssysteme (S.~\pageref{cour_6215.dp_997}) & 3/1 & W & 6 & F. Bellosa\\
\end{courselist}

\begin{styleenv}
\begin{assessment}
Die Erfolgskontrolle erfolgt in Form einer schriftlichen Prüfung im Umfang von 60 Minuten nach § 4 Abs. 2 Nr. 1 SPO sowie eines bewerteten Übungsscheines (Erfolgskontrolle anderer Art nach § 4 Abs. 2 Nr. 3 SPO).

 

Besonderheit: Für den Übungsschein können Bonuspunkte erarbeitet werden. Die Bonuspunkte werden zur Notenverbesserung für eine bestandene Prüfung verwendet.


\end{assessment}

\begin{conditions}Kenntnisse in der Programmierung in C/C++ werden vorausgesetzt.

\end{conditions}

\begin{recommendations}Der vorherige erfolgreiche Abschluss von Modul \emph{Programmieren} [IN1INPROG] ist empfohlen.

\end{recommendations}
\end{styleenv}

\begin{learningoutcomes}
Ziel des Moduls ist es, die Studierenden mit den grundlegenden Systemarchitekturen und Betriebssystemkomponenten vertraut zu machen. Sie sollen die Basismechanismen und Strategien von Betriebs- und Laufzeitsystemen kennen.


\end{learningoutcomes}

\begin{content}
Inhalte:

 \begin{itemize}\item System Structures  \item Processes Management  \item Synchronization  \item Memory Management  \item File Systems  \item I/O Management  \end{itemize}
\end{content}



\end{module}

