% Lehrveranstaltungsbeschreibung
% Informationsgrad : extern
% Sprache: de
\begin{course}

\setdoclanguagegerman
\coursedegreeprogramme{Informatik}
\coursemodulename{Energiewirtschaft (S.~\pageref{mod_2871.dp_997})[IN3WWBWL12]}
\courseID{2581010}
\coursename{Einführung in die Energiewirtschaft}
\coursecoordination{W. Fichtner}

\documentdate{2011-12-23 10:05:51.797705}

\courselevel{3}
\coursecredits{5,5}
\courseterm{Sommersemester}
\coursehours{2/2}
\courseinstructionlanguage{de}

\coursehead

% For index (key word@display). Key word is used for sorting - no Umlauts please.
\index{Einfuehrung in die Energiewirtschaft@Einführung in die Energiewirtschaft}

% For later referencing
\label{cour_7409.dp_997}


\begin{styleenv}
\begin{assessment}
Die Erfolgskontrolle erfolgt in Form einer schriftlichen Prüfung (nach §4 (2), 1 SPO).


\end{assessment}

\begin{conditions}Keine.\end{conditions}


\end{styleenv}

\begin{learningoutcomes}
Der/die Studierende

 \begin{itemize}\item kann die verschiedenen Energieträger und deren Eigenheiten charakterisieren und bewerten,  \end{itemize}\begin{itemize}\item ist in der Lage energiewirtschaftliche Zusammenhänge zu verstehen.  \end{itemize}
\end{learningoutcomes}

\begin{content}
\begin{enumerate}\item Einführung: Begriffe, Einheiten, Umrechnungen  \item Der Energieträger Gas (Reserven, Ressourcen, Technologien)  \item Der Energieträger Öl (Reserven, Ressourcen, Technologien)  \item Der Energieträger Steinkohle (Reserven, Ressourcen, Technologien)  \item Der Energieträger Braunkohle (Reserven, Ressourcen, Technologien)  \item Der Energieträger Uran (Reserven, Ressourcen, Technologien)  \item Der Endenergieträger Elektrizität  \item Der Endenergieträger Wärme  \item Sonstige Endenergieträger (Kälte, Wasserstoff, Druckluft)  \end{enumerate}
\end{content}

\begin{media}Medien werden über die Lernplattform ILIAS bereitgestellt.

\end{media}

\begin{literature}\textbf{Weiterführende Literatur:}

 

Pfaffenberger, Wolfgang. Energiewirtschaft. ISBN 3-486-24315-2

 

Feess, Eberhard. Umweltökonomie und Umweltpolitik. ISBN 3-8006-2187-8

 

Müller, Leonhard. Handbuch der Elektrizitätswirtschaft. ISBN 3-540-67637-6

 

Stoft, Steven. Power System Economics. ISBN 0-471-15040-1

 

Erdmann, Georg. Energieökonomik. ISBN 3-7281-2135-5

\end{literature}



\end{course}