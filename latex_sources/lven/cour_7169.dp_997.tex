% Lehrveranstaltungsbeschreibung
% Informationsgrad : extern
% Sprache: de
\begin{course}

\setdoclanguagegerman
\coursedegreeprogramme{Informatik}
\coursemodulename{Mikroprozessoren I (S.~\pageref{mod_4045.dp_997})[IN3INMP1]}
\courseID{24688}
\coursename{Mikroprozessoren I}
\coursecoordination{W. Karl}

\documentdate{2011-03-04 15:16:16.062740}

\courselevel{3}
\coursecredits{3}
\courseterm{Sommersemester}
\coursehours{2}
\courseinstructionlanguage{de}

\coursehead

% For index (key word@display). Key word is used for sorting - no Umlauts please.
\index{Mikroprozessoren I@Mikroprozessoren I}

% For later referencing
\label{cour_7169.dp_997}


\begin{styleenv}
\begin{assessment}
Die Erfolgskontrolle wird in der Modulbeschreibung näher erläutert.


\end{assessment}

\begin{conditions}Keine.\end{conditions}


\end{styleenv}

\begin{learningoutcomes}
Die Studierenden sollen detaillierte Kenntnisse über den Aufbau und die Organisation von Mikroprozessorsystemen in den verschiedenen Einsatzgebieten erwerben. \newline
Die Studierenden sollen die Fähigkeit erwerben, Mikroprozessoren für verschiedene Einsatzgebiete bewerten und auswählen zu können.\newline
Die Studierenden sollen die Fähigkeit erwerben, systemnahe Funktionen programmieren zu können.\newline
Die Studierenden sollen Architekturmerkmale von Mikroprozessoren zur Beschleunigung von Anwendungen und Systemfunktionen ableiten, bewerten und entwerfen können.\newline
Die Studierenden sollen die Fähigkeiten erwerben, Mikroprozessorsysteme in strukturierter und systematischer Weise entwerfen zu können.


\end{learningoutcomes}

\begin{content}
Das Modul befasst sich im ersten Teil mit Mikroprozessoren, die in Desktops und Servern eingesetzt werden. Ausgehend von den grundlegenden Eigenschaften dieser Rechner und dem Systemaufbau werden die Architekturmerkmale von Allzweck- und Hochleistungs-Mikroprozessoren vermittelt. Insbesondere sollen die Techniken und Mechanismen zur Unterstützung von Betriebssystemfunktionen, zur Beschleunigung durch Ausnützen des Parallelismus auf Maschinenbefehlsebene und Aspekte der Speicherhierarchie vermittelt werden.

 

Der zweite Teil behandelt Mikroprozessoren, die in eingebetteten Systemen eingesetzt werden. Es werden die grundlegenden Eigenschaften von Microcontrollern vermittelt. Eigenschaften von Mikroprozessoren, die auf spezielle Einsatzgebiete zugeschnitten sind, werden ausführlich behandelt.


\end{content}

\begin{media}Vorlesungsfolien

\end{media}





\end{course}