% Lehrveranstaltungsbeschreibung
% Informationsgrad : extern
% Sprache: de
\begin{course}

\setdoclanguagegerman
\coursedegreeprogramme{Informatik}
\coursemodulename{Topics in Finance I (S.~\pageref{mod_1575.dp_997})[IN3WWBWL13], eFinance (S.~\pageref{mod_2729.dp_997})[IN3WWBWL15]}
\courseID{2530296}
\coursename{Börsen}
\coursecoordination{J. Franke}

\documentdate{2012-01-09 19:06:58.675602}

\courselevel{3}
\coursecredits{1,5}
\courseterm{Sommersemester}
\coursehours{1}
\courseinstructionlanguage{de}

\coursehead

% For index (key word@display). Key word is used for sorting - no Umlauts please.
\index{Boersen@Börsen}

% For later referencing
\label{cour_6289.dp_997}


\begin{styleenv}
\begin{assessment}
Die Erfolgskontrolle erfolgt in Form einer schriftlichen Prüfung (60min.) (nach §4(2), 1 SPO).

 

Die Prüfung wird in jedem Semester angeboten und kann zu jedem ordentlichen Prüfungstermin wiederholt werden.


\end{assessment}

\begin{conditions}Keine.\end{conditions}


\end{styleenv}

\begin{learningoutcomes}
Den Studierenden werden aktuelle Entwicklungen rund um die Börsenorganisation und den Wertpapierhandel aufgezeigt.


\end{learningoutcomes}

\begin{content}
\begin{itemize}\item Börsenorganisationen - Zeitgeist im Wandel: “Corporates” anstelle von kooperativen Strukturen?  \item Marktmodelle: Order driven contra market maker: Liquiditätsspender als Retter für umsatzschwache Werte?  \item Handelssysteme - Ende einer Ära: Kein Bedarf mehr an rennenden Händlern?  \item Clearing - Vielfalt statt Einheit: Sicherheit für alle?  \item Abwicklung - wachsende Bedeutung: Sichert effizientes Settlement langfristig den “value added“ der Börsen?  \end{itemize}
\end{content}



\begin{literature}\textbf{Weiterführende Literatur:}

 

Lehrmaterial wird in der Vorlesung ausgegeben.

\end{literature}



\end{course}