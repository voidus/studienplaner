% Modulbeschreibung 
% Informationsgrad : extern
% Sprache: de
\begin{module}

\setdoclanguagegerman
\moduledegreeprogramme{Informatik (B.Sc.)}
\modulesubject{}
\moduleID{IN3INSWT2}
\modulename{Softwaretechnik II}
\modulecoordination{R. Reussner, W. Tichy}

\documentdate{2011-08-08 14:56:05.865665}

\modulecredits{6}
\moduleduration{1}
\modulecycle{Jedes 2. Semester, Wintersemester}



\modulehead

% For index (key word@display). Key word is used for sorting - no Umlauts please.
\index{Softwaretechnik II@Softwaretechnik II (M)}

% For later referencing
\label{mod_3881.dp_997}

\begin{courselist}
24076 & Softwaretechnik II (S.~\pageref{cour_7361.dp_997}) & 3/1 & W & 6 & R. Reussner, W. Tichy\\
\end{courselist}

\begin{styleenv}
\begin{assessment}
Die Erfolgskontrolle erfolgt in Form einer schriftlichen Prüfung im Umfang von i.d.R. 60 Minuten nach § 4 Abs. 2 Nr. 1 SPO.

 

Die Modulnote ist die Note der schriftlichen Prüfung.


\end{assessment}

\begin{conditions}Keine.\end{conditions}

\begin{recommendations}Die Lehrveranstaltung \emph{Softwaretechnik I} sollte bereits gehört worden sein.

\end{recommendations}
\end{styleenv}

\begin{learningoutcomes}
Die Studierenden erlernen Vorgehensweisen und Techniken für systematische Softwareentwicklung, indem fortgeschrittene Themen der Softwaretechnik behandelt werden.


\end{learningoutcomes}

\begin{content}
Requirements Engineering, Softwareprozesse, Software-Qualität, Software-Architekturen, MDD, Enterprise Software Patterns

 

Software-Wartbarkeit, Sicherheit, Verläßlichkeit (Dependability), eingebettete Software, Middleware, statistisches Testen


\end{content}

\begin{remarks}Das Modul \emph{Softwaretechnik II} ist ein Stammmodul.

\end{remarks}

\end{module}

