% Lehrveranstaltungsbeschreibung
% Informationsgrad : extern
% Sprache: de
\begin{course}

\setdoclanguagegerman
\coursedegreeprogramme{Informatik}
\coursemodulename{Biomedizinische Technik I (S.~\pageref{mod_4005.dp_997})[IN3EITBIOM]}
\courseID{23269}
\coursename{Biomedizinische Messtechnik I}
\coursecoordination{A. Bolz}

\documentdate{2010-07-15 11:57:50.500952}

\courselevel{3}
\coursecredits{5}
\courseterm{Wintersemester}
\coursehours{3}
\courseinstructionlanguage{de}

\coursehead

% For index (key word@display). Key word is used for sorting - no Umlauts please.
\index{Biomedizinische Messtechnik I@Biomedizinische Messtechnik I}

% For later referencing
\label{cour_8137.dp_997}


\begin{styleenv}
\begin{assessment}
Die Erfolgskontrolle erfolgt in Form einer mündlichen Prüfung im Umfang von i.d.R. 20 Minuten nach § 4 Abs. 2 Nr. 2 SPO.


\end{assessment}

\begin{conditions}Keine.\end{conditions}


\end{styleenv}

\begin{learningoutcomes}

\end{learningoutcomes}

\begin{content}
Herkunft von Biosignalen: Anatomie und Physiologie der Nervenzelle und des Nervensystems, Ruhezustand der Zelle, elektrische Aktivität erregbarer Zellen, Aufnahmetechniken des Ruhe- und des Aktionspotentials.\newline
\newline
 Elektrodentechnologie: Elektroden-Elektrolyt-Grenzfläche, Polarisation, polarisierbare und nicht polarisierbare Elektroden, Elektrodenverhalten und Ersatzschaltbilder, Elektroden-Haut-Grenzfläche.\newline
\newline
 Biosignalverstärker: Differenzverstärker, Biosignalvorverstärker.\newline
\newline
 Störungen: Störungen im Elektrodensystem, äußere Störungen, galvanisch eingekoppelte Störungen, kapazitiv eingekoppelte Störungen, induktiv eingekoppelte Störungen, Messtechniken für elektrische und magnetische Felder, Methoden der Störunterdrückung.\newline
\newline
 Biosignale des Nervenstems und der Muskel: Anatomie und Funktion, Elektroneurogramm (ENG), Elektromyogramm (EMG), Nervenleitgeschwindigkeit, Diagnose, Aufnahmetechniken.\newline
\newline
 Biosignale des Gehirns: Anatomie und Funktion des zentralen Nervensystems, Elektrokortikogramm (ECoG), Elektroenzephalogramm (EEG), Aufnahmetechniken, Diagnose.\newline
\newline
 Elektrokardiogramm (EKG): Anatomie und Funktion des Herzens, ventrikuläre Zellen, ventrikuläre Aktivierung, Körperolächenpotenziale.\newline
\newline
 Elektrische Sicherheit: physiologische Effekte der Elektrizität, elektrische Schläge, elektrische Sicherheitsregeln und -standards, Sicherheitsmaßnahmen, Testen elektrischer Systeme.


\end{content}







\end{course}