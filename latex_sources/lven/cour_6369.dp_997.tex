% Lehrveranstaltungsbeschreibung
% Informationsgrad : extern
% Sprache: de
\begin{course}

\setdoclanguagegerman
\coursedegreeprogramme{Informatik}
\coursemodulename{Insurance Markets and Management (S.~\pageref{mod_1565.dp_997})[IN3WWBWL7]}
\courseID{2530050}
\coursename{Private and Social Insurance}
\coursecoordination{W. Heilmann, K. Besserer}

\documentdate{2012-01-22 17:23:22.097761}

\courselevel{4}
\coursecredits{2,5}
\courseterm{Wintersemester}
\coursehours{2/0}
\courseinstructionlanguage{de}

\coursehead

% For index (key word@display). Key word is used for sorting - no Umlauts please.
\index{Private and Social Insurance@Private and Social Insurance}

% For later referencing
\label{cour_6369.dp_997}


\begin{styleenv}
\begin{assessment}
Die Erfolgskontrolle erfolgt in Form einer schriftlichen Prüfung (nach §4(2), 1 SPO). Die Prüfung wird in jedem Semester angeboten und kann zu jedem ordentlichen Prüfungstermin wiederholt werden.


\end{assessment}

\begin{conditions}Keine.\end{conditions}


\end{styleenv}

\begin{learningoutcomes}
Kennenlernen der Grundbegriffe und der Funktion von Privat- und Sozialversicherung.


\end{learningoutcomes}

\begin{content}
Grundbegriffe des Versicherungswesens, d.h. Wesensmerkmale, rechtliche und politische Grundlagen und Funktionsweise von Individual- und Sozialversicherung sowie deren einzelwirtschaftliche, gesamtwirtschaftliche und sozialpolitische Bedeutung.


\end{content}



\begin{literature}\textbf{Weiterführende Literatur:}

 \begin{itemize}\item F. Büchner, G. Winter. Grundriss der Individualversicherung. 1995.  \item P. Koch. Versicherungswirtschaft. 2005.  \item Jahrbücher des GDV. Die deutsche Versicherungswirtschaft:\newline
http://www.gdv.de/2011/11/jahrbuch-der-deutschen-versicherungswirtschaft-2011/  \end{itemize}\end{literature}

\begin{remarks}Blockveranstaltung, aus organisatorischen Gründung melden Sie sich bitte im Sekretariat des Lehrstuhls an: thomas.mueller3@kit.edu

\end{remarks}

\end{course}