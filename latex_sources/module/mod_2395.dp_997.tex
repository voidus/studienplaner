% Modulbeschreibung 
% Informationsgrad : extern
% Sprache: de
\begin{module}

\setdoclanguagegerman
\moduledegreeprogramme{Informatik (B.Sc.)}
\modulesubject{}
\moduleID{IN1MATHLA}
\modulename{Lineare Algebra}
\modulecoordination{K. Spitzmüller, S. Kühnlein}

\documentdate{2008-06-23 10:36:18}

\modulecredits{14}
\moduleduration{2}
\modulecycle{Jedes 2. Semester, Wintersemester}



\modulehead

% For index (key word@display). Key word is used for sorting - no Umlauts please.
\index{Lineare Algebra@Lineare Algebra (M)}

% For later referencing
\label{mod_2395.dp_997}

\begin{courselist}
01332 & Lineare Algebra  und Analytische Geometrie I für die Fachrichtung Informatik (S.~\pageref{cour_6283.dp_997}) & 4/2/2 & W & 9 & K. Spitzmüller, S. Kühnlein, Hug\\
01870 & Lineare Algebra II für die Fachrichtung Informatik (S.~\pageref{cour_6281.dp_997}) & 2/1/2 & S & 5 & K. Spitzmüller, S. Kühnlein, Hug\\
\end{courselist}

\begin{styleenv}
\begin{assessment}
Die Erfolgskontrolle erfolgt in Form einer schriftlichen Prüfung nach § 4 Abs. 2. Nr. 1 SPO im Umfang von 210 Minuten und eines bestandenen Leistungsnachweises nach § 4 Abs. 2 Nr. 3 SPO aus den Übungsbetrieben zu \emph{Lineare Algebra und Analytische Geometrie I für die Fachrichtung Informatik }[1332] oder \emph{Lineare Algebra II} \emph{für die Fachrichtung Informatik }[1870].

 

Die Modulnote ist die Note der schriftlichen Prüfung.

 

\textbf{Achtung:} Diese Prüfung oder die Prüfung zum Modul \emph{Höhere Mathematik} [IN1MATHHM] oder zum Modul \emph{Analysis} [IN1MATHANA] oder zum Modul \emph{Lineare Algebra und Analytische Geometrie} [IN1MATHLAAG] ist bis zum Ende des 2. Fachsemesters anzutreten und bis zum Ende des 3. Fachsemesters zu bestehen, da sie Bestandteil der Orientierungsprüfung nach § 8 Abs. 1 SPO ist.


\end{assessment}

\begin{conditions}Keine.\end{conditions}


\end{styleenv}

\begin{learningoutcomes}
Die Studierenden sollen am Ende des Moduls

 \begin{itemize}\item den Übergang von der Schule zur Universität bewältigt haben,  \item mit logischem Denken und strengen Beweisen vertraut sei,   \item die Methoden und grundlegenden Strukturen der Linearen Algebra beherrschen.  \end{itemize}
\end{learningoutcomes}

\begin{content}
\begin{itemize}\item Grundbegriffe (Mengen, Abbildungen, Relationen, Gruppen, Ringe, Körper, Matrizen, Polynome)   \item Lineare Gleichungssysteme (Gauß´sches Eliminationsverfahren, Lösungstheorie)   \item Vektorräume (Beispiele, Unterräume, Quotientenräume, Basis und Dimension)   \item Lineare Abbildungen (Kern, Bild, Rang, Homomorphiesatz, Vektorräume von Abbildungen, Dualraum, Darstellungsmatrizen, Basiswechsel)   \item Determinanten   \item Eigenwerttheorie (Eigenwerte, Eigenvektoren, charakteristisches Polynom, Normalformen  \item Vektorräume mit Skalarprodukt (bilineare Abbildungen, Skalarprodukt, Norm, Orthogonalität, adjungierte Abbildung, selbstadjungierte Endomorphismen, Spektralsatz, Isometrien)  \end{itemize}
\end{content}

\begin{remarks}Moduldauer: 2 Semester

\end{remarks}

\end{module}

