% Lehrveranstaltungsbeschreibung
% Informationsgrad : extern
% Sprache: de
\begin{course}

\setdoclanguagegerman
\coursedegreeprogramme{Informatik}
\coursemodulename{Mobilkommunikation (S.~\pageref{mod_2595.dp_997})[IN3INMK]}
\courseID{24643}
\coursename{Mobilkommunikation}
\coursecoordination{O. Waldhorst}

\documentdate{2010-06-07 10:00:00.290226}

\courselevel{4}
\coursecredits{4}
\courseterm{Sommersemester}
\coursehours{2/0}
\courseinstructionlanguage{de}

\coursehead

% For index (key word@display). Key word is used for sorting - no Umlauts please.
\index{Mobilkommunikation@Mobilkommunikation}

% For later referencing
\label{cour_5385.dp_997}


\begin{styleenv}
\begin{assessment}
Die Erfolgskontrolle wird in der Modulbeschreibung erläutert.


\end{assessment}

\begin{conditions}Keine.\end{conditions}

\begin{recommendations}Inhalte der Vorlesungen \emph{Einführung in Rechnernetze} [24519] (oder vergleichbarer Vorlesungen) und \emph{Telematik }[24128].

\end{recommendations}
\end{styleenv}

\begin{learningoutcomes}
Ziel der Vorlesung ist es, die technischen Grundlagen der Mobilkommunikation (Signalausbreitung, Medienzugriff, etc.) zu vermitteln. Zusätzlich werden aktuelle Entwicklungen in der Forschung (Mobile IP, Ad-hoc Netze, Mobile TCP, etc.) betrachtet.


\end{learningoutcomes}

\begin{content}
Die Vorlesung “Mobilkommunikation” erläutert anhand von typischen Beispielen verschiedene Architekturen für typische Mobilkommunikationssysteme, wie z. B. mobile Telekommunikationssysteme, drahtlose lokale, innerstädtische und persönliche Netze. Die Realisierung von TCP/IP-basierter Kommunikation über mobile Netze sowie die Positionsbestimmung mobiler Geräte sind weitere Themen mit aktuellem Forschungsbezug. Dabei ist das Lernziel nicht die Vermittlung von Wissen über einzelne Architekturen und Standards, sondern vielmehr die Beleuchtung grundlegender Problemstellungen und typischer Lösungsansätze. Die notwendigen Grundlagen der digitalen Signalübertragung wie Frequenzbereiche, Signalausbreitung, Modulation und Multiplextechniken werden in kompakter Form und motiviert aus den Anwendungen ebenfalls vermittelt.


\end{content}

\begin{media}Folien.

\end{media}

\begin{literature}J. Schiller; Mobilkommunikation; Addison-Wesley, 2003.

 

\textbf{Weiterführende Literatur:}

 

C. Eklund, R. Marks, K. Stanwood, S. Wang; IEEE Stadard 802.16: A Technical Overview of the WirelessMANTM Air Interface for the Broadband Wireless Access; IEEE Communications Magazine, June 2002.

 

H. Kaaranen, A. Ahtiainen, et. al., UMTS Networks – Architecture, Mobility and Services, Wiley Verlag, 2001.

 

B. O’Hara, A. Petrick, The IEEE 802.11 Handbook – A Designers Companion IEEE, 1999.

 

B. A. Miller, C. Bisdikian, Bluetooth Revealed, Prentice Hall, 2002

 

J. Rech, Wireless LAN – 802.11-WLAN-Technologien und praktische Umsetzung im Detail, Verlag Heinz Heise, 2004.

 

B. Walke, Mobilfunknetze und ihre Protokolle, 3. Auflage, Teubner Verlag, 2001.

 

R. Read, Nachrichten- und Informationstechnik; Pearson Studium 2004.

 

What You Should Know About the ZigBee Alliance http://www.zigbee.org.

 

C. Perkins, Ad-hoc Networking, Addison Wesley, 2000.

 

H. Holma, WCDMA For UMTS, HSPA Evolution and LTE, 2007

\end{literature}



\end{course}