% Modulbeschreibung 
% Informationsgrad : extern
% Sprache: de
\begin{module}

\setdoclanguagegerman
\moduledegreeprogramme{Informatik (B.Sc.)}
\modulesubject{EF Mathematik}
\moduleID{IN3MATHAG04}
\modulename{Riemannsche Geometrie}
\modulecoordination{E. Leuzinger}

\documentdate{2009-04-15 10:34:03}

\modulecredits{9}
\moduleduration{1}
\modulecycle{Jedes 2. Semester, Wintersemester}



\modulehead

% For index (key word@display). Key word is used for sorting - no Umlauts please.
\index{Riemannsche Geometrie@Riemannsche Geometrie (M)}

% For later referencing
\label{mod_3109.dp_997}

\begin{courselist}
1036 & Riemannsche Geometrie (S.~\pageref{cour_7979.dp_997}) & 4/2 & W & 9 & E. Leuzinger\\
\end{courselist}

\begin{styleenv}
\begin{assessment}
Prüfung: schriftliche oder mündliche Prüfung\newline
Notenbildung: Note der Prüfung


\end{assessment}

\begin{conditions}Das Modul \emph{Proseminar Mathematik} [IN3MATHPS] muss geprüft werden.

 

Das Modul \emph{Einführung in die Algebra und Zahlentheorie}, \emph{Einführung in die Geometrie und Topologie} oder \emph{Algebra} muss geprüft werden. 

\end{conditions}

\begin{recommendations}Kenntnisse aus \emph{Einführung in die Geometrie und Topologie} werden vorausgesetzt.

\end{recommendations}
\end{styleenv}

\begin{learningoutcomes}
Einführung in die Konzepte der Riemannschen Geometrie


\end{learningoutcomes}

\begin{content}
\begin{itemize}\item Mannigfaltigkeiten  \item Riemannsche Metriken  \item Affine Zusammenhänge  \item Geodätische  \item Krümmung  \item Jacobi-Felder  \item Längen-Metrik  \item Krümmung und Topologie  \end{itemize}
\end{content}



\end{module}

