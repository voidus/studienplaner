% Lehrveranstaltungsbeschreibung
% Informationsgrad : extern
% Sprache: de
\begin{course}

\setdoclanguagegerman
\coursedegreeprogramme{Informatik}
\coursemodulename{Betriebssysteme (S.~\pageref{mod_2369.dp_997})[IN2INBS]}
\courseID{24009}
\coursename{Betriebssysteme}
\coursecoordination{F. Bellosa}

\documentdate{2011-02-17 14:19:46.721166}

\courselevel{2}
\coursecredits{6}
\courseterm{Wintersemester}
\coursehours{3/1}
\courseinstructionlanguage{de}

\coursehead

% For index (key word@display). Key word is used for sorting - no Umlauts please.
\index{Betriebssysteme@Betriebssysteme}

% For later referencing
\label{cour_6215.dp_997}


\begin{styleenv}
\begin{assessment}
Die Erfolgskontrolle wird in der Modulbeschreibung erläutert.


\end{assessment}

\begin{conditions}Kenntnisse in der Programmierung in C/C++ werden vorausgesetzt.

\end{conditions}

\begin{recommendations}Der vorherige erfolgreiche Abschluss vom Modul \emph{Programmieren} [IN1INPROG] wird empfohlen.

\end{recommendations}
\end{styleenv}

\begin{learningoutcomes}
Ziel der Vorlesung ist es, die Studierenden mit den grundlegenden Systemarchitekturen und Betriebssystemkomponenten vertraut zu machen. Sie sollen die Basismechanismen und Strategien von Betriebs- und Laufzeitsystemen kennen.


\end{learningoutcomes}

\begin{content}
Inhalte:

 \begin{itemize}\item System Structures  \item Processes Management  \item Synchronization  \item Memory Management  \item File Systems  \item I/O Management  \end{itemize}
\end{content}

\begin{media}Vorlesungsfolien in englischer Sprache.

\end{media}

\begin{literature}\textbf{Operating System Concepts} von Abraham Silberschatz, 8th Edition

 

\textbf{Weiterführende Literatur:}

 

\textbf{Modern Operating Systems }von Andrew S. Tanenbaum, 3rd Edition

\end{literature}



\end{course}