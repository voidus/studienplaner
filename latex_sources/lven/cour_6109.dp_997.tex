% Lehrveranstaltungsbeschreibung
% Informationsgrad : extern
% Sprache: de
\begin{course}

\setdoclanguagegerman
\coursedegreeprogramme{Informatik}
\coursemodulename{Grundlagen der BWL (S.~\pageref{mod_3033.dp_997})[IN3WWBWL]}
\courseID{2600026}
\coursename{Allgemeine Betriebswirtschaftslehre C}
\coursecoordination{M. Ruckes, M. Uhrig-Homburg}

\documentdate{2012-01-10 12:58:16.539295}

\courselevel{1}
\coursecredits{4}
\courseterm{Wintersemester}
\coursehours{2/0/2}
\courseinstructionlanguage{de}

\coursehead

% For index (key word@display). Key word is used for sorting - no Umlauts please.
\index{Allgemeine Betriebswirtschaftslehre C@Allgemeine Betriebswirtschaftslehre C}

% For later referencing
\label{cour_6109.dp_997}


\begin{styleenv}
\begin{assessment}
Die Erfolgskontrolle erfolgt in Form einer schriftlichen Prüfung (Klausur) im Umfang von 90 min nach § 4 Abs. 2 Nr. 1 SPO.

 

Die Prüfungen werden in jedem Semester angeboten und können zu jedem ordentlichen Prüfungstermin wiederholt werden.


\end{assessment}

\begin{conditions}Keine.\end{conditions}


\end{styleenv}

\begin{learningoutcomes}
Ziel der Vorlesung und der sie begleitenden Tutorien ist es, den Studierenden Grundkenntnisse und Basiswissen im Bereich der Investition und Finanzierung sowie des Controllings zu vermitteln. Die Entscheidungsfindung in Bezug auf die BWL-Module im Vertiefungsteil des Bachelorstudiums soll auf dieser Grundlage erleichtert werden.

 
\end{learningoutcomes}

\begin{content}
Die Lehrveranstaltung setzt sich zusammen aus den Teilgebieten:

 

\textbf{Investition und Finanzierung}

 

Das Teilgebiet Investition und Finanzierung vermittelt die Grundlagen der Kapitalmarkttheorie und bietet eine moderne Einführung in die Theorie und Praxis der unternehmerischen Kapitalbeschaffung und -verwendung.

 

\textbf{Controlling}


\end{content}



\begin{literature}Ausführliche Literaturhinweise werden in den Materialen zur Vorlesung BWL C gegeben.

\end{literature}

\begin{remarks}\textbf{\textbf{Wichtige Ankündigung: zum Wintersemester 2012/2013 wird diese Vorlesung überarbeitet. Voraussichtlich werden dann die Teile Investition und Finanzierung als auch Controlling (Managerial Accounting) behandelt.}}

 

Die Schlüsselqualifikation umfasst die aktive Beteiligung in den Tutorien durch Präsentation eigener Lösungen und Einbringung von Diskussionsbeiträgen.

 

Die Teilgebiete werden von den jeweiligen BWL-Fachvertretern präsentiert. Ergänzt wird die Vorlesung durch begleitende Tutorien.

\end{remarks}

\end{course}