% Modulbeschreibung 
% Informationsgrad : extern
% Sprache: de
\begin{module}

\setdoclanguagegerman
\moduledegreeprogramme{Informatik (B.Sc.)}
\modulesubject{}
\moduleID{IN3INWAP}
\modulename{Web-Anwendungen und Praxis}
\modulecoordination{S. Abeck}

\documentdate{2012-01-12 11:56:56.903279}

\modulecredits{9}
\moduleduration{1}
\modulecycle{Jedes 2. Semester, Wintersemester}



\modulehead

% For index (key word@display). Key word is used for sorting - no Umlauts please.
\index{Web-Anwendungen und Praxis@Web-Anwendungen und Praxis (M)}

% For later referencing
\label{mod_14535.dp_997}

\begin{courselist}
24153 & Web-Anwendungen und Serviceorientierte Architekturen (I) (S.~\pageref{cour_14537.dp_997}) & 2/0 & W & 4 & S. Abeck\\
24312 & Praktikum Web-Anwendungen und Serviceorientierte Architekturen (I) (S.~\pageref{cour_14541.dp_997}) & 2/0 & W & 5 & S. Abeck\\
\end{courselist}

\begin{styleenv}
\begin{assessment}
Die Modulprüfung erfolgt in Form von Teilprüfungen über die beiden Lehrveranstaltungen des Moduls.

 

Die Erfolgskontrolle zu Web-Anwendungen und Serviceorientierte Architekturen (I) [24153] erfolgt in Form einer mündlichen Prüfung nach § 4 Abs. 2 Nr. 2 SPO. Die Zulassung zur Prüfung erfolgt nur bei nachgewiesener Mitarbeit an den in der Vorlesung gestellten praktischen Aufgaben.

 

Die Erfolgskontrolle zum Praktikum Web-Anwendungen und Serviceorientierte Architekturen (I) [24312] erfolgt benotet als Erfolgskontrolle anderer Art nach § 4 Abs. 2 Nr. 3 SPO.

 

Die Gesamtnote des Moduls wird aus den mit LP gewichteten Teilnoten gebildet und nach der ersten Kommastelle abgeschnitten.


\end{assessment}

\begin{conditions}Keine.\end{conditions}


\end{styleenv}

\begin{learningoutcomes}
\begin{itemize}\item Die wichtigsten den Stand der Technik repräsentierenden Technologien und Standards zur Entwicklung von traditionellen Web-Anwendungen sind bekannt und können genutzt werden.  \item Die Architektur von traditionellen Web-Anwendungen ist verstanden.  \item Die Softwarearchitektur einer traditionellen Web-Anwendung kann modelliert werden.  \item Die wichtigsten Prinzipien traditioneller Softwareentwicklung und des entsprechenden Entwicklungsprozesses sind bekannt.  \item Die Technologien und Werkzeuge können zur Entwicklung von Beispielszenarien angewendet werden.  \end{itemize}
\end{learningoutcomes}

\begin{content}
Das Internet als Verteilungsplattform und die darauf basierenden Webtechnologien spielen eine große Rolle bei der Entwicklung verteilter Anwendungssysteme. Traditionelle Webanwendungen nutzen standardisierte Technologien zur Kommunikation (u.a. HTTP, TCP) und zur Informationsbeschreibung (u.a. HTML, XML), die in der Vorlesung an einer durchgängigen Beispiel-Anwendung aufgezeigt werden.


\end{content}



\end{module}

