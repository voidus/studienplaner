% Lehrveranstaltungsbeschreibung
% Informationsgrad : extern
% Sprache: de
\begin{course}

\setdoclanguagegerman
\coursedegreeprogramme{Informatik}
\coursemodulename{eBusiness und Service Management (S.~\pageref{mod_1611.dp_997})[IN3WWBWL2]}
\courseID{2540478}
\coursename{Spezialveranstaltung Informationswirtschaft}
\coursecoordination{C. Weinhardt}

\documentdate{2011-12-20 16:08:43.594822}

\courselevel{4}
\coursecredits{4,5}
\courseterm{Winter-/Sommersemester}
\coursehours{3}
\courseinstructionlanguage{de}

\coursehead

% For index (key word@display). Key word is used for sorting - no Umlauts please.
\index{Spezialveranstaltung Informationswirtschaft@Spezialveranstaltung Informationswirtschaft}

% For later referencing
\label{cour_8187.dp_997}


\begin{styleenv}
\begin{assessment}
Die Erfolgskontrolle erfolgt durch das Ausarbeiten einer schriftlichen Dokumentation, einer Präsentation der Ergebnisse der durchgeführten praktischen Komponenten und der aktiven Beteiligung an den Diskussionen (nach §4(2), 3 SPO).

 

Bitte bachten Sie, dass auch eine praktische Komponente wie die Durchführung einer Umfrage, oder die Implementierung einer Applikation neben der schriftlichen Ausarbeitung zum regulären Leistungsumfang der Veranstaltung gehört. Die jeweilige Aufgabenstellung entnehmen Sie bitte der Veranstaltungsbeschreibung.

 

Die Gesamtnote setzt sich zusammen aus den benoteten und gewichteten Erfolgskontrollen (z.B. Dokumentation, mündl. Vortrag, praktische Ausarbeitung sowie aktive Beteiligung).


\end{assessment}

\begin{conditions}Keine.\end{conditions}


\end{styleenv}

\begin{learningoutcomes}
Der Student soll eine gründliche Literaturrecherche ausgehend von einem vorgegebenen Thema der Informationswirtschaft durchführen. Dabei soll er relevante Arbeiten identifizieren und zu einer Analyse und Bewertung der in der Literatur vorgestellten Methoden im Rahmen einer Präsentation und schriftlichen Ausarbeitung auf wissenschaftlichem Niveau gelangen. Die zusätzlichen praktischen Aufgaben sollen Kenntnisse zur wissenschaftlicher Arbeitsweise und den damit verbundenen Methoden vermitteln.

 

Die Dokumentation dient auch der Vorbereitung auf weitere wissenschaftliche Arbeiten wie Master- oder Doktorarbeiten.


\end{learningoutcomes}

\begin{content}
Die Veranstaltung ermöglicht dem Studenten, mit den Methoden des wissenschaftlichen Arbeitens ein vorgegebenes Thema zu bearbeiten. Die angebotenen Themen fokussieren die Problemstellungen der Informationswirtschaft in verschiedenen Branchen, die in der Regel eine interdisziplinäre Betrachtung erfordern. Die konkrete praktische Umsetzung kann dabei eine Fallstudie, ökonomische Experimente oder Softwareentwicklungsarbeit enthalten. Die geleistet Arbeit ist ebenfalls wie bei einer Seminararbeit zu dokumentieren.


\end{content}

\begin{media}\begin{itemize}\item Power Point  \item eLearning Plattform Ilias  \item ggf. Software Tools zur Entwicklung  \end{itemize}\end{media}

\begin{literature}Die Basisliteratur wird entsprechend der zu bearbeitenden Themen bereitgestellt.

\end{literature}

\begin{remarks}Alle angebotenen Seminarpraktika können als Spezialveranstaltung Informationswirtschaft am Lehrstuhl von Prof. Dr. Weinhardt gewählt werden. Das aktuelle Angebot der Seminarpraktikathemen wird auf der Webseite www.iism.kit.edu/im/lehre bekannt gegeben.

 

Die Spezialveranstaltung Informationswirtschaft entspricht dem Seminarpraktikum, wie es bisher nur für den Studiengang Informationswirtschaft angeboten wurde. Mit dieser Veranstaltung wird die Möglichkeit, praktische Erfahrungen zu sammeln bzw. wissenschaftliche Arbeitsweise im Rahmen eines Seminarpraktikums zu erlernen, auch Studierenden des Wirtschaftsingenieurwesens und der Technischen Volkswirtschaftslehre zugänglich gemacht.

 

Die Spezialveranstaltung Informationswirtschaft kann anstelle einer regulären Vorlesung (siehe Modulbeschreibung) gewählt werden. Sie kann aber nur einmal pro Modul angerechnet werden.

 \end{remarks}

\end{course}