% Lehrveranstaltungsbeschreibung
% Informationsgrad : extern
% Sprache: de
\begin{course}

\setdoclanguagegerman
\coursedegreeprogramme{Informatik}
\coursemodulename{CRM und Servicemanagement (S.~\pageref{mod_1601.dp_997})[IN3WWBWL1]}
\courseID{2540508}
\coursename{Customer Relationship Management}
\coursecoordination{A. Geyer-Schulz}

\documentdate{2011-12-30 10:24:43.990044}

\courselevel{4}
\coursecredits{4,5}
\courseterm{Wintersemester}
\coursehours{2/1}
\courseinstructionlanguage{en}

\coursehead

% For index (key word@display). Key word is used for sorting - no Umlauts please.
\index{Customer Relationship Management@Customer Relationship Management}

% For later referencing
\label{cour_4435.dp_997}


\begin{styleenv}
\begin{assessment}
Die Erfolgskontrolle erfolgt in Form einer schriftlichen Prüfung (Klausur) im Umfang von 1h nach §4, Abs. 2, 1 SPO und durch Ausarbeiten von Übungsaufgaben als Erfolgskontrolle anderer Art nach §4, Abs. 2, 3 SPO.\newline
\newline
 Die Lehrveranstaltung ist bestanden, wenn in der Klausur 50 der 100 Punkte erreicht wurden. Im Falle der bestandenen Klausur werden die Punkte der Übungsleistung (maximal 25) zu den Punkten der Klausur addiert. Für die Berechnung der Note gilt folgende Skala:

 

\begin{center}
\begin{tabular}{cc}
Note & Mindestpunkte \\
\hline
1.0 & 113 \\
1.3 & 106 \\
1.7 & 99 \\
2.0 & 92 \\
2.3 & 85 \\
2.7 & 78 \\
3.0 & 71 \\
3.3 & 64 \\
3.7 & 57 \\
4.0 & 50 \\
\hline
4.7 & 40 \\
5.0 & 0 \\
\end{tabular}
\end{center}

 

Bemerkung: Für Diplomstudiengänge gilt eine abweichende Regelung.

    
\end{assessment}

\begin{conditions}Keine.\end{conditions}


\end{styleenv}

\begin{learningoutcomes}
Die Studierenden

 \begin{itemize}\item begreifen Servicemanagement als betriebswirtschaftliche Grundlage für Customer Relationship Management und lernen die sich daraus ergebenden Konsequenzen für die Unternehmensführung, Organisation und die einzelnen betrieblichen Teilbereiche kennen,  \item gestalten und entwickeln Servicekonzepte und Servicesysteme auf konzeptueller Ebene,  \item arbeiten Fallstudien im CRM-Bereich als kleine Projekte in Teamarbeit unter Einhaltung von Zeitvorgaben aus,  \item lernen Englisch als Fachsprache im Bereich CRM und ziehen internationale Literatur aus diesem Bereich zur Bearbeitung der Fallstudien heran.  \end{itemize}
\end{learningoutcomes}

\begin{content}
Das Wachstum des Dienstleistungssektors (Service) als Anteil vom BIP (und die häufig unterschätzte wirtschaftliche Bedeutung von Services durch versteckte Dienstleistungen in Industrie, Landwirtschaft und Bergbau) und die Globalisierung motivieren Servicewettbewerb als Wettbewerbstrategie für Unternehmen. Servicestrategien werden in der Regel mit CRM-Ansätzen implementiert, das intellektuelle Kapital von Mitarbeitern und die Orientierung am langfristigen Unternehmenswert ist dabei von hoher Bedeutung. Gleichzeitig verändert Servicewettbewerb die Marketingfunktion einer Unternehmung.

 

Servicewettbewerb erfordert das Management der Beziehungen zwischen Kunden und Lieferanten als Marketingansatz. Wichtige taktische (direkter Kundenkontakt, Kundeninformationssystem, Servicesystem für Kunden) und strategische (die Definition des Unternehmens als Serviceunternehmen, die Analyse der Organisation aus einer prozessorientierten Perspektive und die Etablierung von Partnernetzen für den Serviceprozess) CRM-Elemente, sowie Begriffe, wie z.B. Relationship, Kunde, Interesse des Kunden an Beziehung, Kundennutzen in Beziehung, Trust, Commitment, Attraction, und Relationship Marketing werden vorgestellt.

 

Die spezielle Natur von Services und ihre Folgen für das Marketing werden mit Hilfe des Marketingdreiecks für Produkt- und Servicemarketing erklärt. Betont wird dabei vor allem der Unterschied zwischen Produkt- und Prozesskonsum. Dieser Unterschied macht die technische Qualität und die funktionale Qualität eines Dienstes zu den Hauptbestandteilen des Modells der von Kunden wahrgenommenen Servicequalität. Erweiterte Qualitätsmodelle für Dienste und Beziehungen werden vorgestellt. Die systematische Analyse von Qualitätsabweichungen ist die Grundlage des Gap-Modells, das ein Modell für ganzheitliches Servicequalitätsmanagement darstellt. Service Recovery wird als Alternative zum traditionellen Beschwerdemanagement diskutiert. Aufbauend auf dem Konzept von Beziehungskosten, das hauptsächlich Qualitätsmängel im Service quantifiziert, wird ein Modell der Profitabilität von Beziehungen entwickelt.

 

Die Entwicklung eines erweiterten Serviceangebots umfasst ein Basisservicepaket, das mit Elementen, die die Zugänglichkeit, die Interaktivität und die Partizipation des Kunden am Service verbessern, zu einem vollen Serviceangebot erweitert wird. Die Prinzipien des Servicemanagements mit ihren Auswirkungen auf Geschäftsmodell, Entscheidungsfindung, Organisationsaufbau, Mitarbeiterführung, Anreizsysteme und Leistungsmessung werden ausführlich vorgestellt. Vertieft wird das Problem der Messung von Servicequalität, die erweiterte Rolle von Marketing in der Organisation in der Form des interaktiven und internen Marketings, die Entwicklung integrierter Marktkommunikation, von Brandrelationships und Image, der Aufbau einer marktorientierten Serviceoroganisation, sowie der Notwendigkeit, eine Servicekultur im Unternehmen zu etablieren.


\end{content}

\begin{media}Folien, Audio, Reader zur Vorlesung.

\end{media}

\begin{literature} 

Christian Grönroos. Service Management and Marketing : A Customer Relationship Management Approach. Wiley, Chichester, 2nd edition, 2000.

 

\textbf{Weiterführende Literatur:}

  

Jill Dyché. The CRM Handbook: A Business Guide to Customer Relationship Management. Addison-Wesley, Boston, 2nd edition, 2002.

  

Ronald S. Swift. Accelerating Customer Relationships: Using CRM and RelationshipTechnologies. Prentice Hall, Upper Saddle River, 2001.

   

Stanley A. Brown. Customer Relationship Management: A Strategic Imperative in theWorld of E-Business. John Wiley, Toronto, 2000.

           \end{literature}



\end{course}