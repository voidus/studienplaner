% Modulbeschreibung 
% Informationsgrad : extern
% Sprache: de
\begin{module}

\setdoclanguagegerman
\moduledegreeprogramme{Informatik (B.Sc.)}
\modulesubject{EF E-Technik}
\moduleID{IN3EITPAI}
\modulename{Praktikum Automation und Information}
\modulecoordination{F. Puente León, G.F. Trommer }

\documentdate{2012-01-17 15:35:45.726470}

\modulecredits{6}
\moduleduration{1}
\modulecycle{Jedes 2. Semester, Sommersemester}



\modulehead

% For index (key word@display). Key word is used for sorting - no Umlauts please.
\index{Praktikum Automation und Information@Praktikum Automation und Information (M)}

% For later referencing
\label{mod_3943.dp_997}

\begin{courselist}
23169 & Praktikum Automation und Information (S.~\pageref{cour_8059.dp_997}) & 0/4 & S & 6 & F. Puente, G.F. Trommer \\
\end{courselist}

\begin{styleenv}
\begin{assessment}
Die Erfolgskontrolle wird in den Lehrveranstaltungsbeschreibungen erläutert.

 

Die Gesamtnote des Moduls wird aus den mit LP gewichteten Noten der Teilprüfungen gebildet und nach der ersten Kommastelle abgeschnitten.


\end{assessment}

\begin{conditions}Der erfolgreiche Besuch vom Modul “Systemtheorie” [IN3EITST] wird vorausgesetzt.

\end{conditions}


\end{styleenv}

\begin{learningoutcomes}
Im Praktikum \textbf{Automation und Information} werden einige grundlegende Verfahren der Automatisierungs- und Informationstechnik behandelt und von den Studierenden selbst erprobt. Das Spektrum umfasst neben Informationstechnischen Inhalten wie Datenerfassung, Messtechnik und Bildverarbeitung auch Automatisierungsaspekte wie die Identifikation, Regelung und Optimierung technischer Laboraufbauten.


\end{learningoutcomes}

\begin{content}
Die einzelnen Versuche und der Ablauf werden vor Beginn des Praktikums auf den Internetseiten des Instituts für Regelungs- und Steuerungssysteme (IRS) bekanntgegeben (http://www.irs.uni-karlsruhe.de/1430.php)


\end{content}

\begin{remarks}\textcolor{red}{Dieses Modul wird nicht mehr angeboten. Prüfungen sind noch bis WS 2012/13 möglich.}

\end{remarks}

\end{module}

