% Lehrveranstaltungsbeschreibung
% Informationsgrad : extern
% Sprache: de
\begin{course}

\setdoclanguagegerman
\coursedegreeprogramme{Informatik}
\coursemodulename{Moderne Physik für Informatiker (S.~\pageref{mod_4247.dp_997})[IN2PHY2]}
\courseID{2400451}
\coursename{Moderne Physik für Informatiker}
\coursecoordination{Evers}

\documentdate{2011-04-08 12:15:41.098206}

\courselevel{2}
\coursecredits{9}
\courseterm{Sommersemester}
\coursehours{4/2}
\courseinstructionlanguage{de}

\coursehead

% For index (key word@display). Key word is used for sorting - no Umlauts please.
\index{Moderne Physik fuer Informatiker@Moderne Physik für Informatiker}

% For later referencing
\label{cour_8651.dp_997}


\begin{styleenv}
\begin{assessment}
Die Erfolgskontrolle erfolgt in Form einer schriftlichen Prüfung im Umfang von 90 Minuten nach § 4 Abs. 2 Nr. 1 SPO sowie einem Leistungsnachweis über die Übungen (nach § 4 Abs. 2 Nr. 3).

 

Die Note ist die Note der schriftlichen Prüfung.


\end{assessment}

\begin{conditions}Keine.\end{conditions}


\end{styleenv}

\begin{learningoutcomes}
Verständnis der grundlegenden experimentellen und Mathematischen Methoden der Quantenphysik (Atome, Moleküle, Festkörper, Kerne und Elementarteilchen)


\end{learningoutcomes}

\begin{content}
Einführung in den Mikrokosmos, spezielle Relativitätstheorie, Wellen- und Teilchencharakter des Lichts, quantisierte Größen, Welleneigenschaften von Teilchen, deBroglie-Beziehung, Heisenberg'sche Unbestimmtheitsrelation, Schrödingergleichung, Quantenmechanische Beschreibung von Atomen, Elektronen-Spin, Pauli-Prinzip, Periodensystem der Elemente, Wechselwirkung von Licht mit Atomen, Laser, Chemische Bindung und Moleküle, Grundprinzipien der Festkörperphysik, Elektronengas, Wärmekapazität, Stromleitung, Bändermodell, Atomkerne, Radioaktivität und Kernkräfte, Kernfusion, Kernmodelle, Kernzerfälle, Kernspaltung, Physik der Sonne, Elementarteilchen als Bausteine der Welt, Astrophysik und Kosmologie


\end{content}







\end{course}