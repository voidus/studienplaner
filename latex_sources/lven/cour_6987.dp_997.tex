% Lehrveranstaltungsbeschreibung
% Informationsgrad : extern
% Sprache: de
\begin{course}

\setdoclanguagegerman
\coursedegreeprogramme{Informatik}
\coursemodulename{Höhere Mathematik (S.~\pageref{mod_1481.dp_997})[IN1MATHHM]}
\courseID{01868}
\coursename{Höhere Mathematik II (Analysis) für die Fachrichtung Informatik}
\coursecoordination{C. Schmoeger}

\documentdate{2008-07-09 11:01:40}

\courselevel{1}
\coursecredits{6}
\courseterm{Sommersemester}
\coursehours{3/1}
\courseinstructionlanguage{de}

\coursehead

% For index (key word@display). Key word is used for sorting - no Umlauts please.
\index{Hoehere Mathematik II (Analysis) fuer die Fachrichtung Informatik@Höhere Mathematik II (Analysis) für die Fachrichtung Informatik}

% For later referencing
\label{cour_6987.dp_997}


\begin{styleenv}
\begin{assessment}
Die Erfolgskontrolle wird in der Modulbeschreibung erläutert.


\end{assessment}

\begin{conditions}Keine.\end{conditions}


\end{styleenv}

\begin{learningoutcomes}
Die Studierenden sollen am Ende des Moduls

 \begin{itemize}\item den Übergang von Schule zu Universität bewältigt haben,  \item mit logischem Denken und strengen Beweisen vertraut sein,  \item die Methoden und grundlegenden Strukturen der (reellen) Analysis beherrschen.  \end{itemize}
\end{learningoutcomes}

\begin{content}
\begin{itemize}\item \textbf{Der Raum R\textsuperscript{n}} (Konvergenz, Grenzwerte bei Funktionen, Stetigkeit)  \item \textbf{Differentialrechnung} im R\textsuperscript{n} (partielle Ableitungen, (totale) Ableitung, Taylorentwicklung, Extremwertberechnungen)  \item \textbf{Das mehrdimensionale Riemann- Integral} (Fubini, Volumenberechnung mit Cavalieri, Substitution, Polar-, Zylinder-, Kugelkoordinaten)  \item \textbf{Differentialgleichungen} (Trennung der Ver., lineare DGL 1. Ordnung, Bernoulli-DGL, Riccati-DGL, lineare Systeme, lineare DGL höherer Ordnung)  \item \textbf{Integraltransformationen}  \end{itemize}
\end{content}

\begin{media}Vorlesungspräsentationen

\end{media}

\begin{literature}wird in der Vorlesung bekannt gegeben

\textbf{Weiterführende Literatur:}

wird in der Vorlesung bekannt gegeben

\end{literature}



\end{course}