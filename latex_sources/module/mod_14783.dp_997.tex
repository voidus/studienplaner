% Modulbeschreibung 
% Informationsgrad : extern
% Sprache: de
\begin{module}

\setdoclanguagegerman
\moduledegreeprogramme{Informatik (B.Sc.)}
\modulesubject{}
\moduleID{IN3INDBSTP}
\modulename{Datenbanksysteme in Theorie und Praxis}
\modulecoordination{K. Böhm, Clemens Heidinger}

\documentdate{2012-01-05 09:16:09.188361}

\modulecredits{9}
\moduleduration{2}
\modulecycle{Jedes Semester}



\modulehead

% For index (key word@display). Key word is used for sorting - no Umlauts please.
\index{Datenbanksysteme in Theorie und Praxis@Datenbanksysteme in Theorie und Praxis (M)}

% For later referencing
\label{mod_14783.dp_997}

\begin{courselist}
dbe & Datenbankeinsatz (S.~\pageref{cour_5111.dp_997}) & 2/1 & S & 5 & K. Böhm\\
24317 & Arbeiten mit Datenbanksystemen (S.~\pageref{cour_14781.dp_997}) & 2 & W & 4 & K. Böhm, Clemens Heidinger\\
\end{courselist}

\begin{styleenv}
\begin{assessment}
Es wird mindestens sechs Wochen im Voraus angekündigt, ob die Erfolgskontrolle der Vorlesung in Form einer schriftlichen Prüfung (Klausur) im Umfang von i.d.R. 1h nach § 4 Abs. 2 Nr. 1 SPO oder in Form einer mündlichen Prüfung im Umfang von i.d.R. 20 Minuten nach § 4 Abs. 2 Nr. 2 SPO stattfindet.

 

Darüber hinaus ist zum Bestehen des Moduls das Bestehen des Praktikums nötig.


\end{assessment}

\begin{conditions}Die LV \emph{Datenbanksysteme} muss geprüft werden. Die Erteilung von Ausnahmegenehmigungen durch den Modulverantwortlichen für Studierende, die eine vergleichbare Lehrveranstaltung an einer anderen Universität besucht haben, ist möglich.

\end{conditions}


\end{styleenv}

\begin{learningoutcomes}
Am Ende der Lehrveranstaltung sollen die Teilnehmer Datenbank-Konzepte (insbesondere Datenmodelle, Anfragesprachen) – breiter, als es in einführenden Datenbank-Veranstaltungen vermittelt wurde – erläutern und miteinander vergleichen können. Sie sollten Alternativen bezüglich der Verwaltung komplexer Anwendungsdaten mit Datenbank-Technologie kennen und bewerten können.

 

Im Praktikum soll das in Vorlesungen wie “Datenbankeinsatz” und „Datenbanksysteme“ erlernte Wissen in der Praxis erprobt werden. Schrittweise sollen die Programmierung von Datenbankanwendungen, Benutzung von Anfragesprachen sowie Datenbankentwurf für überschaubare Realweltszenarien erlernt werden. Darüber hinaus sollen die Studenten lernen, im Team zusammenzuarbeiten und dabei wichtige Werkzeuge zur Teamarbeit kennenlernen


\end{learningoutcomes}

\begin{content}
Diese Vorlesung soll Studierende an den Einsatz moderner Datenbanksysteme heranführen.

 

Dabei werden unterschiedlicher Datenmodelle, insbesondere des relationalen und des semistrukturierten Modells (vulgo XML), und entsprechender Anfragesprachen (SQL, XQuery) gegenübergestellt. Verschiedene Anwendungsszenarien werden dabei untersucht. Die erworbenen Kenntnisse werden in dem Praktikum vertieft.

 

Dabei werden zunächst den Teilnehmern die wesentlichen Bestandteile von Datenbanksystemen in ausgewählten Versuchen mit relationaler Datenbanktechnologie nähergebracht. Sie erproben die klassischen Konzepte des Datenbankentwurfs und von Anfragesprachen an praktischen Beispielen.


\end{content}



\end{module}

