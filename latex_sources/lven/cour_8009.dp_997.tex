% Lehrveranstaltungsbeschreibung
% Informationsgrad : extern
% Sprache: de
\begin{course}

\setdoclanguagegerman
\coursedegreeprogramme{Informatik}
\coursemodulename{Differentialgleichungen und Hilberträume (S.~\pageref{mod_3295.dp_997})[IN3MATHAN03]}
\courseID{1566}
\coursename{Differentialgleichungen und Hilberträume}
\coursecoordination{G. Herzog, M. Plum, W. Reichel, C. Schmoeger, R. Schnaubelt, L. Weis}

\documentdate{2011-10-06 18:26:12.721123}

\courselevel{}
\coursecredits{9}
\courseterm{Sommersemester}
\coursehours{4/2}
\courseinstructionlanguage{}

\coursehead

% For index (key word@display). Key word is used for sorting - no Umlauts please.
\index{Differentialgleichungen und Hilbertraeume@Differentialgleichungen und Hilberträume}

% For later referencing
\label{cour_8009.dp_997}


\begin{styleenv}
\begin{assessment}
Die Erfolgskontrolle wird in der Modulbeschreibung erläutert.


\end{assessment}

\begin{conditions}Keine.\end{conditions}

\begin{recommendations}Folgende Module sollten bereits belegt worden sein (Empfehlung):\newline
Lineare Algebra 1+2\newline
Analysis 1-3

\end{recommendations}
\end{styleenv}

\begin{learningoutcomes}
Vertieftes Verständnis analytischer Konzepte und Methoden


\end{learningoutcomes}

\begin{content}
\begin{itemize}\item Modellierung mit Differentialgleichungen  \item Erste Integrale  \item Phasenebene  \item Stabilität  \item Prinzip der linearisierten Stabilität  \item Randwertprobleme  \item Greensche Funktionen  \item Lösungsmethoden für elementare partielle Differentialgleichungen  \item Hilbert- und Banachräume und stetige lineare Operatoren  \item Grundbegriffe der Sobolevräume  \item Orthonormalbasen und Orthogonalprojektionen  \item Darstellungssätze von Riesz-Fischer und Lax-Milgram  \item Dirichletproblem als Variationsproblem  \item Spektralsatz für kompakte und selbstadjungierte Operatoren  \end{itemize}
\end{content}







\end{course}