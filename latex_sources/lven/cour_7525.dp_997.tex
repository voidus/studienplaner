% Lehrveranstaltungsbeschreibung
% Informationsgrad : extern
% Sprache: de
\begin{course}

\setdoclanguagegerman
\coursedegreeprogramme{Informatik}
\coursemodulename{Basispraktikum TI: Mobile Roboter (S.~\pageref{mod_2993.dp_997})[IN2INTIBP]}
\courseID{24573}
\coursename{TI-Basispraktikum Mobile Roboter}
\coursecoordination{R. Dillmann, Schill, Böge}

\documentdate{2011-08-01 14:29:16.960742}

\courselevel{2}
\coursecredits{4}
\courseterm{Sommersemester}
\coursehours{4}
\courseinstructionlanguage{de}

\coursehead

% For index (key word@display). Key word is used for sorting - no Umlauts please.
\index{TI-Basispraktikum Mobile Roboter@TI-Basispraktikum Mobile Roboter}

% For later referencing
\label{cour_7525.dp_997}


\begin{styleenv}
\begin{assessment}
Die Erfolgskontrolle wird in der Modulbeschreibung erläutert.


\end{assessment}

\begin{conditions}Keine.\end{conditions}

\begin{recommendations}Abschluss des Moduls \emph{Technische Informatik} [IN1INTI].

 

Grundlegende Kenntnisse in C sind hilfreich, aber nicht zwingend erforderlich.

\end{recommendations}
\end{styleenv}

\begin{learningoutcomes}
Ziel dieses Praktikums ist die Vermittlung von Grundlagen der Elektronik und Mikrocontrollerprogrammierung in der Praxis. Dazu zählt das Erlernen von elektromechanischen Grundfertigkeiten (Aufbau der ASURO-Plattform, Löten), das Durchführen einer Fehlersuche, die Programmierung unter Verwendung von Cross-Compilern, und die Ansteuerung von Sensoren und Aktoren.


\end{learningoutcomes}

\begin{content}
Im Rahmen des Praktikums werden Elektronik- und Hardware-Grundlagen vermittelt und die Mikrocontroller in C programmiert. Neben der seriellen Kommunikation werden Sensoren und Aktoren behandelt, und für die Umsetzung von reflexbasiertem Verhalten verwendet.


\end{content}

\begin{media}Versuchsbeschreibungen

\end{media}





\end{course}