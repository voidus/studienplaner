% Lehrveranstaltungsbeschreibung
% Informationsgrad : extern
% Sprache: de
\begin{course}

\setdoclanguagegerman
\coursedegreeprogramme{Informatik}
\coursemodulename{Biomedizinische Technik I (S.~\pageref{mod_4005.dp_997})[IN3EITBIOM]}
\courseID{23262}
\coursename{Bildgebende Verfahren in der Medizin II}
\coursecoordination{O. Dössel}

\documentdate{2010-07-15 12:00:15.825698}

\courselevel{3}
\coursecredits{3}
\courseterm{Sommersemester}
\coursehours{2}
\courseinstructionlanguage{de}

\coursehead

% For index (key word@display). Key word is used for sorting - no Umlauts please.
\index{Bildgebende Verfahren in der Medizin II@Bildgebende Verfahren in der Medizin II}

% For later referencing
\label{cour_8141.dp_997}


\begin{styleenv}
\begin{assessment}
Die Erfolgskontrolle erfolgt in Form einer schriftlichen Prüfung im Umfang von i.d.R. 120 Minuten nach § 4 Abs. 2 Nr. 1 SPO.


\end{assessment}

\begin{conditions}Keine.\end{conditions}


\end{styleenv}

\begin{learningoutcomes}
Umfassendes Verständnis für alle Methoden der medizinischen Bildgebung ohne ionisierende Strahlung.


\end{learningoutcomes}

\begin{content}
\begin{itemize}\item  Ultraschall-Bildgebung  \item  Thermographie  \item  Optische Tomographie  \item  Impedanztomographie  \item  Abbildung bioelektrischer Quellen  \item  Endoskopie  \item  Magnet-Resonanz-Tomographie  \item Bildgebung mit mehreren Modalitäten  \item  Molekulare Bildgebung  \end{itemize}
\end{content}







\end{course}