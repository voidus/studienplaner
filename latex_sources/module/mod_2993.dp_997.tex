% Modulbeschreibung 
% Informationsgrad : extern
% Sprache: de
\begin{module}

\setdoclanguagegerman
\moduledegreeprogramme{Informatik (B.Sc.)}
\modulesubject{}
\moduleID{IN2INTIBP}
\modulename{Basispraktikum TI: Mobile Roboter}
\modulecoordination{R. Dillmann}

\documentdate{2011-02-17 15:06:46.455236}

\modulecredits{4}
\moduleduration{1}
\modulecycle{Jedes 2. Semester, Sommersemester}



\modulehead

% For index (key word@display). Key word is used for sorting - no Umlauts please.
\index{Basispraktikum TI: Mobile Roboter@Basispraktikum TI: Mobile Roboter (M)}

% For later referencing
\label{mod_2993.dp_997}

\begin{courselist}
24573 & TI-Basispraktikum Mobile Roboter (S.~\pageref{cour_7525.dp_997}) & 4 & S & 4 & R. Dillmann, Schill, Böge\\
\end{courselist}

\begin{styleenv}
\begin{assessment}
Die Erfolgskontrolle erfolgt nach § 4 Abs. 2 Nr. 3 SPO als Erfolgskontrolle anderer Art und besteht aus mehreren Teilaufgaben. Die Bewertung erfolgt mit den Noten “bestanden” / “nicht bestanden”.


\end{assessment}

\begin{conditions}Keine.\end{conditions}

\begin{recommendations}Abschluss des Moduls \emph{Technische Informatik} [IN1INTI].

 

Grundlegende Kenntnisse in C sind hilfreich, aber nicht zwingend erforderlich.

\end{recommendations}
\end{styleenv}

\begin{learningoutcomes}
Ziel dieses Praktikums ist die Vermittlung von Grundlagen der Elektronik und Mikrocontrollerprogrammierung in der Praxis. Dazu zählt das Erlernen von elektromechanischen Grundfertigkeiten (Aufbau der ASURO-Plattform, Löten), das Durchführen einer Fehlersuche, die Programmierung unter Verwendung von Cross-Compilern, und die Ansteuerung von Sensoren und Aktoren.


\end{learningoutcomes}

\begin{content}
Im Rahmen des Praktikums werden Elektronik- und Hardware-Grundlagen vermittelt und die Mikrocontroller in C programmiert. Neben der seriellen Kommunikation werden Sensoren und Aktoren behandelt, und für die Umsetzung von reflexbasiertem Verhalten verwendet.


\end{content}



\end{module}

