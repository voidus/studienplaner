% Modulbeschreibung 
% Informationsgrad : extern
% Sprache: de
\begin{module}

\setdoclanguagegerman
\moduledegreeprogramme{Informatik (B.Sc.)}
\modulesubject{}
\moduleID{IN3INDPI}
\modulename{Datenschutz und Privatheit in vernetzten Informationssystemen}
\modulecoordination{K. Böhm}

\documentdate{2010-05-21 11:24:10.537614}

\modulecredits{3}
\moduleduration{1}
\modulecycle{Jedes 2. Semester, Sommersemester}



\modulehead

% For index (key word@display). Key word is used for sorting - no Umlauts please.
\index{Datenschutz und Privatheit in vernetzten Informationssystemen@Datenschutz und Privatheit in vernetzten Informationssystemen (M)}

% For later referencing
\label{mod_3067.dp_997}

\begin{courselist}
24605 & Datenschutz und Privatheit in vernetzten Informationssystemen (S.~\pageref{cour_7393.dp_997}) & 2 & S & 3 & K. Böhm, Buchmann\\
\end{courselist}

\begin{styleenv}
\begin{assessment}
Es wird mind. 6 Wochen im Voraus angekündigt, ob die Erfolgskontrolle in Form einer schriftlichen Prüfung (Klausur) im Umfang von 1h nach § 4 Abs. 2 Nr. 1 SPO oder in Form einer mündlichen Prüfung im Umfang von 20 min nach § 4 Abs. 2 Nr. 2 SPO stattfindet.


\end{assessment}

\begin{conditions}Grundkenntnisse zu Datenbanken, verteilten Informationssystemen, Systemarchitekturen und Komunikationsinfrastrukturen, z.B. aus der Vorlesung \emph{Datenbanksysteme }[24516].

\end{conditions}


\end{styleenv}

\begin{learningoutcomes}
Die Studenten sollen in die Ziele und Grundbegriffe der Informationellen Selbstbestimmung eingeführt werden. Sie sollen dazu die grundlegende Herausforderungen des Datenschutzes und ihre vielfältigen Auswirkungen auf Gesellschaft und Individuen benennen können. Weiterhin sollen die Studenten aktuelle Technologien zum Datenschutz beherrschen und anwenden können, z.B. Methoden des Spatial \& Temporal Cloaking. Die Studenten sollen damit in die Lage versetzt werden, die Risiken unbekannter Technologien für die Privatheit zu analysieren, geeignete Maßnahmen zum Umgang mit diesen Risiken vorzuschlagen und die Effektivität dieser Maßnahmen abzuschätzen.


\end{learningoutcomes}

\begin{content}
In diesem Modul soll vermittelt werden, welchen Einfluss aktuelle und derzeit in der Entwicklung befindliche Informationssysteme auf die Privatheit ausüben. Diesen Herausforderungen werden technische Maßnahmen zum Datenschutz gegenübergestellt, die derzeit in der Forschung diskutiert werden. Ein Exkurs zu den gesellschaftlichen Implikationen von Datenschutzproblen und Datenschutztechniken rundet das Modul ab.


\end{content}

\begin{remarks}\textcolor{red}{Dieses Modul wird ab dem SS 2012 nicht mehr im Bachelor-Studiengang Informatik, sondern nur noch im Master-Studiengang Informatik angeboten.}

 

Im Bachelor-Studiengang kann die Prüfung nur noch nach Antragstellung und Genehmigung durch das Service-Zentrum Studium und Lehre erfolgen.

\end{remarks}

\end{module}

