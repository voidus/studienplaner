% Modulbeschreibung 
% Informationsgrad : extern
% Sprache: de
\begin{module}

\setdoclanguagegerman
\moduledegreeprogramme{Informatik (B.Sc.)}
\modulesubject{Praktische Informatik}
\moduleID{IN1INSWT1}
\modulename{Softwaretechnik I}
\modulecoordination{W. Tichy, R. Reussner}

\documentdate{2010-07-28 10:25:09.491931}

\modulecredits{6}
\moduleduration{1}
\modulecycle{Jedes 2. Semester, Sommersemester}



\modulehead

% For index (key word@display). Key word is used for sorting - no Umlauts please.
\index{Softwaretechnik I@Softwaretechnik I (M)}

% For later referencing
\label{mod_2515.dp_997}

\begin{courselist}
24518 & Softwaretechnik I (S.~\pageref{cour_7119.dp_997}) & 3/1/2 & S & 6 & W. Tichy, Andreas Höfer\\
\end{courselist}

\begin{styleenv}
\begin{assessment}
Die Erfolgskontrolle besteht aus einer schriftlichen Prüfung nach § 4 Abs. 2 Nr. 1 SPO im Umfang von i.d.R. 60 Minuten. \newline
Die Modulnote ist die Note der schriftlichen Prüfung.

 

Zusätzlich muss ein unbenoteter Übungsschein als Erfolgskontrolle anderer Art nach § 4 Abs. 2 Nr. 3 SPO erbracht werden.


\end{assessment}

\begin{conditions}Keine.\end{conditions}

\begin{recommendations}Das Modul \emph{Programmieren} [IN1INPROG] sollte abgeschlossen sein.

\end{recommendations}
\end{styleenv}

\begin{learningoutcomes}
Der/Die Studierende soll

 \begin{itemize}\item Grundwissen über die Prinzipien, Methoden und Werkzeuge der Softwaretechnik erwerben.  \item komplexe Softwaresysteme ingenieurmäßig entwickeln und warten sollen.  \end{itemize}
\end{learningoutcomes}

\begin{content}
Inhalt der Vorlesung ist der gesamte Lebenszyklus von Software von der Projektplanung über die Systemanalyse, die Kostenschätzung, den Entwurf und die Implementierung, die Validation und Verifikation, bis hin zur Wartung von Software. Weiter werden UML, Entwurfsmuster, Software-Werkzeuge, Programmierumgebungen und Konfigurationskontrolle behandelt.


\end{content}



\end{module}

