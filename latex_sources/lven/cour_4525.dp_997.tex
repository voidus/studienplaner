% Lehrveranstaltungsbeschreibung
% Informationsgrad : extern
% Sprache: de
\begin{course}

\setdoclanguagegerman
\coursedegreeprogramme{Informatik}
\coursemodulename{Data Warehousing und Mining (S.~\pageref{mod_2531.dp_997})[IN3INDWM]}
\courseID{24114}
\coursename{Data Warehousing und Mining}
\coursecoordination{K. Böhm}

\documentdate{2011-10-20 16:11:09.026557}

\courselevel{4}
\coursecredits{5}
\courseterm{Wintersemester}
\coursehours{2/1}
\courseinstructionlanguage{de}

\coursehead

% For index (key word@display). Key word is used for sorting - no Umlauts please.
\index{Data Warehousing und Mining@Data Warehousing und Mining}

% For later referencing
\label{cour_4525.dp_997}


\begin{styleenv}
\begin{assessment}
Die Erfolgskontrolle erfolgt in Form einer mündlichen Prüfung nach § 4 Abs. 2 Nr. 2 der SPO.


\end{assessment}

\begin{conditions}Diese Lehrveranstaltung kann nicht belegt werden, wenn die Lehrveranstaltung \emph{Knowledge Discovery} [2511302] oder \emph{Data Mining} [2520375] belegt wurde/wird.

\end{conditions}

\begin{recommendations}Datenbankkenntnisse, z.B. aus der Vorlesung\emph{ Datenbanksysteme}

\end{recommendations}
\end{styleenv}

\begin{learningoutcomes}
Am Ende der Lehrveranstaltung sollen die Teilnehmer die Notwendigkeit von Data Warehousing- und Data-Mining Konzepten gut verstanden haben und erläutern können. Sie sollen unterschiedliche Ansätze zur Verwaltung und Analyse großer Datenbestände hinsichtlich ihrer Wirksamkeit und Anwendbarkeit einschätzen und vergleichen können. Die Teilnehmer sollen verstehen, welche Probleme im Themenbereich Data Warehousing/Data Mining derzeit offen sind, und einen Einblick in den diesbezüglichen Stand der Forschung gewonnen haben.


\end{learningoutcomes}

\begin{content}
Data Warehouses und Data Mining stoßen bei Anwendern mit großen Datenmengen, z.B. in den Bereichen Handel, Banken oder Versicherungen, auf großes Interesse. Hinter beiden Begriffen steht der Wunsch, in sehr großen, z.T. verteilten Datenbeständen die Übersicht zu behalten und mit möglichst geringem Aufwand interessante Zusammenhänge aus dem Datenbestand zu extrahieren. Ein Data Warehouse ist ein Repository, das mit Daten von einer oder mehreren operationalen Datenbanken versorgt wird. Die Daten werden so aufbereitet, dass die schnelle Evaluierung komplexer Analyse-Queries (OLAP, d.h. Online Analytical Processing) möglich wird. Bei Data Mining steht dagegen im Vordergrund, dass das System selbst Muster in den Datenbeständen erkennt.


\end{content}

\begin{media}Folien.

\end{media}

\begin{literature}\begin{itemize}\item Jiawei Han, Micheline Kamber: Data Mining: Concepts and Techniques. 2nd edition, Morgan Kaufmann Publishers, March 2006.  \end{itemize}

\textbf{Weiterführende Literatur:}

 

Weitere aktuelle Angaben in den Folien am Ende eines jeden Kapitels.

\end{literature}

\begin{remarks}\textcolor{red}{Diese Lehrveranstaltung wird im Bachelor-Studiengang Informatik nicht mehr angeboten.}

\end{remarks}

\end{course}