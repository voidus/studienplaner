% Lehrveranstaltungsbeschreibung
% Informationsgrad : extern
% Sprache: de
\begin{course}

\setdoclanguagegerman
\coursedegreeprogramme{Informatik}
\coursemodulename{Rechnerstrukturen (S.~\pageref{mod_2489.dp_997})[IN3INRS]}
\courseID{24570}
\coursename{Rechnerstrukturen}
\coursecoordination{J. Henkel, W. Karl}

\documentdate{2011-11-14 11:27:42.398608}

\courselevel{4}
\coursecredits{6}
\courseterm{Sommersemester}
\coursehours{3/1}
\courseinstructionlanguage{de}

\coursehead

% For index (key word@display). Key word is used for sorting - no Umlauts please.
\index{Rechnerstrukturen@Rechnerstrukturen}

% For later referencing
\label{cour_7099.dp_997}


\begin{styleenv}
\begin{assessment}
Die Erfolgskontrolle wird in der Modulbeschreibung erläutert.


\end{assessment}

\begin{conditions}Die Lehrveranstaltung setzt die Kenntnisse des Moduls Technische Informatik voraus.

\end{conditions}


\end{styleenv}

\begin{learningoutcomes}
Die Lehrveranstaltung soll die Studierenden in die Lage versetzen,

 \begin{itemize}\item grundlegendes Verständnis über den Aufbau, die Organisation und das Operationsprinzip von Rechnersystemen zu erwerben,  \item aus dem Verständnis über die Wechselwirkungen von Technologie, Rechnerkonzepten und Anwendungen die grundlegenden Prinzipien des Entwurfs nachvollziehen und anwenden zu können,  \item Verfahren und Methoden zur Bewertung und Vergleich von Rechensystemen anwenden zu können,  \item grundlegendes Verständnis über die verschiedenen Formen der Parallelverarbeitung in Rechnerstrukturen zu erwerben.  \end{itemize}

Insbesondere soll die Lehrveranstaltung die Voraussetzung liefern, vertiefende Veranstaltungen über eingebettete Systeme, moderne Mikroprozessorarchitekturen, Parallelrechner, Fehlertoleranz und Leistungsbewertung zu besuchen und aktuelle Forschungsthemen zu verstehen.


\end{learningoutcomes}

\begin{content}

\end{content}

\begin{media}Vorlesungsfolien, Aufgabenblätter

\end{media}

\begin{literature}\textbf{Weiterführende Literatur:}

 \begin{itemize}\item Hennessy, J.L., Patterson, D.A.: Computer Architecture: A Quantitative Approach. Morgan Kaufmann, 3.Auflage 2002  \item U. Bringschulte, T. Ungerer: Microcontroller und Microprozessoren, Springer, Heidelberg, 2. Auflage 2007  \item Theo Ungerer: Parallelrechner und parallele Programmierung, Spektrum-Verlag 1997  \end{itemize}\end{literature}



\end{course}