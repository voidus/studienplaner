% Lehrveranstaltungsbeschreibung
% Informationsgrad : extern
% Sprache: de
\begin{course}

\setdoclanguagegerman
\coursedegreeprogramme{Informatik}
\coursemodulename{Lineare Algebra (S.~\pageref{mod_2395.dp_997})[IN1MATHLA]}
\courseID{01332}
\coursename{Lineare Algebra  und Analytische Geometrie I für die Fachrichtung Informatik}
\coursecoordination{K. Spitzmüller, S. Kühnlein, Hug}

\documentdate{2008-06-23 11:17:36}

\courselevel{1}
\coursecredits{9}
\courseterm{Wintersemester}
\coursehours{4/2/2}
\courseinstructionlanguage{de}

\coursehead

% For index (key word@display). Key word is used for sorting - no Umlauts please.
\index{Lineare Algebra  und Analytische Geometrie I fuer die Fachrichtung Informatik@Lineare Algebra  und Analytische Geometrie I für die Fachrichtung Informatik}

% For later referencing
\label{cour_6283.dp_997}


\begin{styleenv}
\begin{assessment}
Die Erfolgskontrolle wird in der Modulbeschreibung erläutert.


\end{assessment}

\begin{conditions}Keine.\end{conditions}


\end{styleenv}

\begin{learningoutcomes}
Die Studierenden sollen am Ende des Moduls

 \begin{itemize}\item den Übergang von der Schule zur Universität bewältigt haben,  \item mit logischem Denken und strengen Beweisen vertraut sei,   \item die Methoden und grundlegenden Strukturen der Linearen Algebra beherrschen.  \end{itemize}
\end{learningoutcomes}

\begin{content}
\begin{itemize}\item Grundbegriffe (Mengen, Abbildungen, Relationen, Gruppen, Ringe, Körper, Matrizen, Polynome)   \item Lineare Gleichungssysteme (Gauß´sches Eliminationsverfahren, Lösungstheorie)   \item Vektorräume (Beispiele, Unterräume, Quotientenräume, Basis und Dimension)   \item Lineare Abbildungen (Kern, Bild, Rang, Homomorphiesatz, Vektorräume von Abbildungen, Dualraum, Darstellungsmatrizen, Basiswechsel)   \item Determinanten   \item Eigenwerttheorie (Eigenwerte, Eigenvektoren, Charakteristisches Polynom, Normalformen)   \end{itemize}
\end{content}



\begin{literature}\textbf{Weiterführende Literatur:}

Skriptum zur Vorlesung,\newline
weitere Literatur wird in der Vorlesung bekannt gegeben

\end{literature}



\end{course}