% Lehrveranstaltungsbeschreibung
% Informationsgrad : extern
% Sprache: de
\begin{course}

\setdoclanguagegerman
\coursedegreeprogramme{Informatik}
\coursemodulename{Energiebewusste Systeme (S.~\pageref{mod_10583.dp_997})[IN3INEBS]}
\courseID{24672}
\coursename{Low Power Design}
\coursecoordination{J. Henkel}

\documentdate{2011-10-21 08:37:17.751210}

\courselevel{4}
\coursecredits{3}
\courseterm{Sommersemester}
\coursehours{2}
\courseinstructionlanguage{de}

\coursehead

% For index (key word@display). Key word is used for sorting - no Umlauts please.
\index{Low Power Design@Low Power Design}

% For later referencing
\label{cour_6269.dp_997}


\begin{styleenv}
\begin{assessment}
Die Erfolgskontrolle wird in der Modulbeschreibung erläutert.


\end{assessment}

\begin{conditions}Keine.\end{conditions}

\begin{recommendations}Modul: “Entwurf und Architekturen für eingebettete Systeme”

 

Grundkenntnisse aus dem Modul „Optimierung und Synthese Eingebetteter Systeme“ sind zum Verständnis dieser Vorlesung hilfreich aber nicht zwingend erforderlich. Die Vorlesung ist gleichermaßen für Informatik-Studenten wie auch für Elektrotechnik-Studenten geeignet.

\end{recommendations}
\end{styleenv}

\begin{learningoutcomes}
Die Studierenden erlernen für alle Ebenen des Entwurfs Eingebetteter Systeme die Berücksichtigung energiesparender Maßnahmen bei gleichzeitiger Erhaltung der Rechenleistung. Nach Abschluss der Vorlesung soll der Student in der Lage sein, den problematischen Energieverbrauch zu erkennen und Maßnahmen zu dessen Beseitigung zu ergreifen.


\end{learningoutcomes}

\begin{content}
Diese Vorlesung gibt einen Überblick über Entwurfsverfahren, Syntheseverfahren, Schätzverfahren, Softwaretechniken, Betriebssystemstrategien etc. mit dem Ziel, den Leistungsverbrauch eingebetteter Systeme zu minimieren unter gleichzeitiger Beibehaltung der geforderten Performance. Sowohl forschungsrelevante als auch bereits etablierte (d.h. in Produkten implementierte) Techniken auf verschiedenen Abstraktionsebenen (vom Schaltkreis zum System) werden in der Verlesung behandelt.


\end{content}

\begin{media}Vorlesungsfolien

\end{media}





\end{course}