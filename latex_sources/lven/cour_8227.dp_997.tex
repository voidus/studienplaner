% Lehrveranstaltungsbeschreibung
% Informationsgrad : extern
% Sprache: de
\begin{course}

\setdoclanguagegerman
\coursedegreeprogramme{Informatik}
\coursemodulename{Industrielle Produktion I (S.~\pageref{mod_1595.dp_997})[IN3WWBWL10]}
\courseID{2581996}
\coursename{Logistik und Supply Chain Management}
\coursecoordination{F. Schultmann}

\documentdate{2012-01-04 10:30:41.668285}

\courselevel{3}
\coursecredits{3,5}
\courseterm{Wintersemester}
\coursehours{2/0}
\courseinstructionlanguage{de}

\coursehead

% For index (key word@display). Key word is used for sorting - no Umlauts please.
\index{Logistik und Supply Chain Management@Logistik und Supply Chain Management}

% For later referencing
\label{cour_8227.dp_997}


\begin{styleenv}
\begin{assessment}
Die Erfolgskontrolle erfolgt in Form einer mündlichen oder schriftlichen Prüfung (nach §4(2), 1 SPO). Die Prüfungen werden in jedem Semester angeboten und können zu jedem ordentlichen Prüfungstermin wiederholt werden.


\end{assessment}

\begin{conditions}Keine.\end{conditions}


\end{styleenv}

\begin{learningoutcomes}
Die Studierenden erlernen die wesentlichen Grundlagen und Charakteristika der betriebswirtschaftlichen Logistik und des Supply Chain Management. Neben betriebswirtschaftlichen Grundfunktionen der Logistik wird deren Zusammenwirken erlernt. Zudem erwerben die Teilnehmer Kenntnisse in der Gestaltung und Steuerung betrieblicher und überbetrieblicher Wertschöfpfungsnetzwerke.


\end{learningoutcomes}

\begin{content}
Im Einzelnen werden folgende Bereiche behandelt:

 \begin{itemize}\item Einführung in die Logistik, Begriffsbestimmungen  \item Aufgaben- und Teilbereiche der Logistik  \item Logistikziele und Logistikkosten  \item Logistikkennzahlen und Logistikperformance  \item Beschaffungslogistik  \item Produktionslogistik  \item Distributionslogistik  \item Reverse Logistics  \item Definition und Ziele des Supply Chain Management  \item Konzepte des Supply Chain Management  \item Modellierung von Supply Chains  \end{itemize}
\end{content}

\begin{media}Medien werden über die Lernplattform bereitgestellt.

\end{media}

\begin{literature}Wird in der Veranstaltung bekannt gegeben.

\end{literature}



\end{course}