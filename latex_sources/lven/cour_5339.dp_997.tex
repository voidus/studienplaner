% Lehrveranstaltungsbeschreibung
% Informationsgrad : extern
% Sprache: de
\begin{course}

\setdoclanguagegerman
\coursedegreeprogramme{Informatik}
\coursemodulename{Netzsicherheit: Architekturen und Protokolle (S.~\pageref{mod_2603.dp_997})[IN3INNAP]}
\courseID{24601}
\coursename{Netzsicherheit: Architekturen und Protokolle}
\coursecoordination{M. Schöller}

\documentdate{2010-06-07 10:04:48.778559}

\courselevel{4}
\coursecredits{4}
\courseterm{Sommersemester}
\coursehours{2/0}
\courseinstructionlanguage{de}

\coursehead

% For index (key word@display). Key word is used for sorting - no Umlauts please.
\index{Netzsicherheit: Architekturen und Protokolle@Netzsicherheit: Architekturen und Protokolle}

% For later referencing
\label{cour_5339.dp_997}


\begin{styleenv}
\begin{assessment}
Die Erfolgskontrolle wird in der Modulbeschreibung erläuert.


\end{assessment}

\begin{conditions}Keine.\end{conditions}

\begin{recommendations}Inhalte der Vorlesungen \emph{Einführung in Rechnernetze} [24519] (oder vergleichbarer Vorlesungen) und \emph{Telematik }[24128].

\end{recommendations}
\end{styleenv}

\begin{learningoutcomes}
Ziel der Vorlesung ist es, die Studenten mit Grundlagen des Entwurfs sicherer Kommunikationsprotokolle vertraut zu machen und Ihnen Kenntnisse bestehender Sicherheitsprotokolle, wie sie im Internet und in lokalen Netzen verwendet werden, zu vermitteln.


\end{learningoutcomes}

\begin{content}
Die Vorlesung „Netzsicherheit: Architekturen und Protokolle“ betrachet Herausforderungen und Techniken im Design sicherer Kommunikationsprotokolle sowie Themen des Datenschutz und der Privatsphäre. Komplexe Systeme wie Kerberos werden detailliert betrachtet und ihre Entwurfsentscheidungen in Bezug auf Sicherheitsaspekte herausgestellt. Spezieller Fokus wird auf PKI-Grundlagen, -Infrastrukturen sowie spezifische PKI-Formate gelegt. Ein weiterer Schwerpunkt stellen die verbreiteten Sicherheitsprotokolle IPSec und TLS/SSL sowie Protokolle zum Infrastrukturschutz dar.


\end{content}

\begin{media}Folien.

\end{media}

\begin{literature}Roland Bless et al. Sichere Netzwerkkommunikation. Springer-Verlag, Heidelberg, Juni 2005.

 

\textbf{Weiterführende Literatur:}

 \begin{itemize}\item Charlie Kaufman, Radia Perlman und Mike Speciner. Network Security: Private Communication in a Public World. 2nd Edition. Prentice Hall, New Jersey, 2002.  \item Carlisle Adams und Steve Lloyd. Understanding PKI. Addison Wesley, 2003  \item Rolf Oppliger. Secure Messaging with PGP and S/MIME. Artech House, Norwood, 2001.  \item Sheila Frankel. Demystifiying the IPsec Puzzle. Artech House, Norwood, 2001.  \item Thomas Hardjono und Lakshminath R. Dondeti. Security in Wireless LANs and MANs. Artech House, Norwood, 2005.  \item Eric Rescorla. SSL and TLS: Designing and Building Secure Systems. Addison Wesley, Indianapolis, 2000.  \end{itemize}\end{literature}



\end{course}