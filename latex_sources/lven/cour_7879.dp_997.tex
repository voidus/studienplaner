% Lehrveranstaltungsbeschreibung
% Informationsgrad : extern
% Sprache: de
\begin{course}

\setdoclanguagegerman
\coursedegreeprogramme{Informatik}
\coursemodulename{Anwendungen des Operations Research (S.~\pageref{mod_3831.dp_997})[IN3WWOR2], Methodische Grundlagen des OR (S.~\pageref{mod_3833.dp_997})[IN3WWOR3]}
\courseID{2550134}
\coursename{Globale Optimierung I}
\coursecoordination{O. Stein}

\documentdate{2011-12-15 17:10:49.702111}

\courselevel{4}
\coursecredits{4,5}
\courseterm{Wintersemester}
\coursehours{2/1}
\courseinstructionlanguage{de}

\coursehead

% For index (key word@display). Key word is used for sorting - no Umlauts please.
\index{Globale Optimierung I@Globale Optimierung I}

% For later referencing
\label{cour_7879.dp_997}


\begin{styleenv}
\begin{assessment}
Die Erfolgskontrolle erfolgt in Form einer schriftlichen Prüfung (60min.) (nach §4(2), 1 SPO).

 

Die Prüfung wird im Vorlesungssemester und dem darauf folgenden Semester angeboten.

 

Zulassungsvoraussetzung zur schriftlichen Prüfung ist der Erwerb von mindestens 30\% der Übungspunkte. Die Prüfungsanmeldung über das Online-Portal für die schriftliche Prüfung gilt somit vorbehaltlich der Erfüllung der Zulassungsvoraussetzung.

 

Die Erfolgskontrolle kann auch zusammen mit der Erfolgskontrolle zu \emph{Globale Optimierung II} [2550136] erfolgen. In diesem Fall beträgt die Dauer der schriftlichen Prüfung 120 min.

 

Bei gemeinsamer Erfolgskontrolle über die Vorlesungen \emph{Globale Optimierung I} [2550134] und \emph{Globale Optimierung II} [2550134] wird bei Erwerb von mindestens 60\% der Übungspunkte die Note der bestandenen Klausur um ein Drittel eines Notenschrittes angehoben.

 

Bei gemeinsamer Erfolgskontrolle über die Vorlesungen \emph{Globale Optimierung I} [2550134] und \emph{Globale Optimierung II} [2550134] wird bei Erwerb von mindestens 60\% der Rechnerübungspunkte die Note der bestandenen Klausur um ein Drittel eines Notenschrittes angehoben.


\end{assessment}

\begin{conditions}Keine.\end{conditions}


\end{styleenv}

\begin{learningoutcomes}
Der/die Studierende soll

 \begin{itemize}\item mit Grundlagen der deterministischen globalen Optimierung vertraut gemacht werden  \item in die Lage versetzt werden, moderne Techniken der deterministischen globalen Optimierung in der Praxis auswählen, gestalten und einsetzen zu können.  \end{itemize}
\end{learningoutcomes}

\begin{content}
Bei vielen Optimierungsproblemen aus Wirtschafts-, Ingenieur- und Naturwissenschaften tritt das Problem auf, dass numerische Lösungsverfahren zwar effizient \emph{lokale }Optimalpunkte finden können, während \emph{globale }Optimalpunkte sehr viel schwerer zu identifizieren sind. Dies entspricht der Tatsache, dass man mit lokalen Suchverfahren zwar gut den Gipfel des nächstgelegenen Berges finden kann, während die Suche nach dem Gipfel des Mount Everest eher aufwändig ist.

 

Teil I der Vorlesung behandelt Verfahren zur globalen Optimierung von konvexen Funktionen unter konvexen Nebenbedingungen. Sie ist wie folgt aufgebaut:

 \begin{itemize}\item Einführende Beispiele und Terminologie  \item Existenzaussagen  \item Optimalität in der konvexen Optimierung  \item Dualität, Schranken und Constraint Qualifications  \item Numerische Verfahren  \end{itemize}

Die Behandlung nichtkonvexer Optimierungsprobleme ist Inhalt von Teil II der Vorlesung.

 

In der parallel zur Vorlesung angebotenen Rechnerübung haben Sie Gelegenheit, die Programmiersprache MATLAB zu erlernen und einige dieser Verfahren zu implementieren und an praxisnahen Beispielen zu testen.


\end{content}



\begin{literature}\textbf{Weiterführende Literatur:}

 \begin{itemize}\item W. Alt \emph{Numerische Verfahren der konvexen, nichtglatten Optimierung }Teubner 2004  \item C.A. Floudas \emph{Deterministic Global Optimization }Kluwer 2000  \item R. Horst, H. Tuy \emph{Global Optimization }Springer 1996  \item A. Neumaier \emph{Interval Methods for Systems of Equations }Cambridge University Press 1990  \end{itemize}\end{literature}

\begin{remarks}Teil I und II der Vorlesung werden nacheinander im \emph{selben }Semester gelesen.

\end{remarks}

\end{course}