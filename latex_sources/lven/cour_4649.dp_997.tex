% Lehrveranstaltungsbeschreibung
% Informationsgrad : extern
% Sprache: de
\begin{course}

\setdoclanguagegerman
\coursedegreeprogramme{Informatik}
\coursemodulename{Mikroökonomische Theorie (S.~\pageref{mod_2461.dp_997})[IN3WWVWL6]}
\courseID{2520525}
\coursename{Spieltheorie I}
\coursecoordination{N.N.}

\documentdate{2011-10-28 15:21:45.920455}

\courselevel{3}
\coursecredits{4,5}
\courseterm{Sommersemester}
\coursehours{2/2}
\courseinstructionlanguage{de}

\coursehead

% For index (key word@display). Key word is used for sorting - no Umlauts please.
\index{Spieltheorie I@Spieltheorie I}

% For later referencing
\label{cour_4649.dp_997}


\begin{styleenv}
\begin{assessment}
Die Erfolgskontrolle erfolgt in Form einer schriftlichen Prüfung (Klausur) im Umfang von i.d.R. 80 Minuten nach § 4 Abs. 2 Nr. 1 SPO.


\end{assessment}

\begin{conditions}Keine.\end{conditions}

\begin{recommendations}Es werden Grundkenntnisse in Mathematik und Statistik vorausgesetzt.

 

Siehe Modulbeschreibung.

\end{recommendations}
\end{styleenv}

\begin{learningoutcomes}
Dieser Kurs vermittelt fundierte Kenntnisse in der Theorie strategischer Entscheidungen. Ein Hörer der Vorlesung soll in der Lage sein, allgemeine strategische Fragestellungen systematisch zu analysieren und gegebenenfalls Handlungsempfehlungen für konkrete volkswirtschaftliche Entscheidungssituationen (wie kooperatives vs. egoistisches Verhalten) zu geben.


\end{learningoutcomes}

\begin{content}
Der inhaltliche Schwerpunkt dieser Vorlesung sind die Grundlagen der nicht-kooperativen Spieltheorie. Modellannahmen, verschiedenste Lösungskonzepte und Anwendungen werden sowohl für simultane Spiele (Normalformspiele) als auch für sequentielle Spiele (Extensivformspiele) detailliert besprochen. Klassische Gleichgewichtskonzepte wie das Nash-Gleichgewicht oder das teilspielperfekte Gleichgewicht, aber auch fortgeschrittene Konzepte werden ausführlich diskutiert. Es wird zudem ggf. ein kurzer Einblick in die kooperative Spieltheorie gegeben.


\end{content}

\begin{media}Folien, Übungsblätter.

\end{media}

\begin{literature}Gibbons, A primer in Game Theory, Harvester-Wheatsheaf, 1992\newline
Holler/Illing, Eine Einführung in die Spieltheorie, 5. Auflage, Springer Verlag, 2003 \newline
Gardner, Games for Business and Economics, 2. Auflage, Wiley, 2003 \newline
Berninghaus/Ehrhart/Güth, Strategische Spiele, 2. Auflage, Springer Verlag 2006

 

\textbf{Weiterführende Literatur:}

 \begin{itemize}\item Binmore, Fun and Games, DC Heath, Lexington, MA, 1991  \end{itemize}\end{literature}



\end{course}