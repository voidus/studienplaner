% Lehrveranstaltungsbeschreibung
% Informationsgrad : extern
% Sprache: de
\begin{course}

\setdoclanguagegerman
\coursedegreeprogramme{Informatik}
\coursemodulename{Biomedizinische Technik I (S.~\pageref{mod_4005.dp_997})[IN3EITBIOM]}
\courseID{23276}
\coursename{Praktikum für biomedizinische Messtechnik }
\coursecoordination{A. Bolz}

\documentdate{2010-07-15 12:02:04.642108}

\courselevel{3}
\coursecredits{6}
\courseterm{Sommersemester}
\coursehours{4}
\courseinstructionlanguage{de}

\coursehead

% For index (key word@display). Key word is used for sorting - no Umlauts please.
\index{Praktikum fuer biomedizinische Messtechnik @Praktikum für biomedizinische Messtechnik }

% For later referencing
\label{cour_8147.dp_997}


\begin{styleenv}
\begin{assessment}
Die Erfolgskontrolle erfolgt in Form einer mündlichen Prüfung im Umfang von i.d.R. 20 Minuten nach § 4 Abs. 2 Nr. 2 SPO.


\end{assessment}

\begin{conditions}Keine.\end{conditions}


\end{styleenv}

\begin{learningoutcomes}

\end{learningoutcomes}

\begin{content}
\begin{itemize}\item  Biomedizinische Signalverarbeitung  \item  Invasive Blutdruckmessung  \item  Nicht-invasive Blutdruckmessung  \item  Elektrokardiographie  \item  Verstärkertechnologien für bioelektrische Signale  \item  Impedanzmessung in menschlichem Gewebe  \item  Elektrostimulation  \item  Elektromyographie und Muskelkontraktionskraft  \item  Hämatologie  \end{itemize}
\end{content}







\end{course}