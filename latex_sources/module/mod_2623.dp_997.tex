% Modulbeschreibung 
% Informationsgrad : extern
% Sprache: de
\begin{module}

\setdoclanguagegerman
\moduledegreeprogramme{Informatik (B.Sc.)}
\modulesubject{}
\moduleID{IN3INMMMK}
\modulename{Multilinguale Mensch-Maschine-Kommunikation}
\modulecoordination{T. Schultz}

\documentdate{2012-01-20 12:28:06.386293}

\modulecredits{6}
\moduleduration{1}
\modulecycle{Jedes 2. Semester, Sommersemester}



\modulehead

% For index (key word@display). Key word is used for sorting - no Umlauts please.
\index{Multilinguale Mensch-Maschine-Kommunikation@Multilinguale Mensch-Maschine-Kommunikation (M)}

% For later referencing
\label{mod_2623.dp_997}

\begin{courselist}
24600 & Multilinguale Mensch-Maschine-Kommunikation (S.~\pageref{cour_7267.dp_997}) & 4 & S & 6 & T. Schultz, F. Putze\\
\end{courselist}

\begin{styleenv}
\begin{assessment}
Es wird 6 Wochen im Voraus angekündigt, ob die Erfolgskontrolle in Form einer schriftlichen Prüfung (Klausur) im Umfang von i.d.R. 2h nach § 4 Abs. 2 Nr. 1 SPO oder in Form einer mündlichen Prüfung im Umfang von i.d.R. 30 min. nach § 4 Abs. 2 Nr. 2 SPO stattfinden wird.

 

Die Modulnote entspricht dieser Note.

 

Terminvereinbarung bitte per E-Mail an: helga.scherer@kit.edu \newline
Es wird empfohlen, sich frühzeitig um einen Prüfungstermin zu kümmern.


\end{assessment}

\begin{conditions}Keine.\end{conditions}


\end{styleenv}

\begin{learningoutcomes}
Die Studierenden werden in die Grundlagen der automatischen Spracherkennung und –verarbeitung eingeführt.

 

Dazu werden zunächst die theoretischen Grundlagen der Signalverarbeitung und der Modellierung von Sprache vorgestellt. Besonderes Augenmerk wird hier auf statistische Modellierungsmethoden gelegt. Der gegenwärtige Stand der Forschung und Entwicklung wird anhand zahlreicher Anwendungsbeispiele veranschaulicht. Nach dem Besuch der Veranstaltung sollten die Studierenden in der Lage sein, das Potential sowie die Herausforderungen und Grenzen moderner Sprachtechnologien und Anwendungen einzuschätzen.


\end{learningoutcomes}

\begin{content}
Die Vorlesung \emph{Multilinguale Mensch-Maschine-Kommunikation} bietet eine Einführung in die automatische Spracherkennung und Sprachverarbeitung. Dazu werden zunächst die theoretischen Grundlagen der Signalverarbeitung und der Modellierung von Sprache vorgestellt. Besonderes Augenmerk wird hier auf statistischen Modellierungsmethoden gelegt. Anschließend werden die wesentlichen praktischen Ansätze und Methoden behandelt, die für eine erfolgreiche Umsetzung der Theorie in die Praxis der sprachlichen Mensch-Maschine Kommunikation relevant sind. Die modernen Anforderungen der Spracherkennung und Sprachverarbeitung im Zuge der Globalisierung werden in der Vorlesung anhand zahlreicher Beispiele von state-of-the-art Systemen illustriert und im Kontext der Multilingualität beleuchtet.

 

Weitere Informationen unter http://csl.anthropomatik.kit.edu.


\end{content}



\end{module}

