% Modulbeschreibung 
% Informationsgrad : extern
% Sprache: de
\begin{module}

\setdoclanguagegerman
\moduledegreeprogramme{Informatik (B.Sc.)}
\modulesubject{Schlüsselqualifikationen}
\moduleID{IN2INSWPS}
\modulename{Teamarbeit in der Software-Entwicklung}
\modulecoordination{G. Snelting}

\documentdate{2011-07-04 13:24:49.231482}

\modulecredits{2}
\moduleduration{1}
\modulecycle{Jedes Semester}



\modulehead

% For index (key word@display). Key word is used for sorting - no Umlauts please.
\index{Teamarbeit in der Software-Entwicklung@Teamarbeit in der Software-Entwicklung (M)}

% For later referencing
\label{mod_10785.dp_997}

\begin{courselist}
24511 & Teamarbeit und Präsentation in der Software-Entwicklung (S.~\pageref{cour_10787.dp_997}) & 1 & W/S & 2 & G. Snelting, Dozenten der Fakultät für Informatik\\
\end{courselist}

\begin{styleenv}
\begin{assessment}
Die Erfolgskontrolle erfolgt als benotete Erfolgskontrolle anderer Art nach § 4 Abs. 2 Nr. 3 SPO.

 

Teilnehmer müssen als Team von ca. 5 Studierenden Präsentationen zu den Software-Entwicklungsphasen Pflichtenheft, Entwurf, Implementierung, Qualitätssicherung sowie eine Abschlusspräsentation von je 15 Minuten erarbeiten. Teilnehmer müssen Dokumente zur Projektplanung, insbesondere Qualitätssicherungsplan und Implementierungsplan vorlegen und umsetzen.


\end{assessment}

\begin{conditions}Das Modul kann nur in Verbindung mit dem Modul \emph{Praxis der Software-Entwicklung} [IN2INSWP] absolviert werden.

 

Der erfolgreiche Abschluss der Module \emph{Grundbegriffe der Informatik} [IN1INGI] und\emph{ Programmieren} [IN1INPROG] wird vorausgesetzt.

\end{conditions}

\begin{recommendations}Die Veranstaltung sollte erst belegt werden, wenn alle Module aus den ersten beiden Semestern abgeschlossen sind.

\end{recommendations}
\end{styleenv}

\begin{learningoutcomes}
Die Teilnehmer erwerben wichtige nicht-technische Kompetenzen zur Durchfühung von Softwareprojekten im Team. Dazu gehören Sprachkompetenz und kommunikative Kompetenz sowie Techniken der Teamarbeit, der Präsentation und der Projektplanung.


\end{learningoutcomes}

\begin{content}
Auseinandersetzung mit der Arbeit im Team, Kommunikations-, Organisations- und Konfliktbehandungsstrategien; Erarbeitung von Präsentationen zu Pflichtenheft, Entwurf, Implementierung, Qualitätssicherung, Abschlusspräsentation; Projektplanungstechniken (z.B. Netzplantechnik, Phasenbeauftragte).


\end{content}

\begin{remarks}Dieses Modul ergänzt das Pflichtmodul \emph{Praxis der Software-Entwicklung} [IN2INSWP]. Es ist ein Pflichtmodul. Studierende, die die Schlüsselqualifikationen bereits in vollem Umfang vorliegen, aber das Modul \emph{Praxis der Software-Entwicklung} [IN2INSWP] noch nicht bestanden haben, kontaktieren bitte das Service-Zentrum Studium und Lehre.

\end{remarks}

\end{module}

