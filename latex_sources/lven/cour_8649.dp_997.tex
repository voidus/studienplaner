% Lehrveranstaltungsbeschreibung
% Informationsgrad : extern
% Sprache: de
\begin{course}

\setdoclanguagegerman
\coursedegreeprogramme{Informatik}
\coursemodulename{Kommunikation und Datenhaltung (S.~\pageref{mod_2517.dp_997})[IN2INKD]}
\courseID{24516}
\coursename{Datenbanksysteme}
\coursecoordination{K. Böhm}

\documentdate{2012-02-06 10:34:21.936868}

\courselevel{3}
\coursecredits{4}
\courseterm{Sommersemester}
\coursehours{2/1}
\courseinstructionlanguage{de}

\coursehead

% For index (key word@display). Key word is used for sorting - no Umlauts please.
\index{Datenbanksysteme@Datenbanksysteme}

% For later referencing
\label{cour_8649.dp_997}


\begin{styleenv}
\begin{assessment}
Die Erfolgskontrolle erfolgt semesterbegleitend als benotete Erfolgskontrolle anderer Art nach § 4 Abs. 2. Nr. 3 SPO durch Bearbeiten von Übungsaufgaben, deren Lösungen benotet werden. Am Ende des Semesters wird eine benotete schriftliche Präsenzübung durchgeführt.\newline
\newline
Die semesterbegleitenden Übungen tragen insgesamt mit ca. 25\% zur Gesamtnote bei. Das Ergebnis der Präsenzübung trägt mit ca. 75\% zur Gesamtnote bei.\newline
\newline
Die Prüfung Datenbanksysteme kann einmal wiederholt werden.


\end{assessment}

\begin{conditions}Im Modul \emph{Kommunikation und Datenhaltung} muss diese Vorlesung gemeinsam mit der Lehrveranstaltung \emph{Einführung in Rechnernetze} [24519] geprüft werden.

\end{conditions}

\begin{recommendations}Der Besuch von Vorlesungen zu Rechnernetzen, Systemarchitektur und Softwaretechnik wird empfohlen, aber nicht vorausgesetzt.

\end{recommendations}
\end{styleenv}

\begin{learningoutcomes}
Der/die Studierende

 \begin{itemize}\item stellt den Nutzen von Datenbank-Technologie dar,  \item definiert die Modelle und Methoden bei der Entwicklung von funktionalen Datenbank-Anwendungen,  \item legt selbstständig einfache Datenbanken an und tätigt Zugriffe auf diese,  \item kennt und versteht die entsprechenden Begrifflichkeiten und die Grundlagen der zugrundeliegenden Theorie.  \end{itemize}
\end{learningoutcomes}

\begin{content}
Datenbanksysteme gehören zu den entscheidenden Softwarebausteinen in modernen Informationssystemen und somit auch zu den Kernfächern in den Universitätsstudiengängen im Gebiet der Informatik. Ziel der Vorlesung ist die Vermittlung von Grundkenntnissen zur Arbeit mit Datenbanken. Schwerpunkte bilden dabei Datenbankmodelle für Entwurf und Implementierung (ER-Modell, Relationenmodell), Sprachen für Datenbanksyteme (SQL) und deren theoretische Basis (relationale Algebra) sowie Aspekte der Transaktionsverwaltung, Datenintegrität und Sichten.


\end{content}

\begin{media}Folien.

\end{media}

\begin{literature}\begin{itemize}\item Andreas Heuer, Kai-Uwe Sattler, Gunther Saake: Datenbanken - Konzepte und Sprachen, 3. Aufl., mitp-Verlag, Bonn, 2007  \item Alfons Kemper, André Eickler: Datenbanksysteme. Eine Einführung, 7. Aufl., Oldenbourg Verlag, 2009  \end{itemize}

\textbf{Weiterführende Literatur:}

 \begin{itemize}\item S. Abeck, P. C. Lockemann, J. Seitz, J. Schiller: Verteilte Informationssysteme, dpunkt-Verlag, 1. Aulage,2002, ISBN-13: 978-3898641883  \item R. Elmasri, S.B. Navathe: Fundamentals of Database Systems, 4. Auflage, Benjamin/Cummings, 2000.  \item Gerhard Weikum, Gottfried Vossen: Transactional Information Systems, Morgan Kaufmann, 2002.  \item C. J. Date: An Introduction to Database Systems, 8. Auflage, Addison-Wesley, Reading, 2003.  \end{itemize}\end{literature}

\begin{remarks}\textbf{Anmerkung zur Erfolgskontrolle:} \newline
Es gibt i.d.R. drei prüfungsrelevante semesterbegleitende Übungsaufgaben. Für die Bearbeitung der Übungsaufgaben werden geeignete Zeitspannen eingeräumt. Eine Verlängerung der Abgabefrist ist ausgeschlossen.\newline
Die Abmeldung von der Prüfung kann bis kurz vor der Präsenzübung stattfinden. Der genaue Termin wird rechtzeitig bekannt gegeben.\newline
Für die Präsenzübung sind zwei Termine vorgesehen, einen nach Ende der Vorlesungszeit und einen kurz vor Semesterende. Studierende, die von vornherein den zweiten Termin wahrnehmen möchten, müssen uns dies eine Woche vor dem ersten Termin schriftlich mitteilen, ansonsten wird die Präsenzübung mit null Punkten bewertet. Studierende, die den ersten Termin wahrnehmen wollten und aus nicht zu vertretenden Gründen dies jedoch nicht konnten, können bei Vorlage eines Attests am zweiten Termin teilnehmen.\newline
Erbrachte Leistungen aus einem früheren Versuch (z.B. in Form von Punkten) werden nicht anerkannt.

 

Zur Lehrveranstaltung Datenbanksysteme ist es möglich als weitergehende Übung im Wahlfach das Modul \textbf{\emph{Weitergehende Übung Datenbanksysteme} [IN3INWDS]} zu belegen (dieses Modul wird zurzeit nicht angeboten).

\end{remarks}

\end{course}