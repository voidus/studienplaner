% Modulbeschreibung 
% Informationsgrad : extern
% Sprache: de
\begin{module}

\setdoclanguagegerman
\moduledegreeprogramme{Informatik (B.Sc.)}
\modulesubject{}
\moduleID{IN3INBSBS}
\modulename{Biosignale und Benutzerschnittstellen}
\modulecoordination{T. Schultz}

\documentdate{2012-01-20 12:28:28.008873}

\modulecredits{6}
\moduleduration{1}
\modulecycle{Jedes 2. Semester, Wintersemester}



\modulehead

% For index (key word@display). Key word is used for sorting - no Umlauts please.
\index{Biosignale und Benutzerschnittstellen@Biosignale und Benutzerschnittstellen (M)}

% For later referencing
\label{mod_2627.dp_997}

\begin{courselist}
24105 & Biosignale und Benutzerschnittstellen (S.~\pageref{cour_7303.dp_997}) & 4 & W & 6 & T. Schultz, M. Wand\\
\end{courselist}

\begin{styleenv}
\begin{assessment}
Es wird 6 Wochen im Voraus angekündigt, ob die Erfolgskontrolle in Form einer schriftlichen Prüfung (Klausur) im Umfang von i.d.R. 2h nach § 4 Abs. 2 Nr. 1 SPO oder in Form einer mündlichen Prüfung im Umfang von i.d.R. 30 min. nach § 4 Abs. 2 Nr. 2 SPO stattfinden wird.

 

Die Modulnote entspricht dieser Note.

 

Terminvereinbarung bitte per E-Mail an: helga.scherer@kit.edu

 

Es wird empfohlen, sich frühzeitig um einen Prüfungstermin zu kümmern.


\end{assessment}

\begin{conditions}Keine.\end{conditions}


\end{styleenv}

\begin{learningoutcomes}
Die Studierenden sollen in die Grundlagen der Biosignale, deren Entstehung, Erfassung, und Interpretation eingeführt werden und deren Potential für die Anwendung im Zusammenhang mit Mensch-Maschine Benutzerschnittstellen verstehen. Dabei sollen sie auch lernen, die Probleme, Herausforderungen und Chancen von Biosignalen für Benutzerschnittstellen zu analysieren und formal zu beschreiben. Dazu werden die Studierenden mit den grundlegenden Verfahren zum Messen von Biosignalen, der Signalverarbeitung, und Erkennung und Identifizierung mittels statistischer Methoden vertraut gemacht. Der gegenwärtige Stand der Forschung und Entwicklung wird anhand zahlreicher Anwendungsbeispiele veranschaulicht. Nach dem Besuch der Veranstaltung sollten die Studierenden in der Lage sein, die vorgestellten Anwendungsbeispiele auf neue moderne Anforderungen von Benutzerschnittstellen zu übertragen.


\end{learningoutcomes}

\begin{content}
Dieses Modul bietet eine Einführung in Technologien, die verschiedenste Biosignale des Menschen zur Übertragung von Information einsetzen und damit das Design von Benutzerschnittstellen revolutionieren. Hauptaugenmerk liegt dabei auf der Interaktion zwischen Mensch und Maschine. Dazu vermitteln wir zunächst einen Überblick über das Spektrum menschlicher Biosignale, mit Fokus auf diejenigen Signale, die äußerlich abgeleitet werden können, wie etwa die Aktivität des Gehirns von der Kopfoberfläche (Elektroencephalogramm - EEG), die Muskelaktivität von der Hautoberfläche (Elektromyogramm - EMG), die Aktivität der Augen (Elektrookulogramm - EOG) und Parameter wie Hautleitwert, Puls und Atemfrequenz. Daran anschließend werden die Grundlagen zur Ableitung, Vorverarbeitung, Erkennung und Interpretation dieser Signale vermittelt. Zur Erläuterung und Veranschaulichung werden zahlreiche Anwendungsbeispiele aus der Literatur und eigenen Forschungsarbeiten vorgestellt.

 

Weitere Informationen unter http://csl.anthropomatik.kit.edu.


\end{content}



\end{module}

