% Lehrveranstaltungsbeschreibung
% Informationsgrad : extern
% Sprache: de
\begin{course}

\setdoclanguagegerman
\coursedegreeprogramme{Informatik}
\coursemodulename{Programmieren (S.~\pageref{mod_2373.dp_997})[IN1INPROG]}
\courseID{24004}
\coursename{Programmieren}
\coursecoordination{A. Pretschner}

\documentdate{2011-07-29 12:39:12.497915}

\courselevel{1}
\coursecredits{5}
\courseterm{Wintersemester}
\coursehours{2/0/2}
\courseinstructionlanguage{de}

\coursehead

% For index (key word@display). Key word is used for sorting - no Umlauts please.
\index{Programmieren@Programmieren}

% For later referencing
\label{cour_6247.dp_997}


\begin{styleenv}
\begin{assessment}
Die Erfolgskontrolle wird in der Modulbeschreibung erläutert.


\end{assessment}

\begin{conditions}Keine.\end{conditions}

\begin{recommendations}Vorkenntnisse in Java-Programmierung können hilfreich sein, werden aber nicht vorausgesetzt.

\end{recommendations}
\end{styleenv}

\begin{learningoutcomes}
Der/die Studierende soll

 \begin{itemize}\item grundlegender Strukturen der Programmiersprache Java kennen und anwenden, insbesondere Kontrollstrukturen, einfache Datenstrukturen, Umgang mit Objekten, und Implementierung elementarer Algorithmen.  \item grundlegende Kenntnisse in Programmiermethodik und die Fähigkeit zur autonomen Erstellung kleiner bis mittlerer, lauffähiger Java-Programme erwerben.  \end{itemize}
\end{learningoutcomes}

\begin{content}
\begin{itemize}\item Objekte und Klassen  \item Typen, Werte und Variablen  \item Methoden  \item Kontrollstrukturen  \item Rekursion  \item Referenzen, Listen  \item Vererbung   \item Ein/-Ausgabe  \item Exceptions  \item Programmiermethodik  \item Implementierung elementarer Algorithmen (z.B. Sortierverfahren) in Java  \end{itemize}
\end{content}

\begin{media}Beamer, Folien, Tafel, Übungsblätter

\end{media}

\begin{literature}P. Pepper, Programmieren Lernen, Springer, 3. Auflage 2007

 

\textbf{Weiterführende Literatur:}

 

B. Eckels: Thinking in Java. Prentice Hall 2006\newline
J. Bloch: Effective Java, Addison-Wesley 2008

\end{literature}



\end{course}