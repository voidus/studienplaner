% Lehrveranstaltungsbeschreibung
% Informationsgrad : extern
% Sprache: de
\begin{course}

\setdoclanguagegerman
\coursedegreeprogramme{Informatik}
\coursemodulename{Datenbankeinsatz (S.~\pageref{mod_2543.dp_997})[IN3INDBE], Datenbanksysteme in Theorie und Praxis (S.~\pageref{mod_14783.dp_997})[IN3INDBSTP]}
\courseID{dbe}
\coursename{Datenbankeinsatz}
\coursecoordination{K. Böhm}

\documentdate{2012-02-01 10:32:32.918571}

\courselevel{4}
\coursecredits{5}
\courseterm{Sommersemester}
\coursehours{2/1}
\courseinstructionlanguage{de}

\coursehead

% For index (key word@display). Key word is used for sorting - no Umlauts please.
\index{Datenbankeinsatz@Datenbankeinsatz}

% For later referencing
\label{cour_5111.dp_997}


\begin{styleenv}
\begin{assessment}
Die Erfolgskontrolle wird in der Modulbeschreibung erläutert.


\end{assessment}

\begin{conditions}Datenbankkenntnisse, z.B. aus der Vorlesungen \emph{Datenbanksysteme }[24516] und \emph{Einführung in Rechnernetze} [24519].

\end{conditions}


\end{styleenv}

\begin{learningoutcomes}
Am Ende der Lehrveranstaltung sollen die Teilnehmer Datenbank-Konzepte (insbesondere Datenmodelle, Anfragesprachen) – breiter, als es in einführenden Datenbank-Veranstaltungen vermittelt wurde – erläutern und miteinander vergleichen können. Sie sollten Alternativen bezüglich der Verwaltung komplexer Anwendungsdaten mit Datenbank-Technologie kennen und bewerten können.


\end{learningoutcomes}

\begin{content}
Diese Vorlesung soll Studierende an den Einsatz moderner Datenbanksysteme heranführen, in Breite und Tiefe. ’Breite’ erreichen wir durch die ausführliche Betrachtung und die Gegenüberstellung unterschiedlicher Datenmodelle, insbesondere des relationalen und des semistrukturierten Modells (vulgo XML), und entsprechender Anfragesprachen (SQL, XQuery). ’Tiefe’ erreichen wir durch die Betrachtung mehrerer nichttrivialer Anwendungen. Dazu gehören beispielhaft die Verwaltung von XML-Datenbeständen oder E-Commerce Daten, die Implementierung von Retrieval-Modellen mit relationaler Datenbanktechnologie oder die Verwendung von SQL für den Zugriff auf Sensornetze. Diese Anwendungen sind von allgemeiner Natur und daher auch isoliert betrachtet bereits interessant.


\end{content}

\begin{media}Folien.

\end{media}

\begin{literature}\begin{itemize}\item Andreas Heuer, Gunther Saake: Datenbanken - Konzepte und Sprachen. 2. Aufl., mitp-Verlag, Bonn, Januar 2000.  \item Alfons Kemper, Andre Eickler: Datenbanksysteme. 6. Aufl., Oldenbourg Verlag, 2006.  \end{itemize}

\textbf{Weiterführende Literatur:}

 \begin{itemize}\item Hector Garcia-Molina, Jeffrey D. Ullman, Jennifer Widom: Database Systems: The Complete Book. Prentice Hall, 2002  \item Ramez Elmasri, Shamkant B. Navathe: Fundamentals of Database Systems.  \end{itemize}\end{literature}



\end{course}