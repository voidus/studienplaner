% Modulbeschreibung 
% Informationsgrad : extern
% Sprache: de
\begin{module}

\setdoclanguagegerman
\moduledegreeprogramme{Informatik (B.Sc.)}
\modulesubject{EF Betriebswirtschaftslehre}
\moduleID{IN3WWBWL2}
\modulename{eBusiness und Service Management}
\modulecoordination{C. Weinhardt}

\documentdate{2011-12-20 16:37:04.928129}

\modulecredits{9}
\moduleduration{1}
\modulecycle{Jedes Semester}



\modulehead

% For index (key word@display). Key word is used for sorting - no Umlauts please.
\index{eBusiness und Service Management@eBusiness und Service Management (M)}

% For later referencing
\label{mod_1611.dp_997}

\begin{courselist}
2595466 & eServices (S.~\pageref{cour_6027.dp_997}) & 2/1 & S & 5 & C. Weinhardt, H. Fromm, J. Kunze von Bischhoffshausen\\
2590452 & Management of Business Networks (S.~\pageref{cour_4811.dp_997}) & 2/1 & W & 4,5 & C. Weinhardt, J. Kraemer\\
2540454 & eFinance: Informationswirtschaft für den Wertpapierhandel (S.~\pageref{cour_4851.dp_997}) & 2/1 & W & 4,5 & R. Riordan\\
2540478 & Spezialveranstaltung Informationswirtschaft (S.~\pageref{cour_8187.dp_997}) & 3 & W/S & 4,5 & C. Weinhardt\\
\end{courselist}

\begin{styleenv}
\begin{assessment}
Die Erfolgskontrolle erfolgt in Form von Teilprüfungen (nach §4(2), 1-3 SPO) über die Lehrveranstaltungen des Moduls im Umfang von insgesamt 9 LP. Die Erfolgskontrolle wird bei jeder Lehrveranstaltung dieses Moduls beschrieben.

 

Die Gesamtnote des Moduls wird aus den mit LP gewichteten Noten der Teilprüfungen gebildet und nach der ersten Nachkommastelle abgeschnitten.


\end{assessment}

\begin{conditions}Nur prüfbar in Kombination mit dem Modul \emph{Grundlagen der BWL}.

 \begin{itemize}\item Das Modul ist nur zusammen mit dem Pflichtmodul \emph{Grundlagen der BWL} [IN3WWBWL] prüfbar.  \end{itemize}\end{conditions}


\end{styleenv}

\begin{learningoutcomes}
Die Studierenden

 \begin{itemize}\item verstehen die strategischen und operativen Gestaltungen von Informationen und Informationsprodukten,  \item analysieren die Rolle von Informationen auf Märkten,  \item evaluieren Fallbeispiele bzgl. Informationsprodukte,  \item erarbeiten Lösungen in Teams.  \end{itemize}
\end{learningoutcomes}

\begin{content}
Dieses Modul vermittelt einen Überblick über die gegenseitigen Abhängigkeiten von strategischem Management und Informationssystemen. Es wird eine klare Unterscheidung in der Betrachtung von Information als Produktions- und Wettbewerbsfaktor sowie als Wirtschaftsgut eingeführt. Die zentrale Rolle von Informationen wird durch das Konzept des \emph{Informationslebenszyklus} erläutert, deren einzelne Phasen vor allem aus betriebswirtschaftlicher und mikroökonomischer Perspektive analysiert werden. Über diesen Informationslebenszyklus hinweg wird jeweils der Stand der Forschung in der ökonomischen Theorie dargestellt. Die Veranstaltung wird durch begleitende Übungen ergänzt.

 

Die Vorlesungen “Management of Business Networks”, “eFinance: Informationswirtschaft für den Wertpapierhandel” und “eServices” bilden drei Vertiefungs- und Anwendungsbereiche für die Inhalte der Pflichtveranstaltung. In der Veranstaltung “Management of Business Networks” wird insbesondere auf die strategischen Aspekte des Managements und der Informationsunterstützung abgezielt. Über den englischsprachigen Vorlesungsteil hinaus, vermittelt der Kurs das Wissen anhand einer Fallstudie, in der die Studenten das erlernte Wissen in einem “Business-Rollenspiel” anwenden sollen. In diesem Zusammenhang werden auch internationale Gastdozenten von der Universität Montreal bzw. Rotterdam einen internationalen Einblick in die Materie der strategischen Unternehmensnetzwerke vermitteln.

 

Die Vorlesung “eFinance: Informationswirtschaft für den Wertpapierhandel” vermittelt tiefgehende und praxisrelevante Inhalte über den börslichen und außerbörslichen Wertpapierhandel. Der Fokus liegt auf der ökonomischen und technischen Gestaltung von Märkten als informationsverarbeitenden Systemen.

 

In “eServices” wird die zunehmende Entwicklung von elektronischen Dienstleistungen im Gegensatz zu den klassischen Diensleistungen hervorgehoben. Die Informations- und Kommunikationstechnologie ermöglicht die Bereitstellung von Diensten, die durch Interaktivität und Individualität gekennzeichnet sind. In dieser Veranstaltung werden die Grundlagen für die Entwicklung und das Management IT-basierter Dienstleistungen gelegt.

 

Die Veranstaltung “Spezialveranstaltung Informationswirtschaft” festigt die theoretischen Grundlagen und ermöglicht weitergehende praktische Erfahrungen im Bereich der Informationswirtschaft. Seminarpraktika des IM können als Spezialveranstaltung Informationswirtschaft belegt werden.


\end{content}

\begin{remarks}Als Spezialveranstaltung Informationswirtschaft können alle Seminarpraktika des IM belegt werden. Aktuelle Informationen zum Angebot sind unter: www.iism.kit.edu/im/lehre zu finden.

\end{remarks}

\end{module}

