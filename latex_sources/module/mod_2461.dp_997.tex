% Modulbeschreibung 
% Informationsgrad : extern
% Sprache: de
\begin{module}

\setdoclanguagegerman
\moduledegreeprogramme{Informatik (B.Sc.)}
\modulesubject{EF Volkswirtschaftslehre}
\moduleID{IN3WWVWL6}
\modulename{Mikroökonomische Theorie}
\modulecoordination{C. Puppe}

\documentdate{2012-01-09 17:01:01.625460}

\modulecredits{9}
\moduleduration{1}
\modulecycle{Jedes 2. Semester, Sommersemester}



\modulehead

% For index (key word@display). Key word is used for sorting - no Umlauts please.
\index{Mikrooekonomische Theorie@Mikroökonomische Theorie (M)}

% For later referencing
\label{mod_2461.dp_997}

\begin{courselist}
2520527 & Advanced Topics in Economic Theory (S.~\pageref{cour_7079.dp_997}) & 2/1 & S & 4,5 & C. Puppe, M. Hillebrand, K. Mitusch\\
2520517 & Wohlfahrtstheorie (S.~\pageref{cour_7075.dp_997}) & 2/1 & S & 4,5 & C. Puppe\\
2520525 & Spieltheorie I (S.~\pageref{cour_4649.dp_997}) & 2/2 & S & 4,5 & N.N.\\
26240 & Wettbewerb in Netzen (S.~\pageref{cour_4959.dp_997}) & 2/1 & W & 4,5 & K. Mitusch\\
\end{courselist}

\begin{styleenv}
\begin{assessment}
Die Modulprüfung erfolgt in Form von Teilprüfungen (nach §4(2), 1 o. 2 SPO) über die gewählten Lehrveranstaltungen des Moduls, mit denen in Summe die Mindestanforderung an Leistungspunkten erfüllt ist. Die Erfolgskontrolle wird bei jeder Lehrveranstaltung dieses Moduls beschrieben.

 

Die Gesamtnote des Moduls wird aus den mit LP gewichteten Noten der Teilprüfungen gebildet und nach der ersten Nachkommastelle abgeschnitten.


\end{assessment}

\begin{conditions}Nur prüfbar in Kombination mit dem Modul \emph{Grundlagen der VWL}.

 

Das Modul ist nur zusammen mit dem Pflichtmodul \emph{Grundlagen der VWL} [IN3WWVWL] prüfbar.

\end{conditions}


\end{styleenv}

\begin{learningoutcomes}
Der/die Studierende

 \begin{itemize}\item beherrscht den Umgang mit fortgeschrittenen Konzepten der mikroökonomischen Theorie - beispielsweise der allgemeinen Gleichgewichtstheorie oder der Preistheorie - und kann diese auf reale Probleme, z. B. der Allokation auf Faktor- und Gütermärkten, anwenden. (Lehrveranstaltung „Fortgeschrittene Mikroökonomische Theorie”),  \item versteht Konzepte und Methoden der Wohlfahrtstheorie und kann sie auf Probleme der Verteilungsgerechtigkeit, Chancengleichheit und gesellschaftliche Fairness anwenden, (Lehrveranstaltung „Wohlfahrtstheorie”)  \item erlangt fundierte Kenntnisse in der Theorie strategischer Entscheidungen. Ein Hörer der Vorlesung „Spieltheorie” soll in der Lage sein, allgemeine strategische Fragestellungen systematisch zu analysieren und gegebenenfalls Handlungsempfehlungen für konkrete volkswirtschaftliche Entscheidungssituationen (wie kooperatives vs. egoistisches Verhalten) zu geben. (Lehrveranstaltung „Spieltheorie”).  \end{itemize}
\end{learningoutcomes}

\begin{content}
Hauptziel des Moduls ist die Vertiefung der Kenntnisse in verschiedenen Anwendungsgebieten der mikroökonomischen Theorie. Die Teilnehmer sollen die Konzepte und Methoden der mikroökonomischen Analyse zu beherrschen lernen und in die Lage versetzt werden, diese auf reale Probleme anzuwenden.


\end{content}



\end{module}

