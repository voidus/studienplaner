% Lehrveranstaltungsbeschreibung
% Informationsgrad : extern
% Sprache: de
\begin{course}

\setdoclanguagegerman
\coursedegreeprogramme{Informatik}
\coursemodulename{Technische Informatik (S.~\pageref{mod_2409.dp_997})[IN1INTI]}
\courseID{24007}
\coursename{Digitaltechnik und Entwurfsverfahren}
\coursecoordination{T. Asfour, R. Dillmann, U. Hanebeck, J. Henkel, W. Karl}

\documentdate{2011-02-21 11:51:56.306589}

\courselevel{1}
\coursecredits{6}
\courseterm{Wintersemester}
\coursehours{3/1/2}
\courseinstructionlanguage{de}

\coursehead

% For index (key word@display). Key word is used for sorting - no Umlauts please.
\index{Digitaltechnik und Entwurfsverfahren@Digitaltechnik und Entwurfsverfahren}

% For later referencing
\label{cour_7007.dp_997}


\begin{styleenv}
\begin{assessment}
Die Erfolgskontrolle wird in der Modulbeschreibung erläutert.


\end{assessment}

\begin{conditions}Die Lehrveranstaltung Digitaltechnik und Entwurfsverfahren (Technische Informatik II) kann nur mit der Lehrveranstaltung Rechnerorganisation (Technische Informatik I) geprüft werden.

\end{conditions}


\end{styleenv}

\begin{learningoutcomes}
Studierende sollen durch diese Lehrveranstaltung folgende Kompetenzen erwerben:

 \begin{itemize}\item Verständnis der verschiedenen Darstellungsformen von Zahlen und Alphabeten in Rechnern,  \item Fähigkeiten der formalen und programmiersprachlichen Schaltungsbeschreibung,  \item Kenntnisse der technischen Realisierungsformen von Schaltungen,  \item basierend auf dem Verständnis für Aufbau und Funktion aller wichtigen Grundschaltungen und Rechenwerke die Fähigkeit, unbekannte Schaltungen zu analysieren und zu verstehen, sowie eigene Schaltungen zu entwickeln,  \item Kenntnisse der relevanten Speichertechnologien,  \item Kenntnisse verschiedener Realisierungsformen komplexer Schaltungen.  \end{itemize}
\end{learningoutcomes}

\begin{content}
Der Inhalt der Lehrveranstaltung umfasst die Grundlagen der Informationsdarstellung, Zahlensysteme, Binärdarstellungen negativer Zahlen, Gleitkomma-Zahlen, Alphabete, Codes; Rechnertechnologie: MOS-Transistoren, CMOS-Schaltungen; formale Schaltungsbeschreibungen, boolesche Algebra, Normalformen, Schaltungsoptimierung; Realisierungsformen von digitalen Schaltungen: Gatter, PLDs, FPGAs, ASICs; einfache Grundschaltungen: FlipFlop-Typen, Multiplexer, Halb/Voll-Addierer; Rechenwerke: Addierer-Varianten, Multiplizier-Schaltungen, Divisionsschaltungen; Mikroprogrammierung.


\end{content}

\begin{media}Vorlesungsfolien, Aufgabenblätter, Skript.

\end{media}





\end{course}