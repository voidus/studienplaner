% Lehrveranstaltungsbeschreibung
% Informationsgrad : extern
% Sprache: de
\begin{course}

\setdoclanguagegerman
\coursedegreeprogramme{Informatik}
\coursemodulename{Programmierparadigmen (S.~\pageref{mod_2971.dp_997})[IN3INPROGP]}
\courseID{24030}
\coursename{Programmierparadigmen}
\coursecoordination{G. Snelting, R. Reussner}

\documentdate{2010-09-13 11:04:27.631314}

\courselevel{3}
\coursecredits{6}
\courseterm{Wintersemester}
\coursehours{3/1}
\courseinstructionlanguage{de}

\coursehead

% For index (key word@display). Key word is used for sorting - no Umlauts please.
\index{Programmierparadigmen@Programmierparadigmen}

% For later referencing
\label{cour_7363.dp_997}


\begin{styleenv}
\begin{assessment}
Die Erfolgskontrolle wird in der Modulbeschreibung erläutert.


\end{assessment}

\begin{conditions}Die Voraussetzungen werden in der Modulbeschreibung erläutert.

\end{conditions}


\end{styleenv}

\begin{learningoutcomes}
Der/Die Studierenden erlernen

 \begin{itemize}\item Grundlagen und Anwendung von funktionaler Programmierung, Logischer Programmierung, Parallelprogrammierung;   \item elementare Grundlagen des Übersetzerbaus.  \end{itemize}
\end{learningoutcomes}

\begin{content}
Die Teilnehmer sollen nichtimperative Programmierung und ihre Anwendungsgebiete kennenlernen. Im einzelnen werden behandelt:

 \begin{enumerate}\item Funktionale Programmierung - rekursive Funktionen und Datentypen, Funktionen höherer Ordnung, Kombinatoren, lazy Evaluation, lambda-Kalkül, Typsysteme, Anwendungsbeispiele.  \item Logische Programmierung - Terme, Hornklauseln, Unifikation, Resolution, regelbasierte Programmierung, constraint logic programming, Anwendungen.  \item Parallelprogrammierung - message passing, verteilte Software, Aktorkonzept, Anwendungsbeispiele.  \item Elementare Grundlagen des Compilerbaus.  \end{enumerate}

Es werden folgende Programmiersprachen (teils nur kurz) vorgestellt: Haskell, Scala, Prolog, CLP, C++, X10, Java Byte Code.


\end{content}

\begin{media}Vorlesungsfolien, Sekundärliteratur

\end{media}

\begin{literature}Wird in der Vorlesung bekanntgegeben.

\end{literature}



\end{course}