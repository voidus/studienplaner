% Lehrveranstaltungsbeschreibung
% Informationsgrad : extern
% Sprache: de
\begin{course}

\setdoclanguagegerman
\coursedegreeprogramme{Informatik}
\coursemodulename{Datenbanksysteme in Theorie und Praxis (S.~\pageref{mod_14783.dp_997})[IN3INDBSTP]}
\courseID{24317}
\coursename{Arbeiten mit Datenbanksystemen}
\coursecoordination{K. Böhm, Clemens Heidinger}

\documentdate{2011-08-15 14:01:01.455810}

\courselevel{3}
\coursecredits{4}
\courseterm{Wintersemester}
\coursehours{2}
\courseinstructionlanguage{de}

\coursehead

% For index (key word@display). Key word is used for sorting - no Umlauts please.
\index{Arbeiten mit Datenbanksystemen@Arbeiten mit Datenbanksystemen}

% For later referencing
\label{cour_14781.dp_997}


\begin{styleenv}
\begin{assessment}
Die Erfolgskontrolle erfolgt in Form einer “Erfolgskontrolle anderer Art” und besteht aus mehreren Teilaufgaben (Projekten, Experimenten, Vorträgen und Berichten, siehe § 4 Abs. 2 Nr. 3 SPO). Die Veranstaltung wird mit “bestanden” oder “nicht bestanden” bewertet (siehe § 7 Abs. 3 SPO). Zum Bestehen des Praktikums müssen alle Teilaufgaben erfolgreich bestanden werden. Im Falle eines Abbruchs des Praktikums nach der ersten Praktikumssitzung wird dieses mit „nicht bestanden“ bewertet.


\end{assessment}

\begin{conditions}Nachweis von Datenbankkenntnissen durch eine bestandene Prüfung zur Vorlesung „Datenbanksysteme“ oder einer vergleichbaren Veranstaltung.

 

Hinweis: Für Studierende, die an diesem Praktikum für den Bachelor-Studiengang teilgenommen haben, ist eine spätere Teilnahme am Datenbankpraktikum für den Master-Studiengang nicht mehr möglich.

\end{conditions}


\end{styleenv}

\begin{learningoutcomes}
Im Praktikum soll das in Vorlesungen wie “Datenbankeinsatz” und „Datenbanksysteme“ erlernte Wissen in der Praxis erprobt werden. Schrittweise sollen die Programmierung von Datenbankanwendungen, Benutzung von Anfragesprachen sowie Datenbankentwurf für überschaubare Realweltszenarien erlernt werden. Darüber hinaus sollen die Studenten lernen, im Team zusammenzuarbeiten und dabei wichtige Werkzeuge zur Teamarbeit kennenlernen.


\end{learningoutcomes}

\begin{content}
Das Datenbankpraktikum bietet Studierenden einen Einstieg in das Arbeiten mit Datenbanksystemen, als Ergänzung zu den Inhalten der Datenbankvorlesungen. Zunächst werden den Teilnehmern die wesentlichen Bestandteile von Datenbanksystemen in ausgewählten Versuchen mit relationaler Datenbanktechnologie nähergebracht. Sie erproben die klassischen Konzepte des Datenbankentwurfs und von Anfragesprachen an praktischen Beispielen. Darauf aufbauend führen Sie die folgenden Versuche durch:

 \begin{itemize}\item Zugriff auf Datenbanken aus Anwendungsprogrammen heraus,  \item Verwaltung großer Datenbestände interessanter Anwendungsgebiete,  \item Performanceoptimierungen bei der Anfragebearbeitung.  \end{itemize}

Arbeiten im Team ist ein wichtiger Aspekt bei allen Versuchen.


\end{content}







\end{course}