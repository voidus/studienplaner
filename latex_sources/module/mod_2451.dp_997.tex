% Modulbeschreibung 
% Informationsgrad : extern
% Sprache: de
\begin{module}

\setdoclanguagegerman
\moduledegreeprogramme{Informatik (B.Sc.)}
\modulesubject{EF Volkswirtschaftslehre}
\moduleID{IN3WWVWL8}
\modulename{Makroökonomische Theorie}
\modulecoordination{M. Hillebrand}

\documentdate{2012-01-09 17:01:43.636594}

\modulecredits{9}
\moduleduration{2}
\modulecycle{Jedes Semester}



\modulehead

% For index (key word@display). Key word is used for sorting - no Umlauts please.
\index{Makrooekonomische Theorie@Makroökonomische Theorie (M)}

% For later referencing
\label{mod_2451.dp_997}

\begin{courselist}
2520543 & Wachstumstheorie (S.~\pageref{cour_7081.dp_997}) & 2/1 & S & 4,5 & M. Hillebrand\\
25549 & Konjunkturtheorie (Theory of Business Cycles) (S.~\pageref{cour_8375.dp_997}) & 2/1 & W & 4,5 & M. Hillebrand\\
\end{courselist}

\begin{styleenv}
\begin{assessment}
Die Modulprüfung erfolgt in Form von Teilprüfungen (nach §4(2), 1 o. 2 SPO) über die gewählten Lehrveranstaltungen des Moduls, mit denen in Summe die Mindestanforderung an Leistungspunkten erfüllt ist. Die Erfolgskontrolle wird bei jeder Lehrveranstaltung dieses Moduls beschrieben.

 

Die Gesamtnote des Moduls wird aus den mit LP gewichteten Noten der Teilprüfungen gebildet und nach der ersten Nachkommastelle abgeschnitten.


\end{assessment}

\begin{conditions}Das Modul ist nur zusammen mit dem Pflichtmodul \emph{Grundlagen der VWL} [IN3WWVWL] prüfbar.

\end{conditions}

\begin{recommendations}Grundlegende mikro- und makroökonomische Kenntnisse, wie sie beispielsweise in den Veranstaltungen \emph{Volkswirtschaftslehre I (Mikroökonomie)} [2600012] und \emph{Volkswirtschaftslehre II (Makroökonomie)} [2600014] vermittelt werden, werden vorausgesetzt.

 

Aufgrund der inhaltlichen Ausrichtung der Veranstaltung wird ein Interesse an quantitativ-mathematischer Modellierung vorausgesetzt.

\end{recommendations}
\end{styleenv}

\begin{learningoutcomes}
Der/die Studierende

 \begin{itemize}\item beherrscht die grundlegenden Konzepte der makroökonomischen Theorie, insbesondere der dynamischen Gleichgewichtstheorie, und kann diese auf aktuelle politische Fragestellungen, wie beispielsweise Fragen der optimalen Besteuerung, Ausgestaltung von Rentenversicherungssystemen sowie fiskal- und geldpolitische Maßnahmen zur Stabilisierung von Konjunkturzyklen und Wirtschaftswachstum anwenden,  \item kennt die wesentlichen Techniken zur Analyse von intertemporalen makroökonomischen Modellen mit Unsicherheit,  \item beherrscht die dynamischen Gleichgewichtskonzepte, die zur Beschreibung von Preisen und Allokationen auf Güter- und Finanzmärkten sowie deren zeitlicher Entwicklung erforderlich sind,  \item besitzt Kenntnisse bezüglich der grundlegenden Interaktionsmechanismen zwischen Realökonomie und Finanzmärkten.  \end{itemize}
\end{learningoutcomes}

\begin{content}
Hauptziel des Moduls ist die Vertiefung der Kenntnisse der Hörer in Fragestellungen und Konzepte der makroökonomischen Theorie. Die Teilnehmer sollen die Konzepte und Methoden der makroökonomischen Theorie zu beherrschen lernen und in die Lage versetzt werden, makroökonomische Fragestellungen selbstständig beurteilen zu können.


\end{content}



\end{module}

