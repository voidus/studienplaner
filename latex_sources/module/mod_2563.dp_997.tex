% Modulbeschreibung 
% Informationsgrad : extern
% Sprache: de
\begin{module}

\setdoclanguagegerman
\moduledegreeprogramme{Informatik (B.Sc.)}
\modulesubject{}
\moduleID{IN3INSTR}
\modulename{Steuerungstechnik für Roboter }
\modulecoordination{H. Wörn}

\documentdate{2010-06-04 11:33:39.326127}

\modulecredits{3}
\moduleduration{1}
\modulecycle{Jedes 2. Semester, Wintersemester}



\modulehead

% For index (key word@display). Key word is used for sorting - no Umlauts please.
\index{Steuerungstechnik fuer Roboter @Steuerungstechnik für Roboter  (M)}

% For later referencing
\label{mod_2563.dp_997}

\begin{courselist}
24151 & Steuerungstechnik für Roboter  (S.~\pageref{cour_5593.dp_997}) & 2 & W & 3 & H. Wörn\\
\end{courselist}

\begin{styleenv}
\begin{assessment}
Die Erfolgskontrolle erfolgt in Form einer mündlichen Prüfung im Umfang von i.d.R. 30 Minuten nach § 4 Abs. 2 Nr. 2 der SPO.

 

Die Modulnote ist die Note der mündlichen Prüfung.


\end{assessment}

\begin{conditions}Der erfolgreiche Abschluss der folgenden Module wird vorausgesetzt:

 

\emph{Theoretical }\emph{Theoretische Grundlagen der Informatik} [IN2INTHEOG], \emph{Programmieren }[IN1INPROG], \emph{Höhere Mathematik }[IN1INMATHHM] oder \emph{Analysis} [IN1INMATHANA].

\end{conditions}


\end{styleenv}

\begin{learningoutcomes}
Der/Die Studierende soll

 \begin{itemize}\item Bauformen und Komponenten eines Roboters verstehen.  \item grundlegende Verfahren für die Vorwärts- und Rückwärtskinematik, die Bahnplanung, die Bewegungsführung, die Interpolation, die Roboter-Roboter-Kooperation und die achs- und modellbasierte Regelung sowie für die modellbasierte Kalibration kennenlernen und anwenden können.  \item in die Lage versetzt werden, Hard- und Softwarearchitekturen mit Schnittstellen zu Peripherie und zu Sensoren für Roboter zu entwerfen.  \end{itemize}
\end{learningoutcomes}

\begin{content}
Zunächst werden verschiedene Typen von Robotersystemen erläutert und anhand von Beispielen klassifiziert. Es wird auf die möglichen kinematischen Formen eingegangen und die Kenngrößen Freiheitsgrad, kinematische Kette, Arbeitsraum und Traglast eingeführt. Der kinematische Aufbau sowie die Komponenten von Robotern wie Getriebe, Motoren und Wegmeßsysteme werden behandelt. Anhand von Beispielen werden der prinzipielle Aufbau von Greifern und Werkzeugen und eine Übersicht über die verschiedenen Kinematiken gegeben. Ausführlich wird auf die Architektur von Robotersteuerungen eingegangen.

 

Ausgehend von den Kernaufgaben werden Robotersteuerungsarchitekturen vorgestellt. Dies umfasst auf der Hardwareseite insbesondere modulare busbasierte Mehrprozessorsteuerungssysteme und PC-basierte Steuerungssysteme. Softwareseitig werden verschiedene Architekturen basierend auf Echtzeitbetriebssystemen teilweise kombiniert mit PC-Betriebssystemen vorgestellt. Die Bewegungssteuerung von Robotern wird behandelt mit Geschwindigkeitsprofilerzeugung, Interpolation (Linear-, Zirkular-, Splineinterpolation), Transformation und Achsregelung.

 

Ausführlich werden verschiedene Roboterkoordinatensysteme, homogene Transformationen und Framearithmetik sowie Verfahren für die Vorwärts- und Rückwärtstransformation vorgestellt. Anschließend wird auf die Grundkonzepte der Roboterregelung mit PID-Kaskadenregler, modellbasier-tem und adaptivem Roboterregler eingegangen. Es wird eine Einführung in die Roboterdynamik gegeben. Die wesentlichen Programmierverfahren für Roboter werden vorgestellt. Beginnend mit der klassischen Programmierung über Computersprachen, die um Roboterbefehle erweitert sind, werden neue Trends z.B. Icon-Programmierung, Sensorgestützte Programmierung bzw. automatische Offline-Programmierung mit Kollisionsvermeidung behandelt. Ausgehend von den Sensorprinzipien werden unterschiedliche Sensorsysteme für Roboter beispielhaft erläutert und deren Einsatzgebiete aufgezeigt. Neue Anwendungsgebiete von Robotern, z.B. Mensch-Roboter-Kooperation, Chirurgieroboter und Mikroroboter werden erläutert.


\end{content}



\end{module}

