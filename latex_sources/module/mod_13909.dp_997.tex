% Modulbeschreibung 
% Informationsgrad : extern
% Sprache: de
\begin{module}

\setdoclanguagegerman
\moduledegreeprogramme{Informatik (B.Sc.)}
\modulesubject{}
\moduleID{IN3INGO}
\modulename{Geometrische Optimierung}
\modulecoordination{H. Prautzsch}

\documentdate{2011-03-22 10:51:35.117488}

\modulecredits{3}
\moduleduration{3}
\modulecycle{Jedes 2. Semester, Sommersemester}



\modulehead

% For index (key word@display). Key word is used for sorting - no Umlauts please.
\index{Geometrische Optimierung@Geometrische Optimierung (M)}

% For later referencing
\label{mod_13909.dp_997}

\begin{courselist}
24657 & Geometrische Optimierung (S.~\pageref{cour_13907.dp_997}) & 2 & S & 3 & H. Prautzsch\\
\end{courselist}

\begin{styleenv}
\begin{assessment}
Die Erfolgskontrolle erfolgt in Form einer mündlichen Prüfung im Umfang von i.d.R. 15-20 Minuten nach § 4 Abs. 2 Nr. 2 der SPO.

 

Die Modulnote ist die Note der mündlichen Prüfung.


\end{assessment}

\begin{conditions}Keine.\end{conditions}


\end{styleenv}

\begin{learningoutcomes}
Die Hörer und Hörerinnen der Vorlesung sollen Grundlagen der Optimierung bei geometrischen Anwendungsaufgaben kennenlernen


\end{learningoutcomes}

\begin{content}
Siehe Lehrveranstaltungsbeschreibung.


\end{content}



\end{module}

