% Modulbeschreibung 
% Informationsgrad : extern
% Sprache: de
\begin{module}

\setdoclanguagegerman
\moduledegreeprogramme{Informatik (B.Sc.)}
\modulesubject{}
\moduleID{IN3INBP}
\modulename{Basispraktikum zum ICPC-Programmierwettbewerb}
\modulecoordination{D. Wagner}

\documentdate{2012-01-09 09:50:12.660292}

\modulecredits{4}
\moduleduration{1}
\modulecycle{Jedes 2. Semester, Sommersemester}



\modulehead

% For index (key word@display). Key word is used for sorting - no Umlauts please.
\index{Basispraktikum zum ICPC-Programmierwettbewerb@Basispraktikum zum ICPC-Programmierwettbewerb (M)}

% For later referencing
\label{mod_15669.dp_997}

\begin{courselist}
24876 & Basispraktikum zum ICPC Programmierwettbewerb (S.~\pageref{cour_7491.dp_997}) & 4 & S & 4 & D. Wagner, W. Tichy, I. Rutter, Meder, Krug\\
\end{courselist}

\begin{styleenv}
\begin{assessment}
Für den erfolgreichen Abschluss dieses Moduls ist das Bestehen einer Erfolgskontrolle anderer Art nach § 4 Abs. 2 Nr. 3 SPO notwendig. Diese erfolgt kontinuierlich in Form von Programmieraufgaben sowie einem Abschlussvortrag im Umfang von ca. 20 Minuten.

 

Die Bewertung erfolgt mit den Noten “bestanden”/”nicht bestanden”.


\end{assessment}

\begin{conditions}Keine.\end{conditions}

\begin{recommendations}Programmierkenntnisse in C++ oder Java, algorithmische Grundkenntnisse sind wünschenswert.

\end{recommendations}
\end{styleenv}

\begin{learningoutcomes}
Der/Die Studierende soll

 \begin{itemize}\item vertiefte und erweiterte Kompetenzen in den Bereichen Problemanalyse, Softwareentwicklung und Teamarbeit erwerben.  \item die Fähigkeit, in einem vorgegebenen Zeitrahmen zu einer vorgegebenen Aufgabe eine Lösung selbständig erarbeiten und praktiksch umsetzen zu koennen, erwerben.  \end{itemize}
\end{learningoutcomes}

\begin{content}
Der \emph{ACM International Collegiate Programming Contest} (ICPC) ist ein jährlich stattfindender, weltweiter Programmierwettbewerb für Studierende. Der Wettbewerb findet in zwei Runden statt. Im Herbst jedes Jahres treten Teams aus jeweils drei Studierenden, die sich in den ersten vier Jahren ihres Studiums befinden müssen, in weltweit 32 Regional Contests gegeneinander an. Das Gewinnerteam jedes Regionalwettbewerbs hat im Frühjahr des Folgejahres die Möglichkeit, an den \emph{World Finals} teilzunehmen.

 

Im Praktikum werden zu allen für den Wettbewerb relevanten Themengebieten die wichtigsten theoretisch Grundlagen vermittelt und an praktischen Übungsaufgaben erprobt. Höhepunkte des Praktikums sind Local Contests, in denen sich die Praktikumsteilnehmer unter Wettbewerbsbedingungen miteinander messen.

 

Aus den Teilnehmern des Praktikums werden außerdem die Teams ausgewählt, die die Universität Karlsruhe beim \emph{ACM ICPC Regionalwettbewerb der Region Südwesteuropa} (SWERC) im Herbst vertreten werden.


\end{content}

\begin{remarks}Dieses Basispraktikum soll auf den ACM-ICPC Programmierwettberwerb vorbereiten, an welchem Studierende bis zum 9. Hochschulsemester (inklusive) unter der Leitung der Forschungsgruppe Prof. Wagner teilnehmen können.

\end{remarks}

\end{module}

