% Lehrveranstaltungsbeschreibung
% Informationsgrad : extern
% Sprache: de
\begin{course}

\setdoclanguagegerman
\coursedegreeprogramme{Informatik}
\coursemodulename{Insurance Markets and Management (S.~\pageref{mod_1565.dp_997})[IN3WWBWL7]}
\courseID{ INSGAME}
\coursename{Unternehmensplanspiel Versicherungen – INSGAME}
\coursecoordination{U. Werner}

\documentdate{2012-01-22 17:34:57.884894}

\courselevel{}
\coursecredits{3}
\courseterm{Wintersemester}
\coursehours{0/2}
\courseinstructionlanguage{de}

\coursehead

% For index (key word@display). Key word is used for sorting - no Umlauts please.
\index{Unternehmensplanspiel Versicherungen – INSGAME@Unternehmensplanspiel Versicherungen – INSGAME}

% For later referencing
\label{cour_14347.dp_997}


\begin{styleenv}
\begin{assessment}
Die Erfolgskontrolle setzt sich zusammen aus Vorträgen und der aktiven Teilnahme in den konkurrierenden Teilnehmergruppen während der Vorlesungszeit (nach §4 (2), 3 SPO)


\end{assessment}

\begin{conditions}Keine.\end{conditions}


\end{styleenv}

\begin{learningoutcomes}
Der/die Studierende

 \begin{itemize}\item lernt den komplexen Charakter der Produktion von Versicherungsschutz in Abhängigkeit von zufallsbestimmten Schadenereignissen kennen,  \item entscheidet über absatzpolitische Alternativen und Kapitalanlagemöglichkeiten auf Basis von Marktkennzahlen und Jahresabschlussangaben über das eigene Geschäft,  \item verhandelt mit weiteren „Versicherungsunternehmen“ über Rückversicherungsverträge und deren Konditionen,  \end{itemize}

berücksichtigt dabei organisatorische Beschränkungen und die Wettbewerbssituation, welche sich durch den von den Teilnehmergruppen gebildeten Markt und deren Entscheidungen dynamisch verändert.


\end{learningoutcomes}

\begin{content}
Simulation eines (Rück)Versicherungsmarktes und der Wirkungen strategischer Entscheidungen für im Wettbewerb stehende Unternehmen im Rahmen eines mehrperiodigen Planspiels.


\end{content}







\end{course}