% Lehrveranstaltungsbeschreibung
% Informationsgrad : extern
% Sprache: de
\begin{course}

\setdoclanguagegerman
\coursedegreeprogramme{Informatik}
\coursemodulename{Computergraphik (S.~\pageref{mod_4261.dp_997})[IN3INCG]}
\courseID{24081 }
\coursename{Computergraphik}
\coursecoordination{C. Dachsbacher}

\documentdate{2011-11-14 11:28:02.605089}

\courselevel{4}
\coursecredits{6}
\courseterm{Wintersemester}
\coursehours{4}
\courseinstructionlanguage{de}

\coursehead

% For index (key word@display). Key word is used for sorting - no Umlauts please.
\index{Computergraphik@Computergraphik}

% For later referencing
\label{cour_8671.dp_997}


\begin{styleenv}
\begin{assessment}
Die Erfolgskontrolle wird in der Modulbeschreibung erläutert.


\end{assessment}

\begin{conditions}Keine.\end{conditions}


\end{styleenv}

\begin{learningoutcomes}
Die Studierenden sollen grundlegende Konzepte und Algorithmen der Computergraphik verstehen und anwenden lernen. Die erworbenen Kenntnisse ermöglichen einen erfolgreichen Besuch weiterführender Veranstaltungen im Vertiefungsgebiet Computergraphik.


\end{learningoutcomes}

\begin{content}
Grundlegende Algorithmen der Computergraphik, Farbmodelle, Beleuchtungsmodelle, Bildsynthese-Verfahren (Ray Tracing, Rasterisierung), Geometrische Transformationen und Abbildungen, Texturen, Graphik-Hardware und APIs, Geometrisches Modellieren, Dreiecksnetze.


\end{content}



\begin{literature}Wird in der Vorlesung bekanntgegeben.

\end{literature}

\begin{remarks}Diese Vorlesung wird erstmals im WS 10/11 angeboten.

\end{remarks}

\end{course}