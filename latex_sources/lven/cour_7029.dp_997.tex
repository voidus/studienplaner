% Lehrveranstaltungsbeschreibung
% Informationsgrad : extern
% Sprache: de
\begin{course}

\setdoclanguagegerman
\coursedegreeprogramme{Informatik}
\coursemodulename{Analysis (S.~\pageref{mod_2415.dp_997})[IN1MATHANA]}
\courseID{01501}
\coursename{Analysis 2}
\coursecoordination{R. Schnaubelt, Plum, Reichel, Weis}

\documentdate{2008-07-25 13:35:53}

\courselevel{1}
\coursecredits{9}
\courseterm{Sommersemester}
\coursehours{4/2/2}
\courseinstructionlanguage{de}

\coursehead

% For index (key word@display). Key word is used for sorting - no Umlauts please.
\index{Analysis 2@Analysis 2}

% For later referencing
\label{cour_7029.dp_997}


\begin{styleenv}
\begin{assessment}
Die Erfolgskontrolle wird in der Modulbeschreibung erläutert.


\end{assessment}

\begin{conditions}Keine.\end{conditions}


\end{styleenv}

\begin{learningoutcomes}
Die Studierenden sollen am Ende des Moduls

 \begin{itemize}\item den Übergang von der Schule zur Universität bewältigt haben,  \item mit logischem Denken und strengen Beweisen vertraut sein,  \item die Grundlagen der Differential- und Integralrechnung von Funktionen einer reellen Variablen und der Differentialrechnung von Funktionen in mehreren Variablen beherrschen.  \end{itemize}
\end{learningoutcomes}

\begin{content}
Normierte Vektorräume und topologische Grundbegriffe, Fixpunktsatz von Banach. Mehrdimensionale Differentiation (lineare Approximation, partielle Ableitungen, Satz von Schwarz), Satz von Taylor, Umkehrsatz, implizit definierte Funktionen, Extrema ohne/mit Nebenbedingungen. Kurvenintegral, Wegunabhängigkeit. Iterierte Riemannintegrale, Volumenberechnung. Einführung in gewöhnliche Differentialgleichungen: Trennung der Variablen, Satz von Picard und Lindelöf, Systeme linearer Differentialgleichungen und ihre Stabilität.


\end{content}



\begin{literature}Wird in der Vorlesung bekannt gegeben.

\end{literature}



\end{course}