% Modulbeschreibung 
% Informationsgrad : extern
% Sprache: de
\begin{module}

\setdoclanguagegerman
\moduledegreeprogramme{Informatik (B.Sc.)}
\modulesubject{EF Mathematik}
\moduleID{IN3MATHAN03}
\modulename{Differentialgleichungen und Hilberträume}
\modulecoordination{R. Schnaubelt}

\documentdate{2011-10-06 18:26:16.512407}

\modulecredits{9}
\moduleduration{1}
\modulecycle{Jedes 2. Semester, Sommersemester}



\modulehead

% For index (key word@display). Key word is used for sorting - no Umlauts please.
\index{Differentialgleichungen und Hilbertraeume@Differentialgleichungen und Hilberträume (M)}

% For later referencing
\label{mod_3295.dp_997}

\begin{courselist}
1566 & Differentialgleichungen und Hilberträume (S.~\pageref{cour_8009.dp_997}) & 4/2 & S & 9 & G. Herzog, M. Plum, W. Reichel, C. Schmoeger, R. Schnaubelt, L. Weis\\
\end{courselist}

\begin{styleenv}
\begin{assessment}
Prüfung: schriftliche oder mündliche Prüfung\newline
Notenbildung: Note der Prüfung


\end{assessment}

\begin{conditions}Das Modul \emph{Proseminar Mathematik} [IN3MATHPS] muss geprüft werden.

 

Das Modul muss mit dem Modul \emph{Funktionalanalysis} [IN3MATHAN05] oder mit dem Modul \emph{Analysis 3} [IN3MATHAN02] geprüft werden.

\end{conditions}

\begin{recommendations}Es wird empfohlen, das Modul \emph{Funktionalanaly}sis [IN3MATH05] zu kombinieren und das Modul \emph{Funktionalanalysis }zuerst zu belegen.

\end{recommendations}
\end{styleenv}

\begin{learningoutcomes}
Vertieftes Verständnis analytischer Konzepte und Methoden


\end{learningoutcomes}

\begin{content}
\begin{itemize}\item Modellierung mit Differentialgleichungen  \item Erste Integrale  \item Phasenebene  \item Stabilität  \item Prinzip der linearisierten Stabilität  \item Randwertprobleme  \item Greensche Funktionen  \item Lösungsmethoden für elementare partielle Differentialgleichungen  \item Hilbert- und Banachräume und stetige lineare Operatoren  \item Grundbegriffe der Sobolevräume  \item Orthonormalbasen und Orthogonalprojektionen  \item Darstellungssätze von Riesz-Fischer und Lax-Milgram  \item Dirichletproblem als Variationsproblem  \item Spektralsatz für kompakte und selbstadjungierte Operatoren  \end{itemize}
\end{content}



\end{module}

