% Lehrveranstaltungsbeschreibung
% Informationsgrad : extern
% Sprache: de
\begin{course}

\setdoclanguagegerman
\coursedegreeprogramme{Informatik}
\coursemodulename{Seminarmodul Recht (S.~\pageref{mod_4185.dp_997})[IN3JURASEM]}
\courseID{AFDsem}
\coursename{Seminar: Aktuelle Fragen des Datenschutzrechts}
\coursecoordination{I. Spiecker genannt Döhmann}

\documentdate{2011-11-14 11:19:36.683128}

\courselevel{3}
\coursecredits{2}
\courseterm{Sommersemester}
\coursehours{2}
\courseinstructionlanguage{de}

\coursehead

% For index (key word@display). Key word is used for sorting - no Umlauts please.
\index{Seminar: Aktuelle Fragen des Datenschutzrechts@Seminar: Aktuelle Fragen des Datenschutzrechts}

% For later referencing
\label{cour_8501.dp_997}


\begin{styleenv}
\begin{assessment}
Die Erfolgskontrolle erfolgt durch Ausarbeiten einer schriftlichen Seminararbeit sowie der Präsentation und Diskussion derselbigen als Erfolgskontrolle anderer Art nach § 4 Abs. 2 Nr. 3 SPO.

 

Gewichtung: 55\% Seminararbeit, 25\% Präsentation, 20\% Diskussionsbeiträge zu anderen Beiträgen


\end{assessment}

\begin{conditions}Keine.\end{conditions}

\begin{recommendations}Parallel zu den Veranstaltungen werden begleitende Tutorien angeboten, die insbesondere der Vertiefung der juristischen Arbeitsweise dienen. Ihr Besuch wird nachdrücklich empfohlen.\newline
Während des Semesters wird eine Probeklausur zu jeder Vorlesung mit ausführlicher Besprechung gestellt. Außerdem wird eine Vorbereitungsstunde auf die Klausuren in der vorlesungsfreien Zeit angeboten.\newline
Details dazu auf der Homepage des ZAR (www.kit.edu/zar).

\end{recommendations}
\end{styleenv}

\begin{learningoutcomes}
Aktuelle Entwicklungen des nationalen und europäischen Datenschutzrechts werden in Seminararbeiten wissenschaftlich erarbeitet und dann präsentiert.


\end{learningoutcomes}

\begin{content}
Aktuelle Entwicklungen des nationalen und europäischen Datenschutzrechts


\end{content}

\begin{media}Ausführliches Skript mit Fällen, Gliederungsübersichten, Unterlagen in den Veranstaltungen.

\end{media}

\begin{literature}Wird bekanntgegeben.

 

\textbf{Weiterführende Literatur:}

 

Wird bekanntgegeben.

\end{literature}

\begin{remarks}Diese Lehrveranstaltung wird zurzeit nicht angeboten.

\end{remarks}

\end{course}