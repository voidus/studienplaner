% Lehrveranstaltungsbeschreibung
% Informationsgrad : extern
% Sprache: de
\begin{course}

\setdoclanguagegerman
\coursedegreeprogramme{Informatik}
\coursemodulename{Insurance Markets and Management (S.~\pageref{mod_1565.dp_997})[IN3WWBWL7]}
\courseID{2530353}
\coursename{International Risk Transfer}
\coursecoordination{W. Schwehr}

\documentdate{2012-01-22 17:20:51.637052}

\courselevel{4}
\coursecredits{2,5}
\courseterm{Sommersemester}
\coursehours{2/0}
\courseinstructionlanguage{de}

\coursehead

% For index (key word@display). Key word is used for sorting - no Umlauts please.
\index{International Risk Transfer@International Risk Transfer}

% For later referencing
\label{cour_5121.dp_997}


\begin{styleenv}
\begin{assessment}
Die Erfolgskontrolle erfolgt in Form einer schriftlichen Prüfung (nach §4(2), 1 SPO). Die Prüfung wird in jedem Semester angeboten und kann zu jedem ordentlichen Prüfungstermin wiederholt werden.


\end{assessment}

\begin{conditions}Keine.\end{conditions}


\end{styleenv}

\begin{learningoutcomes}
Hintergründe und Funktionsweisen verschiedener Möglichkeiten internationalen Risikotransfers verstehen lernen.


\end{learningoutcomes}

\begin{content}
Wie werden potentielle Schäden größeren Ausmaßes finanziert bzw. global getragen/umverteilt? Traditionell sind hier Erst- und vor allem Rückversicherer weltweit aktiv, Lloyd's of London ist eine Drehscheibe für internationale Risiken, globale Industrieunternehmen bauen Captives zur Selbstversicherung auf, für bisher als schwer versicherbar geltende Risiken (z.B. Wetterrisiken) entwickeln die Versicherungs- und Kapitalmärkte innovative Lösungen. Die Vorlesung beleuchtet Hintergründe und Funktionsweisen dieser verschiedenen Möglichkeiten internationalen Risiko Transfers.


\end{content}



\begin{literature}\begin{itemize}\item P. Liebwein. Klassische und moderne Formen der Rückversicherung. Karlsruhe 2000.  \item Brühwiler/ Stahlmann/ Gottschling. Innovative Risikofinanzierung - Neue Wege im Risk Management. Wiesbaden 1999.  \item Becker/ Bracht. Katastrophen- und Wetterderivate. . Finanzinnovationan auf der Basis von Naturkatastrophen und Wettererscheinungen, Wien 1999  \end{itemize}\end{literature}

\begin{remarks}Blockveranstaltung, aus organisatorischen Gründen ist eine Anmeldung erforderlich im Sekretariat des Lehrstuhls: thomas.mueller3@kit.edu.

\end{remarks}

\end{course}