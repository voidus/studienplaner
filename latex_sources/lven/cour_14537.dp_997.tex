% Lehrveranstaltungsbeschreibung
% Informationsgrad : extern
% Sprache: de
\begin{course}

\setdoclanguagegerman
\coursedegreeprogramme{Informatik}
\coursemodulename{Web-Anwendungen (S.~\pageref{mod_14533.dp_997})[IN3INWA], Web-Anwendungen und Praxis (S.~\pageref{mod_14535.dp_997})[IN3INWAP]}
\courseID{24153}
\coursename{Web-Anwendungen und Serviceorientierte Architekturen (I)}
\coursecoordination{S. Abeck}

\documentdate{2012-01-12 12:03:00.430873}

\courselevel{3}
\coursecredits{4}
\courseterm{Wintersemester}
\coursehours{2/0}
\courseinstructionlanguage{de}

\coursehead

% For index (key word@display). Key word is used for sorting - no Umlauts please.
\index{Web-Anwendungen und Serviceorientierte Architekturen (I)@Web-Anwendungen und Serviceorientierte Architekturen (I)}

% For later referencing
\label{cour_14537.dp_997}


\begin{styleenv}
\begin{assessment}
Die Erfolgskontrolle erfolgt in Form einer mündlichen Prüfung im Umfang von i.d.R. 20 Minuten nach § 4 Abs. 2 Nr. 2 SPO.

 

Die Zulassung zur Prüfung erfolgt nur bei nachgewiesener Mitarbeit an den in der Vorlesung gestellten praktischen Aufgaben.


\end{assessment}

\begin{conditions}Keine.\end{conditions}


\end{styleenv}

\begin{learningoutcomes}
\begin{itemize}\item Die wichtigsten den Stand der Technik repräsentierenden Technologien und Standards zur Entwicklung von traditionellen Web-Anwendungen sind bekannt und können genutzt werden.  \item Die Architektur von traditionellen und dienstorientierten Web-Anwendungen ist verstanden.  \item Die Softwarearchitektur einer Web-Anwendung kann modelliert werden.  \item Die wichtigsten Prinzipien traditioneller und dienstorientierter Softwareentwicklung und des entsprechenden Entwicklungsprozesses sind bekannt.  \item Die Technologien und Werkzeuge können zur Entwicklung von Beispielszenarien angewendet werden.  \end{itemize}
\end{learningoutcomes}

\begin{content}
Das Internet als Verteilungsplattform und die darauf basierenden Webtechnologien spielen eine große Rolle bei der Entwicklung verteilter Anwendungssysteme. Traditionelle Webanwendungen nutzen standardisierte Technologien zur Kommunikation (u.a. HTTP, TCP) und zur Informationsbeschreibung (u.a. HTML, XML), die in der Vorlesung an einer durchgängigen Beispiel-Anwendung aufgezeigt werden. Fortgeschrittene Webanwendungen folgen dem Paradigma der Dienstorientierung, indem diese Funktionalität in Form von Webservices über das Internet bereitstellen. Die Webservice-Technologie und die dazu bestehenden wichtigsten Standards werden eingeführt und deren Einsatz wird anhand des Beispiels aufgez


\end{content}

\begin{media}Vorlesungsfolien, Skript

\end{media}

\begin{literature}\begin{itemize}\item Bernd Bruegge, Allen H. Dutoit: Object-Oriented Software Engineering Using UML, Patterns and Java, Pearson Prentice Hall, 2004.  \item Y. Daniel Liang: Introdcution to Java Programming; Companion Website: www.prenhall.com/liang, Pearson Prentice Hall, 2005.  \item James F. Kurose, Keith W. Ross: Computer Networking – A Top-down Approach Featuring the Internet, 2nd Edition, Addison Wesley, 2003.  \end{itemize}\end{literature}



\end{course}