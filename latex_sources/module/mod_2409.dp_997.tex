% Modulbeschreibung 
% Informationsgrad : extern
% Sprache: de
\begin{module}

\setdoclanguagegerman
\moduledegreeprogramme{Informatik (B.Sc.)}
\modulesubject{Technische Informatik}
\moduleID{IN1INTI}
\modulename{Technische Informatik}
\modulecoordination{W. Karl}

\documentdate{2011-02-17 10:40:56.068568}

\modulecredits{12}
\moduleduration{2}
\modulecycle{Jedes 2. Semester, Sommersemester}



\modulehead

% For index (key word@display). Key word is used for sorting - no Umlauts please.
\index{Technische Informatik@Technische Informatik (M)}

% For later referencing
\label{mod_2409.dp_997}

\begin{courselist}
24502 & Rechnerorganisation (S.~\pageref{cour_7005.dp_997}) & 3/1/2 & S & 6 & T. Asfour, R. Dillmann, J. Henkel, W. Karl\\
24007 & Digitaltechnik und Entwurfsverfahren (S.~\pageref{cour_7007.dp_997}) & 3/1/2 & W & 6 & T. Asfour, R. Dillmann, U. Hanebeck, J. Henkel, W. Karl\\
\end{courselist}

\begin{styleenv}
\begin{assessment}
Die Erfolgskontrolle erfolgt in Form einer schriftlichen Gesamtprüfung (120 Minuten) nach § 4 Abs. 2 Nr. 1 SPO über die Lehrveranstaltungen \emph{Rechnerorganisation} und \emph{Digitaltechnik und Entwurfsverfahren}.

 

Die Modulnote ist die Note der Klausur.

 

Besonderheit: In beiden Lehrveranstaltungen werden Zwischenprüfungen angeboten, in denen jeweils bis zu drei Bonuspunkte erarbeitet werden können. Die Bonuspunkte werden zur Notenverbesserung für eine bestandene Prüfung verwendet. Die Teilnahme ist freiwillig.


\end{assessment}

\begin{conditions}Keine.\end{conditions}

\begin{recommendations}Es wird empfohlen, das Modul nach dem Modul \emph{Grundbegriffe der Informatik} [IN1INGI] abzulegen.

\end{recommendations}
\end{styleenv}

\begin{learningoutcomes}
Studierende sollen durch dieses Modul folgende Kompetenzen erwerben:

 \begin{itemize}\item Verständnis der verschiedenen Darstellungsformen von Zahlen und Alphabeten in Rechnern,  \item Fähigkeiten der formalen und programmiersprachlichen Schaltungsbeschreibung,  \item Kenntnisse der technischen Realisierungsformen von Schaltungen,  \item basierend auf dem Verständnis für Aufbau und Funktion aller wichtigen Grundschaltungen und Rechenwerke die Fähigkeit, unbekannte Schaltungen zu analysieren und zu verstehen, sowie eigene Schaltungen zu entwickeln,  \item Kenntnisse der relevanten Speichertechnologien,  \item Verständnis verschiedener Realisierungsformen komplexer Schaltungen,  \item Verständnis über den Aufbau, die Organisation und das Operationsprinzip von Rechnersystemen,  \item den Zusammenhang zwischen Hardware-Konzepten und den Auswirkungen auf die Software zu verstehen, um effiziente Programme erstellen zu können,   \item aus dem Verständnis über die Wechselwirkungen von Technologie, Rechnerkonzepten und Anwendungen die grundlegenden Prinzipien des Entwurfs nachvollziehen und anwenden zu können,  \item einen Rechner aus Grundkomponenten aufbauen zu können.  \end{itemize}
\end{learningoutcomes}

\begin{content}
Das Modul vermittelt eine systematische Heranführung an die Technische Informatik. Sie beinhalten neben den Grundlagen der Mikroelektronik den Entwurf und den Aufbau von einfachen informationsverarbeitenden Systemen, logischen Schaltnetzen und Schaltwerken bis hin zum funktionellen Aufbau digitaler Rechenanlagen. Die Inhalte umfassen:

 \begin{itemize}\item Informationsdarstellung, Zahlensysteme, Binärdarstellungen negativer Zahlen, Gleitkomma-Zahlen, Alphabete, Codes  \item Rechnertechnologie: MOS-Transistoren, CMOS-Schaltungen  \item Formale Schaltungsbeschreibungen, boolesche Algebra, Normalformen, Schaltungsoptimierung  \item Realisierungsformen von digitalen Schaltungen: Gatter, PLDs, FPGAs, ASICs  \item Einfache Grundschaltungen: FlipFlop-Typen, Multiplexer, Halb/Voll-Addierer  \item Rechenwerke: Addierer-Varianten, Multiplizier-Schaltungen Divisionsschaltungen  \item Mikroprogramierung  \item Grundlagen des Aufbaus und der Organisation von Rechnern  \item Befehlssatzarchitektur, Diskussion RISC – CISC  \item Pipelining des Maschinenbefehlszyklus, Pipeline-Hemmnisse, Methoden zur Auflösung von Pipeline-Konflikten  \item Speicherkomponenten, Speicherorganisation, Cache-Speicher  \item Ein-/Ausgabe-System, Schnittstellen, Interrupt-Verarbeitung  \item Bus-Systeme  \item Unterstützung von Betriebssystemfunktionen: virtuelle Speicherverwaltung, Schutzfunktionen  \end{itemize}
\end{content}



\end{module}

