% Lehrveranstaltungsbeschreibung
% Informationsgrad : extern
% Sprache: de
\begin{course}

\setdoclanguagegerman
\coursedegreeprogramme{Informatik}
\coursemodulename{Insurance Markets and Management (S.~\pageref{mod_1565.dp_997})[IN3WWBWL7]}
\courseID{2530350}
\coursename{Current Issues in the Insurance Industry}
\coursecoordination{W. Heilmann}

\documentdate{2012-01-22 17:24:06.804039}

\courselevel{4}
\coursecredits{2,5}
\courseterm{Sommersemester}
\coursehours{2/0}
\courseinstructionlanguage{de}

\coursehead

% For index (key word@display). Key word is used for sorting - no Umlauts please.
\index{Current Issues in the Insurance Industry@Current Issues in the Insurance Industry}

% For later referencing
\label{cour_6373.dp_997}


\begin{styleenv}
\begin{assessment}
Die Erfolgskontrolle erfolgt in Form einer schriftlichen Prüfung (nach §4(2), 1 SPO). Die Prüfung wird in jedem Semester angeboten und kann zu jedem ordentlichen Prüfungstermin wiederholt werden.


\end{assessment}

\begin{conditions}Keine.\end{conditions}

\begin{recommendations}Für das Verständnis von der Lehrveranstaltung ist die Kenntnis des Stoffes von \emph{Private and Social Insurance} [2530050] Voraussetzung.

\end{recommendations}
\end{styleenv}

\begin{learningoutcomes}
Lernziel ist das Kennenlernen und Verstehen wichtiger (und möglichst aktueller) Besonderheiten des Versicherungswesens, z.B. Versicherungsmärkte, -sparten, -produkte, Kapitalanlage, Betriebliche Altersversorgung, Organisation und Controlling.


\end{learningoutcomes}

\begin{content}
Wechselnde Inhalte zu aktuellen Fragestellungen.


\end{content}



\begin{literature}\textbf{Weiterführende Literatur:}

 

Farny, D. Versicherungsbetriebslehre. Verlag Versicherungswirtschaft; Auflage: 5. 2011

 

Koch, P. Versicherungswirtschaft - Ein einführender Überblick. Verlag Versicherungswirtschaft. 2005

 

Tonndorf, F., Horn, G., and Bohner, N. Lebensversicherung von A-Z. Verlag Versicherungswirtschaft. 1999

 

Fürstenwerth, J., andWeiß, A. Versicherungsalphabet (VA). Verlag Versicherungswirtschaft. 2001

 

Buttler, A. Einführung in die betriebliche Altersversorgung. Verlag Versicherungswirtschaft. 2008

 

Liebwein, P. Klassische und moderne Formen der Rückversicherung. Verlag Versicherungswirtschaft. 2009

 

Gesamtverband der Deutschen Versicherungswirtschaft. \emph{Jahrbuch 2011 }\emph{Die deutsche Versicherungswirtschaft.} \newline
http://www.gdv.de/wp-content/uploads/2011/11/GDV\_Jahrbuch\_2011.pdf. 2011

 

Deutsch, E. Das neue Versicherungsvertragsrecht. Verlag Versicherungswirtschaft. 2008

 

Schwebler, Knauth, Simmert. Kapitalanlagepolitik im Versicherungsbinnenmarkt. 1994

 

Seng. Betriebliche Altersversorgung. 1995\newline
von Treuberg, Angermayer. Jahresabschluss von Versicherungsunternehmen. 1995

\end{literature}

\begin{remarks}Blockveranstaltung; aus organisatorischen Gründen ist eine Anmeldung erforderlich beithomas.mueller3@kit.edu(Sekretariat des Lehrstuhls).

\end{remarks}

\end{course}