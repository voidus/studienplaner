% Modulbeschreibung 
% Informationsgrad : extern
% Sprache: de
\begin{module}

\setdoclanguagegerman
\moduledegreeprogramme{Informatik (B.Sc.)}
\modulesubject{EF Mathematik}
\moduleID{IN3MATHAG05}
\modulename{Algebra}
\modulecoordination{F. Herrlich}

\documentdate{2011-10-06 17:47:54.572231}

\modulecredits{9}
\moduleduration{1}
\modulecycle{Jedes 2. Semester, Wintersemester}



\modulehead

% For index (key word@display). Key word is used for sorting - no Umlauts please.
\index{Algebra@Algebra (M)}

% For later referencing
\label{mod_3107.dp_997}

\begin{courselist}
1031 & Algebra (S.~\pageref{cour_7973.dp_997}) & 4/2 & W & 9 & F. Herrlich, S. Kühnlein, C. Schmidt, G. Weitze-Schmithüsen\\
\end{courselist}

\begin{styleenv}
\begin{assessment}
Prüfung: schriftliche oder mündliche Prüfung\newline
Notenbildung: Note der Prüfung


\end{assessment}

\begin{conditions}Das Modul \emph{Einführung in die Algebra und Zahlentheorie}, \emph{Einführung in die Geometrie und Topologie} oder \emph{Riemannsche Geometrie} muss geprüft werden.

 

Das Modul \emph{Proseminar Mathematik} [IN3MATHPS] muss geprüft werden.

\end{conditions}

\begin{recommendations}Kenntnisse aus \emph{Einführung in die Algebra werden} vorausgesetzt.

\end{recommendations}
\end{styleenv}

\begin{learningoutcomes}
\begin{itemize}\item Konzepte und Methoden der Algebra  \item Vorbereitung auf Seminare und weiterführende Vorlesungen im Bereich Algebraische Geometrie und Zahlentheorie  \end{itemize}
\end{learningoutcomes}

\begin{content}
\begin{itemize}\item Körper: \newline
Körpererweiterungen, Galoistheorie, Einheitswurzeln und Kreisteilung  \item Bewertungen: \newline
Beträge, Bewertungsringe, Betragsfortsetzung, lokale Körper  \item Dedekindringe: \newline
ganze Ringerweiterungen, Normalisierung, noethersche Ringe  \end{itemize}
\end{content}



\end{module}

