% Lehrveranstaltungsbeschreibung
% Informationsgrad : extern
% Sprache: de
\begin{course}

\setdoclanguagegerman
\coursedegreeprogramme{Informatik}
\coursemodulename{Robotik in der Medizin (S.~\pageref{mod_8829.dp_997})[IN3INROBM]}
\courseID{24681}
\coursename{Robotik in der Medizin }
\coursecoordination{H. Wörn, Raczkowsky}

\documentdate{2010-06-07 11:38:20.372516}

\courselevel{4}
\coursecredits{3}
\courseterm{Sommersemester}
\coursehours{2}
\courseinstructionlanguage{de}

\coursehead

% For index (key word@display). Key word is used for sorting - no Umlauts please.
\index{Robotik in der Medizin @Robotik in der Medizin }

% For later referencing
\label{cour_7135.dp_997}


\begin{styleenv}
\begin{assessment}
Die Erfolgskontrolle wird in der Modulbeschreibung näher erläutert.


\end{assessment}

\begin{conditions}Keine.\end{conditions}


\end{styleenv}

\begin{learningoutcomes}
Der Student soll die spezifischen Anforderungen der Chirurgie an die Automatisierung mit Robotern verstehen. Zusätzlich soll er grundlegende Verfahren für die Registrierung von Bilddaten unterschiedlicher Modalitäten und die physikalische mit ihren verschiedenen Flexibilisierungsstufen kennenlernen und anwenden können. Der Student soll in die Lage versetzt werden, den kompletten Workflow für einen robotergestützten Eingriff zu entwerfen.


\end{learningoutcomes}

\begin{content}
Zur Motivation werden die verschiedenen Szenarien des Robotereinsatzes im chirurgischen Umfeld erläutert und anhand von Beispielen klassifiziert. Es wird auf Grundlagen der Robotik mit den verschiedenen kinematischen Formen eingegangen und die Kenngrößen Freiheitsgrad, kinematische Kette, Arbeitsraum und Traglast eingeführt. Danach werden die verschiedenen Module der Prozesskette für eine robotergestützte Chirurgie vorgestellt. Diese beginnt mit der Bildgebung $\pi{}$, mit den verschiedenen tomographischen Verfahren. Sie werden anhand der physikalischen Grundlagen und ihrer meßtechnischen Aussagen zur Anatomie und Pathologie erläutert. In diesem Kontext spielen die Datenformate und Kommunikation eine wesentliche Rolle. Die medizinische Bildverarbeitung mit Schwerpunkt auf Segmentierung schliesst sich an. Dies führt zur geometrischen 3D-Rekonstruktion anatomischer Strukturen, die die Grundlage für ein attributiertes Patientenmodell bilden. Dazu werden die Methoden für die Registrierung der vorverarbeiteten Meßdaten aus verschiedenen tomographischen Modalitäten beschrieben. Die verschiedenen Ansätze für die Modellierung von Gewebeparametern ergänzen die Ausführungen zu einem vollständigen Patientenmodell. Die Anwendungen des Patientenmodells in der Visualisierung und Operationsplanung ist das nächste Thema. Am Begriff der Planung wird die sehr unterschiedliche Sichtweise von Medizinern und Ingenieuren verdeutlicht. Neben der geometrischen Planung wird die Rolle der Ablaufplanung erarbeitet, die im klinischen Alltag immer wichiger wird. Im wesentlichen unter dem Gesichtspunkt der Verifikation der Operationsplanung wird das Thema Simulation behandelt. Unterthemen sind hierbei die funktionale anatomiebezogene Simulation, die Robotersimulation mit Standortverifikation sowie Trainingssysteme. Der intraoperative Teil der Prozesskette beinhaltet die Registrierung, Navigation, Erweiterte Realität und Chirurgierobotersysteme. Diese werden mit Grundlagen und Anwendungsbeispielen erläutert. Als wichtige Punkte werden hier insbesondere Techniken zum robotergestützten Gewebeschneiden und die Ansätze zu Mikro- und Nanochirurgie behandelt. Die Vorlesung schliesst mit einem kurzen Diskurs zu den speziellen Sicherheitsfragen und den rechtlichen Aspekten von Medizinprodukten.


\end{content}

\begin{media}PowerPoint-Folien als pdf im Internet

\end{media}

\begin{literature}\textbf{Weiterführende Literatur:}

 

- Springer Handbook of Robotics, Siciliano, Bruno; Khatib, Oussama (Eds.) 2008, LX, 1611 p. 1375 illus., 422 in color. With DVD., Hardcover, ISBN:978-3-540-23957-4

 

- Heinz Wörn, Uwe Brinkschulte “Echtzeitsysteme”, Springer, 2005, ISBN: 3-540-20588-8

 

- Proccedings of Medical image computing and computer-assisted intervention (MICCAI ab 2005)

 

- Proccedings of Computer assisted radiologiy and surgery (CARS ab 2005)

 

- Tagungsbände Bildverarbeitung für die Medizin (BVM ab 2005)

\end{literature}



\end{course}