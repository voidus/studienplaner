% Lehrveranstaltungsbeschreibung
% Informationsgrad : extern
% Sprache: de
\begin{course}

\setdoclanguagegerman
\coursedegreeprogramme{Informatik}
\coursemodulename{Konzepte und Anwendungen von Workflowsystemen (S.~\pageref{mod_10173.dp_997})[IN3INKAW]}
\courseID{24111}
\coursename{Konzepte und Anwendungen von Workflowsystemen}
\coursecoordination{J. Mülle, Silvia von Stackelberg}

\documentdate{2011-08-23 16:30:52.470354}

\courselevel{4}
\coursecredits{5}
\courseterm{Wintersemester}
\coursehours{3}
\courseinstructionlanguage{de}

\coursehead

% For index (key word@display). Key word is used for sorting - no Umlauts please.
\index{Konzepte und Anwendungen von Workflowsystemen@Konzepte und Anwendungen von Workflowsystemen}

% For later referencing
\label{cour_10171.dp_997}


\begin{styleenv}
\begin{assessment}
Es wird im Voraus angekündigt, ob die Erfolgskontrolle in Form einer schriftlichen Prüfung (Klausur) im Umfang von 1h nach § 4, Abs. 2 Nr. 1 SPO oder in Form einer mündlichen Prüfung im Umfang von 20 min. nach § 4 Abs. 2 Nr. 2 SPO stattfindet.


\end{assessment}

\begin{conditions}Keine.\end{conditions}

\begin{recommendations}Datenbankkenntnisse, z.B. aus der Vorlesung \emph{Datenbanksysteme} [24516].

\end{recommendations}
\end{styleenv}

\begin{learningoutcomes}
Am Ende des Kurses sollen die Teilnehmer in der Lage sein, Workflows zu modellieren, die Modellierungsaspekte und ihr Zusammenspiel zu erläutern, Modellierungsmethoden miteinander zu vergleichen und ihre Anwendbarkeit in unterschiedlichen Anwendungsbereichen einzuschätzen. Sie sollten den technischen Aufbau eines Workflow-Management-Systems mit den wichtigsten Komponenten kennen und verschiedene Architekturen bewerten können. Schließlich sollten die Teilnehmer einen Einblick in die aktuellen relevanten Standards und in den Stand der Forschung durch aktuelle Forschungsthemen gewonnen haben.


\end{learningoutcomes}

\begin{content}
Workflow-Management-Systeme (WFMS) unterstützen die Abwicklung von Geschäftsprozessen entsprechend vorgegebener Arbeitsabläufe. Immer wichtiger wird die Unterstützung von Abläufen im Service-orientierten Umfeld.

 \begin{itemize}\item Die Vorlesung beginnt mit der Einordnung von WFMS in betriebliche Informationssysteme und stellt den Zusammenhang mit der Geschäftsprozessmodellierung her.  \item Es werden formale Grundlagen für WFMS eingeführt (Petri- Netze, Pi-Kalkül).  \item Modellierungsmethoden für Workflows und der Entwicklungsprozess von Workflow-Management-Anwendungen werden vorgestellt und in Übungen vertieft.  \item Insbesondere der Einsatz von Internettechniken speziell von Web Services und Standardisierungen für Prozessmodellierung, Orchestrierung und Choreographie werden in diesem Kontext vorgestellt.  \item Im Teil Realisierung von Workflow-Management-Systemen werden verschiedene Architekturen sowie Systemtypen und beispielhaft konkrete Systeme behandelt.  \item Weiterhin wird auf anwendungsgetriebene Vorgehensweisen zur Änderung von Workflows, speziell Geschäftsprozess-Reengineering und kontinuierliche Prozessverbesserung eingegangen.  \item  Abschließend werden Ergebnisse aus aktuellen Forschungsrichtungen, wie Methoden und Konzepte zur Unterstützung flexibler, adaptiver Workflows, Security für Workflows und Prozess-Mining behandelt.  \end{itemize}
\end{content}

\begin{media}Vorlesungsfolien.

\end{media}

\begin{literature}\textbf{Pflichtliteratur}

 \begin{itemize}\item Matthias Weske: Business Process Management. Springer, 2007  \item Frank Leymann, Dieter Roller: Production Workflows - Concepts and Techniques. Prentice-Hall, 2000  \item W.M.P. van der Aalst: Workflow Management: Models, Methods, and Systems. MIT Press, 368 pp., 2002  \item W.M.P. van der Aalst: Workflow Management: Models, Methods, and Systems. MIT Press, 368 pp., \$40.00, ISBN 0-262-01189-1, 2002  \item Michael Havey: Essential Business Process Modeling. O´Reilly Media, Inc., 2005  \item S. Jablonski, M. Böhm, W. Schulze (Hrsg.): Workflow-Management - Entwicklung von Anwendungen und Systemen. dpunkt-Verlag, Heidelberg, 1997  \end{itemize}

\textbf{Ergänzungsliteratur}

 

Weitere aktuelle Angaben in den Folien am Ende eines jeden Kapitels.

\end{literature}

\begin{remarks}\textcolor{red}{Diese Lehrveranstaltung wird im Bachelor-Studiengang Informatik nicht mehr angeboten.}

\end{remarks}

\end{course}