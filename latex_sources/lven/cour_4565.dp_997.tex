% Lehrveranstaltungsbeschreibung
% Informationsgrad : extern
% Sprache: de
\begin{course}

\setdoclanguagegerman
\coursedegreeprogramme{Informatik}
\coursemodulename{Energiewirtschaft (S.~\pageref{mod_2871.dp_997})[IN3WWBWL12]}
\courseID{2581959}
\coursename{Energiepolitik}
\coursecoordination{M. Wietschel}

\documentdate{2011-12-23 10:02:44.191580}

\courselevel{3}
\coursecredits{3,5}
\courseterm{Sommersemester}
\coursehours{2/0}
\courseinstructionlanguage{de}

\coursehead

% For index (key word@display). Key word is used for sorting - no Umlauts please.
\index{Energiepolitik@Energiepolitik}

% For later referencing
\label{cour_4565.dp_997}


\begin{styleenv}
\begin{assessment}
Die Erfolgskontrolle erfolgt in Form einer schriftlichen Prüfungen (nach §4(2), 1 SPO). Die Prüfungen werden in jedem Semester angeboten und können zu jedem ordentlichen Prüfungstermin wiederholt werden.


\end{assessment}

\begin{conditions}Keine.\end{conditions}


\end{styleenv}

\begin{learningoutcomes}
Der/die Studierende

 \begin{itemize}\item benennt Problemstellungen aus dem Bereich der Stoff- und Energiepolitik,  \item kennt Lösungsansätze für die benannten Probleme und kann diese anwenden.  \end{itemize}
\end{learningoutcomes}

\begin{content}
Die Vorlesung beschäftigt sich mit der Stoff- und Energiepolitik, wobei diese im Sinne eines Managements von Stoff- und Energieströmen durch hoheitliche Akteure sowie die daraus resultierenden Rückwirkungen auf Betriebe behandelt wird. Zu Beginn wird die traditionelle Umweltökonomie mit den Erkenntnissen zur Problembewusstseinsschaffung - Anerkennung von Marktversagen bei öffentlichen Gütern und der Internalisierung externer Effekte - diskutiert. Aufbauend auf den neueren Erkenntnissen, dass viele natürliche Ressourcen für die menschliche Zivilisation existenziell und nicht durch technische Produkte substituierbar sind und künftigen Generationen nicht der Anspruch auf eine gleichwertige Lebensgrundlage verwehrt werden darf, wird die traditionelle Umweltökonomie kritisch hinterfragt und anschließend das Konzept der Nachhaltigen Entwicklung als neues Leitbild vorgestellt. Nach der Diskussion des Konzeptes wird auf die z.T. problematische Operationalisierung des Ansatzes eingegangen. Darauf aufbauend werden die Aufgaben einer Stoff- und Energiepolitik entscheidungsorientiert dargestellt. Die Wirtschaftshandlungen werden zunehmend durch positive und negative Anreize der staatlichen Umweltpolitik gezielt beeinflusst. Deshalb werden im Folgenden ausführlich umweltpolitische Instrumente vorgestellt und diskutiert. Diese Diskussion bezieht sich auf aktuelle Instrumente wie die ökologische Steuerreform, freiwillige Selbstverpflichtungserklärungen oder den Emissionshandel.


\end{content}



\begin{literature}Wird in der Vorlesung bekannt gegeben.

\end{literature}



\end{course}