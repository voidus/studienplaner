% Lehrveranstaltungsbeschreibung
% Informationsgrad : extern
% Sprache: de
\begin{course}

\setdoclanguagegerman
\coursedegreeprogramme{Informatik}
\coursemodulename{Supply Chain Management (S.~\pageref{mod_2721.dp_997})[IN3WWBWL14], eBusiness und Service Management (S.~\pageref{mod_1611.dp_997})[IN3WWBWL2]}
\courseID{2590452}
\coursename{Management of Business Networks}
\coursecoordination{C. Weinhardt, J. Kraemer}

\documentdate{2011-12-20 16:47:17.277731}

\courselevel{3}
\coursecredits{4,5}
\courseterm{Wintersemester}
\coursehours{2/1}
\courseinstructionlanguage{en}

\coursehead

% For index (key word@display). Key word is used for sorting - no Umlauts please.
\index{Management of Business Networks@Management of Business Networks}

% For later referencing
\label{cour_4811.dp_997}


\begin{styleenv}
\begin{assessment}
Die Erfolgskontrolle erfolgt in Form einer schriftlichen Prüfung (nach §4(2), 1 SPO) und durch Ausarbeiten von Übungsaufgaben (nach §4(2), 3 SPO).

 
\end{assessment}

\begin{conditions}Keine.\end{conditions}


\end{styleenv}

\begin{learningoutcomes}
Der Studierende

 \begin{itemize}\item identifiziert die Koordinationsprobleme in einem Business Netzwerk  \item erklärt die Theorie des strategischen und operativen Managements  \item analysiert Fallstudien aus der Logistik unter Berücksichtigung der Organisationslehre und Netzwerkanalyse  \item argumentiert und konstruiert neue Lösungen für die Fallstudien mit Hilfe von elektronischen Werkzeugen  \end{itemize}
\end{learningoutcomes}

\begin{content}
Der bedeutende und anhaltende Einfluss web-basierter Business-to-Business (B2B) Netzwerke wird erst in letzter Zeit deutlich. Die explorative Phase während des ersten Internet-Hypes hat eine Vielzahl von Ansätzen hervorgebracht welche mutige Geschäftsideen darstellten, deren Systemarchitektur jedoch meist einfach und unfundiert war. Nur wenige Modelle haben diese erste Phase überlebt und sich als nachhaltig erwiesen. Heute treten Web-basierte B2B Netzwerke verstärkt wieder auf und werden sogar durch große traditionelle Unternehmen und Regierungen vorangetrieben. Diese neue Welle von Netzwerken ist jedoch ausgereifter und bietet mehr Funktionalität als ihre Vorgänger. Als solche bieten sie nicht nur Auktionssysteme an, sondern erleichtern auch elektronische Verhandlungen. Dies bringt ein Umschwenken von einem preisorientierten zu einem beziehungsorientierten Handel mit sich. Doch was motiviert diesen Umschwung? Warum treten Firmen in Geschäftsnetzwerke ein? Wie können diese Netzwerke am besten durch IT unterstützt werden? Die Vorlesung behandelt genau diese Fragen. Zuerst wird eine Einführung in die Organisationslehre gegeben. Danach werden Netzwerk-Probleme adressiert. Zuletzt wird untersucht, wie IT diese Probleme verringern kann.


\end{content}

\begin{media}\begin{itemize}\item Folien  \item Aufzeichnung der Vorlesung im Internet  \item ggf. Videokonferenz.  \end{itemize}\end{media}

\begin{literature}\begin{itemize}\item Milgrom, P., Roberts, J., Economics, Organisation and Management. Prentice-Hall, 1992.  \item Shy, O., The Economics of Network Industries. Cambridge, Cambridge University Press, 2001.  \item Bichler, M. The Future of e-Markets - Multi-Dimensional Market Mechanisms. Cambridge, Cambridge University Press, 2001.  \end{itemize}\end{literature}

\begin{remarks}Die Veranstaltung wird zum SS2012 aus den Master Modulen entfernt und ist nur noch im Bachelor zu belegen. 

\end{remarks}

\end{course}