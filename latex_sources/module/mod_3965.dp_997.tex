% Modulbeschreibung 
% Informationsgrad : extern
% Sprache: de
\begin{module}

\setdoclanguagegerman
\moduledegreeprogramme{Informatik (B.Sc.)}
\modulesubject{EF E-Technik}
\moduleID{IN3EITGNT}
\modulename{Grundlagen der Nachrichtentechnik}
\modulecoordination{F.  Jondral}

\documentdate{2010-08-02 12:20:52.117222}

\modulecredits{21}
\moduleduration{2}
\modulecycle{Jedes Semester}



\modulehead

% For index (key word@display). Key word is used for sorting - no Umlauts please.
\index{Grundlagen der Nachrichtentechnik@Grundlagen der Nachrichtentechnik (M)}

% For later referencing
\label{mod_3965.dp_997}

\begin{courselist}
23109 & Signale und Systeme (S.~\pageref{cour_7991.dp_997}) & 2/1 & W & 5 & F. Puente León\\
23506 & Nachrichtentechnik I (S.~\pageref{cour_8097.dp_997}) & 3/1 & S & 6 & F. Jondral\\
23155 & Systemdynamik und Regelungstechnik (S.~\pageref{cour_6891.dp_997}) & 2/1 & S & 5 & M. Kluwe\\
23616 & Communication Systems and Protocols  (S.~\pageref{cour_8099.dp_997}) & 2/1 & S & 5 & Leuthold, Becker, Hübner\\
\end{courselist}

\begin{styleenv}
\begin{assessment}
Die Erfolgskontrolle wird in den Lehrveranstaltungsbeschreibungen erläutert.

 

Die Gesamtnote des Moduls wird aus den mit LP gewichteten Noten der Teilprüfungen gebildet und nach der ersten Kommastelle abgeschnitten.


\end{assessment}

\begin{conditions}Keine.\end{conditions}

\begin{recommendations}Es werden mathematische Grundlagen und Kenntnisse der Wahrscheinlichkeitstheorie dringend empfohlen.

\end{recommendations}
\end{styleenv}

\begin{learningoutcomes}
Der Studierende erlernt die Beschreibung von Systemen mittels Systemtheorie. Diese Konzepte werden verwendet, um damit Vorgänge bei der Relegungstechnik und der Nachrichtenübertragung zu verstehen. Nach Besuch des Moduls ist der Studierende über die Methoden der Nachrichtenübertragung und deren Realisierung in realen Systemen informiert.


\end{learningoutcomes}

\begin{content}
Dieses Modul vermittelt Studierenden die theoretischen und praktischen Aspekte der Nachrichtenübertragung. Hierzu sind Grundkenntnisse in den Bereichen Systemtheorie, Regelungstechnik und Nachrichtentechnik unerlässlich. Zur weiteren Beschreibung siehe detaillierte Darstellung der einzelnen Lehrveranstaltungen.


\end{content}



\end{module}

