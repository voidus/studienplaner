% Modulbeschreibung 
% Informationsgrad : extern
% Sprache: de
\begin{module}

\setdoclanguagegerman
\moduledegreeprogramme{Informatik (B.Sc.)}
\modulesubject{EF Mathematik}
\moduleID{IN3MATHAN05}
\modulename{Funktionalanalysis}
\modulecoordination{R. Schnaubelt}

\documentdate{2011-10-06 18:24:35.558011}

\modulecredits{9}
\moduleduration{1}
\modulecycle{Jedes 2. Semester, Wintersemester}



\modulehead

% For index (key word@display). Key word is used for sorting - no Umlauts please.
\index{Funktionalanalysis@Funktionalanalysis (M)}

% For later referencing
\label{mod_3343.dp_997}

\begin{courselist}
01048 & Funktionalanalysis (S.~\pageref{cour_7763.dp_997}) & 4/2 & W & 8 & G. Herzog, M. Plum, W. Reichel, C. Schmoeger, R. Schnaubelt, L. Weis\\
\end{courselist}

\begin{styleenv}
\begin{assessment}
Prüfung: schriftliche oder mündliche Prüfung\newline
Notenbildung: Note der Prüfung


\end{assessment}

\begin{conditions}Das Modul \emph{Proseminar Mathematik} [IN3MATHPS] muss geprüft werden.

 

Das Modul muss mit dem Modul \emph{Differentialgleichungen und Hilberträume} [IN3MATHAN03] oder mit dem Modul \emph{Analysis 3} [IN3MATHAN02] geprüft werden.

\end{conditions}

\begin{recommendations}Folgende Module sollten bereits belegt worden sein (Empfehlung):\newline
Lineare Algebra 1+2\newline
Analysis 1-3

\end{recommendations}
\end{styleenv}

\begin{learningoutcomes}
Einführung in funktionalanalytische Konzepte und Denkweisen


\end{learningoutcomes}

\begin{content}
\begin{itemize}\item Metrische Räume (topologische Grundbegriffe, Kompaktheit)  \item Stetige lineare Operatoren auf Banachräumen (Prinzip der gleichmäßigen Beschränktheit, Homomorphiesatz)  \item Dualräume mit Darstellungssätzen, Satz von Hahn-Banach, schwache Konvergenz, Reflexivität  \item Distributionen, schwache Ableitung, Fouriertransformation, Satz von Plancherel, Sobolevräume in L\textsuperscript{2}, partielle Differentialgleichungen mit konstanten Koeffizienten  \end{itemize}
\end{content}



\end{module}

