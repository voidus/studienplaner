% Lehrveranstaltungsbeschreibung
% Informationsgrad : extern
% Sprache: de
\begin{course}

\setdoclanguagegerman
\coursedegreeprogramme{Informatik}
\coursemodulename{CRM und Servicemanagement (S.~\pageref{mod_1601.dp_997})[IN3WWBWL1]}
\courseID{2540522}
\coursename{Analytisches CRM}
\coursecoordination{A. Geyer-Schulz}

\documentdate{2011-12-30 10:24:36.057226}

\courselevel{3}
\coursecredits{4,5}
\courseterm{Sommersemester}
\coursehours{2/1}
\courseinstructionlanguage{de}

\coursehead

% For index (key word@display). Key word is used for sorting - no Umlauts please.
\index{Analytisches CRM@Analytisches CRM}

% For later referencing
\label{cour_4901.dp_997}


\begin{styleenv}
\begin{assessment}
Die Erfolgskontrolle wird in der Modulbeschreibung erläutert.


\end{assessment}

\begin{conditions}Keine.\end{conditions}

\begin{recommendations}Kenntnisse über Datenmodelle und Modellierungssprachen (UML) aus dem Bereich der Informationssysteme werden vorausgesetzt.

\end{recommendations}
\end{styleenv}

\begin{learningoutcomes}
Der Student

 \begin{itemize}\item wendet die wesentlichen im analytischen CRM eingesetzten wissenschaftlichen Methoden (Statistik, Informatik) und ihre Anwendung auf betriebliche Entscheidungsprobleme verstehen und selbständig auf Standardfälle an,  \item hat einen Überblick über die Erstellung und Verwaltung eines Datawarehouse aus operativen Systemen, versteht die dabei notwendigen Prozesse und Schritte und wendet diese auf ein einfaches Beispiel an,  \item führt mit seinen Kenntnissen eine Standard CRM-Analyse für ein betriebliches Entscheidungsproblem mit betrieblichen Daten durch und leitet eine entsprechende Handlungsempfehlung begründet daraus ab.  \item versteht den Modellbildungsprozess und setzt diesen mit Hilfe eines Statistikpaketes (z.B. R) zur Lösung von Anwendungsproblemen ein.  \end{itemize}
\end{learningoutcomes}

\begin{content}
In der Vorlesung Analytisches CRM werden Analysemethoden und -techniken behandelt, die zur Verwaltung und Verbesserung von Kundenbeziehungen verwendet werden können. Wissen über Kunden wird auf aggregierter Ebene für betriebliche Entscheidungen (z.B. Sortimentsplanung, Kundenloyalität, ...) nutzbar gemacht.

 

Voraussetzung dafür ist die Überführung der in den operativen Systemen erzeugten Daten in ein einheitliches Datawarehouse, das der Sammlung aller für Analysezwecke wichtigen Daten dient. Die nötigen Modellierungsschritte und Prozesse zur Erstellung und Verwaltung eines Datawarehouse werden behandelt (u.a. ETL-Prozesse, Datenqualität und Monitoring). Die Generierung von kundenorientierten, flexiblen Reports für verschiedene betriebswirtschaftliche Zwecke wird behandelt.

 

Zwei Analyseverfahren der multivariaten Statistik bilden die methodische Basis, auf der zahlreiche Anwendungen des analystischen CRM aufbauen:

 \begin{enumerate}\item Clusteranalyse. Clusteranalyseverfahren werden zur Segmentierung von Märkten und Kunden eingesetzt und bilden die Grundlage für Personalisierung. Die Ergebnisse dienen einerseits als empirische Grundlage strategischer Marketingentscheidungen und andererseits für operative Zwecke im Rahmen der Vertriebssteuerung bzw. für innovative Kunden/Produktberatungsdienste.  \item Regressionsanalyse. Regressionsmodelle werden häufig als Prognosemodelle eingesetzt. Prognosen reichen dabei von Umsatzprognosen, Kundenwertprognosen, ..., bis zur Prognose von Kundenrisiken. Solche Prognosemodelle werden häufig zur Entscheidungsunterstützung bzw. -automation herangezogen.  \end{enumerate}

Als externe Datenquellen werden Kundenumfragen behandelt.


\end{content}

\begin{media}digitale Folien

\end{media}

\begin{literature} 

Ponnia, Paulraj. Data Warehousing Fundamentals: A Comprehensive Guide for IT Professionals. Wiley, New York, 2001.

  

Duda, Richard O. und Hart, Peter E. und Stork, David G. Pattern Classification. Wiley-Interscience, New York, 2. Ausgabe, 2001.

  

Maddala, G. S. Introduction to Econometrics. Wiley, Chichester, 3rd Ed., 2001.

 

Theil, H. Principles of Econometrics. Wiley, New York, 1971.

\end{literature}



\end{course}