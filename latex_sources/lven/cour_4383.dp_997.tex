% Lehrveranstaltungsbeschreibung
% Informationsgrad : extern
% Sprache: de
\begin{course}

\setdoclanguagegerman
\coursedegreeprogramme{Informatik}
\coursemodulename{Einführung in das Privatrecht (S.~\pageref{mod_2653.dp_997})[IN3INJUR1]}
\courseID{24012}
\coursename{BGB für Anfänger}
\coursecoordination{T. Dreier, P. Sester}

\documentdate{2005-12-21 12:08:41}

\courselevel{1}
\coursecredits{4}
\courseterm{Wintersemester}
\coursehours{4/0}
\courseinstructionlanguage{de}

\coursehead

% For index (key word@display). Key word is used for sorting - no Umlauts please.
\index{BGB fuer Anfaenger@BGB für Anfänger}

% For later referencing
\label{cour_4383.dp_997}


\begin{styleenv}
\begin{assessment}
Die Erfolgskontrolle erfolgt in Form einer schriftlichen Prüfung (Klausur) nach § 4 Abs. 2 Nr. 1 der SPO. Zeitdauer: 90 min.


\end{assessment}

\begin{conditions}Keine.\end{conditions}


\end{styleenv}

\begin{learningoutcomes}
Die Vorlesung soll den Studenten zunächst eine allgemeine
Einführung in das Recht geben und ihr Verständnis
für Problemstellungen und rechtliche Lösungsmuster sowohl
in rechtspolitischer Hinsicht wie auch in Bezug auf konkrete
Streitfälle wecken. Die Studenten sollen die Grundzüge
des Rechts und die Unterschiede von Privatrecht, öffentlichem
Recht und Strafrecht kennen und verstehen lernen. Vor allem sollen
sie Kenntnisse in Bezug auf die Grundbegriffe des Bürgerlichen
Rechts erwerben und deren Ausformung im deutschen Bürgerlichen
Gesetzbuch (BGB) kennen lernen (Rechtssubjekte, Rechtsobjekte,
Willenserklärung, Vertragsschluß, allgemeine
Geschäftsbedingungen, Verbraucherschutz,
Leistungstörungen usw.). Die Studenten sollen ein
Grundverständnis für rechtliche Problemlagen und
juristische Lösungsstrategien entwickeln. Sie sollen rechtlich
relevante Sachverhalte erkennen lernen und einfache Fälle
lösen können.
\end{learningoutcomes}

\begin{content}
Die Vorlesung beginnt mit einer allgemeinen Einführung ins
Recht. Was ist Recht, warum gilt Recht und was will Recht im
Zusammenspiel mit Sozialverhalten, Technikentwicklung und Markt?
Welche Beziehung besteht zwischen Recht und Gerechtigkeit?
Ebenfalls einführend wird die Unterscheidung von Privatrecht,
öffentlichem Recht und Strafrecht vorgestellt sowie die
Grundzüge der gerichtlichen und außergerichtlichen
einschließlich der internationalen Rechtsdurchsetzung
erlüutert. Anschließend werden die Grundbegriffe des
Rechts in ihrer konkreten Ausformung im deutschen Bürgerlichen
Gesetzbuch (BGB) besprochen. Das betrifft insbesondere
Rechtssubjekte, Rechtsobjekte, Willenserklärung, die
Einschaltung Dritter (insbes. Stellvertretetung),
Vertragsschluß (einschließlich Trennungs- und
Abstraktionsprinzip), allgemeine Geschäftsbedingungen,
Verbraucherschutz, Leistungsstörungen. Abschließend
erfolgt ein Ausblick auf das Schuld- und das Sachenrecht.
Schließlich wird eine Einführung in die
Subsumtionstechnik gegeben
\end{content}

\begin{media}Folien\end{media}

\begin{literature}Wird in der Vorlesung bekannt gegeben

\textbf{Weiterführende Literatur:}

Literaturangaben werden in den Vorlesungsfolien angekündigt.

\end{literature}



\end{course}