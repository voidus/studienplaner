% Lehrveranstaltungsbeschreibung
% Informationsgrad : extern
% Sprache: de
\begin{course}

\setdoclanguagegerman
\coursedegreeprogramme{Informatik}
\coursemodulename{Markovsche Ketten (S.~\pageref{mod_3641.dp_997})[IN3MATHST03]}
\courseID{1602}
\coursename{Markovsche Ketten}
\coursecoordination{N. Bäuerle, N. Henze, B. Klar, G. Last}

\documentdate{2011-10-06 17:59:38.719460}

\courselevel{}
\coursecredits{6}
\courseterm{Sommersemester}
\coursehours{3/1}
\courseinstructionlanguage{}

\coursehead

% For index (key word@display). Key word is used for sorting - no Umlauts please.
\index{Markovsche Ketten@Markovsche Ketten}

% For later referencing
\label{cour_8045.dp_997}


\begin{styleenv}
\begin{assessment}
Die Erfolgskontrolle wird in der Modulbeschreibung erläutert.


\end{assessment}

\begin{conditions}Keine.\end{conditions}

\begin{recommendations}Folgende Module sollten bereits belegt worden sein (Empfehlung):\newline
Einführung in die Stochastik

\end{recommendations}
\end{styleenv}

\begin{learningoutcomes}
Einführung in grundlegende Aussagen und Methoden für Markovsche Ketten.


\end{learningoutcomes}

\begin{content}
\begin{itemize}\item Markov-Eigenschaft  \item Übergangswahrscheinlichkeiten  \item Simulationsdarstellung  \item Irreduzibilität und Aperiodizität  \item Stationäre Verteilungen  \item Ergodensätze  \item Reversible Markovsche Ketten  \item Warteschlangen  \item Jackson-Netzwerke  \item Irrfahrten  \item Markov Chain Monte Carlo  \item Markovsche Ketten in stetiger Zeit  \item Übergangsintensitäten  \item Geburts-und Todesprozesse  \item Poissonscher Prozess  \end{itemize}
\end{content}







\end{course}