% Lehrveranstaltungsbeschreibung
% Informationsgrad : extern
% Sprache: de
\begin{course}

\setdoclanguagegerman
\coursedegreeprogramme{Informatik}
\coursemodulename{Lineare Algebra (S.~\pageref{mod_2395.dp_997})[IN1MATHLA]}
\courseID{01870}
\coursename{Lineare Algebra II für die Fachrichtung Informatik}
\coursecoordination{K. Spitzmüller, S. Kühnlein, Hug}

\documentdate{2008-06-23 10:30:13}

\courselevel{1}
\coursecredits{5}
\courseterm{Sommersemester}
\coursehours{2/1/2}
\courseinstructionlanguage{de}

\coursehead

% For index (key word@display). Key word is used for sorting - no Umlauts please.
\index{Lineare Algebra II fuer die Fachrichtung Informatik@Lineare Algebra II für die Fachrichtung Informatik}

% For later referencing
\label{cour_6281.dp_997}


\begin{styleenv}
\begin{assessment}
Die Erfolgskontrolle wird in der Modulbeschreibung erläutert.


\end{assessment}

\begin{conditions}Keine.\end{conditions}


\end{styleenv}

\begin{learningoutcomes}
Die Studierenden sollen am Ende des Moduls

 \begin{itemize}\item den Übergang von der Schule zur Universität bewältigt haben,  \item mit logischem Denken und strengen Beweisen vertraut sei,   \item die Methoden und grundlegenden Strukturen der Linearen Algebra beherrschen.  \end{itemize}
\end{learningoutcomes}

\begin{content}
\begin{itemize}\item Eigenwerttheorie (Eigenwerte, Eigenvektoren, Charakteristisches Polynom, Normalformen)   \item Vektorräume mit Skalarprodukt (bilineare Abbildungen, Skalarprodukt, Norm, Orthogonalität, adjungierte Abbildung, selbstadjungierte Endomorphismen, Spektralsatz, Isometrien)   \end{itemize}
\end{content}



\begin{literature}Skriptum zur Vorlesung.

 

Weitere Literatur wird in der Vorlesung bekannt gegeben.

\end{literature}



\end{course}