% Modulbeschreibung 
% Informationsgrad : extern
% Sprache: de
\begin{module}

\setdoclanguagegerman
\moduledegreeprogramme{Informatik (B.Sc.)}
\modulesubject{}
\moduleID{IN3INWAWT}
\modulename{Web-Anwendungen und Web-Technologien}
\modulecoordination{S. Abeck}

\documentdate{2011-08-01 12:37:02.691342}

\modulecredits{9}
\moduleduration{1}
\modulecycle{Jedes Semester}



\modulehead

% For index (key word@display). Key word is used for sorting - no Umlauts please.
\index{Web-Anwendungen und Web-Technologien@Web-Anwendungen und Web-Technologien (M)}

% For later referencing
\label{mod_2925.dp_997}

\begin{courselist}
24604/24153 & Advanced Web Applications (S.~\pageref{cour_5663.dp_997}) & 2/0 & W/S & 4 & S. Abeck\\
24304/24873 & Praktikum Web-Technologien (S.~\pageref{cour_5681.dp_997}) & 2/0 & W/S & 5 & S. Abeck, Gebhart, Hoyer, Link, Pansa\\
\end{courselist}

\begin{styleenv}
\begin{assessment}
Die Modulprüfung erfolgt in Form von Teilprüfungen über die beiden Lehrveranstaltungen des Moduls.

 

Die Erfolgskontrolle zu \emph{Advanced Web Applications} [24604/24153] erfolgt in Form einer mündlichen Prüfung nach § 4 Abs. 2 Nr. 2 SPO.

 

Die Erfolgskontrolle zum \emph{Praktikum Web-Technologien }[24304/24873] erfolgt benotet als Erfolgskontrolle anderer Art nach § 4 Abs. 2 Nr. 3 SPO.

 

Die Gesamtnote des Moduls wird aus den mit LP gewichteten Teilnoten gebildet und nach der ersten Kommastelle abgeschnitten.


\end{assessment}

\begin{conditions}Keine.\end{conditions}


\end{styleenv}

\begin{learningoutcomes}
Die Architektur von mehrschichtigen und dienstorientierten Anwendungssystemen ist verstanden. \newline
 Die Softwarearchitektur einer Web-Anwendung kann modelliert werden.\newline
 Die wichtigsten Prinzipien traditioneller Softwareentwicklung und des entsprechenden Entwicklungsprozesses sind bekannt.\newline
 Die Verfeinerung höherstufiger Prozessmodelle sowie deren Abbildung auf eine dienstorientierte Architektur sind verstanden.\newline
Die Technologien und Werkzeuge können zur Entwicklung von Beispielszenarien angewendet werden.\newline
Die erzielten Ergebnisse können in Form einer vorgegebenen Dokumentenvorlage klar und verständlich dokumentiert werden.\newline
Die erzielten Ergebnisse können präsentiert und in einer Diskussion vertreten werden.


\end{learningoutcomes}

\begin{content}
Dieses Modul umfasst zunächst die Inhalte der Lehrveranstaltung „AdvancedWeb Applications”, die die modellgetriebene Entwicklung von dienstorientierten Web-Anwendungen behandelt. Hierbei wird der durch die Web-Anwendung zu unterstützende Geschäftsprozess in einem Modell so beschrieben, dass er auf eine dienstorientierte Architektur (Service-Oriented Architecture, SOA) abgebildet werden kann.\newline
Im Rahmen des ergänzend zur Vorlesung angebotenen Praktikums werden die Teilnehmer in eines der in der Forschungsgruppe laufenden Projektteams integriert. Inhaltlich ist das Praktikum Web-Technologien in zwei Abschnitte unterteilt. In den ersten 4-5 Wochen erhält jeder Praktikumsteilnehmer eine grundlegende Einführung in die traditionelle und fortgeschrittene dienstorientierte Software-Entwicklung. Anschließend werden individuelle Aufgaben im Kontext des jeweiligen Projektteams vergeben, welche in der verbleibenden\newline
Zeit zu bearbeiten und dokumentieren sind


\end{content}

\begin{remarks}\textcolor{red}{Dieses Modul wurde letztmalig im SS 11 angeboten, Prüfungen werden noch bis WS 2012/13 für Wiederholer angeboten.}

\end{remarks}

\end{module}

