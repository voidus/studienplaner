% Lehrveranstaltungsbeschreibung
% Informationsgrad : extern
% Sprache: de
\begin{course}

\setdoclanguagegerman
\coursedegreeprogramme{Informatik}
\coursemodulename{Bauökologie (S.~\pageref{mod_1639.dp_997})[IN3WWBWL16]}
\courseID{2585404/2586404}
\coursename{Bauökologie II}
\coursecoordination{T. Lützkendorf}

\documentdate{2011-06-27 15:17:05.719382}

\courselevel{3}
\coursecredits{4,5}
\courseterm{Sommersemester}
\coursehours{2/1}
\courseinstructionlanguage{de}

\coursehead

% For index (key word@display). Key word is used for sorting - no Umlauts please.
\index{Bauoekologie II@Bauökologie II}

% For later referencing
\label{cour_6881.dp_997}


\begin{styleenv}
\begin{assessment}
Die Erfolgskontrolle erfolgt in Form einer schriftlichen Prüfung (nach §4(2), 1 SPO) oder mündlichen Prüfung (nach §4(2), 2 SPO) im Umfang von 20 min.\newline
Die Prüfung wird in jedem Semester angeboten und kann zu jedem ordentlichen Prüfungstermin wiederholt werden.


\end{assessment}

\begin{conditions}Es wird eine Kombination mit dem Modul \emph{Real Estate Management} [IN3WWBWL17] und mit einem ingenieurwissenschaftlichem Modul aus den Bereichen Bauphysik oder Baukonstruktion empfohlen.

\end{conditions}


\end{styleenv}

\begin{learningoutcomes}
Kenntnisse im Bereich der ökonomischen und ökologischen Bewertung von Gebäuden


\end{learningoutcomes}

\begin{content}
Es werden Fragestellungen einer ökonomisch-ökologischen Bewertung entlang des Lebenszyklusses von Bauwerken herausgearbeitet und geeignete Methoden und Hilfsmittel zur Unterstützung der Entscheidungsfindung diskutiert. Behandelt werden u.a. die Themenbereiche Nachhaltigkeit in der Bau-, Wohnungs- und Immobilienwirtschaft, Ökobilanzierung sowie der heute im Bereich Bauökologie verfügbaren Planungs- und Bewertungshilfsmittel (u.a. Element-Kataloge, Datenbanken, Zeichen, Tools) und Bewertungsverfahren (u.a. KEA, effektorientierte Kriterien und Wirkungskategorien, MIPS, ökologischer Fußabdruck)


\end{content}



\begin{literature}\textbf{Weiterführende Literatur:}

 \begin{itemize}\item Schmidt-Bleek: „Das MIPS-Konzept“. Droemer 1998  \item Wackernagel et.al: „Unser ökologischer Fußabdruck“. Birkhäuser 1997  \item Braunschweig: „Methode der ökologischen Knappheit“. BUWAL 1997  \item Hohmeyer et al.: „Social Costs and Sustainability”. Springer 1997  \item Hofstetter: „Perspectives in Life Cycle Impact Assessment”. Kluwer Academic Publishers 1998  \end{itemize}\end{literature}



\end{course}