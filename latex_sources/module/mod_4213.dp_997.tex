% Modulbeschreibung 
% Informationsgrad : extern
% Sprache: de
\begin{module}

\setdoclanguagegerman
\moduledegreeprogramme{Informatik (B.Sc.)}
\modulesubject{}
\moduleID{IN3INPFB}
\modulename{Seminar Proofs from THE BOOK}
\modulecoordination{D. Wagner}

\documentdate{2012-01-05 12:31:08.548528}

\modulecredits{4}
\moduleduration{1}
\modulecycle{Unregelmäßig}



\modulehead

% For index (key word@display). Key word is used for sorting - no Umlauts please.
\index{Seminar Proofs from THE BOOK@Seminar Proofs from THE BOOK (M)}

% For later referencing
\label{mod_4213.dp_997}

\begin{courselist}
24842 & Seminar Proofs from THE BOOK (S.~\pageref{cour_8499.dp_997}) & 2 & S & 4 & M. Krug, I. Rutter\\
\end{courselist}

\begin{styleenv}
\begin{assessment}
Die Erfolgskontrolle erfolgt unbenotet als Erfolgskontrolle anderer Art nach § 4 Abs. 2 Nr. 3 SPO in Form regelmäßiger Vorträge von ca. 20 Minuten Dauer.


\end{assessment}

\begin{conditions}Keine.\end{conditions}

\begin{recommendations}Kenntnisse zu grundlegenden Beweistechniken sind hilfreich.

\end{recommendations}
\end{styleenv}

\begin{learningoutcomes}
Die Studierenden erlernen wissenschaftliches Arbeiten im Umgang mit verschiedenen Beweistechniken. Es werden grundlegende Beweistechniken anhand wichtiger Resultate aus den Bereichen Geometrie, Kombinatorik und Graphentheorie vorgestellt und vertieft. Die Studierenden erschließen sich verschiedene Themengebiete in selbständiger Arbeit und bereitet ihre Erkenntnisse im Rahmen eines wissenschaftlichen Vortrags auf. Dabei wird Transferwissen vor allem aus Mathematik und Logik umgesetzt. Mit dem Besuch der Seminarveranstaltungen werden neben Techniken des wissenschaftlichen Arbeitens auch Schlüsselqualifikationen integrativ vermittelt.


\end{learningoutcomes}

\begin{content}
Dem 1996 verstorbenen ungarischen Mathematiker Paul Erds zufolge, hält Gott ein Buch - nämlich das BUCH - mit den schönsten und elegantesten mathematischen Beweisen unter Verschluss. Erds' höchstes Ziel war es, eben solche Beweise aus dem BUCH zu finden. \newline
\newline
 Martin Aigner und Günter Ziegler veröffentlichten nach Erds' Tod 1998 das Buch „Proofs from THE BOOK”, das inzwischen auch in deutscher Sprache unter dem Titel „Das BUCH der Beweise” erschienen ist. In ihrer Sammlung, die zum Teil gemeinsam mit Paul Erds entstanden ist, findet man 35 Beweise, die wegen ihrer Eleganz als vielversprechende Kandidaten für BUCH-Beweise gelten. \newline
\newline
 In diesem Seminar werden die Teilnehmer eine Auswahl der Probleme aus dem Buch der Beweise vorstellen und diskutieren.


\end{content}



\end{module}

