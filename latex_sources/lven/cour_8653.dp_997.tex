% Lehrveranstaltungsbeschreibung
% Informationsgrad : extern
% Sprache: de
\begin{course}

\setdoclanguagegerman
\coursedegreeprogramme{Informatik}
\coursemodulename{Algorithmen II (S.~\pageref{mod_2847.dp_997})[IN3INALG2]}
\courseID{24079}
\coursename{Algorithmen II}
\coursecoordination{P. Sanders}

\documentdate{2012-02-07 08:37:04.973782}

\courselevel{3}
\coursecredits{6}
\courseterm{Wintersemester}
\coursehours{3/1}
\courseinstructionlanguage{de}

\coursehead

% For index (key word@display). Key word is used for sorting - no Umlauts please.
\index{Algorithmen II@Algorithmen II}

% For later referencing
\label{cour_8653.dp_997}


\begin{styleenv}
\begin{assessment}
Die Erfolgskontrolle erfolgt in Form einer schriftlichen Prüfung im Umfang von 120 Minuten nach § 4 Abs. 2 Nr. 1 SPO.


\end{assessment}

\begin{conditions}Siehe Modubeschreibung.

\end{conditions}


\end{styleenv}

\begin{learningoutcomes}
Der/die Studierende

 \begin{itemize}\item besitzt einen vertieften Einblick in die wichtigsten Teilgebiete der Algorithmik,  \item identifiziert die algorithmische Probleme in verschiedenen Anwendungsgebieten und kann diese entsprechend formal formulieren,  \item versteht und bestimmt die Laufzeiten von Algorithmen,  \item kennt fundamentale Algorithmen und Datenstrukturen und transferiert diese auf unbekannte Probleme.  \end{itemize}
\end{learningoutcomes}

\begin{content}
Diese Lehrveranstaltung soll Studierenden die grundlegenden theoretischen und praktischen Aspekte der Algorithmentechnik vermitteln. Es werden generelle Methoden zum Entwurf und der Analyse von Algorithmen für grundlegende algorithmische Probleme vermittelt sowie die Grundzüge allgemeiner algorithmischer Methoden wie Approximationsalgorithmen, Lineare Programmierung, Randomisierte Algorithmen, Parallele Algorithmen und parametrisierte Algorithmen behandelt.


\end{content}





\begin{remarks}Diese Lehrveranstaltung ersetzt die ausgelaufene Lehrveranstaltung \emph{Algorithmentechnik}.

\end{remarks}

\end{course}