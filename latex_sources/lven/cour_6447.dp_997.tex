% Lehrveranstaltungsbeschreibung
% Informationsgrad : extern
% Sprache: de
\begin{course}

\setdoclanguagegerman
\coursedegreeprogramme{Informatik}
\coursemodulename{Topics in Finance I (S.~\pageref{mod_1575.dp_997})[IN3WWBWL13], eFinance (S.~\pageref{mod_2729.dp_997})[IN3WWBWL15]}
\courseID{2530570}
\coursename{Internationale Finanzierung}
\coursecoordination{M. Uhrig-Homburg, Walter}

\documentdate{2011-12-23 18:39:45.784736}

\courselevel{3}
\coursecredits{3}
\courseterm{Sommersemester}
\coursehours{2}
\courseinstructionlanguage{de}

\coursehead

% For index (key word@display). Key word is used for sorting - no Umlauts please.
\index{Internationale Finanzierung@Internationale Finanzierung}

% For later referencing
\label{cour_6447.dp_997}


\begin{styleenv}
\begin{assessment}
Die Erfolgskontrolle erfolgt in Form einer schriftlichen Prüfung (60min.) (nach §4(2), 1 SPO).

 

Die Prüfung wird in jedem Semester angeboten und kann zu jedem ordentlichen Prüfungstermin wiederholt werden.


\end{assessment}

\begin{conditions}Keine.\end{conditions}


\end{styleenv}

\begin{learningoutcomes}
Ziel der Vorlesung ist es, die Studierenden mit Investitions- und Finanzierungsentscheidungen auf den internationalen Märkten vertraut zu machen und sie in die Lage zu versetzen, Wechselkursrisiken zu managen.


\end{learningoutcomes}

\begin{content}
Im Zentrum der Veranstaltung stehen die Chancen und die Risiken, welche mit einem internationalen Agieren einhergehen. Dabei erfolgt die Analyse aus zwei Perspektiven: Zum einen aus dem Blickwin-kel eines internationalen Investors, zum anderen aus der Sicht eines international agierenden Unter-nehmens. Hierbei gilt es mögliche Handlungsalternativen, insbesondere für das Management von Wechselkursrisiken, aufzuzeigen. Auf Grund der zentralen Bedeutung des Wechselkursrisikos wird zu Beginn auf den Devisenmarkt eingegangen. Darüber hinaus werden die gängigen Wechselkurstheo-rien vorgestellt.


\end{content}



\begin{literature}\textbf{Weiterführende Literatur:}

 \begin{itemize}\item D. Eiteman et al. (2004): Multinational Business Finance, 10. Auflage  \end{itemize}\end{literature}

\begin{remarks}Die Veranstaltung wird 14-tägig oder als Blockveranstaltung angeboten.

\end{remarks}

\end{course}