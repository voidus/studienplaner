% Modulbeschreibung 
% Informationsgrad : extern
% Sprache: de
\begin{module}

\setdoclanguagegerman
\moduledegreeprogramme{Informatik (B.Sc.)}
\modulesubject{EF Betriebswirtschaftslehre}
\moduleID{IN3WWBWL11}
\modulename{Strategie und Organisation}
\modulecoordination{H. Lindstädt}

\documentdate{2011-12-15 15:43:50.869706}

\modulecredits{9}
\moduleduration{2}
\modulecycle{Jedes Semester}



\modulehead

% For index (key word@display). Key word is used for sorting - no Umlauts please.
\index{Strategie und Organisation@Strategie und Organisation (M)}

% For later referencing
\label{mod_1659.dp_997}

\begin{courselist}
2577900 & Unternehmensführung und Strategisches Management (S.~\pageref{cour_5015.dp_997}) & 2/0 & S & 4 & H. Lindstädt\\
2577902 & Organisationsmanagement (S.~\pageref{cour_5017.dp_997}) & 2/0 & W & 4 & H. Lindstädt\\
2577907 & Spezielle Fragestellungen der Unternehmensführung: Unternehmensführung und IT aus Managementperspektive (S.~\pageref{cour_5977.dp_997}) & 1/0 & W/S & 2 & H. Lindstädt\\
\end{courselist}

\begin{styleenv}
\begin{assessment}
Die Modulprüfung erfolgt in Form von schriftlichen Teilprüfungen (nach §4(2), 1 SPO) über die einzelnen Lehrveranstaltungen des Moduls, mit denen in Summe die Mindestabforderung an LP erfüllt wird. Die Prüfungen werden jedes Semester angeboten und können zu jedem ordentlichen Prüfungstermin wiederholt werden. Die Erfolgskontrolle wird bei jeder Lehrveranstaltung dieses Moduls beschrieben.

 

Die Note der einzelnen Teilprüfungen entspricht der jeweiligen Klausurnote.\newline
Die Gesamtnote des Moduls wird aus den mit LP gewichteten Noten der Teilprüfungen gebildet und nach der ersten Nachkommastelle abgeschnitten.


\end{assessment}

\begin{conditions}Nur prüfbar in Kombination mit dem Modul \emph{Grundlagen der BWL}.

 

Das Modul ist nur zusammen mit dem Pflichtmodul \emph{Grundlagen der BWL} [IN3WWBWL] prüfbar.

\end{conditions}


\end{styleenv}

\begin{learningoutcomes}
\begin{itemize}\item Der/die Studierende wird sowohl zentrale Konzepte des strategischen Managements als auch Konzepte und Modelle für die Gestaltung organisationaler Strukturen beschreiben können.  \item Er/sie wird die Stärken und Schwächen existierender organisationaler Strukturen und Regelungen anhand systematischer Kriterien bewerten können.  \item Die Steuerung organisationaler Veränderungen werden die Studierenden anhand von Fallbeispielen diskutieren und überprüfen können, inwieweit sich die Modelle in der Praxis einsetzen lassen und welche Bedingungen dafür gelten müssen.  \item Zudem werden die Studierenden den Einsatz von IT zur Unterstützung der Unternehmesführung planen können.  \end{itemize}
\end{learningoutcomes}

\begin{content}
Das Modul ist praxisnah und handlungsorientiert aufgebaut und vermittelt dem Studierenden einen aktuellen Überblick grundlegender Konzepte und Modelle des strategischen Managements und ein realistisches Bild von Möglichkeiten und Grenzen rationaler Gestaltungsansätze der Organisation.

 

Im Mittelpunkt stehen erstens interne und externe strategische Analyse, Konzept und Quellen von Wettbewerbsvorteilen, Formulierung von Wettbewerbs- und von Unternehmensstrategien sowie Strategiebewertung und -implementierung. Zweitens werden Stärken und Schwächen organisationaler Strukturen und Regelungen anhand systematischer Kriterien beurteilt. Dabei werden Konzepte für die Gestaltung organisationaler Strukturen, die Regulierung organisationaler Prozesse und die Steuerung organisationaler Veränderungen vorgestellt.


\end{content}



\end{module}

