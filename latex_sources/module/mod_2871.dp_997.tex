% Modulbeschreibung 
% Informationsgrad : extern
% Sprache: de
\begin{module}

\setdoclanguagegerman
\moduledegreeprogramme{Informatik (B.Sc.)}
\modulesubject{EF Betriebswirtschaftslehre}
\moduleID{IN3WWBWL12}
\modulename{Energiewirtschaft}
\modulecoordination{W. Fichtner}

\documentdate{2011-12-23 10:04:39.029900}

\modulecredits{9}
\moduleduration{1}
\modulecycle{Jedes Semester}



\modulehead

% For index (key word@display). Key word is used for sorting - no Umlauts please.
\index{Energiewirtschaft@Energiewirtschaft (M)}

% For later referencing
\label{mod_2871.dp_997}

\begin{courselist}
2581010 & Einführung in die Energiewirtschaft (S.~\pageref{cour_7409.dp_997}) & 2/2 & S & 5,5 & W. Fichtner\\
2581012 & Erneuerbare Energien - Technologien und Potenziale (S.~\pageref{cour_7417.dp_997}) & 2/0 & W & 3,5 & R. McKenna\\
2581005 & Unternehmensführung in der Energiewirtschaft (S.~\pageref{cour_13749.dp_997}) & 2/0 & S & 3,5 & H. Villis\\
2581959 & Energiepolitik (S.~\pageref{cour_4565.dp_997}) & 2/0 & S & 3,5 & M. Wietschel\\
\end{courselist}

\begin{styleenv}
\begin{assessment}
Die Modulprüfung erfolgt in Form von schriftlichen Teilprüfungen (nach §4(2), 1 SPO) über die Vorlesungen \emph{Einführung in die Energiewirtschaft} und eine der drei Ergänzungsveranstaltungen \emph{Erneuerbare Energien - Technologien und Potenziale,} \emph{Unternehmensführung in der Energiewirtschaft }oder \emph{Energiepolitik. }Die Prüfungen werden in jedem Semester angeboten und können zu jedem ordentlichen Prüfungstermin wiederholt werden. Die Erfolgskontrolle wird bei jeder Lehrveranstaltung dieses Moduls beschrieben.

 

Die Gesamtnote des Moduls wird aus den mit LP gewichteten Noten der Teilprüfungen gebildet und nach der ersten Nachkommastelle abgeschnitten.


\end{assessment}

\begin{conditions}Nur prüfbar in Kombination mit dem \emph{Modul Grundlagen der BWL}.

 \end{conditions}

\begin{recommendations}Die Lehrveranstaltungen sind so konzipiert, dass sie unabhängig voneinander gehört werden können. Daher kann sowohl im Winter- als auch im Sommersemester mit dem Modul begonnen werden.

\end{recommendations}
\end{styleenv}

\begin{learningoutcomes}
Der/die Studierende

 \begin{itemize}\item ist in der Lage, energiewirtschaftliche Zusammenhänge zu verstehen und ökologische Auswirkungen der Energieversorgung zu beurteilen,  \item kann die verschiedenen Energieträger und deren Eigenheiten bewerten,  \item kennt die energiepolitischen Rahmenvorgaben,  \item besitzt Kenntisse hinsichtlich der neuen marktwirtschaftlichen Gegebenheiten der Energiewirtschaft und insbesondere der Kosten und Potenziale Erneuerbarer Energien.  \end{itemize}
\end{learningoutcomes}

\begin{content}
\emph{Einführung in die Energiewirtschaft}: Charakterisierung (Reserven, Anbieter, Kosten, Technologien) verschiedener Energieträger (Kohle, Gas, Erdöl, Elektrizität, Wärme etc.)

 

\emph{Erneuerbare Energien - Technologien und Potenziale:} Charakterisierung der verschiedenen erneuerbaren Energieträger (Wind, Sonne, Wasser, Erdwärme etc.)

 

\emph{Unternehmensführung in der Energiewirtschaft}: Fragestellungen des Managements eines großen Unternehmens der Energiewirtschaft in Deutschland (übergeordnete Leitungsfunktionen, Strukturen, Prozesse und Projekte aus der Führungsperspektive etc.)

 

\emph{Energiepolitik}: Energiestrommanagement, energiepolitische Ziele und Instrumente (Emissionshandel etc.)


\end{content}

\begin{remarks}Auf Antrag beim Institut können auch zusätzliche Studienleistungen (z.B. von anderen Universitäten) im Modul angerechnet werden.

\end{remarks}

\end{module}

