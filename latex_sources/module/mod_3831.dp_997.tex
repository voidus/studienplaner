% Modulbeschreibung 
% Informationsgrad : extern
% Sprache: de
\begin{module}

\setdoclanguagegerman
\moduledegreeprogramme{Informatik (B.Sc.)}
\modulesubject{EF Operations Research}
\moduleID{IN3WWOR2}
\modulename{Anwendungen des Operations Research}
\modulecoordination{S. Nickel}

\documentdate{2011-12-01 13:16:35.614534}

\modulecredits{9}
\moduleduration{1}
\modulecycle{Jedes Semester}



\modulehead

% For index (key word@display). Key word is used for sorting - no Umlauts please.
\index{Anwendungen des Operations Research@Anwendungen des Operations Research (M)}

% For later referencing
\label{mod_3831.dp_997}

\begin{courselist}
2550486 & Standortplanung und strategisches Supply Chain Management (S.~\pageref{cour_7813.dp_997}) & 2/1 & S & 4,5 & S. Nickel\\
2550488 & Taktisches und operatives Supply Chain Management (S.~\pageref{cour_7815.dp_997}) & 2/1 & W & 4,5 & S. Nickel\\
2550490 & Software-Praktikum: OR-Modelle I (S.~\pageref{cour_7845.dp_997}) & 1/2 & W & 4,5 & S. Nickel\\
2550134 & Globale Optimierung I (S.~\pageref{cour_7879.dp_997}) & 2/1 & W & 4,5 & O. Stein\\
2550662 & Simulation I (S.~\pageref{cour_4641.dp_997}) & 2/1/2 & W & 4,5 & K. Waldmann\\
\end{courselist}

\begin{styleenv}
\begin{assessment}
Die Modulprüfung erfolgt in Form von Teilprüfungen(nach § 4(2), 1 SPO) über die gewählten Lehrveranstaltungen des Moduls, mit denen in Summe die Mindestanforderungen an Leistungspunkten erfüllt ist.

 

Die Erfolgskontrolle wird bei jeder Lehrveranstaltung beschrieben.

 

Die Gesamtnote des Moduls wird aus den mit Leistungspunkten gewichteten Noten der Teilprüfungen gebildet und nach der ersten Nachkommastelle abgeschnitten.


\end{assessment}

\begin{conditions}Nur prüfbar in Kombination mit dem Modul \emph{Grundlagen des OR}.

 

Mindestens eine der Veranstaltungen \emph{Standortplanung und strategisches Supply Chain Management} [2550486] und \emph{Taktisches und operatives Supply Chain Management }[2550488] muss absolviert werden.

 \end{conditions}


\end{styleenv}

\begin{learningoutcomes}
Der/ die Studierende

 \begin{itemize}\item ist vertraut mit wesentlichen Konzepten und Begriffen des Supply Chain Managements,  \item kennt die verschiedenen Teilgebiete des Supply Chain Managements und die zugrunde liegenden Optimierungsprobleme,  \item ist mit den klassischen Standortmodellen (in der Ebene, auf Netzwerken und diskret), sowie mit den grundlegenden Methoden zur Ausliefer- und Transportplanung, Warenlagerplanung und Lagermanagement vertraut,  \item ist in der Lage praktische Problemstellungen mathematisch zu modellieren und kann deren Komplexität abschätzen sowie geeignete Lösungsverfahren auswählen und anpassen.  \end{itemize}
\end{learningoutcomes}

\begin{content}
Supply Chain Management befasst sich mit der Planung und Optimierung des gesamten, unternehmensübergreifenden Beschaffungs-, Herstellungs- und Distributionsprozesses mehrerer Produkte zwischen allen beteiligten Geschäftspartnern (Lieferanten, Logistikdienstleistern, Händlern). Ziel ist es, unter Berücksichtigung verschiedenster Rahmenbedingungen die Befriedigung der (Kunden-) Bedarfe, so dass die Gesamtkosten minimiert werden.

 

Dieses Modul befasst sich mit mehreren Teilgebieten des Supply Chain Management. Zum einen mit der Bestimmung optimaler Standorte innerhalb von Supply Chains. Diese strategischen Entscheidungen über die die Platzierung von Anlagen wie Produktionsstätten, Vertriebszentren und Lager u.ä., sind von großer Bedeutung für die Rentabilität von Supply Chains. Sorgfältig durchgeführte Standortplanungen erlauben einen effizienteren Materialfluss und führen zu verringerten Kosten und besserem Kundenservice. Einen weiteren Schwerpunkt bildet die Planung des Materialtransports im Rahmen des Supply Chain Managements. Durch eine Aneinanderreihung von Transportverbindungen und Zwischenstationen wird die Lieferstelle (Produzent) mit der Empfangsstelle (Kunde) verbunden. Es wird betrachtet, wie für vorgegebene Warenströme oder Sendungen aus den möglichen Logistikketten die optimale Liefer- und Transportkette auszuwählen ist, die bei Einhaltung der geforderten Lieferzeiten und Randbedingungen zu den geringsten Kosten führt.

 

Darüber hinaus bietet das Modul die Möglichkeit verschiedene Aspekte der taktischen und operativen Planungsebene im Supply Chain Management kennenzulernen. Hierzu gehören v.a. Methoden des Schedulings sowie verschiedene Vorgehensweisen in der Beschaffungs- und Distributionslogistik. Fragestellungen der Warenhaltung und des Lagerhaltungsmanagements werden ebenfalls angesprochen.


\end{content}

\begin{remarks}Das für drei Studienjahre im Voraus geplante Lehrangebot kann im Internet nachgelesen werden.

\end{remarks}

\end{module}

