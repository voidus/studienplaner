% Lehrveranstaltungsbeschreibung
% Informationsgrad : extern
% Sprache: de
\begin{course}

\setdoclanguagegerman
\coursedegreeprogramme{Informatik}
\coursemodulename{Effiziente Kreativität (S.~\pageref{mod_13271.dp_997})[IN3MACHEK]}
\courseID{2122371}
\coursename{Effiziente Kreativität - Prozesse und Methoden in der Automobilindustrie}
\coursecoordination{Lamberti}

\documentdate{2011-02-17 13:23:19.492905}

\courselevel{4}
\coursecredits{4}
\courseterm{Sommersemester}
\coursehours{2}
\courseinstructionlanguage{de}

\coursehead

% For index (key word@display). Key word is used for sorting - no Umlauts please.
\index{Effiziente Kreativitaet - Prozesse und Methoden in der Automobilindustrie@Effiziente Kreativität - Prozesse und Methoden in der Automobilindustrie}

% For later referencing
\label{cour_8493.dp_997}


\begin{styleenv}
\begin{assessment}

\end{assessment}

\begin{conditions}Keine.\end{conditions}


\end{styleenv}

\begin{learningoutcomes}
Der/die Studierende

 \begin{itemize}\item kennt die marktbezogenen und technischen Herausforderungen der Entwicklung innovativer Produkte  \item kennt die Ausprägungen des Produktentwicklungsprozesses und die Gründe der Notwendigkeit der Standardisierung  \item kennt die Begriffe, Methoden und Vorgehensweisen bei der Prozessgestaltung  \item kennt exemplarische Methoden, Prozesse und Systeme des Projektmanagements, des Designs und der Gestaltung, des Anforderungsmanagements, des Änderungsmanagements, der Kostensteuerung und des Controllings, der Konstruktion, der Berechnung und Absicherung, der Produktionsplanung, der Datenverwaltung, der Integrationsplattformen, der Variantensteuerung, des Qualitätsmanagements, des Wissensmanagements und der Visualisierungstechnologien  \end{itemize}
\end{learningoutcomes}

\begin{content}
Ziel der Vorlesung ist die Vermittlung von Prozessen und Methoden bei der systematischen Entwicklung innovativer, komplexer und variantenreicher Produkte. Aufgaben, Gestaltung, Zusammenspiel und Koordination dieser Prozesse und Methoden werden am Beispiel der Automobilindustrie dargestellt.\newline
Die Studenten werden ausgehend von historischen, gegenwärtigen und absehbaren technologischen und marktbedingten Entwicklungen im automobilen Umfeld an die Varianten des systematischen Produktentwicklungsprozesses herangeführt. Ausgehend vom standardisierten Produktentwicklungsprozess werden dann die spezifischen und übergreifenden Prozesse und Methoden und deren IT-seitige Abbildung näher beleuchtet.


\end{content}







\end{course}