% Modulbeschreibung 
% Informationsgrad : extern
% Sprache: de
\begin{module}

\setdoclanguagegerman
\moduledegreeprogramme{Informatik (B.Sc.)}
\modulesubject{EF Mathematik}
\moduleID{IN3MATHAN02}
\modulename{Analysis 3}
\modulecoordination{W. Reichel}

\documentdate{2011-10-06 18:34:41.295619}

\modulecredits{9}
\moduleduration{1}
\modulecycle{Jedes 2. Semester, Wintersemester}



\modulehead

% For index (key word@display). Key word is used for sorting - no Umlauts please.
\index{Analysis 3@Analysis 3 (M)}

% For later referencing
\label{mod_3293.dp_997}

\begin{courselist}
01005 & Analysis 3 (S.~\pageref{cour_7269.dp_997}) & 4/2 & W & 9 & G. Herzog, M. Plum, W. Reichel, C. Schmoeger, R. Schnaubelt, L. Weis\\
\end{courselist}

\begin{styleenv}
\begin{assessment}
Prüfung: schriftliche Prüfung\newline
Notenbildung: Note der Prüfung


\end{assessment}

\begin{conditions}Das Modul \emph{Proseminar Mathematik} [IN3MATHPS] muss geprüft werden.

 

Das Modul muss mit dem Modul \emph{Funktionalanalysis} [IN3MATHAN05] oder mit dem Modul \emph{Differentialgleichungen und Hilberträume} [IN3MATHAN03] geprüft werden.

\end{conditions}

\begin{recommendations}Folgende Module sollten bereits belegt worden sein (Empfehlung):\newline
Analysis 1+2\newline
Lineare Algebra 1+2

\end{recommendations}
\end{styleenv}

\begin{learningoutcomes}
\begin{itemize}\item Einführung in die Konzepte der Lebesgueschen Maß- und Integrationstheorie  \item Vertrautheit mit Integrationstechniken  \end{itemize}
\end{learningoutcomes}

\begin{content}
\begin{itemize}\item Lebesgueintegral  \item Messbarkeit  \item Konvergenzsätze  \item Satz von Fubini  \item Transformationssatz  \item Divergenzsatz  \item Satz von Stokes  \item Fourierreihen  \item Beispiele von Rand- und Eigenwertproblemen gewöhnlicher Differentialgleichungen  \end{itemize}
\end{content}



\end{module}

