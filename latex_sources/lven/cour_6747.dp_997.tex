% Lehrveranstaltungsbeschreibung
% Informationsgrad : extern
% Sprache: de
\begin{course}

\setdoclanguagegerman
\coursedegreeprogramme{Informatik}
\coursemodulename{Insurance Markets and Management (S.~\pageref{mod_1565.dp_997})[IN3WWBWL7]}
\courseID{2530323}
\coursename{Insurance Marketing}
\coursecoordination{E. Schwake}

\documentdate{2012-01-22 17:18:21.242293}

\courselevel{4}
\coursecredits{4,5}
\courseterm{Sommersemester}
\coursehours{3/0}
\courseinstructionlanguage{de}

\coursehead

% For index (key word@display). Key word is used for sorting - no Umlauts please.
\index{Insurance Marketing@Insurance Marketing}

% For later referencing
\label{cour_6747.dp_997}


\begin{styleenv}
\begin{assessment}
Die Erfolgskontrolle setzt sich zusammen aus einer mündlichen Prüfung (nach §4(2), 2 SPO) und Vorträgen und Ausarbeitungen im Rahmen der Veranstaltung (nach §4(2), 3 SPO).

 

Die Note setzt sich zu je 50\% aus den Vortragsleistungen (inkl. Ausarbeitungen) und der mündlichen Prüfung zusammen.


\end{assessment}

\begin{conditions}Keine.\end{conditions}


\end{styleenv}

\begin{learningoutcomes}
Grundlegende Bedeutung der Absatzpolitik für die Erstellung der verschiedenen, mitunter komplexen, Dienstleistungen von Versicherungsunternehmen kennen; Beitrag des Kunden als externem Produktionsfaktor über das Marketing steuern; absatzpolitische Instrumente in ihrer charakteristischen Prägung durch das Versicherungsgeschäft kundenorientiert gestalten.


\end{learningoutcomes}

\begin{content}
\begin{enumerate}\item Absatzpolitik als Teil der Unternehmenspolitik von Versicherungsunternehmen  \item Konstituenten der Absatzmärkte von Versicherungsunternehmen  \item Produkt- oder Programmpolitik (kundenorientiert)  \item Entgeltpolitik: Variablen und Restriktionen der Preispolitik  \item Distributionspolitik: Absatzwege, Absatzorgane und deren Vergütung  \item Kommunikationspolitik: Werbung, Verkaufsförderung, PR  \end{enumerate}
\end{content}



\begin{literature}\textbf{Weiterführende Literatur:}

 \begin{itemize}\item Farny, D.. Versicherungsbetriebslehre (Kapitel III.3 sowie V.4). Karlsruhe 2011  \item Kurtenbach / Kühlmann / Käßer-Pawelka. Versicherungsmarketing…. Frankfurt 2001  \item Wiedemann, K.-P./Klee, A. Ertragsorientiertes Zielkundenmanagement für Finanzdienstleister, Wiesbaden 2003  \end{itemize}\end{literature}

\begin{remarks}Aus organisatorischen Gründen ist für die Teilnahme an der Veranstaltung eine Anmeldung erforderlich im Sekretariat des Lehrstuhls: thomas.mueller3@kit.edu.

\end{remarks}

\end{course}