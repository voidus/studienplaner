% Modulbeschreibung 
% Informationsgrad : extern
% Sprache: de
\begin{module}

\setdoclanguagegerman
\moduledegreeprogramme{Informatik (B.Sc.)}
\modulesubject{}
\moduleID{IN3INCG}
\modulename{Computergraphik}
\modulecoordination{C. Dachsbacher}

\documentdate{2011-02-17 12:11:00.701545}

\modulecredits{6}
\moduleduration{1}
\modulecycle{Jedes 2. Semester, Wintersemester}



\modulehead

% For index (key word@display). Key word is used for sorting - no Umlauts please.
\index{Computergraphik@Computergraphik (M)}

% For later referencing
\label{mod_4261.dp_997}

\begin{courselist}
24081  & Computergraphik (S.~\pageref{cour_8671.dp_997}) & 4 & W & 6 & C. Dachsbacher\\
\end{courselist}

\begin{styleenv}
\begin{assessment}
Die Erfolgskontrolle erfolgt in Form einer schriftlichen Prüfung im Umfang von 1 h nach § 4 Abs. 2 Nr. 1 SPO. Für den erfolgreichen Abschluss dieses Moduls ist ein bestandener Leistungsnachweis für die Übung (Erfolgskontrolle anderer Art nach § 4 Abs. 2 Nr. 3 SPO) notwendig.


\end{assessment}

\begin{conditions}Keine.\end{conditions}


\end{styleenv}

\begin{learningoutcomes}
Die Studierenden sollen grundlegende Konzepte und Algorithmen der Computergraphik verstehen und anwenden lernen. Die erworbenen Kenntnisse ermöglichen einen erfolgreichen Besuch weiterführender Veranstaltungen im Vertiefungsgebiet Computergraphik.


\end{learningoutcomes}

\begin{content}
Grundlegende Algorithmen der Computergraphik, Farbmodelle, Beleuchtungsmodelle, Bildsynthese-Verfahren (Ray Tracing, Rasterisierung), Geometrische Transformationen und Abbildungen, Texturen, Graphik-Hardware und APIs, Geometrisches Modellieren, Dreiecksnetze


\end{content}

\begin{remarks}Dieses Modul ist ein Stammmodul.

\end{remarks}

\end{module}

