% Lehrveranstaltungsbeschreibung
% Informationsgrad : extern
% Sprache: de
\begin{course}

\setdoclanguagegerman
\coursedegreeprogramme{Informatik}
\coursemodulename{Stochastische Methoden und Simulation (S.~\pageref{mod_3855.dp_997})[IN3WWOR4], Anwendungen des Operations Research (S.~\pageref{mod_3831.dp_997})[IN3WWOR2]}
\courseID{2550662}
\coursename{Simulation I}
\coursecoordination{K. Waldmann}

\documentdate{2011-10-28 15:11:25.245765}

\courselevel{4}
\coursecredits{4,5}
\courseterm{Wintersemester}
\coursehours{2/1/2}
\courseinstructionlanguage{de}

\coursehead

% For index (key word@display). Key word is used for sorting - no Umlauts please.
\index{Simulation I@Simulation I}

% For later referencing
\label{cour_4641.dp_997}


\begin{styleenv}
\begin{assessment}

\end{assessment}

\begin{conditions}Es werden Kentnisse in folgenden Bereichen vorausgesetzt:

 \begin{itemize}\item Operations Research, wie sie in den Veranstaltungen \emph{Einführung in das Operations Research I} [2550040] und \emph{Einführung in das Operations Research II} [2530043] vermittelt werden.  \item Statistik, wie sie in den Veranstaltungen \emph{Statistik I} [25008/25009] and \emph{Statistik II} [25020/25021] vermittelt werden.  \end{itemize}\end{conditions}


\end{styleenv}

\begin{learningoutcomes}
Die Vorlesung vermittelt die typische Vorgehensweise bei der Planung und Durchführung einer Simulationsstudie. Im Rahmen einer praxisnahen Darstellung werden Modellbildung und statistische Analyse der simulierten Daten erlernt.


\end{learningoutcomes}

\begin{content}
In einer immer komplexer werdenden Welt ist es oft nicht möglich, interessierende Kenngrößen von Systemen analytisch zu ermitteln, ohne das reale Problem allzu sehr zu vereinfachen. Deshalb werden effiziente Simulationsverfahren immer wichtiger. Ziel dieser Vorlesung ist es, die wichtigsten Grundideen der Simulation vorzustellen und anhand ausgewählter Fallstudien zu erläutern.

 

Überblick über den Inhalt: Diskrete Simulation, Erzeugung von Zufallszahlen, Erzeugung von Zufallszahlen diskreter und stetiger Zufallsvariablen, statistische Analyse simulierter Daten.


\end{content}

\begin{media}Tafel, Folien, Flash-Animationen, Simulationssoftware

\end{media}

\begin{literature}\begin{itemize}\item Skript  \item K.-H. Waldmann / U. M. Stocker: Stochastische Modelle - Eine anwendungsorientierte Einführung; Springer (2004).  \end{itemize}

\textbf{Weiterführende Literatur:}

 \begin{itemize}\item A. M. Law / W. D. Kelton: Simulation Modeling and Analysis (3rd ed); McGraw Hill (2000)  \end{itemize}\end{literature}

\begin{remarks}Die Lehrveranstaltung wird nicht regelmäßig angeboten. Das für zwei Studienjahre im voraus geplante Lehrangebot kann im Internet nachgelesen werden.

\end{remarks}

\end{course}