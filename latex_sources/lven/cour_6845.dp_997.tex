% Lehrveranstaltungsbeschreibung
% Informationsgrad : extern
% Sprache: de
\begin{course}

\setdoclanguagegerman
\coursedegreeprogramme{Informatik}
\coursemodulename{Grundlagen der VWL (S.~\pageref{mod_3035.dp_997})[IN3WWVWL]}
\courseID{2600014}
\coursename{Volkswirtschaftslehre II: Makroökonomie}
\coursecoordination{B. Wigger}

\documentdate{2011-12-16 15:09:44.134117}

\courselevel{1}
\coursecredits{6}
\courseterm{Sommersemester}
\coursehours{3/0/2}
\courseinstructionlanguage{de}

\coursehead

% For index (key word@display). Key word is used for sorting - no Umlauts please.
\index{Volkswirtschaftslehre II: Makrooekonomie@Volkswirtschaftslehre II: Makroökonomie}

% For later referencing
\label{cour_6845.dp_997}


\begin{styleenv}
\begin{assessment}
Die Erfolgskontrolle erfolgt in Form einer schriftlichen Prüfung (120min.) nach § 4 Abs. 2 Nr. 1 SPO.


\end{assessment}

\begin{conditions}Keine.\end{conditions}


\end{styleenv}

\begin{learningoutcomes}
Die Studierenden erwerben die Fähigkeit:

 \begin{itemize}\item den Einfluss ökonomischer Vorgänge auf die gesamtwirtschaftlichen Zielgrößen zu analysieren und zu identifizieren.  \end{itemize}\begin{itemize}\item die Determinanten von Wachstum und Konjunktur zu erkennen und zu erklären, warum verschiedene Ökonomien unterschiedliche Wachstumsgeschwindigkeiten aufweisen, warum es zu Unterauslastung von Produktionspotenzialen kommt, und warum die Arbeitslosigkeit in manchen Ökonomien höher ist als in anderen.  \end{itemize}\begin{itemize}\item die Auswirkung fixer oder flexibler Wechselkurse zu beurteilen und den Einfluss einer unabhängigen Zentralbank zu bewerten.  \end{itemize}\begin{itemize}\item den Einsatz und die Auswirkungen von Geld- und Fiskalpolitik zu beurteilen.  \end{itemize}
\end{learningoutcomes}

\begin{content}
Die Vorlesung verschafft zunächst einen Überblick über die elementaren volkswirtschaftlichen Indikatoren und entwickelt ein erstes Verständnis für makroökonomische Zusammenhänge in einzelnen Volkswirtschaften und in der globalisierten Welt. In verschiedenen Gleichgewichtsmodellen geschlossener und offener Volkswirtschaften wird der Einfluss wirtschaftpolitscher Maßnahmen auf Preise, Zinsen, Beschäftigung und Produktion analysiert. Dynamische Prozesse wie Inflation, Wachstum und Konjunktur sowie die Notwendigkeit und die Grenzen wirtschaftspolitischer Maßnahmen werden untersucht.

 

Kapitel 1: Gesamtwirtschaftliche Zielgrößen

 

Kapitel 2: Bruttoinlandsprodukt: Ein klassisches Modell

 

Kapitel 3: Wachstum

 

Kapitel 4: Geld und Inflation

 

Kapitel 5: Die offene Volkswirtschaft

 

Kapitel 6: IS-LM Modell und Konjunktur

 

Kapitel 7: Mundell-Fleming Modell

 

Kapitel 8: Gesamtwirtschaftliches Gleichgewicht

 

Kapitel 9: Arbeitslosigkeit


\end{content}



\begin{literature}\textbf{Weiterführende Literatur:}

 

Sieg, G. (2008): \emph{Volkswirtschaftslehre;} 2. Aufl., Oldenbourg.

\end{literature}



\end{course}