% Lehrveranstaltungsbeschreibung
% Informationsgrad : extern
% Sprache: de
\begin{course}

\setdoclanguagegerman
\coursedegreeprogramme{Informatik}
\coursemodulename{Teamarbeit in der Software-Entwicklung (S.~\pageref{mod_10785.dp_997})[IN2INSWPS]}
\courseID{24511}
\coursename{Teamarbeit und Präsentation in der Software-Entwicklung}
\coursecoordination{G. Snelting, Dozenten der Fakultät für Informatik}

\documentdate{2011-07-04 13:25:46.342904}

\courselevel{2}
\coursecredits{2}
\courseterm{Winter-/Sommersemester}
\coursehours{1}
\courseinstructionlanguage{de}

\coursehead

% For index (key word@display). Key word is used for sorting - no Umlauts please.
\index{Teamarbeit und Praesentation in der Software-Entwicklung@Teamarbeit und Präsentation in der Software-Entwicklung}

% For later referencing
\label{cour_10787.dp_997}


\begin{styleenv}
\begin{assessment}
Die Erfolgskontrolle erfolgt als Erfolgskontrolle anderer Art nach § 4 Abs. 2 Nr. 3 SPO.

 

Teilnehmer müssen als Team von ca. 5 Studierenden Präsentationen zu den Software-Entwicklungsphasen Pflichtenheft, Entwurf, Implementierung, Qualitätssicherung sowie eine Abschlusspräsentation von je 15 Minuten erarbeiten. Teilnehmer müssen Dokumente zur Projektplanung, insbesondere Qualitätssicherungsplan und Implementierungsplan vorlegen und umsetzen.


\end{assessment}

\begin{conditions}Der erfolgreiche Abschluss der Module \emph{Grundbegriffe der Informatik} [IN1INGI] und\emph{ Programmieren} [IN1INPROG] wird vorausgesetzt.

\end{conditions}

\begin{recommendations}Die Veranstaltung sollte erst belegt werden, wenn alle Scheine aus den ersten beiden Semestern erworben wurden.

\end{recommendations}
\end{styleenv}

\begin{learningoutcomes}
Die Teilnehmer erwerben wichtige nicht-technische Kompetenzen zur Durchfühung von Softwareprojekten im Team. Dazu gehören Sprachkompetenz und kommunikative Kompetenz sowie Techniken der Teamarbeit, der Präsentation und der Projektplanung.


\end{learningoutcomes}

\begin{content}
Auseinandersetzung mit der Arbeit im Team, Kommunikations-, Organisations- und Konfliktbehandungsstrategien; Erarbeitung von Präsentationen zu Pflichtenheft, Entwurf, Implementierung, Qualitätssicherung, Abschlusspräsentation; Projektplanungstechniken (z.B. Netzplantechnik, Phasenbeauftragte).


\end{content}







\end{course}