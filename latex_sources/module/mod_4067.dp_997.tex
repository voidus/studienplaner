% Modulbeschreibung 
% Informationsgrad : extern
% Sprache: de
\begin{module}

\setdoclanguagegerman
\moduledegreeprogramme{Informatik (B.Sc.)}
\modulesubject{}
\moduleID{IN3INAWA}
\modulename{Advanced Web Applications}
\modulecoordination{S. Abeck}

\documentdate{2011-08-22 11:33:06.666046}

\modulecredits{4}
\moduleduration{1}
\modulecycle{Jedes Semester}



\modulehead

% For index (key word@display). Key word is used for sorting - no Umlauts please.
\index{Advanced Web Applications@Advanced Web Applications (M)}

% For later referencing
\label{mod_4067.dp_997}

\begin{courselist}
24604/24153 & Advanced Web Applications (S.~\pageref{cour_5663.dp_997}) & 2/0 & W/S & 4 & S. Abeck\\
\end{courselist}

\begin{styleenv}
\begin{assessment}
Die Erfolgskontrolle erfolgt in Form einer mündlichen Prüfung im Umfang von i.d.R. 20 Minuten nach § 4 Abs. 2 Nr. 2 SPO.

 

Die Modulnote ist die Note der mündlichen Prüfung.


\end{assessment}

\begin{conditions}Keine.\end{conditions}

\begin{recommendations}1. Fundierte Telematik-Kenntnisse, insbes. zu Schichentenarchitekturen, Kommunikationsprotokollen (insbes. Anwendungsschicht), Extensible Markup Language. \newline
2. Fundierte Softwaretechnik-Kenntnisse, insbes. zu Softwarearchitekturen und deren Modellierung mittels Unified Modeling Language.

\end{recommendations}
\end{styleenv}

\begin{learningoutcomes}
\begin{itemize}\item Die Architektur von mehrschichtigen und dienstorientierten Anwendungssystemen ist verstanden.  \item Die Softwarearchitektur einer Web-Anwendung kann modelliert werden.  \item Die wichtigsten Prinzipien traditioneller Softwareentwicklung und des entsprechenden Entwicklungsprozesses sind bekannt.  \item Die Verfeinerung höherstufiger Prozessmodelle sowie deren Abbildung auf eine dienstorientierte Architektur sind verstanden.  \end{itemize}
\end{learningoutcomes}

\begin{content}
Das Modul setzt sich aus den folgenden Kurseinheiten zusammen:

 \begin{itemize}\item GRUNDLAGEN FORTGESCHRITTENER WEBANWENDUNGEN: Mehrschichtige Anwendungsarchitekturen, insbesondere die dienstorientierte Architektur (Service-Oriented Architecture, SOA) basiered auf Webservice-Standards wie XML (Extensible Markup Language) und WSDL (Web Services Description Language) werden beschrieben.   \item DIENSTENTWURF: Der Entwicklunsprozess wird um Ansätze zur Abbildung von Geschäftsprozessen auf dienstorientierte Web-Anwendungen und zum Entwurf der dabei notwendigen Dienste erweitert.  \item BENUTZERINTERAKTION: Diese Kursseinheit behandelt die modellgetriebene Sofwareentwicklung von fortgeschrittenen, benutzerzentrierten Web-Anwendungen basierend auf UML (Unified Modeling Language) und MDA (Model-driven Architecture).   \item IDENTITÄTSMANAGEMENT: Die wichtigsten Funktionsbausteine eines Identitätsmanagements werden eingeführt und die spezifischen Anforderungen an eine dienstorientierte Lösung werden abgeleitet.  \item IT-MANAGEMENT: Die Kurseinheit betrachtet prozessorientierte Managementstandards, die durch standardisierte Managementkomponenten umgesetzt werden können.  \end{itemize}
\end{content}

\begin{remarks}\textcolor{red}{Dieses Modul wurde in dieser Form letztmalig im SS 11 angeboten, Prüfungen werden noch bis WS 2012/13 für Wiederholer an- \newline
geboten.}

\end{remarks}

\end{module}

