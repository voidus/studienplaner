% Lehrveranstaltungsbeschreibung
% Informationsgrad : extern
% Sprache: de
\begin{course}

\setdoclanguagegerman
\coursedegreeprogramme{Informatik}
\coursemodulename{Biomedizinische Technik I (S.~\pageref{mod_4005.dp_997})[IN3EITBIOM]}
\courseID{23270}
\coursename{Biomedizinische Messtechnik II}
\coursecoordination{A. Bolz}

\documentdate{2010-07-15 11:58:17.879035}

\courselevel{3}
\coursecredits{5}
\courseterm{Sommersemester}
\coursehours{3}
\courseinstructionlanguage{de}

\coursehead

% For index (key word@display). Key word is used for sorting - no Umlauts please.
\index{Biomedizinische Messtechnik II@Biomedizinische Messtechnik II}

% For later referencing
\label{cour_8149.dp_997}


\begin{styleenv}
\begin{assessment}
Die Erfolgskontrolle erfolgt in Form einer mündlichen Prüfung im Umfang von i.d.R. 20 Minuten nach § 4 Abs. 2 Nr. 2 SPO.


\end{assessment}

\begin{conditions}Keine.\end{conditions}


\end{styleenv}

\begin{learningoutcomes}

\end{learningoutcomes}

\begin{content}
Blutdruckmessung: Physikalische und physiologische Grundlagen, Analyse der Blutdruckkurven. Nicht-invasive Methoden: Korotkow- und oszillometrische Blutdruckmessung. Invasive Methoden: Dynamische Eigenschaften des Messsystems, Übertragungsfunktion, Messung der Systemantwort, Einflüsse der Systemeigenschafften auf die Systemantwort, Einflüsse auf die Druckmessung, Tip-Katheter.\newline
\newline
 Blutflussmessung: Physikalische und physiologische Grundlagen, elektromagnetische Flussmessgeräte: DC-, AC- Erregung, Ultraschallflussmessgeräte: Laufzeit-, Dopplermessgeräte.\newline
\newline
 Messung des Herzzeitvolumens: Physikalische und physiologische Grundlagen, Fick'sches Prinzip, Indikatorverdünnungsmethode, elektrische Impedanzplethysmographie, Diagnose.\newline
\newline
 Elektrostimulation: Physikalische und physiologische Grundlagen, DC-, Nieder- und Mittelfrequenzströme, lokale und Systemkompatibilität, physiologische Schwelle, Spannungs- und Stromquellen, Analyse unterschiedlicher Wellenformen.\newline
\newline
 Defibrillation: Elektrophysiologische Grundlagen, normaler und krankhafter kardialer Rhythmus, technische Realisierung: Externe und implantierbare Defibrillatoren, halbautomatische und automatische Systeme, Sicherheitsüberlegungen.\newline
\newline
 Herzschrittmacher: Elektrophysiologische Grundlagen, Indikationen, Einkammer und Zweikammersysteme: V00 ... DDDR, Schrittmachertechnologie: Elektroden, Gehäuse, Energie, Elektronik


\end{content}







\end{course}