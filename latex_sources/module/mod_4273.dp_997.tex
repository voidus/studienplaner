% Modulbeschreibung 
% Informationsgrad : extern
% Sprache: de
\begin{module}

\setdoclanguagegerman
\moduledegreeprogramme{Informatik (B.Sc.)}
\modulesubject{EF Informationsmanagement im Ingenieurwesen}
\moduleID{IN3MACHPLM}
\modulename{Product Lifecycle Management}
\modulecoordination{Thomas Maier}

\documentdate{2011-07-25 11:03:46.579596}

\modulecredits{6}
\moduleduration{1}
\modulecycle{Jedes 2. Semester, Wintersemester}



\modulehead

% For index (key word@display). Key word is used for sorting - no Umlauts please.
\index{Product Lifecycle Management@Product Lifecycle Management (M)}

% For later referencing
\label{mod_4273.dp_997}

\begin{courselist}
2121350 & Product Lifecycle Management (S.~\pageref{cour_7495.dp_997}) & 3/1 & W & 6 & J. Ovtcharova\\
\end{courselist}

\begin{styleenv}
\begin{assessment}
Die Erfolgskontrolle wird in der Lehrveranstaltungsbeschreibung erläutert.


\end{assessment}

\begin{conditions}Die Module \textbf{\emph{Product Lifecycle Managemet}} [N3MACHPLM] und \textbf{\emph{Technische Informationssysteme}} [IN3INMACHTI] müssen im Ergänzungsfach Informationsmanagement im Ingenieurwesen geprüft werden.

\end{conditions}


\end{styleenv}

\begin{learningoutcomes}
Der/ die Studierende

 \begin{itemize}\item besitzt grundlegende Informationen über vielfältigen Informationen, die während des gesamten Produktlebenszyklus entstehen,   \item beherrscht Methoden des PLM zur Durchführung von Geschäftsprozessen,   \item versteht die Planung und Steuerung von Ressourcen, basierend auf den verwendeten Methoden der Informationsverarbeitung (Informationsflussgestaltung und Datenmodellierung).  \end{itemize}
\end{learningoutcomes}

\begin{content}
In der Vorlesung wird der Management- und Organisationsansatz des Product Lifecycle Management dargestellt. Dabei wird auf folgende grundlegende Problemstellungen eingegangen:

 \begin{itemize}\item Welche Anforderungen werden an PLM gestellt?  \item Welche Funktionen und Aufgaben muss ein PLM-System aufgrund der Anforderungen erfüllen?  \item Wie werden diese Funktionen und Aufgaben auf der IT-Ebene umgesetzt?  \item Welches Nutzenpotential bietet PLM heutigen Unternehmen?  \end{itemize}

Welche Kosten verursacht die Einführung von PLM in einem Unternehmen?


\end{content}



\end{module}

