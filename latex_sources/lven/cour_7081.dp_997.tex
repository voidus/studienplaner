% Lehrveranstaltungsbeschreibung
% Informationsgrad : extern
% Sprache: de
\begin{course}

\setdoclanguagegerman
\coursedegreeprogramme{Informatik}
\coursemodulename{Makroökonomische Theorie (S.~\pageref{mod_2451.dp_997})[IN3WWVWL8]}
\courseID{2520543}
\coursename{Wachstumstheorie}
\coursecoordination{M. Hillebrand}

\documentdate{2012-01-09 16:32:28.369446}

\courselevel{4}
\coursecredits{4,5}
\courseterm{Sommersemester}
\coursehours{2/1}
\courseinstructionlanguage{en}

\coursehead

% For index (key word@display). Key word is used for sorting - no Umlauts please.
\index{Wachstumstheorie@Wachstumstheorie}

% For later referencing
\label{cour_7081.dp_997}


\begin{styleenv}
\begin{assessment}
Die Erfolgskontrolle erfolgt in Abhängigkeit der Teilnehmerzahl in Form einer schriftlichen (60min.) oder mündlichen (20min.) Prüfung (nach §4(2), 1 o. 2) zu Beginn der vorlesungsfreien Zeit des Semesters.

 

Die Prüfung wird in jedem Semester angeboten und kann zu jedem ordentlichen Prüfungstermin wiederholt werden.


\end{assessment}

\begin{conditions}Keine.\end{conditions}

\begin{recommendations}Grundlegende mikro- und makroökonomische Kenntnisse, wie sie beispielsweise in den Veranstaltungen \emph{Volkswirtschaftslehre I (Mikroökonomie)} [2600012] und \emph{Volkswirtschaftslehre II (Makroökonomie)} [2600014] vermittelt werden, werden vorausgesetzt.

 

Aufgrund der inhaltlichen Ausrichtung der Veranstaltung wird ein Interesse an quantitativ-mathematischer Modellierung vorausgesetzt.

\end{recommendations}
\end{styleenv}

\begin{learningoutcomes}
Der/die Studierende

 \begin{itemize}\item ist in der Lage, mit Hilfe eines analytischen Instrumentariums grundlegende Fragestellungen der Wachstums zu bearbeiten,  \item kann sich selbstständig ein fundiertes Urteil über ökonomische Fragestellungen bilden.  \end{itemize}
\end{learningoutcomes}

\begin{content}
Gegenstand der Wachstumstheorie ist die Erklärung und Untersuchung des langfristigen Wachstums von Volkswirtschaften. Im Rahmen der Vorlesung werden Modelle entwickelt, die eine mathematische Beschreibung des Wachstumsprozesses und seiner strukturellen Determinanten liefern. Unter Verwendung der Theorie zeitdiskreter dynamischer Systeme kann das Langfristverhalten solcher Modelle analysiert werden. So können beispielsweise Bedingungen für das Auftreten stabiler, zyklischer oder irregulär schwankender (chaotischer) Wachstumspfade abgeleitet werden. Aufbauend auf den dabei gewonnenen Erkenntnissen werden im Rahmen der Vorlesung wirtschaftspolitische Möglichkeiten zur Erhöhung bzw. Stabilisierung des Wirtschaftswachstums und beispielsweise die Auswirkungen von Umverteilungs- und Rentenversicherungssystemen auf den Wachstumsprozess diskutiert.


\end{content}





\begin{remarks}Nach Absprache mit den Studierenden besteht die Möglichkeit, die Lehrveranstaltung in englischer Sprache zu halten.

\end{remarks}

\end{course}