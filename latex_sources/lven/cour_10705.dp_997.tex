% Lehrveranstaltungsbeschreibung
% Informationsgrad : extern
% Sprache: de
\begin{course}

\setdoclanguagegerman
\coursedegreeprogramme{Informatik}
\coursemodulename{Proseminar (S.~\pageref{mod_2385.dp_997})[IN2INPROSEM]}
\courseID{24059/24544}
\coursename{Anthropomatik: Von der Theorie zur Anwendung }
\coursecoordination{J. Beyerer, U. Hanebeck}

\documentdate{2011-08-29 14:02:37.351416}

\courselevel{2}
\coursecredits{3}
\courseterm{Winter-/Sommersemester}
\coursehours{2}
\courseinstructionlanguage{}

\coursehead

% For index (key word@display). Key word is used for sorting - no Umlauts please.
\index{Anthropomatik: Von der Theorie zur Anwendung @Anthropomatik: Von der Theorie zur Anwendung }

% For later referencing
\label{cour_10705.dp_997}


\begin{styleenv}
\begin{assessment}
Die Erfolgskontrolle erfolgt durch Ausarbeiten einer schriftlichen Proseminararbeit sowie der Präsentation derselbigen als Erfolgskontrolle anderer Art nach § 4 Abs. 2 Nr. 3 SPO.\newline
\newline
 Die Proseminarnote setzt sich aus der Note der schriftlichen Ausarbeitung und der Präsentation zusammen.


\end{assessment}

\begin{conditions}Keine.\end{conditions}

\begin{recommendations}Kenntnisse aus der Vorlesung Kognitive Systeme sind hilfreich, aber nicht zwingend erforderlich.

\end{recommendations}
\end{styleenv}

\begin{learningoutcomes}
\begin{itemize}\item Die Studierenden wenden ihre Kenntnisse aus dem Bereich Kognitive Systeme/Robotik an und vertiefen diese gleichzeitig.  \item Den Studierenden wird der Übergang von der Grundlagenforschung hin zur konkreten Anwendung vermittelt.  \item Das Verfassen der Proseminararbeit liefert erste Erfahrungen mit dem Umgang fremdverfasster wissenschaftlicher Arbeiten. Dazu gehört neben der selbstständigen Literaturrecherche zu einem vorgegebenen Thema auch die Bewertung der gefundenen Literatur auf ihre Relevanz für die Aufgabenstellung.  \item Durch Vermittlung von Präsentationstechniken und Anleitung zur Erstellung einer wissenschaftlichen Ausarbeitung bereitet das Proseminar zudem auf die Abfassung der Bachelorarbeit vor. Zu diesem Zweck findet ein Workshop \emph{Einführung in das wissenschaftliche Schreiben und Vortragen} statt.  \end{itemize}
\end{learningoutcomes}

\begin{content}
Das Forschungsgebiet Anthropomatik umfasst wichtige Themen wie zum Beispiel die multimodale Interaktion von Menschen mit technischen Systemen, humanoide Roboter, Bildverstehen, Lernen, Erkennen und Verstehen von Situationen oder die Sensordatenverarbeitung. Ziel der Anthropomatik in diesem Umfeld ist die Erforschung und Entwicklung menschgerechter und menschzentrierter Systeme mit den Mitteln der Informatik. Voraussetzung dafür ist ein grundlegendes Verständnis und die Modellierung des Menschen, z.B. bezüglich seiner Anatomie, seiner Motorik, seiner Wahrnehmung und Informationsverarbeitung und seines Verhaltens.

 

Im Rahmen dieses Proseminars sollen ausgewählte theoretische Arbeiten aus der Anthropomatik einerseits und deren Umsetzung in praktikable Anwendungen andererseites präsentiert werden. Um das breite Spektrum von Grundlagenforschung und angewandter Forschung abzudecken wird das Proseminar gemeinsam vom Lehrstuhl für Intelligente Sensor-Aktor-Systeme (ISAS), dem Lehrstuhl für Interaktive Echtzeitsyteme (IES) und dem Fraunhofer-Institut für Optronik, Systemtechnik und Bildauswertung IOSB angeboten. Beide Lehrstühle gehören dem Institut für Anthropomatik der Fakultät für Informatik an.


\end{content}







\end{course}