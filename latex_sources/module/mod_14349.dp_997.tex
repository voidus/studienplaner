% Modulbeschreibung 
% Informationsgrad : extern
% Sprache: de
\begin{module}

\setdoclanguagegerman
\moduledegreeprogramme{Informatik (B.Sc.)}
\modulesubject{EF Mathematik}
\moduleID{MATHAG26}
\modulename{Graphentheorie}
\modulecoordination{M. Axenovich}

\documentdate{2011-10-06 17:52:58.948966}

\modulecredits{9}
\moduleduration{1}
\modulecycle{Unregelmäßig}



\modulehead

% For index (key word@display). Key word is used for sorting - no Umlauts please.
\index{Graphentheorie@Graphentheorie (M)}

% For later referencing
\label{mod_14349.dp_997}

\begin{courselist}
GraphTH & Graphentheorie (S.~\pageref{cour_14353.dp_997}) & 4+2 & W/S & 8 & M. Axenovich\\
\end{courselist}

\begin{styleenv}
\begin{assessment}
Prüfung: schriftliche oder mündliche Prüfung

 

Notenbildung: Note der Prüfung


\end{assessment}

\begin{conditions}Keine.\end{conditions}

\begin{recommendations}Folgende Module sollten bereits belegt worden sein (Empfehlung):

 

Lineare Algebra 1+2, Analysis 1+2

\end{recommendations}
\end{styleenv}

\begin{learningoutcomes}
Die Lernziele umfassen: Verständnis struktureller und algorithmischer Eigenschaften von Graphen, Kenntnisse über Färbung von Graphen, unvermeidliche Strukturen in Graphen, probabilistische Methoden, Eigenschaften großer Graphen


\end{learningoutcomes}

\begin{content}
Der Kurs über Graphentheorie spannt den Bogen von den grundlegenden Grapheneigenschaften, die auf Euler zurückgehen, bis hin zu modernen Resultaten und Techniken in der extremalen Graphentheorie. Insbesondere werden die folgenden Themen behandelt: Struktur von Bäumen, Pfade, Zykel, Wege in Graphen, unvermeidliche Teilgraphen in dichten Graphen, planare Graphen, Graphenfärbung, Ramsey-Theorie, Regularität in Graphen.


\end{content}



\end{module}

