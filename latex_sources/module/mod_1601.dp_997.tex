% Modulbeschreibung 
% Informationsgrad : extern
% Sprache: de
\begin{module}

\setdoclanguagegerman
\moduledegreeprogramme{Informatik (B.Sc.)}
\modulesubject{EF Betriebswirtschaftslehre}
\moduleID{IN3WWBWL1}
\modulename{CRM und Servicemanagement}
\modulecoordination{A. Geyer-Schulz}

\documentdate{2011-12-30 10:22:22.157757}

\modulecredits{9}
\moduleduration{1}
\modulecycle{Jedes Semester}



\modulehead

% For index (key word@display). Key word is used for sorting - no Umlauts please.
\index{CRM und Servicemanagement@CRM und Servicemanagement (M)}

% For later referencing
\label{mod_1601.dp_997}

\begin{courselist}
2540508 & Customer Relationship Management (S.~\pageref{cour_4435.dp_997}) & 2/1 & W & 4,5 & A. Geyer-Schulz\\
2540522 & Analytisches CRM (S.~\pageref{cour_4901.dp_997}) & 2/1 & S & 4,5 & A. Geyer-Schulz\\
2540520 & Operatives CRM (S.~\pageref{cour_4899.dp_997}) & 2/1 & W & 4,5 & A. Geyer-Schulz\\
\end{courselist}

\begin{styleenv}
\begin{assessment}
Die Modulprüfung erfolgt in Form von Teilprüfungen (nach §4 (2) SPO) zu den gewählten Lehrveranstaltungen, mit denen in Summe die Mindestanforderungen an Leistungspunkten erfüllt wird. Dabei wird jede Lehrveranstaltung in Form einer 60min. Klausur (nach §4(2), 1 SPO) und durch Ausarbeiten von Übungsaufgaben (nach §4(2), 3 SPO) geprüft.

 

Die Noten der einzelnen Teilprüfungen setzen sich zu ungefähr 90\% aus der Klausurnote (100 von 112 Punkte) und zu ungefähr 10\% aus der Übungsleistung (12 von 112 Punkte) zusammen. Im Falle der bestandenen Klausur (50 Punkte) werden für die Berechnung der Note die Punkte der Übungsleistung zu den Punkten der Klausur addiert. Die Erfolgskontrolle wird bei jeder Lehrveranstaltung dieses Moduls beschrieben.

 

Die Gesamtnote des Moduls wird aus den mit Leistungspunkten gewichteten Teilnoten der einzelnen Lehrveranstaltungen gebildet und nach der ersten Nachkommastelle abgeschnitten.


\end{assessment}

\begin{conditions}Das Modul ist nur zusammen mit dem Pflichtmodul \emph{Grundlagen der BWL} [IN3WWBWL] prüfbar.

\end{conditions}


\end{styleenv}

\begin{learningoutcomes}
 

Der/die Studierende

 \begin{itemize}\item versteht Servicemanagement als betriebswirtschaftliche Grundlage für Customer Relationship Management und kennt die sich daraus ergebenden Konsequenzen für die Unternehmensführung, Organisation und die einzelnen betrieblichen Teilbereiche,  \item entwickelt und gestaltet Servicekonzepte und Servicesysteme auf konzeptueller Ebene,  \item bearbeitet Fallstudien im Team unter Einhaltung von Zeitvorgaben und zieht dabei internationale Literatur aus dem Bereich heran,  \item kennt die aktuellen Entwicklungen im CRM-Bereich in Wissenschaft und Praxis,  \item versteht die wichtigsten wissenschaftlichen Methoden (BWL, Statistik, Informatik) des analytischen CRM und kann diese Methoden selbständig auf Standardfälle anwenden,  \item gestaltet, implementiert und analysiert operative CRM-Prozesse in konkreten Anwendungsbereichen (wie Marketing Kampagnen Management, Call Center Management, …).  \end{itemize}
\end{learningoutcomes}

\begin{content}
Im Modul \emph{CRM und Servicemanagement} [IN3WWBWL1] werden die Grundlagen moderner kunden- und serviceorientierter Unternehmensführung und ihre praktische Unterstützung durch Systemarchitekturen und CRM-Softwarepakete vermittelt. Customer Relationship Management (CRM) als Unternehmensstrategie erfordet Servicemanagement und dessen konsequente Umsetzung in allen Unternehmensbereichen.

 

Im \emph{operativen CRM} [2540520] wird die Gestaltung kundenorientierter IT-gestützter Geschäftsprozesse auf der Basis der Geschäftsprozessmodellierung an konkreten Anwendungsszenarien erläutert (z.B. Kampagnenmanagement, Call Center Management, Sales Force Management, Field Services, ... ).

 

Im \emph{analytischen CRM} [2540522] wird Wissen über Kunden auf aggregierter Ebene für betriebliche Entscheidungen (z.B. Sortimentsplanung, Kundenloyalität, Kundenwert, ...) und zur Verbesserung von Services nutzbar gemacht. Voraussetzung dafür ist die enge Integration der operativen Systeme mit einem Datawarehouse, die Entwicklung eines kundenorientierten und flexiblen Reportings, sowie die Anwendung statistischer Analysemethoden (z.B. Clustering, Regression, stochastische Modelle, ...).

 
\end{content}

\begin{remarks}Die Lehrveranstaltung \emph{Customer Relationship Management} [2540508] wird auf Englisch gehalten.

\end{remarks}

\end{module}

