% Lehrveranstaltungsbeschreibung
% Informationsgrad : extern
% Sprache: de
\begin{course}

\setdoclanguagegerman
\coursedegreeprogramme{Informatik}
\coursemodulename{Stochastische Methoden und Simulation (S.~\pageref{mod_3855.dp_997})[IN3WWOR4], Methodische Grundlagen des OR (S.~\pageref{mod_3833.dp_997})[IN3WWOR3]}
\courseID{2550679}
\coursename{Stochastische Entscheidungsmodelle I}
\coursecoordination{K. Waldmann}

\documentdate{2011-02-25 15:22:23.715753}

\courselevel{4}
\coursecredits{5}
\courseterm{Wintersemester}
\coursehours{2/1/2}
\courseinstructionlanguage{de}

\coursehead

% For index (key word@display). Key word is used for sorting - no Umlauts please.
\index{Stochastische Entscheidungsmodelle I@Stochastische Entscheidungsmodelle I}

% For later referencing
\label{cour_5703.dp_997}


\begin{styleenv}
\begin{assessment}

\end{assessment}

\begin{conditions}Keine.\end{conditions}


\end{styleenv}

\begin{learningoutcomes}
Die Studierenden erwerben die Kenntnis moderner Methoden der stochastischen Modellbildung und werden dadurch in die Lage versetzt, einfache stochastische Systeme adäquat zu beschreiben und zu analysieren.


\end{learningoutcomes}

\begin{content}
Aufbauend auf dem Modul \emph{Einführung in das Operations Research} werden quantitative Verfahren zur Planung, Analyse und Optimierung von dynamischen Systemen vorgestellt. Einen Schwerpunkt bilden dabei stochastische Methoden und Modelle. Das bedeutet, dass Problemstellungen betrachtet werden, bei denen zufällige Einflüsse eine wesentliche Rolle spielen. Es wird untersucht, wie solche Systeme sich modellieren lassen, welche Eigenschaften und Kenngrößen zur Beschreibung der Modelle verwendet werden können und was für typische Problemstellungen in diesem Zusammenhang auftreten.

 

Überblick über den Inhalt: Markov Ketten, Poisson Prozesse, Markov Ketten in stetiger Zeit, Wartesysteme.


\end{content}

\begin{media}Tafel, Folien, Flash-Animationen, Simulationssoftware

\end{media}

\begin{literature}Waldmann, K.H. , Stocker, U.M. (2004): Stochastische Modelle - eine anwendungsorientierte Einführung; Springer

 

\textbf{Weiterführende Literatur:}

 

Norris, J.R. (1997): Markov Chains; Cambridge University Press

 

Bremaud, P. (1999): Markov Chains, Gibbs Fields, Monte Carlo Simulation, and Queues; Springer

\end{literature}



\end{course}