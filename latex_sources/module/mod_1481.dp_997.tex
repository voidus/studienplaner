% Modulbeschreibung 
% Informationsgrad : extern
% Sprache: de
\begin{module}

\setdoclanguagegerman
\moduledegreeprogramme{Informatik (B.Sc.)}
\modulesubject{}
\moduleID{IN1MATHHM}
\modulename{Höhere Mathematik}
\modulecoordination{C. Schmoeger}

\documentdate{2008-04-14 09:32:27}

\modulecredits{15}
\moduleduration{2}
\modulecycle{Jedes 2. Semester, Wintersemester}



\modulehead

% For index (key word@display). Key word is used for sorting - no Umlauts please.
\index{Hoehere Mathematik@Höhere Mathematik (M)}

% For later referencing
\label{mod_1481.dp_997}

\begin{courselist}
01330 & Höhere Mathematik I (Analysis) für die Fachrichtung Informatik (S.~\pageref{cour_6173.dp_997}) & 4/2 & W & 9 & C. Schmoeger\\
01868 & Höhere Mathematik II (Analysis) für die Fachrichtung Informatik (S.~\pageref{cour_6987.dp_997}) & 3/1 & S & 6 & C. Schmoeger\\
\end{courselist}

\begin{styleenv}
\begin{assessment}
Die Erfolgskontrolle erfolgt in Form einer schriftlichen Gesamtprüfung im Umfang von 240 Minuten nach § 4 Abs. 2 Nr. 1 SPO und einer Erfolgskontrolle anderer Art nach § 4 Abs. 2 Nr. 3 SPO (mindestens ein Übungsschein aus den Lehrveranstaltungen \emph{Höhere Mathematik I [1330] oder Höhere Mathematik II} [1868]).

 

Die Modulnote ist die Note der schriftlichen Prüfung.

 

\textbf{Achtung:} Diese Prüfung oder die Prüfung zum Modul \emph{Lineare Algebra} [IN1MATHLA] oder zum Modul \emph{Analysis} [IN1MATHANA] oder zum Modul \emph{Lineare Algebra und Analytische Geometrie} [IN1MATHLAAG] ist bis zum Ende des 2. Fachsemesters anzutreten und bis zum Ende des 3. Fachsemesters zu bestehen, da sie Bestandteil der Orientierungsprüfung nach § 8 Abs. 1 SPO ist.


\end{assessment}

\begin{conditions}Keine.\end{conditions}


\end{styleenv}

\begin{learningoutcomes}
Die Studierenden sollen am Ende des Moduls

 \begin{itemize}\item den Übergang von Schule zu Universität bewältigt haben,  \item mit logischem Denken und strengen Beweisen vertraut sein,  \item die Methoden und grundlegenden Strukturen der (reellen) Analysis beherrschen.  \end{itemize}
\end{learningoutcomes}

\begin{content}
\textbf{HM I:}

 \begin{itemize}\item \textbf{Reelle Zahlen} (Körpereigenschaften, natürliche Zahlen, Induktion)  \item \textbf{Konvergenz in R} ( Folgen, Reihen, Potenzreihen, elementare Funktionen, q-adische Entwicklung reeller Zahlen)  \item \textbf{Funktionen }(Grenzwerte bei Funktionen, Stetigkeit, Funktionenfolgen und -reihen)  \item \textbf{Differentialrechnung }(Ableitungen, Mittelwertsätze, Regel v. de l'Hospital, Satz von Taylor)  \item \textbf{Integralrechnung} (Riemann- Integral, Hauptsätze, Substitution, part. Integration, uneigentliche Integrale)  \item \textbf{Fourierreihen}  \end{itemize}

\textbf{HM II:}

 \begin{itemize}\item \textbf{Der Raum R\textsuperscript{n}} (Konvergenz, Grenzwerte bei Funktionen, Stetigkeit)  \item \textbf{Differentialrechnung im R\textsuperscript{n}} (partielle Ableitungen, (totale) Ableitung, Taylorentwicklung, Extremwertberechnungen)  \item \textbf{Das mehrdimensionale Riemann- Integral} (Fubini, Volumenberechnung mit Cavalieri, Substitution, Polar-, Zylinder-, Kugelkoordinaten)  \item \textbf{Differentialgleichungen} (Trennung der Ver., lineare DGL 1. Ordnung, Bernoulli-DGL, Riccati-DGL, lineare Systeme, lineare DGL höherer Ordnung)  \item \textbf{Integraltransformationen}  \end{itemize}
\end{content}

\begin{remarks}Moduldauer: 2 Semester

\end{remarks}

\end{module}

