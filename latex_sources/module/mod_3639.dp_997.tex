% Modulbeschreibung 
% Informationsgrad : extern
% Sprache: de
\begin{module}

\setdoclanguagegerman
\moduledegreeprogramme{Informatik (B.Sc.)}
\modulesubject{EF Mathematik}
\moduleID{IN3MATHST02}
\modulename{Wahrscheinlichkeitstheorie}
\modulecoordination{N. Bäuerle}

\documentdate{2011-10-06 18:00:55.538437}

\modulecredits{6}
\moduleduration{1}
\modulecycle{Jedes 2. Semester, Sommersemester}



\modulehead

% For index (key word@display). Key word is used for sorting - no Umlauts please.
\index{Wahrscheinlichkeitstheorie@Wahrscheinlichkeitstheorie (M)}

% For later referencing
\label{mod_3639.dp_997}

\begin{courselist}
1598 & Wahrscheinlichkeitstheorie (S.~\pageref{cour_8039.dp_997}) & 3/1 & S & 6 & N. Bäuerle, N. Henze, B. Klar, G. Last\\
\end{courselist}

\begin{styleenv}
\begin{assessment}
Prüfung: schriftliche oder mündliche Prüfung \newline
Notenbildung: Note der Prüfung


\end{assessment}

\begin{conditions}Die Module \emph{Einführung in die Stochastik} [IN3MATHST01] und \emph{Markovsche Ketten} [IN3MATHST03] müssen geprüft werden.

 

Das Modul \emph{Proseminar Mathematik} [IN3MATHPS] muss geprüft werden.

\end{conditions}

\begin{recommendations}Folgende Module sollten bereits belegt worden sein (Empfehlung):\newline
Analysis 3\newline
Einführung in die Stochastik

\end{recommendations}
\end{styleenv}

\begin{learningoutcomes}
Die Studierenden sollen am Ende des Moduls:

 \begin{itemize}\item mit modernen wahrscheinlichkeitstheoretischen Methoden vertraut sein,  \item Grundlagen für die Stochastik, Statistik und die moderne Finanzmathematik erworben haben.  \end{itemize}
\end{learningoutcomes}

\begin{content}
\begin{itemize}\item Maß-Integral  \item Monotone und majorisierte Konvergenz  \item Lemma von Fatou  \item Nullmengen u. Maße mit Dichten  \item Satz von Radon-Nikodym  \item Produkt-$\sigma{}$-Algebra  \item Familien von unabhängigen Zufallsvariablen  \item Transformationssatz für Dichten  \item Schwache Konvergenz  \item Charakteristische Funktion  \item Zentraler Grenzwertsatz  \item Bedingte Erwartungswerte  \item Zeitdiskrete Martingale und Stoppzeiten  \end{itemize}
\end{content}



\end{module}

