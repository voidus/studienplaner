% Modulbeschreibung 
% Informationsgrad : extern
% Sprache: de
\begin{module}

\setdoclanguagegerman
\moduledegreeprogramme{Informatik (B.Sc.)}
\modulesubject{EF Operations Research}
\moduleID{IN3WWOR3}
\modulename{Methodische Grundlagen des OR}
\modulecoordination{O. Stein}

\documentdate{2011-12-15 17:07:19.714146}

\modulecredits{9}
\moduleduration{1}
\modulecycle{Jedes Semester}



\modulehead

% For index (key word@display). Key word is used for sorting - no Umlauts please.
\index{Methodische Grundlagen des OR@Methodische Grundlagen des OR (M)}

% For later referencing
\label{mod_3833.dp_997}

\begin{courselist}
2550111 & Nichtlineare Optimierung I (S.~\pageref{cour_7885.dp_997}) & 2/1 & S & 4,5 & O. Stein\\
2550113 & Nichtlineare Optimierung II (S.~\pageref{cour_7883.dp_997}) & 2/1 & S & 4,5 & O. Stein\\
2550134 & Globale Optimierung I (S.~\pageref{cour_7879.dp_997}) & 2/1 & W & 4,5 & O. Stein\\
2550136 & Globale Optimierung II (S.~\pageref{cour_7881.dp_997}) & 2/1 & W & 4,5 & O. Stein\\
2550486 & Standortplanung und strategisches Supply Chain Management (S.~\pageref{cour_7813.dp_997}) & 2/1 & S & 4,5 & S. Nickel\\
2550679 & Stochastische Entscheidungsmodelle I (S.~\pageref{cour_5703.dp_997}) & 2/1/2 & W & 5 & K. Waldmann\\
\end{courselist}

\begin{styleenv}
\begin{assessment}
Die Modulprüfung erfolgt in Form von schriftlichen Teilprüfungen(nach § 4(2), 1 SPO) über die gewählten Lehrveranstaltungen des Moduls, mit denen in Summe die Mindestanforderungen an Leistungspunkten erfüllt ist. Die Erfolgskontrolle wird bei jeder Lehrveranstaltung beschrieben.


\end{assessment}

\begin{conditions}Nur prüfbar in Kombination mit dem Modul \emph{Grundlagen des OR}.

 \end{conditions}


\end{styleenv}

\begin{learningoutcomes}
Der/die Studierende

 \begin{itemize}\item benennt und beschreibt die Grundbegriffe von Optimierungsverfahren, insbesondere aus der nichtlinearen und aus der globalen Optimierung,  \item kennt die für eine quantitative Analyse unverzichtbaren Methoden und Modelle,  \item modelliert und klassifiziert Optimierungsprobleme und wählt geeignete Lösungsverfahren aus, um auch anspruchsvolle Optimierungsprobleme selbständig und gegebenenfalls mit Computerhilfe zu lösen,  \item validiert, illustriert und interpretiert erhaltene Lösungen.  \end{itemize}
\end{learningoutcomes}

\begin{content}
Der Schwerpunkt des Moduls liegt auf der Vermittlung sowohl theoretischer Grundlagen als auch von Lösungsverfahren für Optimierungsprobleme mit kontinuierlichen Entscheidungsvariablen. Die Vorlesungen zur nichtlinearen Optimierung behandeln lokale Lösungskonzepte, die Vorlesungen zur globalen Optimierung die Möglichkeiten zur globalen Lösung.


\end{content}

\begin{remarks}Das für drei Studienjahre im voraus geplante Lehrangebot kann im Internet unter http://www.ior.kit.edu nachgelesen werden.

 

Bei den Vorlesungen von Professor Stein ist jeweils eine Prüfungsvorleistung (30\% der Übungspunkte) zu erbringen. Die jeweiligen Lehrveranstaltungsbeschreibungen enthalten \newline
weitere Einzelheiten.

\end{remarks}

\end{module}

