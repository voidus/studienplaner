% Lehrveranstaltungsbeschreibung
% Informationsgrad : extern
% Sprache: de
\begin{course}

\setdoclanguagegerman
\coursedegreeprogramme{Informatik}
\coursemodulename{Energiewirtschaft (S.~\pageref{mod_2871.dp_997})[IN3WWBWL12]}
\courseID{2581012}
\coursename{Erneuerbare Energien - Technologien und Potenziale}
\coursecoordination{R. McKenna}

\documentdate{2011-12-23 09:56:54.574639}

\courselevel{3}
\coursecredits{3,5}
\courseterm{Wintersemester}
\coursehours{2/0}
\courseinstructionlanguage{de}

\coursehead

% For index (key word@display). Key word is used for sorting - no Umlauts please.
\index{Erneuerbare Energien - Technologien und Potenziale@Erneuerbare Energien - Technologien und Potenziale}

% For later referencing
\label{cour_7417.dp_997}


\begin{styleenv}
\begin{assessment}
Die Erfolgskontrolle erfolgt in Form einer schriftlichen Prüfung (nach §4 (2), 1 SPO).


\end{assessment}

\begin{conditions}Keine.\end{conditions}


\end{styleenv}

\begin{learningoutcomes}
Der/die Studierende

 \begin{itemize}\item versteht die Motivation und globale Zusammenhänge für Erneuerbare Energieresourcen,  \item besitzt detaillierte Kenntnisse zu den verschiedenen Erneuerbaren Ressourcen und Techniken, sowie ihren Potenzialen,  \item versteht die systemische Zusammenhänge und Wechselwirkung die aus eines erhöhten Anteils erneuerbarer Stromerzeugung resultieren,  \item versteht die wesentliche wirtschaftliche Aspekte der Erneuerbaren Energien, inklusive Stromgestehungskosten, politische Förderung, und Vermarktung von Erneuerbaren Strom,  \item ist in der Lage, diese Technologien zu charakterisieren und ggf. zu berechnen.  \end{itemize}
\end{learningoutcomes}

\begin{content}
1. Allgemeine Einleitung: Motivation, Globaler Stand

 

2. Grundlagen der Erneuerbaren Energien: Energiebilanz der Erde, Potenzialbegriffe

 

3. Wasser

 

4. Wind

 

5. Sonne

 

6. Biomasse

 

7. Erdwärme

 

8. Sonstige erneuerbare Energien

 

9. Förderung erneuerbarer Energien

 

10. Wechselwirkungen im Systemkontext

 

11. Ausflug zum Energieberg in Mühlburg


\end{content}

\begin{media}Medien werden über die Lernplattform ILIAS bereitgestellt.

\end{media}

\begin{literature}\textbf{Weiterführende Literatur:}

 \begin{itemize}\item Kaltschmitt, M., 2006, Erneuerbare Energien : Systemtechnik, Wirtschaftlichkeit, Umweltaspekte, aktualisierte, korrigierte und ergänzte Auflage Berlin, Heidelberg : Springer-Verlag Berlin Heidelberg.  \item Kaltschmitt, M., Streicher, W., Wiese, A. (eds.), 2007, Renewable Energy: Technology, Economics and Environment, Springer, Heidelberg.  \item Quaschning, V., 2010, Erneuerbare Energien und Klimaschutz : Hintergründe - Techniken - Anlagenplanung – Wirtschaftlichkeit München : Hanser, Ill.2., aktualis. Aufl.  \item Harvey, D., 2010, Energy and the New Reality 2: Carbon-Free Energy Supply, Eathscan, London/Washington.  \item Boyle, G. (ed.), 2004, Renewable Energy: Power for a Sustainable Future, 2\textsuperscript{nd} Edition, Open University Press, Oxford.  \end{itemize}\end{literature}



\end{course}