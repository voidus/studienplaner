% Lehrveranstaltungsbeschreibung
% Informationsgrad : extern
% Sprache: de
\begin{course}

\setdoclanguagegerman
\coursedegreeprogramme{Informatik}
\coursemodulename{CRM und Servicemanagement (S.~\pageref{mod_1601.dp_997})[IN3WWBWL1]}
\courseID{2540520}
\coursename{Operatives CRM}
\coursecoordination{A. Geyer-Schulz}

\documentdate{2011-12-30 10:24:50.889300}

\courselevel{3}
\coursecredits{4,5}
\courseterm{Wintersemester}
\coursehours{2/1}
\courseinstructionlanguage{de}

\coursehead

% For index (key word@display). Key word is used for sorting - no Umlauts please.
\index{Operatives CRM@Operatives CRM}

% For later referencing
\label{cour_4899.dp_997}


\begin{styleenv}
\begin{assessment}
Die Erfolgskontrolle wird in der Modulbeschreibung erläutert.


\end{assessment}

\begin{conditions}Keine.\end{conditions}

\begin{recommendations}Der Besuch der Vorlesungen \emph{Customer Relationship Management }[2540508] und \emph{Analytisches CRM} [2540522] wird als sinnvoll erachtet.

\end{recommendations}
\end{styleenv}

\begin{learningoutcomes}
Der/die Studierende

 \begin{itemize}\item versteht die Theorie zu Methoden der Prozess- und Datenanalyse und wendet diese zur Gestaltung und Implementierung operativer CRM-Prozesse im komplexen Kontext eines Unternehmens an,  \item berücksichtigt die dabei entstehenden Privacy-Probleme,  \item evaluieren bestehende operative CRM-Prozesse in Unternehmen kritisch und geben Empfehlungen zu deren Verbesserung. Dies bedingt die Kenntnise von operativen CRM-Beispielsprozessen und die Fähigkeit, diese für einen solchen Einsatz entsprechend zu transformieren, um neue Lösungen zu entwickeln.  \item nutzen zur Lösung von Fallstudien zur Gestaltung operativer CRM-Prozesse über die Vorlesung hinausgehend fach- und branchenspezifische Literatur, kommunizieren kompetenz mit Fachleuten und fassen ihre Empfehlungen und Entwürfe als präzise und kohärente Berichte zusammen.  \end{itemize}
\end{learningoutcomes}

\begin{content}
Die Vorlesung Operatives CRM ist der Gestaltung und Umsetzung der operativen CRM-Prozesse in Unternehmen bzw. Organisationen gewidmet. Dazu wird zunächst die CRM-Prozesslandschaft in einem Unternehmen vorgestellt und ein Vorgehensmodell zur Prozessinnovation im CRM vorgestellt. Prozessmodellierung auf der Basis von höheren Petrinetzen und Datenmodellierung sind die theoretischen Grundlagen für die formale Spezifikation operativer CRM-Prozesse. Die Verwendung von UML-Diagrammen und ihre Beziehung zu Petrinetzen und Datenbanken wird vorgestellt. UML-Diagramme werden anschließend zur Modellierung von operativen CRM-Prozessen herangezogen. Die zur Bewertung von operativen CRM-Prozessen notwendigen Key Performance Indikatoren (Kennzahlen) und deren Wechselwirkung mit den Unternehmenszielen wird angeschnitten.

 

In der Vorlesung werden operative CRM-Prozesse wie z.B. Marketingmanagement, Kampagnenmanagement, Eventmanagement, Call Center Management, Sales Force Management, Permission Marketing, Direct Marketing, eBusiness, B2B, Sortimentsmanagement, Field Services ..., und industriespezifische Datenmodelle für solche Prozesse vorgestellt und diskutiert. Privacy Probleme werden angeschnitten.

 

Abschließend wird ein kurzer Überblick über den Markt von CRM-Softwarepaketen gegeben.


\end{content}

\begin{media}Folien

\end{media}

\begin{literature} 

Jill Dyché. The CRM Handbook: A Business Guide to Customer Relationship Management. Addison-Wesley, Boston, 2 edition, 2002.

  

Ronald S. Swift. Accelerating Customer Relationships: Using CRM and RelationshipTechnologies. Prentice Hall, Upper Saddle River, 2001.

 

\textbf{Weiterführende Literatur:}

  

Alex Berson, Kurt Thearling, and Stephen J. Smith. Building Data Mining Applications for CRM. Mc Graw-Hill, New York, 2000.

  

Stanley A. Brown. Customer Relationship Management: A Strategic Imperative in theWorld of E-Business. John Wiley, Toronto, 2000.

  

Dimitris N. Chorafas. Integrating ERP, CRM, Supply Chain Management, and SmartMaterials. Auerbach Publications, Boca Raton, Florida, 2001.

  

Keith Dawson. Call Center Handbook: The Complete Guide to Starting, Running, and Improving Your Call Center. CMP Books, Gilroy, CA, 4 edition, 2001.

  

Andreas Eggert and Georg Fassot. eCRM – Electronic Customer Relationship Management: Anbieter von CRM-Software im Vergleich. Schäffer-Poeschel, Stuttgart, 2001.

  

Seth Godin. Permission Marketing. Kunden wollen wählen können. FinanzBuch Verlag, München, 1999.

  

Paul Greenberg. CRM at the Speed of Light: Capturing and Keeping Customers in Internet Real Time. Osborne/McGraw-Hill, 3rd ed. edition, Aug 2004.

  

Philip Kotler. Marketing Management: Millennium Edition. Prentice Hall, Upper Saddle River, 10 edition, 2000.

  

Don Peppers and Martha Rogers. The One To One Future. Currency Doubleday, New York, 1997.

  

Duane E. Sharp. Customer Relationship Management Systems Handbook. Auerbach, 2002.

  

Len Silverston. The Data Model Resource Book: A Library of Universal Data Models for All Entreprises, volume 1. John Wiley \& Sons, 2001.

  

Toby J. Teorey. Database Modeling and Design. Morgan Kaufmann, San Francisco, 3 edition, 1999.

  

Chris Todman. Designing a Data Warehouse : Supporting Customer Relationship Management. Prentice Hall, Upper Saddle River, 1 edition, 2001.

\end{literature}



\end{course}