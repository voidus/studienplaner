% Modulbeschreibung 
% Informationsgrad : extern
% Sprache: de
\begin{module}

\setdoclanguagegerman
\moduledegreeprogramme{Informatik (B.Sc.)}
\modulesubject{}
\moduleID{IN3INMMK}
\modulename{Multimediakommunikation}
\modulecoordination{M. Zitterbart}

\documentdate{2010-06-04 12:00:19.672947}

\modulecredits{4}
\moduleduration{1}
\modulecycle{Jedes 2. Semester, Wintersemester}



\modulehead

% For index (key word@display). Key word is used for sorting - no Umlauts please.
\index{Multimediakommunikation@Multimediakommunikation (M)}

% For later referencing
\label{mod_2589.dp_997}

\begin{courselist}
24132 & Multimediakommunikation (S.~\pageref{cour_5363.dp_997}) & 2/0 & W & 4 & R. Bless\\
\end{courselist}

\begin{styleenv}
\begin{assessment}
Die Erfolgskontrolle erfolgt in Form einer mündlichen Prüfung im Umfang von i.d.R. 20 Minuten nach § 4 Abs. 2 Nr. 2 SPO.

 

Die Modulnote ist die Note der mündlichen Prüfung.


\end{assessment}

\begin{conditions}Inhalte der Vorlesung \emph{Einführung in Rechnernetze} [24519] (Teil des Pflichtmoduls \emph{Kommunikation und Datenhaltung} [IN3INKD]) und des Stammmoduls \emph{Telematik }[IN3INTM] werden vorausgesetzt.

 

Das Stammmodul Telematik muss belegt und geprüft werden.

\end{conditions}


\end{styleenv}

\begin{learningoutcomes}
Ziel der Vorlesung ist es, aktuelle Techniken und Protokolle für multimediale Kommunikation in – überwiegend Internet-basierten – Netzen zu vermitteln. Insbesondere vor dem Hintergrund der zunehmenden Sprachkommunikation über das Internet (Voice over IP) werden die Schlüsseltechniken und -protokolle wie RTP und SIP ausführlich erläutert, so dass deren Möglichkeiten und ihre Funktionsweise verstanden werden.


\end{learningoutcomes}

\begin{content}
Diese Vorlesung beschreibt Techniken und Protokolle, um beispielsweise Audio- und Videodaten im Internet zu übertragen. Behandelte Themen sind unter anderem: Audio- und Videokonferenzen, Audio/Video-Transportprotokolle, Voice over IP (VoIP), SIP zur Signalisierung und Aufbau sowie Steuerung von Multimedia-Sitzungen, RTP zum Transport von Multimediadaten über das Internet, RTSP zur Steuerung von A/V-Strömen, Enum zur Rufnummernabbildung, A/V-Streaming, Middleboxes und Caches, DVB und Video on Demand.


\end{content}



\end{module}

