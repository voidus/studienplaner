% Lehrveranstaltungsbeschreibung
% Informationsgrad : extern
% Sprache: de
\begin{course}

\setdoclanguagegerman
\coursedegreeprogramme{Informatik}
\coursemodulename{Proseminar (S.~\pageref{mod_2385.dp_997})[IN2INPROSEM]}
\courseID{prosemis}
\coursename{Proseminar Informationssysteme}
\coursecoordination{K. Böhm}

\documentdate{2011-08-29 14:31:13.217261}

\courselevel{3}
\coursecredits{3}
\courseterm{Sommersemester}
\coursehours{2}
\courseinstructionlanguage{de}

\coursehead

% For index (key word@display). Key word is used for sorting - no Umlauts please.
\index{Proseminar Informationssysteme@Proseminar Informationssysteme}

% For later referencing
\label{cour_6111.dp_997}


\begin{styleenv}
\begin{assessment}
Die Erfolgskontrolle erfolgt durch Ausarbeiten einer schriftlichen Seminararbeit sowie durch Präsentation derselbigen als benotete Erfolgskontrolle anderer Art nach § 4 Abs. 2 Nr. 3 SPO. \newline
Die Seminarnote entspricht dabei der schriftlichen Leistung, kann aber durch die Präsentationsleistung um bis zu zwei Notenstufen gesenkt bzw. angehoben werden. Die Notenvergabe basiert auf einem Bewertungssystem, in das sich die Teilnehmer selbst einbringen. Im Falle eines Abbruchs der Seminararbeit nach Ausgabe des des Themas wird das Seminar mit der Note 5,0 bewertet..


\end{assessment}

\begin{conditions}Keine.\end{conditions}

\begin{recommendations}Zum Thema des Seminars passende Vorlesungen am Lehrstuhl für Systeme der Informationsverwaltung werden empfohlen.

\end{recommendations}
\end{styleenv}

\begin{learningoutcomes}
Selbständige Bearbeitung und Präsentation eines Themas aus dem Bereich Informationssysteme nach wissenschaftlichen Maßstäben.


\end{learningoutcomes}

\begin{content}
Am Lehrstuhl für Systeme der Informationsverwaltung wird jedes Sommersemester ein Proseminar zu einem ausgewählten Thema der Informationssysteme angeboten (jedes Proseminar am “Lehrstuhl für Systeme der Informationsverwaltung” zählt als “Proseminar Informationssysteme”). Beispielsweise kann das Seminarthema aus folgenden Bereichen sein: Peer-to-Peer Netzwerke, Datenbanken, Data Mining, Sensornetze, Workflow Management. Details werden jedes Semester bekannt gegeben (Aushänge und Homepage des Lehrstuhls für Systeme der Informationsverwaltung).


\end{content}

\begin{media}Folien.

\end{media}

\begin{literature}Wird für jedes Seminar bekannt gegeben.

 

\textbf{Weiterführende Literatur:}

 

Literatur aus Vorlesungen zu dem Seminarthema.

\end{literature}



\end{course}