% Lehrveranstaltungsbeschreibung
% Informationsgrad : extern
% Sprache: de
\begin{course}

\setdoclanguagegerman
\coursedegreeprogramme{Informatik}
\coursemodulename{Supply Chain Management (S.~\pageref{mod_2721.dp_997})[IN3WWBWL14]}
\courseID{2118090}
\coursename{Quantitatives Risikomanagement von Logistiksystemen}
\coursecoordination{ A. Cardeneo}

\documentdate{2011-09-22 10:52:52.805313}

\courselevel{4}
\coursecredits{6}
\courseterm{Wintersemester}
\coursehours{3/1}
\courseinstructionlanguage{de}

\coursehead

% For index (key word@display). Key word is used for sorting - no Umlauts please.
\index{Quantitatives Risikomanagement von Logistiksystemen@Quantitatives Risikomanagement von Logistiksystemen}

% For later referencing
\label{cour_7063.dp_997}


\begin{styleenv}
\begin{assessment}
Die Erfolgskontrolle erfolgt in Form einer mündlichen Prüfung (nach §4(2), 2 SPO). Bei großer Teilnehmerzahl wird die Prüfung (nach §4(2), 1 SPO) schriftlich durchgeführt.


\end{assessment}

\begin{conditions}Vorkenntnisse in Logistik und idealerweise Operations Research sind empfehlenswert, u.a. Kenntnisse der linearen und gemischt-ganzzahligen Optimierung, einfacher Graphentheorie und Grundkenntnisse der Statistik.

\end{conditions}


\end{styleenv}

\begin{learningoutcomes}
Der/die Studierende

 \begin{itemize}\item identifiziert, analysiert und bewertet Risiken von Logistiksystemen  \item plant Standort und Transporte unter Unsicherheit  \item kennt risikorelevante Elemente und beherrscht entsprechende Methoden im Umgang mit Planungsprozessen (Beschaffung, Nachfrage, Infrastruktur, Kontinuitätsmanagement)  \end{itemize}
\end{learningoutcomes}

\begin{content}
Die Planung und der Betrieb von Logistiksystemen sind in großem Maße mit Unsicherheit verbunden: Sei es die unbekannte Nachfrage, schwankende Transportzeiten, unerwartete Verzögerungen, ungleichmäßige Produktionsausbeute oder volatile Wechselkurse: Mengen, Zeitpunkte, Qualitäten und Preise sind unsichere Größen. Es ist daher notwendig sich mit den aus dieser Unsicherheit ergebenden Folgen zu befassen, um insbesondere negative Auswirkungen zu beherrschen. Dies ist Aufgabe des Risikomanagements der Logistik und Gegenstand dieser Vorlesung.\newline
\newline
 In dieser Vorlesung befassen wir uns mit größtenteils mathematischen Modellen und Methoden, mit denen die unterschiedlichsten Risikoarten beherrscht werden können.

 

Themen umfassen:

 \begin{itemize}\item Risikoidentifikation, -analyse und -bewertung  \item Grundtechniken: Prognose, robuste Optimierung, Szenarioplanung und Simulation  \item Entscheidungsmodelle für Risikomanagementstrategien: Schadensbegrenzung oder Vorbeugung  \item Standortplanung unter Unsicherheit: Robuste Standortplanung  \item Transportplanung unter Unsicherheit: Robuste Transportnetzwerke  \item Produktion: Robuste Produktionsplanung  \item Beschaffung: Multi-Sourcing-Strategien, Kapazitätsoptionen, Umgang mit Preisrisiken  \item Nachfrage: Gestaltung der Nachfrage durch Revenue Management  \item Infrastrukturschutz: Schutz von Standorten gegen äußere Einwirkungen  \item Kontinuitätsmanagement: Schutz der Unternehmens-IT  \end{itemize}
\end{content}



\begin{literature}Wird in der Vorlesung bekannt gegeben.

\end{literature}



\end{course}