% Modulbeschreibung 
% Informationsgrad : extern
% Sprache: de
\begin{module}

\setdoclanguagegerman
\moduledegreeprogramme{Informatik (B.Sc.)}
\modulesubject{EF Betriebswirtschaftslehre}
\moduleID{IN3WWBWL16}
\modulename{Bauökologie}
\modulecoordination{T. Lützkendorf}

\documentdate{2011-06-27 15:17:05.719382}

\modulecredits{9}
\moduleduration{2}
\modulecycle{Jedes 2. Semester, Wintersemester}



\modulehead

% For index (key word@display). Key word is used for sorting - no Umlauts please.
\index{Bauoekologie@Bauökologie (M)}

% For later referencing
\label{mod_1639.dp_997}

\begin{courselist}
26404w & Bauökologie I (S.~\pageref{cour_6851.dp_997}) & 2/1 & W & 4,5 & T. Lützkendorf\\
2585404/2586404 & Bauökologie II (S.~\pageref{cour_6881.dp_997}) & 2/1 & S & 4,5 & T. Lützkendorf\\
\end{courselist}

\begin{styleenv}
\begin{assessment}
Die Modulprüfung erfolgt in Form von Teilprüfungen (nach §4(2), 1 o. 2 SPO) zu den einzelnen Lehrveranstaltungen des Moduls, mit denen in Summe die Mindestanforderung an Leistungspunkten erfüllt wird. Die Erfolgskontrolle wird bei jeder Lehrveranstaltung dieses Moduls beschrieben.

 

Die Gesamtnote des Moduls wird aus den mit LP gewichtete Noten der Teilprüfungen gebildet und nach der ersten Nachkommastelle abgeschnitten.

 

Innerhalb des Moduls kann fakultativ eine Seminararbeit aus dem Bereich “Bauökologie” angefertigt werden, die mit einer Gewichtung von 20\% in die Modulnote eingerechnet werden kann (nach §4(2), 3 SPO).


\end{assessment}

\begin{conditions}Nur prüfbar in Kombination mit dem Modul \emph{Grundlagen der BWL}.

 \end{conditions}

\begin{recommendations}Es wird eine Kombination mit dem Modul \emph{Real Estate Management} [IN3WWBWL17] empfohlen.\newline
Weiterhin empfehlenswert ist die Kombination mit Lehrveranstaltungen aus den Bereichen

 \begin{itemize}\item Industrielle Produktion (Stoff- und Energieflüsse in der Ökonomie, Stoff- und Energiepolitik, Emissionen in die Umwelt)  \item Bauingenieurwesen und Architektur (Bauphysik, Baukonstruktion)  \end{itemize}\end{recommendations}
\end{styleenv}

\begin{learningoutcomes}
Der/die Studierende

 \begin{itemize}\item kennt die Grundlagen des nachhaltigen Planens, Bauens und Betreibens von Gebäuden mit einem Schwerpunkt im Themenbereich Bauökologie  \item besitzt Kenntnisse über die bauökologischen Bewertungsmethoden sowie Hilfsmittel zur Planung und Bewertung von Gebäuden  \item ist in der Lage, diese Kenntnisse zur Beurteilung der ökologischen Vorteilhaftigkeit sowie des Beitrages zu einer nachhaltigen Entwicklung von Immobilien einzusetzen.  \end{itemize}
\end{learningoutcomes}

\begin{content}
Nachhaltiges Planen, Bauen und Betreiben von Immobilien sowie “green buildings” und “sustainable buildings” sind z.Z. die beherrschenden Themen in der Immobilienbranche. Diese Themen sind nicht nur für Planer sondern insbesondere auch für Akteure von Interesse, die sich künftig mit der Entwicklung, Finanzierung und Versicherung von Immobilien beschäftigen oder mit der Steuerung von Gebäudebeständen und Immobilienfonds betraut sind.\newline
Das Lehrangebot vermittelt einerseits die Grundlagen des energiesparenden, ressourcenschonenden und gesundheitsgerechten Planens, Bauens und Betreibens. Andererseits werden bewertungsmethodische Grundlagen für die Analyse und Kommunikation der ökologischen Vorteilhaftigkeit von Lösungen erörtert. Mit den Grundlagen für die Zertifizierung der Nachhaltigkeit von Gebäuden werden Kenntnisse erworben, die momentan stark nachgefragt werden.

 

Zur Veranschaulichung der Lehrinhalte des Moduls werden Videos und Simulationstools eingesetzt.


\end{content}



\end{module}

