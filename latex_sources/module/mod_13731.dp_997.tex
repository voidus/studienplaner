% Modulbeschreibung 
% Informationsgrad : extern
% Sprache: de
\begin{module}

\setdoclanguagegerman
\moduledegreeprogramme{Informatik (B.Sc.)}
\modulesubject{}
\moduleID{IN3INMBV}
\modulename{Methoden der Biosignalverarbeitung}
\modulecoordination{T. Schultz}

\documentdate{2012-01-20 12:31:31.251865}

\modulecredits{3}
\moduleduration{1}
\modulecycle{Jedes 2. Semester, Sommersemester}



\modulehead

% For index (key word@display). Key word is used for sorting - no Umlauts please.
\index{Methoden der Biosignalverarbeitung@Methoden der Biosignalverarbeitung (M)}

% For later referencing
\label{mod_13731.dp_997}

\begin{courselist}
24641 & Methoden der Biosignalverarbeitung (S.~\pageref{cour_13729.dp_997}) & 2 & S & 3 & M. Wand, T. Schultz\\
\end{courselist}

\begin{styleenv}
\begin{assessment}
Es wird 6 Wochen im Voraus angekündigt, ob die Erfolgskontrolle in Form einer schriftlichen Prüfung (Klausur) im Umfang von i.d.R. 1h nach § 4 Abs. 2 Nr. 1 SPO oder in Form einer mündlichen Prüfung im Umfang von i.d.R. 15 min. nach § 4 Abs. 2 Nr. 2 SPO stattfinden wird.

 

Die Modulnote entspricht dieser Note.

 

Terminvereinbarung bitte per E-Mail an: helga.scherer@kit.edu. Es wird empfohlen, sich frühzeitig um einen Prüfungstermin zu kümmern.


\end{assessment}

\begin{conditions}Keine.\end{conditions}

\begin{recommendations}Die Inhalte der Vorlesung „Biosignale und Benutzerschnittstellen“ \emph{oder} „Multilinguale Mensch-Maschine-Kommunikation“ oder einer gleichwertigen Vorlesung werden vorausgesetzt.

\end{recommendations}
\end{styleenv}

\begin{learningoutcomes}
Diese Vorlesung vermittelt den Studierenden einen vertieften Einblick in die Algorithmik der Biosignalverarbeitung. Fokus ist insbesondere der Umgang mit aus mehreren Komponenten zusammengesetzten Signalen sowie die Fusion von Entscheidungen in multimodalen Systemen.

 

Der Besuch der Veranstaltung soll die Studierenden in die Lage versetzen, die behandelten Verfahren und Methoden selbständig auf Problemstellungen der modernen Biosignalverarbeitung anwenden zu können.


\end{learningoutcomes}

\begin{content}
Diese Vorlesung behandelt algorithmische Methoden der modernen Biosignalverarbeitung. Vertieft wird unter anderem die Quellenseparierung von Biosignalen, also die Analyse von Messreihen, die sich aus mehreren überlagerten Komponenten zusammensetzen. Ein weiteres Thema ist die Fusion von Informationen, die z.B. von verschiedenen Bestandteilen eines multimodalen Klassifikationssystems stammen können.

 

Die theoretischen Grundlagen werden durch Anwendungsbeispiele aus Literatur und eigener Forschung veranschaulicht.

 

Hinweis: Die Inhalte der Vorlesung „Biosignale und Benutzerschnittstellen“ \emph{oder} „Multilinguale Mensch-Maschine-Kommunikation“ oder einer gleichwertigen Vorlesung werden vorausgesetzt


\end{content}



\end{module}

