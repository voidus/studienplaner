% Lehrveranstaltungsbeschreibung
% Informationsgrad : extern
% Sprache: de
\begin{course}

\setdoclanguagegerman
\coursedegreeprogramme{Informatik}
\coursemodulename{Wirtschaftsprivatrecht (S.~\pageref{mod_2655.dp_997})[IN3INJUR2]}
\courseID{24506/24017}
\coursename{Privatrechtliche Übung}
\coursecoordination{P. Sester, T. Dreier}

\documentdate{2005-12-21 12:17:56}

\courselevel{1}
\coursecredits{3}
\courseterm{Winter-/Sommersemester}
\coursehours{2/0}
\courseinstructionlanguage{de}

\coursehead

% For index (key word@display). Key word is used for sorting - no Umlauts please.
\index{Privatrechtliche UEbung@Privatrechtliche Übung}

% For later referencing
\label{cour_4397.dp_997}


\begin{styleenv}
\begin{assessment}
Die Erfolgskontrolle erfolgt in Form schriftlicher Prüfungen (Klausuren) im Umfang von je 90 min. nach § 4, Abs. 2 Nr. 3 SPO. Angeboten werden insgesamt 5 Klausuren, von denen die Studenten mindestens 2 Klausuren bestehen müssen. Sind mehr als 2 Klausuren bestanden, so werden die beiden Klausuren mit den besten Noten für den benoteten Schein gewertet.


\end{assessment}

\begin{conditions}Der Besuch der Vorlesung \emph{BGB für Anfänger} [24012] oder einer vergleichbaren Einführung in das Zivilrecht ist Voraussetzung; der Besuch der Vorlesungen \emph{BGB für Fortgeschrittene} [24504] sowie \emph{Handels- und Gesellschaftsrecht} [24011] wird sehr empfohlen.

\end{conditions}


\end{styleenv}

\begin{learningoutcomes}
Ziel der Übung ist die vertiefende Einübung der Fallösungstechnik (Anspruchsaufbau, Gutachtenstil). Zugleich wird das rechtliche Grundlagenwissen, das die Studenten im Rahmen der Vorlesungen “BGB für Fortgeschrittene” und “Handels- und Gesellschaftsrecht” erworben haben, wiederholt und vertieft und im Rahmen der Klausuren abgeprüft. Auf diese Weise sollen die Studenten die Befähigung erwerben, juristische Problemfälle der Praxis mit juristischen Mitteln methodisch sauber zu lösen.


\end{learningoutcomes}

\begin{content}
In 5 Übungsterminen wird der Stoff der Veranstaltungen „BGB für Fortgeschrittene” und „Handels- und Gesellschaftsrecht” wiederholt und die juristische Fallösungsmethode vertiefend eingeübt. Weiterhin werden im Rahmen der Übung 5 Klausuren geschrieben, die sich über den gesamten bisher im Privatrecht erlernetn Stoff erstrecken. Weitere Termine sind für die Klaussurrückgabe und die Besprechungen der einzelnen Klausuren reserviert.


\end{content}

\begin{media}Folien

\end{media}

\begin{literature}Wird in der Vorlesung bekannt gegeben

\end{literature}



\end{course}