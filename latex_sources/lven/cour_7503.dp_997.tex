% Lehrveranstaltungsbeschreibung
% Informationsgrad : extern
% Sprache: de
\begin{course}

\setdoclanguagegerman
\coursedegreeprogramme{Informatik}
\coursemodulename{Virtual Engineering I (S.~\pageref{mod_4267.dp_997})[IN3MACHVE1]}
\courseID{2121352}
\coursename{Virtual Engineering I}
\coursecoordination{J. Ovtcharova}

\documentdate{2012-01-17 12:16:19.042179}

\courselevel{4}
\coursecredits{6}
\courseterm{Wintersemester}
\coursehours{2/3}
\courseinstructionlanguage{de}

\coursehead

% For index (key word@display). Key word is used for sorting - no Umlauts please.
\index{Virtual Engineering I@Virtual Engineering I}

% For later referencing
\label{cour_7503.dp_997}


\begin{styleenv}
\begin{assessment}
Die Erfolgskontrolle erfolgt in Form einer mündlichen Prüfung um Umfang von 40 min über die Inhalte der Veranstaltung \emph{Virtual Engineering I} [21352] und \emph{Virtual Engineering II} [21378].

 

Die mündliche Prüfung kann auch nur über die Inhalte der Veranstaltung \emph{Virtual Engineering I} [21352] erfolgen. In diesem Fall verkürzt sich die Zeit der Prüfung auf 20 min.


\end{assessment}

\begin{conditions}Diese Lehrveranstaltung ist Pflicht im Modul \emph{Virtual Engineering} A [WW4INGMB19] und muss erfolgreich geprüft werden.

\end{conditions}


\end{styleenv}

\begin{learningoutcomes}
Die Studenten erhalten eine Einführung in Produkt Lifecycle Mangement (PLM) und verstehen den Einsatz von PLM im Rahmen von Virtual Engineering. Sie können CAD/PLM-Systeme in den einzelnen Phasen des Produktentstehungsprozesses einsetzen.

 

Desweiteren erwerben sie ein fundiertes Wissen über die Datenmodelle, die einzelnen Module und die Funktionen von CAD. Sie kennen die informationstechnischen Hintergründe von CAX-Systemen, deren Integrationsprobleme und mögliche Lösungsansätze.

 

Sie erlangen eine Übersicht über verschiedene Analysemethoden des CAE und deren Anwendungsmöglichkeiten, Randbedingungen und Grenzen. Sie kennen die unterschiedlichen Funktionalitäten von Preprozessor, Solver und Postprozessor in CAE-Systemen. Sie kennen die unterschiedlichen Integrationsarten von CAD/CAE-Systemen und die damit einhergehenden Vor- und Nachteile.

 

Sie wissen wie CAM-Module (oder Systeme) mit CAD-Systemen integriert werden und können Fertigungsprozesse im CAM-Modul definieren und simulieren. Sie verstehen die Philosophie von Virtual Engineering und Virtueller Fabrik. Sie sind in der Lage die Vorteile des Virtual Engineering gegenüber der herkömmlichen Herangehensweise zu identifizieren.


\end{learningoutcomes}

\begin{content}
Die Vorlesung vermittelt die informationstechnischen Aspekte und Zusammenhänge der Virtuellen Produktentstehung. Im Mittelpunkt stehen die verwendeten IT-Systeme zur Unterstützung der Prozesskette des Virtual Engineerings:

 \begin{itemize}\item Product Lifecycle Management ist ein Ansatz der Verwaltung von produktbezogenen Daten und Informationen über den gesamten Lebenszyklus hinweg, von der Konzeptphase bis zur Demontage und zum Recycling.  \item CAx-Systeme ermöglichen die Modellierung des digitalen Produktes im Hinblick auf die Planung, Konstruktion, Fertigung, Montage und Wartung.  \item Validierungssysteme ermöglichen die Überprüfung der Konstruktion im Hinblick auf Statik, Dynamik, Fertigung und Montage.  \end{itemize}

Ziel der Vorlesung ist es, die Verknüpfung von Konstruktions- und Validierungstätigkeiten unter Nutzung Virtueller Prototypen und VR/AR-Visualisierungstechniken in Verbindung mit PDM/PLM-Systemen zu verdeutlichen. Ergänzt wird dies durch Einführungen in die jeweiligen Systeme anhand praxisbezogener Aufgaben.


\end{content}

\begin{media}Vorlesungsskript

\end{media}





\end{course}