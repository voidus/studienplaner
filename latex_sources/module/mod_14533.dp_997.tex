% Modulbeschreibung 
% Informationsgrad : extern
% Sprache: de
\begin{module}

\setdoclanguagegerman
\moduledegreeprogramme{Informatik (B.Sc.)}
\modulesubject{}
\moduleID{IN3INWA}
\modulename{Web-Anwendungen}
\modulecoordination{S. Abeck}

\documentdate{2012-01-12 11:56:22.774295}

\modulecredits{4}
\moduleduration{1}
\modulecycle{Jedes 2. Semester, Wintersemester}



\modulehead

% For index (key word@display). Key word is used for sorting - no Umlauts please.
\index{Web-Anwendungen@Web-Anwendungen (M)}

% For later referencing
\label{mod_14533.dp_997}

\begin{courselist}
24153 & Web-Anwendungen und Serviceorientierte Architekturen (I) (S.~\pageref{cour_14537.dp_997}) & 2/0 & W & 4 & S. Abeck\\
\end{courselist}

\begin{styleenv}
\begin{assessment}
Die Erfolgskontrolle erfolgt in Form einer \textbf{mündlichen} Prüfung im Umfang von i.d.R. \textbf{20} Minuten nach § 4 Abs. 2 Nr. 2 SPO.

 

Die Zulassung zur Prüfung erfolgt nur bei nachgewiesener Mitarbeit an den in der Vorlesung gestellten praktischen Aufgaben.

 

Die Modulnote ist die Note der \textbf{mündlichen} Prüfung.


\end{assessment}

\begin{conditions}Keine.\end{conditions}


\end{styleenv}

\begin{learningoutcomes}
\begin{itemize}\item Die wichtigsten den Stand der Technik repräsentierenden Technologien und Standards zur Entwicklung von traditionellen Web-Anwendungen sind bekannt.  \item Die Architektur von traditionellen Web-Anwendungen ist verstanden.  \item Die Softwarearchitektur einer traditionellen Web-Anwendung kann modelliert werden.  \item Die wichtigsten Prinzipien traditioneller Softwareentwicklung und des entsprechenden Entwicklungsprozesses sind bekannt  \end{itemize}
\end{learningoutcomes}

\begin{content}
Das Internet als Verteilungsplattform und die darauf basierenden Webtechnologien spielen eine große Rolle bei der Entwicklung verteilter Anwendungssysteme. Traditionelle Webanwendungen nutzen standardisierte Technologien zur Kommunikation (u.a. HTTP, TCP) und zur Informationsbeschreibung (u.a. HTML, XML), die in der Vorlesung an einer durchgängigen Beispiel-Anwendung aufgezeigt werden.


\end{content}



\end{module}

