% Lehrveranstaltungsbeschreibung
% Informationsgrad : extern
% Sprache: de
\begin{course}

\setdoclanguagegerman
\coursedegreeprogramme{Informatik}
\coursemodulename{Product Lifecycle Management (S.~\pageref{mod_4273.dp_997})[IN3MACHPLM]}
\courseID{2121350}
\coursename{Product Lifecycle Management}
\coursecoordination{J. Ovtcharova}

\documentdate{2012-01-17 12:10:11.262114}

\courselevel{4}
\coursecredits{6}
\courseterm{Wintersemester}
\coursehours{3/1}
\courseinstructionlanguage{de}

\coursehead

% For index (key word@display). Key word is used for sorting - no Umlauts please.
\index{Product Lifecycle Management@Product Lifecycle Management}

% For later referencing
\label{cour_7495.dp_997}


\begin{styleenv}
\begin{assessment}
Die Erfolgskontrolle erfolgt in Form einer schriftlichen Prüfung im Umfang von 90 Minuten nach § 4 Abs. 2 Nr. 1 SPO.


\end{assessment}

\begin{conditions}Werden in der Modulbeschreibung erläutert.

\end{conditions}


\end{styleenv}

\begin{learningoutcomes}
Ziel der Vorlesung PLM ist es, den Management- und Organisationsansatz Product Lifecycle Management darzustellen. Der/die Studierenden:

 \begin{itemize}\item kennen das Managementkonzept PLM, seine Ziele und sind in der Lage, den wirtschaftlichen Nutzen des PLM-Konzeptes herauszustellen.  \item kennen Anbieter von PLM Systemlösungen und können die aktuelle Marktsituation darstellen.  \item Verstehen die Notwendigkeit für einen durchgängigen und abteilungsübergreifenden Unternehmensprozess - angefangen von der Portfolioplanung über die Konstruktion und Rückführung von Kundeninformationen aus der Nutzungsphase bis hin zur Wartung und zum Recycling der Produkte.  \item kennen Prozesse und Funktionen, die zur Unterstützung des gesamten Produktlebenszyklus benötigt werden.  \item erlangen Kenntnis über die wichtigsten betrieblichen Softwaresysteme (PDM, ERP, SCM, CRM) und die durchgängige Integration dieser Systeme.  \item erarbeiten Vorgehensweisen zur erfolgreichen Einführung des Managementkonzeptes PLM.  \end{itemize}
\end{learningoutcomes}

\begin{content}
Bei Product Lifecycle Management (PLM) handelt es sich um einen Ansatz zur ganzheitlichen und unternehmensübergreifenden Verwaltung und Steuerung aller produktbezogenen Prozesse und Daten über den gesamten Lebenszyklus entlang der erweiterten Logistikkette – von der Konstruktion und Produktion über den Vertrieb bis hin zur Demontage und dem Recycling.

 

Das Product Lifecycle Management ist ein umfassendes Konzept zur effektiven und effizienten Gestaltung von Informationen von der „Wiege“ bis zum „Grab“ eines Produktes. Basierend auf der Gesamtheit an Produktinformationen, die über die gesamte Wertschöpfungskette und verteilt über mehrere Partner anfallen, werden Prozesse, Methoden und Werkzeuge zur Verfügung gestellt, um die richtigen Informationen in der richtigen Zeit, Qualität und am richtigen Ort bereitzustellen.

 

Die Vorlesung umfasst:

 \begin{itemize}\item Eine durchgängige Beschreibung sämtlicher Geschäftsprozesse, die während des Produktlebenzyklus auftreten (Entwicklung, Produktion, Vertrieb, Demontage, …),  \item die Darstellung von Methoden des PLM zur Erfüllung der Geschäftsprozesse,  \item die Erläuterung der wichtigsten betrieblichen Informationssysteme zur Unterstützung des Lebenszyklus (PDM, ERP, SCM, CRM-Systeme) am Beispiel des Softwareherstellers SAP  \end{itemize}
\end{content}

\begin{media}Skript zur Veranstaltung, Passwort wird in Vorlesung bekanntgegeben.

\end{media}





\end{course}