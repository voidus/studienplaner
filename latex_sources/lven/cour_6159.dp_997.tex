% Lehrveranstaltungsbeschreibung
% Informationsgrad : extern
% Sprache: de
\begin{course}

\setdoclanguagegerman
\coursedegreeprogramme{Informatik}
\coursemodulename{Algorithmen für planare Graphen (S.~\pageref{mod_2937.dp_997})[IN3INALGPG]}
\courseID{24614}
\coursename{Algorithmen für planare Graphen}
\coursecoordination{D. Wagner}

\documentdate{2011-06-03 12:12:53.975890}

\courselevel{4}
\coursecredits{5}
\courseterm{Winter-/Sommersemester}
\coursehours{2/1}
\courseinstructionlanguage{de}

\coursehead

% For index (key word@display). Key word is used for sorting - no Umlauts please.
\index{Algorithmen fuer planare Graphen@Algorithmen für planare Graphen}

% For later referencing
\label{cour_6159.dp_997}


\begin{styleenv}
\begin{assessment}

\end{assessment}

\begin{conditions}Keine.\end{conditions}

\begin{recommendations}Kenntnisse zu Grundlagen der Graphentheorie und Algorithmentechnik sind hilfreich.

\end{recommendations}
\end{styleenv}

\begin{learningoutcomes}
Ziel der Vorlesung ist es, den Studierenden einen Überblick über das Gebiet der planaren Graphen zu geben, dabei wird insbesondere auf algorithmische Fragestellungen eingegangen. Die Studierenden erwerben ein systematisches Verständnis der zentralen Konzepte und Techniken zur Behandlung algorithmischer Fragestellungen auf planaren Graphen, das auf dem bestehenden Wissen der Studierenden in den Themenbereichen Graphentheorie und Algorithmik aufbaut. Die auftretenden Fragestellungen werden auf ihren algorithmischen Kern reduziert und anschließend, soweit aus komplexitätstheoretischer Sicht möglich, effizient gelöst. Studierende lernen die vorgestellten Methoden und Techniken autonom auf verwandte Fragestellungen anzuwenden und können mit dem erworbenen Wissen an aktuellen Forschungsthemen im Bereich planare Graphen arbeiten.


\end{learningoutcomes}

\begin{content}
Ein planarer Graph ist ein Graph, der in der Ebene gezeichnet werden, ohne dass die Kanten sich kreuzen. Planare Graphen haben viele schöne Eigenschaften, die benutzt werden können um für zahlreiche Probleme besonders einfache, schnelle und schöne Algorithmen zu entwerfen. Oft können sogar Probleme, die auf allgemeinen Graphen (NP-)schwer sind auf planaren Graphen sehr effizient gelöst werden. In dieser Vorlesung werden einige dieser Probleme und Algorithmen zu ihrer Lösung vorgestellt.


\end{content}

\begin{media}Tafel, Skript

\end{media}

\begin{literature}\textbf{Weiterführende Literatur:}

 

Takao Nishizeki and Norishige Chiba. Planar Graphs: Theory and Algorithms, volume 32 of Annals of Discrete Mathematics. North-Holland, 1988.

\end{literature}

\begin{remarks}\textcolor{red}{Die Lehrveranstaltung wird unregelmäßig angeboten.}

\end{remarks}

\end{course}