% Modulbeschreibung 
% Informationsgrad : extern
% Sprache: de
\begin{module}

\setdoclanguagegerman
\moduledegreeprogramme{Informatik (B.Sc.)}
\modulesubject{}
\moduleID{IN3INFS}
\modulename{Formale Systeme}
\modulecoordination{B. Beckert, P. Schmitt}

\documentdate{2010-09-13 10:20:16.395904}

\modulecredits{6}
\moduleduration{1}
\modulecycle{Jedes 2. Semester, Wintersemester}



\modulehead

% For index (key word@display). Key word is used for sorting - no Umlauts please.
\index{Formale Systeme@Formale Systeme (M)}

% For later referencing
\label{mod_2403.dp_997}

\begin{courselist}
24086 & Formale Systeme (S.~\pageref{cour_6057.dp_997}) & 3/2 & W & 6 & B. Beckert, P. Schmitt\\
\end{courselist}

\begin{styleenv}
\begin{assessment}
Die Erfolgskontrolle erfolgt in Form einer schriftlichen Prüfung (Klausur) im Umfang von 60 Minuten, (§ 4 Abs. 2 Nr. 1 der SPO). Es besteht die Möglichkeit, einen Übungsschein (Erfolgskontrolle anderer Art nach § 4 Abs. 2 Nr. 3 SPO) zu erwerben. Für diesen werden Bonuspunkte vergeben, die auf eine bestandene Klausur angerechnet werden.

 

Die Modulnote ist die Note der Klausur.


\end{assessment}

\begin{conditions}Der erfolgreiche Abschluss des Moduls \emph{Theoretische Grundlagen der Informatik }[IN2INTHEOG] ist Voraussetzung.

\end{conditions}


\end{styleenv}

\begin{learningoutcomes}
\begin{itemize}\item Der Studierende soll in die Grundbegriffe der formalen Modellierung und Verifikation von Informatiksystemen eingeführt werden.   \item Der Studierende soll die grundlegende Definitionen und ihre wechselseitigen Abhängigkeiten verstehen und anwenden lernen.  \item Der Studierende soll für kleine Beispiele eigenständige Lösungen von Spezifikationsaufgaben finden können, gegebenfalls mit Unterstützung entsprechender Softwarewerkzeuge.  \item Der Studierende soll für kleine Beispiele selbständig Verifikationsaufgaben lösen können, gegebenfalls mit Unterstützung entsprechender Softwarewerkzeuge.  \end{itemize}
\end{learningoutcomes}

\begin{content}
Diese Vorlesung soll die Studierenden einerseits in die Grundlagen der formalen Modellierung und Verifikation einführen und andererseits vermitteln, wie der Transfer von der Theorie zu einer praktisch einsetzbaren Methode betrieben werden kann.\newline
Es wird unterschieden zwischen der Behandlung statischer und dynamischer Aspekte von Informatiksystemen.

 \begin{itemize}\item \textbf{Statische Modellierung und Verifikation}\newline
Anknüpfend an Vorkenntnisse der Studierenden in der Aussagenlogik, werden Kalküle für die aussagenlogische Deduktion vorgestellt und Beweise für deren Korrektheit und Vollständigkeit besprochen. Es soll den Studierenden vermittelt werden, dass solche Kalküle zwar alle dasselbe Problem lösen, aber unterschiedliche Charakteristiken haben können. Beispiele solcher Kalküle können sein: der Resolutionskalkül. Tableaukalkül, Sequenzen- oder Hilbertkalkül. Weiterhin sollen Kalküle für Teilklassen der Aussagenlogik vorgestellt werden, z.B. für universelle Hornformeln.\newline
Die Brücke zwischen Theorie und Praxis soll geschlagen werden durch die Behandlung von Programmen zur Lösung aussagenlogischer Erfüllbarkeitsprobleme (SAT-solver).\newline
\newline
Aufbauend auf den aussagenlogischen Fall werden Syntax, Semantik der Prädikatenlogik eingeführt. Es werden zwei Kalküle behandelt, z.B. Resolutions-, Sequenzen-, Tableau- oder Hilbertkalkül. Wobei in einem Fall ein Beweis der Korrektheit und Vollständigkeit geführt wird.\newline
Die Brücke zwischen Theorie und Praxis soll geschlagen werden durch die Behandlung einer gängigen auf der Prädikatenlogik fußenden Spezifikationssprache, wie z.B. OCL, JML oder ähnliche. Zusätzlich kann auf automatische oder interaktive Beweise eingegangen werden.  \end{itemize}\begin{itemize}\item \textbf{Dynamische Modellierung und Verifikation}\newline
Als Einstieg in Logiken zur Formalisierung von Eigenschaften dynamischer Systeme werden aussagenlogische Modallogiken betrachtet in Syntax und Semantik (Kripke Strukturen) jedoch ohne Berücksichtigung der Beweistheorie.\newline
Aufbauend auf dem den Studenten vertrauten Konzept endlicher Automaten werden omega-Automaten zur Modellierung nicht terminierender Prozesse eingeführt, z.B. Büchi Automaten oder Müller Automaten. Zu den dabei behandelten Themen gehören insbesondere die Abschlusseigenschaften von Büchi Automaten.\newline
Als Spezialisierung der modalen Logiken wird eine temporale modale Logik in Syntax und Semantik eingeführt, z.B. LTL oder CTL.\newline
Es wird der Zusammenhang hergestellt zwischen Verhaltensbeschreibungen durch omega-Automaten und durch Formeln temporalen Logiken.\newline
Die Brücke zwischen Theorie und Praxis soll geschlagen werden durch die Behandlung eines Modellprüfungsverfahrens (model checking).  \end{itemize}
\end{content}

\begin{remarks}Das Modul \emph{Formale Systeme} ist ein Stammmodul.

\end{remarks}

\end{module}

