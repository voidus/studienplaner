% Lehrveranstaltungsbeschreibung
% Informationsgrad : extern
% Sprache: de
\begin{course}

\setdoclanguagegerman
\coursedegreeprogramme{Informatik}
\coursemodulename{Verfassungs- und Verwaltungsrecht (S.~\pageref{mod_2661.dp_997})[IN3INJUR3]}
\courseID{24520}
\coursename{Öffentliches Recht II - Öffentliches Wirtschaftsrecht}
\coursecoordination{I. Spiecker genannt Döhmann}

\documentdate{2012-01-19 11:29:39.975672}

\courselevel{2}
\coursecredits{3}
\courseterm{Sommersemester}
\coursehours{2/0}
\courseinstructionlanguage{de}

\coursehead

% For index (key word@display). Key word is used for sorting - no Umlauts please.
\index{OEffentliches Recht II - OEffentliches Wirtschaftsrecht@Öffentliches Recht II - Öffentliches Wirtschaftsrecht}

% For later referencing
\label{cour_4395.dp_997}


\begin{styleenv}
\begin{assessment}
Die Erfolgskontrolle erfolgt in Form einer schriftlichen Prüfung im Umfang von i.d.R. 60 Minuten nach § 4 Abs. 2 Nr. 1 SPO.


\end{assessment}

\begin{conditions}Keine.\end{conditions}

\begin{recommendations}Parallel zu den Veranstaltungen werden begleitende Tutorien angeboten, die insbesondere der Vertiefung der juristischen Arbeitsweise dienen. Ihr Besuch wird nachdrücklich empfohlen.\newline
Während des Semesters wird eine Probeklausur zu jeder Vorlesung mit ausführlicher Besprechung gestellt. Außerdem wird eine Vorbereitungsstunde auf die Klausuren in der vorlesungsfreien Zeit angeboten.\newline
Details dazu auf der Homepage des ZAR (www.kit.edu/zar).

\end{recommendations}
\end{styleenv}

\begin{learningoutcomes}
Das öffentliche Wirtschaftsrecht ist für die Steuerung der deutschen Wirtschaft von erheblicher Bedeutung. Wer die Funktionsweise hoheitlicher Eingriffe in die Marktmechanismen in einer durchnormierten Rechtsordnung verstehen will, braucht entsprechende Kenntnisse. Diese sollen in der Vorlesung vermittelt werden. Dabei soll vertieft das materielle Recht behandelt werden. Besondere formale Voraussetzungen, insb. Zuständigkeiten von Behörden, Aufsichtsmaßnahmen und die Rechtsschutzmöglichkeiten werden nur im Überblick behandelt (ergänzend zu der Veranstaltung \emph{Öffentliches Recht I}). Die Vorlesung verfolgt primär das Ziel, den Umgang mit den einschlägigen spezialgesetzlichen Rechtsnormen einzuüben. Sie baut auf der Vorlesung \emph{Öffentliches Recht I} auf.


\end{learningoutcomes}

\begin{content}
In einem ersten Schritt werden die wirtschaftsverfassungsrechtlichen Grundlagen (wie die Finanzverfassung und die Eigentums- und Berufsfreiheit) dargestellt. In diesem Rahmen wird auch das Zusammenspiel zwischen dem Grundgesetz und den Vorgaben des europäischen Gemeinschaftsrechts näher erläutert. Sodann werden die verwaltungsrechtlichen Steuerungsinstrumente analysiert. Als besondere Materien werden u.a. die Gewerbeordnung, das sonstige Gewerberecht (Handwerksordnung; Gaststättenrecht), die Grundzüge des Telekommunikationsgesetzes, die Förderregulierung und das Vergaberecht behandelt. Ein letzter Teil widmet sich der institutionellen Ausgestaltung der hoheitlichen Wirtschaftsregulierung.


\end{content}

\begin{media}Ausführliches Skript mit Fällen, Gliederungsübersichten, Unterlagen in den Veranstaltungen.

\end{media}

\begin{literature}Wird in der Vorlesung bekannt gegeben.

 

\textbf{Weiterführende Literatur:}

 

Wird in der Vorlesung bekannt gegeben.

\end{literature}

\begin{remarks}Zum WS 08/09 wurde der Vorlesungsturnus der Veranstaltung Öffentliches Recht I+II von SS/WS auf WS/SS umgestellt.\newline
 D.h.:

 \begin{enumerate}\item Im Wintersemester 08/09 fand die Vorlesung ÖRecht I statt.  \item Im Sommersemester 09 findet die Vorlesung ÖRecht II statt.  \end{enumerate}\end{remarks}

\end{course}