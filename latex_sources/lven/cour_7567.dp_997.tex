% Lehrveranstaltungsbeschreibung
% Informationsgrad : extern
% Sprache: de
\begin{course}

\setdoclanguagegerman
\coursedegreeprogramme{Informatik}
\coursemodulename{Lineare Algebra und Analytische Geometrie (S.~\pageref{mod_3027.dp_997})[IN1MATHLAAG]}
\courseID{01505}
\coursename{Lineare Algebra und Analytische Geometrie 2}
\coursecoordination{F. Herrlich, E. Leuzinger, C. Schmidt, W. Tuschmann}

\documentdate{2009-02-16 11:12:45}

\courselevel{1}
\coursecredits{9}
\courseterm{Sommersemester}
\coursehours{4/2/2}
\courseinstructionlanguage{de}

\coursehead

% For index (key word@display). Key word is used for sorting - no Umlauts please.
\index{Lineare Algebra und Analytische Geometrie 2@Lineare Algebra und Analytische Geometrie 2}

% For later referencing
\label{cour_7567.dp_997}


\begin{styleenv}
\begin{assessment}
Die Erfolgskontrolle wird in der Modulbeschreibung erläutert.


\end{assessment}

\begin{conditions}Keine.\end{conditions}


\end{styleenv}

\begin{learningoutcomes}
Die Studierenden sollen am Ende des Moduls

 \begin{itemize}\item den Übergang von der Schule zur Universität bewältigt haben,  \item mit logischem Denken und strengen Beweisen vertraut sein,   \item die Methoden und grundlegenden Strukturen der Linearen Algebra und Analytischen Geometrie beherrschen.  \end{itemize}
\end{learningoutcomes}

\begin{content}
\begin{itemize}\item Vektorräume mit Skalarprodukt (bilineare Abbildungen, Skalarprodukt, Norm, Orthogonalität, adjungierte Abbildung, normale und selbstadjungierte Endomorphismen, Spektralsatz, Isometrien und Normalformen)  \item Affine Geometrie (Affine Räume, Unterräume, Affine Abbildungen, affine Gruppe, Fixelemente)  \item Euklidische Räume (Unterräume, Bewegungen, Klassifikation, Ähnlichkeitsabbildungen)  \item Quadriken (Affine Klassifikation, Euklidische Klassifikation)  \end{itemize}
\end{content}



\begin{literature}Wird in der Vorlesung bekannt gegeben.

\end{literature}



\end{course}