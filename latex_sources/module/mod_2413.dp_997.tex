% Modulbeschreibung 
% Informationsgrad : extern
% Sprache: de
\begin{module}

\setdoclanguagegerman
\moduledegreeprogramme{Informatik (B.Sc.)}
\modulesubject{Theoretische Informatik}
\moduleID{IN1INALG1}
\modulename{Algorithmen I}
\modulecoordination{P. Sanders, D. Wagner}

\documentdate{2012-02-01 10:21:12.545542}

\modulecredits{6}
\moduleduration{1}
\modulecycle{Jedes 2. Semester, Sommersemester}



\modulehead

% For index (key word@display). Key word is used for sorting - no Umlauts please.
\index{Algorithmen I@Algorithmen I (M)}

% For later referencing
\label{mod_2413.dp_997}

\begin{courselist}
24500 & Algorithmen I (S.~\pageref{cour_7017.dp_997}) & 3/1/2 & S & 6 & M. Zitterbart\\
\end{courselist}

\begin{styleenv}
\begin{assessment}
Die Erfolgskontrolle besteht aus einer schriftlichen Abschlussprüfung nach § 4 Abs. 2 Nr. 1 SPO im Umfang von 120 Minuten.

 

Die Modulnote ist die Note der Abschlussprüfung.


\end{assessment}

\begin{conditions}Keine.\end{conditions}


\end{styleenv}

\begin{learningoutcomes}
Der/die Studierende

 \begin{itemize}\item kennt und versteht grundlegende, häufig benötigte Algorithmen, ihren Entwurf, Korrektheits- und Effizienzanalyse,  \item Implementierung, Dokumentierung und Anwendung,  \item kann mit diesem Verständnis auch neue algorithmische Fragestellungen bearbeiten,  \item wendet die im Modul Grundlagen der Informatik (Bachelor Informationswirtschaft) erworbenen Programmierkenntnisse  \item auf nichttriviale Algorithmen an,  \item wendet die in Grundbegriffe der Informatik (Bachelor Informatik) bzw. Grundlagen der Informatik (Bachelor Informationswirtschaft) und den Mathematikvorlesungen erworbenen mathematischen Herangehensweise an die Lösung von Problemen an. Schwerpunkte sind hier formale Korrektheitsargumente und eine mathematische Effizienzanalyse.  \end{itemize}
\end{learningoutcomes}

\begin{content}
Dieses Modul soll Studierenden grundlegende Algorithmen und Datenstrukturen vermitteln.

 

Die Vorlesung behandelt unter anderem:

 \begin{itemize}\item Grundbegriffe des Algorithm Engineering  \item Asymptotische Algorithmenanalyse (worst case, average case, probabilistisch, amortisiert)  \item Datenstrukturen z.B. Arrays, Stapel, Warteschlangen und Verkettete Listen  \item Hashtabellen  \item Sortieren: vergleichsbasierte Algorithmen (z.B. quicksort, insertionsort), untere Schranken, Linearzeitalgorithmen (z.B. radixsort)  \item Prioritätslisten  \item Sortierte Folgen,Suchbäume und Selektion  \item Graphen (Repräsentation, Breiten-/Tiefensuche, Kürzeste Wege,Minimale Spannbäume)  \item Generische Optimierungsalgorithmen (Greedy, Dynamische Programmierung, systematische Suche, Lokale Suche)  \item Geometrische Algorithmen  \end{itemize}
\end{content}

\begin{remarks}Für Studierende, die das Modul im SS 09 begonnen haben und die Mittsemesterklausur nicht mitgeschrieben haben, besteht im SS 10 letztmalig die Möglichkeit, diese Erfolgskontrolle abzulegen. Studierende, die das Modul im SS 10 begonnen haben, legen die Mittsemesterklausur nur noch im Rahmen des Übungsscheines unbenotet ab.

 

Ab SS 2011 wird die Erfolgskontrolle ohne unbenoteten Übungsschein erbracht.

\end{remarks}

\end{module}

