% Lehrveranstaltungsbeschreibung
% Informationsgrad : extern
% Sprache: de
\begin{course}

\setdoclanguagegerman
\coursedegreeprogramme{Informatik}
\coursemodulename{Strategie und Organisation (S.~\pageref{mod_1659.dp_997})[IN3WWBWL11]}
\courseID{2577900}
\coursename{Unternehmensführung und Strategisches Management}
\coursecoordination{H. Lindstädt}

\documentdate{2011-12-15 15:48:01.159955}

\courselevel{4}
\coursecredits{4}
\courseterm{Sommersemester}
\coursehours{2/0}
\courseinstructionlanguage{de}

\coursehead

% For index (key word@display). Key word is used for sorting - no Umlauts please.
\index{Unternehmensfuehrung und Strategisches Management@Unternehmensführung und Strategisches Management}

% For later referencing
\label{cour_5015.dp_997}


\begin{styleenv}
\begin{assessment}
Die Erfolgskontrolle erfolgt in Form einer schriftlichen Prüfung (60min.) (nach §4(2), 1 SPO) zu Beginn der vorlesungsfreien Zeit des Semesters.

 

Die Prüfung wird in jedem Semester angeboten und kann zu jedem ordentlichen Prüfungstermin wiederholt werden.


\end{assessment}

\begin{conditions}Keine.\end{conditions}


\end{styleenv}

\begin{learningoutcomes}
Die Teilnehmer lernen zentrale Konzepte des strategischen Managements entlang des idealtypischen Strategieprozesses kennen: interne und externe strategische Analyse, Konzept und Quellen von Wettbewerbsvorteilen, ihre Bedeutung bei der Formulierung von Wettbewerbs- und von Unternehmensstrategien sowie Strategiebewertung und -implementierung. Dabei soll vor allem ein Überblick grundlegender Konzepte und Modelle des strategischen Managements gegeben, also besonders eine handlungsorientierte Integrationsleistung erbracht werden.


\end{learningoutcomes}

\begin{content}
\begin{itemize}\item Grundlagen der Unternehmensführung  \item Grundlagen des Strategischen Managements  \item Strategische Analyse  \item Wettbewerbsstrategie: Formulierung und Auswahl auf Geschäftsfeldebene  \item Strategien in Oligopolen und Netzwerken: Antizipation von Abhängigkeiten  \item Unternehmensstrategie: Formulierung und Auswahl auf Unternehmensebene  \item Strategieimplementierung  \end{itemize}
\end{content}

\begin{media}Folien.

\end{media}

\begin{literature}\begin{itemize}\item Grant, R.M.: \emph{Strategisches Management}. 5. aktualisierte Aufl., München 2006.  \item Lindstädt, H.; Hauser, R.: \emph{Strategische Wirkungsbereiche des Unternehmens}. Wiesbaden 2004.  \end{itemize}

Die relevanten Auszüge und zusätzliche Quellen werden in der Veranstaltung bekannt gegeben.

\end{literature}



\end{course}