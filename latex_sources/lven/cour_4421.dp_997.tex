% Lehrveranstaltungsbeschreibung
% Informationsgrad : extern
% Sprache: de
\begin{course}

\setdoclanguagegerman
\coursedegreeprogramme{Informatik}
\coursemodulename{Seminarmodul Recht (S.~\pageref{mod_4185.dp_997})[IN3JURASEM]}
\courseID{rechtsem}
\coursename{Seminar aus Rechtswissenschaften}
\coursecoordination{T. Dreier, P. Sester, I. Spiecker genannt Döhmann}

\documentdate{2011-07-28 11:58:22.981942}

\courselevel{3}
\coursecredits{2}
\courseterm{Winter-/Sommersemester}
\coursehours{2}
\courseinstructionlanguage{de}

\coursehead

% For index (key word@display). Key word is used for sorting - no Umlauts please.
\index{Seminar aus Rechtswissenschaften@Seminar aus Rechtswissenschaften}

% For later referencing
\label{cour_4421.dp_997}


\begin{styleenv}
\begin{assessment}
Die Erfolgskontrolle erfolgt durch Ausarbeiten einer schriftlichen Seminararbeit sowie der Präsentation derselbigen als Erfolgskontrolle anderer Art nach § 4 Abs. 2 Nr. 3 SPO.


\end{assessment}

\begin{conditions}Keine.\end{conditions}

\begin{recommendations}Parallel zu den Veranstaltungen werden begleitende Tutorien angeboten, die insbesondere der Vertiefung der juristischen Arbeitsweise dienen. Ihr Besuch wird nachdrücklich empfohlen.\newline
Während des Semesters wird eine Probeklausur zu jeder Vorlesung mit ausführlicher Besprechung gestellt. Außerdem wird eine Vorbereitungsstunde auf die Klausuren in der vorlesungsfreien Zeit angeboten.\newline
Details dazu auf der Homepage des ZAR (www.kit.edu/zar).

\end{recommendations}
\end{styleenv}

\begin{learningoutcomes}
Ziel des Seminars ist es, die Studenten zur selbständigen wissenschaftlichen Bearbeitung eines rechtlichen Themas aus dem Gebiet der Informationswirtschaft zu befähigen. Thematisch erfasst das Seminar sämtliche Rechtsfragen des Informationsrechts und des Wirtschaftsrechts, vom Internetrecht über das Recht des geistigen Eigentums, das Wettbewerbsrecht und das Datenschutzrecht bis hin zum Vertragsrecht. Die Themen umfassen das nationale, das europäische und das internationale Recht. Die Seminararbeiten sollen in der Regel auch die informationstechnischen und die ökonomischen Bezüge der behandelten rechtlichen Fragestellungen beleuchten.


\end{learningoutcomes}

\begin{content}
Das Seminar befasst sich mit den Rechtsfragen des Informationsrechts, vom Internetrecht über das Recht des geistigen Eigentums, das Wettbewerbsrecht und das Datenschutzrecht bis hin zum Vertragsrecht. Die Themen umfassen das nationale, das europäische und das internationale Recht. Dabei haben die einzelnen Seminare unterschiedliche Schwerpunktsetzungen. Die Seminararbeiten sollen in der Regel auch die informationstechnischen und die ökonomischen Bezüge der behandelten rechtlichen Fragestellungen beleuchten. Die aktuelle Thematik des jeweiligen Seminars inklusive der zu bearbeitenden Themenvorschläge wird vor Semesterbeginn im Internet bekannt gegeben.

 

Absolviert werden können hier die vom ZAR/IIR angebotenen Seminare (Masterseminare, Seminare im Rahmen der Kooperation mit der Universität Freiburg und sonstige eigens gekennzeichnete Seminare können nur nach gesonderter Voranmeldung besucht werden).


\end{content}

\begin{media}Ausführliches Skript mit Fällen, Gliederungsübersichten, Unterlagen in den Veranstaltungen.

\end{media}

\begin{literature}Literatur wird in der Vorlesung bekannt gegeben.

\end{literature}



\end{course}