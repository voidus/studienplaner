% Lehrveranstaltungsbeschreibung
% Informationsgrad : extern
% Sprache: de
\begin{course}

\setdoclanguagegerman
\coursedegreeprogramme{Informatik}
\coursemodulename{Energiewirtschaft (S.~\pageref{mod_2871.dp_997})[IN3WWBWL12]}
\courseID{2581005}
\coursename{Unternehmensführung in der Energiewirtschaft}
\coursecoordination{H. Villis}

\documentdate{2011-07-01 17:09:20.548824}

\courselevel{3}
\coursecredits{3,5}
\courseterm{Sommersemester}
\coursehours{2/0}
\courseinstructionlanguage{de}

\coursehead

% For index (key word@display). Key word is used for sorting - no Umlauts please.
\index{Unternehmensfuehrung in der Energiewirtschaft@Unternehmensführung in der Energiewirtschaft}

% For later referencing
\label{cour_13749.dp_997}


\begin{styleenv}
\begin{assessment}
Die Erfolgskontrolle erfolgt in Form einer schriftlichen Prüfung (60min.) (nach §4(2), 1 SPO). Die Prüfung wird in jedem Semester angeboten und kann zu jedem ordentlichen Prüfungstermin wiederholt werden.


\end{assessment}

\begin{conditions}Keine.\end{conditions}


\end{styleenv}

\begin{learningoutcomes}
Der/ die Studierende

 \begin{itemize}\item Einblicke in die Führung eines großen Unternehmens der Energiewirtschaft erhalten.  \end{itemize}\begin{itemize}\item lernen, wie in einem solchen Unternehmen konkrete Fragestellungen aufgefasst, analysiert, bearbeitet und gelöst werden.  \item Strukturen, Prozesse und Projekte des Unternehmens anhand von konkreten Beispielen kennenlernen.  \item ihr energiewirtschaftliches Wissen vertiefen und sich mit seiner Umsetzung in die betriebliche Praxis vertraut machen.   \end{itemize}
\end{learningoutcomes}

\begin{content}
Gegenstand der Vorlesung sind Fragestellungen des Managements eines großen Unternehmens der Ener-giewirtschaft in Deutschland. Ausgehend von übergeordneten Leitungsfunktionen wie Unternehmensplanung, Strategie, Finanzen, Controlling, Regulierungsmanagement usw. werden im Anschluss anhand der energiewirtschaftlichen Wertschöpfungskette (Erzeugung, Handel, Netze, Vertrieb) Strukturen, Prozesse und Projekte aus der Führungsperspektive dargestellt. Zur inhaltlichen Abrundung ist eine Exkursion zur Baustelle des Rheinhafen-Dampfkraftwerks (RDK 8) geplant, einem der derzeit größten Projekte der EnBW.


\end{content}







\end{course}