% Lehrveranstaltungsbeschreibung
% Informationsgrad : extern
% Sprache: de
\begin{course}

\setdoclanguagegerman
\coursedegreeprogramme{Informatik}
\coursemodulename{Mikroökonomische Theorie (S.~\pageref{mod_2461.dp_997})[IN3WWVWL6]}
\courseID{2520517}
\coursename{Wohlfahrtstheorie}
\coursecoordination{C. Puppe}

\documentdate{2012-01-09 16:27:06.692906}

\courselevel{4}
\coursecredits{4,5}
\courseterm{Sommersemester}
\coursehours{2/1}
\courseinstructionlanguage{de}

\coursehead

% For index (key word@display). Key word is used for sorting - no Umlauts please.
\index{Wohlfahrtstheorie@Wohlfahrtstheorie}

% For later referencing
\label{cour_7075.dp_997}


\begin{styleenv}
\begin{assessment}
Die Erfolgskontrolle erfolgt in Form einer schriftlichen (60min.) Prüfung nach § 4 Abs. 2 Nr. 1 SPO am Ende des Semesters sowie am Ende des auf die LV folgenden Semesters.


\end{assessment}

\begin{conditions}Die Veranstaltung \emph{Volkswirtschaftslehre I (Mikroökonomie)} [2600012] muss erfolgreich abgeschlossen sein.

\end{conditions}

\begin{recommendations}Kenntnisse aus der Veranstaltung \emph{Volkswirtschaftslehre II (Makroökonomie)} [2600014] werden empfohlen.

\end{recommendations}
\end{styleenv}

\begin{learningoutcomes}
Der/die Studierende

 \begin{itemize}\item beherrscht den Umgang mit grundlegenden Konzepten und Methoden der Wohlfahrtstheorie und kann diese auf reale Probleme anwenden.  \end{itemize}
\end{learningoutcomes}

\begin{content}
Die Vorlesung \emph{Wohlfahrtstheorie }beschäftigt sich mit der Frage nach der Effizienz und den Verteilungseigenschaften von ökonomischen Allokationen, insbesondere von Marktgleichgewichten. Ausgangspunkt der Vorlesung sind die beiden Wohlfahrtssätze: Das 1.Wohlfahrtstheorem besagt, dass (unter schwachen Voraussetzungen) jedes Wettbewerbsgleichgewicht effizient ist. Gemäß des 2.Wohlfahrtstheorems kann umgekehrt (unter stärkeren Voraussetzungen) jede effiziente Allokation als ein Wettbewerbsgleichgewicht durch geeignete Wahl der Anfangsausstattung erhalten werden. Anschließend werden die Begriffe der Neidfreiheit sowie das verwandte Konzept der egalitären Äquivalenz im Rahmen der allgemeinen Gleichgewichtstheorie diskutiert. Der zweite Teil der Vorlesung kreist um den Begriff der „sozialen Gerechtigkeit” (d.h. Verteilungsgerechtigkeit). Es werden die grundlegenden Prinzipien des Utilitarismus, der Rawls'schen Theorie der Gerechtigkeit sowie John Roemers Theorie von Chancengleichheit vorgestellt und kritisch beleuchtet.


\end{content}



\begin{literature}\textbf{Weiterführende Literatur:}

 \begin{itemize}\item J. Rawls: \emph{A Theory of Justice}. Harvard University Press (1971)  \item J. Roemer: \emph{Theories of Distributive Justice}. Harvard University Press (1996)  \end{itemize}\end{literature}

\begin{remarks}Die Veranstaltung wird voraussichtlich wieder im Sommersemester 2013 angeboten.

\end{remarks}

\end{course}