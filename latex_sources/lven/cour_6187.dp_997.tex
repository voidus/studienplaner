% Lehrveranstaltungsbeschreibung
% Informationsgrad : extern
% Sprache: de
\begin{course}

\setdoclanguagegerman
\coursedegreeprogramme{Informatik}
\coursemodulename{Grundbegriffe der Informatik (S.~\pageref{mod_2355.dp_997})[IN1INGI]}
\courseID{24001}
\coursename{Grundbegriffe der Informatik}
\coursecoordination{T. Schultz}

\documentdate{2010-08-20 11:07:49.401952}

\courselevel{1}
\coursecredits{4}
\courseterm{Wintersemester}
\coursehours{2/1/2}
\courseinstructionlanguage{de}

\coursehead

% For index (key word@display). Key word is used for sorting - no Umlauts please.
\index{Grundbegriffe der Informatik@Grundbegriffe der Informatik}

% For later referencing
\label{cour_6187.dp_997}


\begin{styleenv}
\begin{assessment}
Die Erfolgskontrolle wird in der Modulbeschreibung erläutert.


\end{assessment}

\begin{conditions}Keine.\end{conditions}


\end{styleenv}

\begin{learningoutcomes}
Der/die Studierende soll

 \begin{itemize}\item grundlegende Definitionsmethoden erlernen und in die Lage versetzt werden, entsprechende Definitionen zu lesen und zu verstehen.  \item den Unterschied zwischen Syntax und Semantik kennen.  \item die grundlegenden Begriffe aus diskreter Mathematik und Informatik kennen und die Fähigkeit haben, sie im Zusammenhang mit der Beschreibung von Problemen und Beweisen anzuwenden.  \end{itemize}
\end{learningoutcomes}

\begin{content}
\begin{itemize}\item Algorithmen informell, Grundlagen des Nachweises ihrer Korrektheit\newline
Berechnungskomplexität, „schwere“ Probleme\newline
O-Notation, Mastertheorem  \item Alphabete, Wörter, formale Sprachen\newline
endliche Akzeptoren, kontextfreie Grammatiken  \item induktive/rekursive Definitionen, vollständige und strukturelle Induktion\newline
Hüllenbildung  \item Relationen und Funktionen  \item Graphen  \end{itemize}
\end{content}

\begin{media}Vorlesungsskript (Pdf), Folien (Pdf).

\end{media}

\begin{literature}\textbf{Weiterführende Literatur:}

 \begin{itemize}\item Goos: Vorlesungen über Informatik, Band 1, Springer, 2005  \item Abeck: Kursbuch Informatik I, Universitätsverlag Karlsruhe, 2005  \end{itemize}\end{literature}



\end{course}