% Modulbeschreibung 
% Informationsgrad : extern
% Sprache: de
\begin{module}

\setdoclanguagegerman
\moduledegreeprogramme{Informatik (B.Sc.)}
\modulesubject{}
\moduleID{IN2INPROSEM}
\modulename{Proseminar}
\modulecoordination{B. Beckert}

\documentdate{2012-01-10 09:52:21.647403}

\modulecredits{3}
\moduleduration{1}
\modulecycle{Jedes Semester}



\modulehead

% For index (key word@display). Key word is used for sorting - no Umlauts please.
\index{Proseminar@Proseminar (M)}

% For later referencing
\label{mod_2385.dp_997}

\begin{courselist}
PROSEM & Proseminar (S.~\pageref{cour_6261.dp_997}) & 2 & W/S & 3 & Dozenten der Fakultät für Informatik\\
ProSemSWT & Proseminar Softwaretechnik (S.~\pageref{cour_7351.dp_997}) & 2 & W/S & 3 & R. Reussner, G. Snelting\\
prosemis & Proseminar Informationssysteme (S.~\pageref{cour_6111.dp_997}) & 2 & S & 3 & K. Böhm\\
24059/24544 & Anthropomatik: Von der Theorie zur Anwendung  (S.~\pageref{cour_10705.dp_997}) & 2 & W/S & 3 & J. Beyerer, U. Hanebeck\\
ProsemAT & Proseminar: Algorithmen-Theorie (S.~\pageref{cour_13745.dp_997}) & 2 & W/S & 3 & D. Wagner\\
24530 & Proseminar Zellularautomaten und diskrete komplexe Systeme (S.~\pageref{cour_13895.dp_997}) & 2 & S & 3 & R. Vollmar, T. Worsch\\
24050 & Proseminar Algorithmentechnik (S.~\pageref{cour_14479.dp_997}) & 2 & W & 3 & P. Sanders, Veit Batz, Timo Bingmann, Christian Schulz\\
\end{courselist}

\begin{styleenv}
\begin{assessment}
Die Erfolgskontrolle erfolgt als Erfolgskontrolle anderer Art nach § 4 Abs. 2 Nr. 3 SPO und wird bei der Lehrveranstaltung dieses Moduls beschrieben.


\end{assessment}

\begin{conditions}Die im Rahmen dieses Moduls besuchten Seminarveranstaltungen müssen von Fachvertretern der Fakultät für Informatik angeboten sein.

\end{conditions}


\end{styleenv}

\begin{learningoutcomes}
\begin{itemize}\item Die Studierenden erhalten eine erste Einführung in das wissenschaftliche Arbeiten auf einem speziellen Fachgebiet.  \item Die Bearbeitung der Proseminararbeit bereitet zudem auf die Abfassung der Bachelorarbeit vor.  \item Mit dem Besuch der Proseminarveranstaltungen werden neben Techniken des wissenschaftlichen Arbeitens auch Schlüsselqualifikationen integrativ vermittelt.  \end{itemize}
\end{learningoutcomes}

\begin{content}
Das Proseminarmodul behandelt in den angebotenen Proseminaren spezifische Themen, die teilweise in entsprechenden Vorlesungen angesprochen wurden und vertieft diese. In der Regel ist die Voraussetzung für das Bestehen des Moduls die Anfertigung einer schriftlichen Ausarbeitung von max. 15 Seiten sowie eine mündliche Präsentation von 20 - 45 Minunten. Dabei ist auf ein ausgewogenes Verhältnis zu achten.


\end{content}

\begin{remarks}Die Anmeldung erfolgt über das Studienbüro. Der erworbene Seminarschein ist im Studienbüro vorzulegen.

 

Das \emph{Proseminar Operation Systems Internals }wird nicht mehr angeboten.

\end{remarks}

\end{module}

