% Modulbeschreibung 
% Informationsgrad : extern
% Sprache: de
\begin{module}

\setdoclanguagegerman
\moduledegreeprogramme{Informatik (B.Sc.)}
\modulesubject{EF Mathematik}
\moduleID{IN3MATHST03}
\modulename{Markovsche Ketten}
\modulecoordination{G. Last}

\documentdate{2011-10-06 18:00:19.995234}

\modulecredits{6}
\moduleduration{1}
\modulecycle{Jedes 2. Semester, Sommersemester}



\modulehead

% For index (key word@display). Key word is used for sorting - no Umlauts please.
\index{Markovsche Ketten@Markovsche Ketten (M)}

% For later referencing
\label{mod_3641.dp_997}

\begin{courselist}
1602 & Markovsche Ketten (S.~\pageref{cour_8045.dp_997}) & 3/1 & S & 6 & N. Bäuerle, N. Henze, B. Klar, G. Last\\
\end{courselist}

\begin{styleenv}
\begin{assessment}
Prüfung: schriftliche oder mündliche Prüfung \newline
Notenbildung: Note der Prüfung


\end{assessment}

\begin{conditions}Die Module \emph{Einführung in die Stochastik} [IN3MATHST01] und \emph{Wahrscheinlichkeitstheorie} [IN3MATHST02] müssen geprüft werden.

 

Das Modul \emph{Proseminar Mathematik} [IN3MATHPS] muss geprüft werden.

\end{conditions}

\begin{recommendations}Folgende Module sollten bereits belegt worden sein (Empfehlung):\newline
Einführung in die Stochastik

\end{recommendations}
\end{styleenv}

\begin{learningoutcomes}
Einführung in grundlegende Aussagen und Methoden für Markovsche Ketten.


\end{learningoutcomes}

\begin{content}
\begin{itemize}\item Markov-Eigenschaft  \item Übergangswahrscheinlichkeiten  \item Simulationsdarstellung  \item Irreduzibilität und Aperiodizität  \item Stationäre Verteilungen  \item Ergodensätze  \item Reversible Markovsche Ketten  \item Warteschlangen  \item Jackson-Netzwerke  \item Irrfahrten  \item Markov Chain Monte Carlo  \item Markovsche Ketten in stetiger Zeit  \item Übergangsintensitäten  \item Geburts-und Todesprozesse  \item Poissonscher Prozess  \end{itemize}
\end{content}



\end{module}

