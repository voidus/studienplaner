% Lehrveranstaltungsbeschreibung
% Informationsgrad : extern
% Sprache: de
\begin{course}

\setdoclanguagegerman
\coursedegreeprogramme{Informatik}
\coursemodulename{Multimediakommunikation (S.~\pageref{mod_2589.dp_997})[IN3INMMK]}
\courseID{24132}
\coursename{Multimediakommunikation}
\coursecoordination{R. Bless}

\documentdate{2010-06-07 10:07:05.343365}

\courselevel{4}
\coursecredits{4}
\courseterm{Wintersemester}
\coursehours{2/0}
\courseinstructionlanguage{de}

\coursehead

% For index (key word@display). Key word is used for sorting - no Umlauts please.
\index{Multimediakommunikation@Multimediakommunikation}

% For later referencing
\label{cour_5363.dp_997}


\begin{styleenv}
\begin{assessment}
Die Erfolgskontrolle wird in der Modulbeschreibung erläutert.


\end{assessment}

\begin{conditions}Keine.\end{conditions}

\begin{recommendations}Inhalte der Vorlesungen \emph{Einführung in Rechnernetze} [24519] (oder vergleichbarer Vorlesungen) und \emph{Telematik }[24128].

\end{recommendations}
\end{styleenv}

\begin{learningoutcomes}
Ziel der Vorlesung ist es, aktuelle Techniken und Protokolle für multimediale Kommunikation in – überwiegend Internet-basierten – Netzen zu vermitteln. Insbesondere vor dem Hintergrund der zunehmenden Sprachkommunikation über das Internet (Voice over IP) werden die Schlüsseltechniken und -protokolle wie RTP und SIP ausführlich erläutert, so dass deren Möglichkeiten und ihre Funktionsweise verstanden wird.


\end{learningoutcomes}

\begin{content}
Diese Vorlesung beschreibt Techniken und Protokolle, um beispielsweise Audio- und Videodaten im Internet zu übertragen. Behandelte Themen sind unter anderem: Audio- und Videokonferenzen, Audio/Video-Transportprotokolle, Voice over IP (VoIP), SIP zur Signalisierung und Aufbau sowie Steuerung von Multimedia-Sitzungen, RTP zum Transport von Multimediadaten über das Internet, RTSP zur Steuerung von A/V-Strömen, Enum zur Rufnummernabbildung, A/V-Streaming, Middleboxes und Caches, DVB und Video on Demand.


\end{content}

\begin{media}Folien. Mitschnitte von Protokolldialogen.

\end{media}

\begin{literature}James F. Kurose, and Keith W. Ross \emph{Computer Networking} 4th edition, Addison-Wesley/Pearson, 2007, ISBN 0-321-49770-8, Chapter Mulitmedia Networking.

 

\textbf{Weiterführende Literatur:}

 

Stephen Weinstein \emph{The Multimedia Internet} Springer, 2005, ISBN 0-387-23681-3

 

Alan B. Johnston \emph{SIP – understanding the Session Initiation Protocol} 2nd ed., Artech House, 2004

 

R. Steinmetz, K. Nahrstedt \emph{Multimedia Systems} Springer 2004, ISBN 3-540-40867-3

 

Ulrick Trick, Frank Weber: \emph{SIP, TPC/IP und Telekommunkationsnetze}, Oldenbourg, 3.\newline
Auflage, 2007

\end{literature}



\end{course}