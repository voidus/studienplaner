% Modulbeschreibung 
% Informationsgrad : extern
% Sprache: de
\begin{module}

\setdoclanguagegerman
\moduledegreeprogramme{Informatik (B.Sc.)}
\modulesubject{EF Mathematik}
\moduleID{IN3MATHAG03}
\modulename{Einführung in Geometrie und Topologie}
\modulecoordination{E. Leuzinger}

\documentdate{2011-10-06 18:08:31.658176}

\modulecredits{9}
\moduleduration{1}
\modulecycle{Jedes 2. Semester, Wintersemester}



\modulehead

% For index (key word@display). Key word is used for sorting - no Umlauts please.
\index{Einfuehrung in Geometrie und Topologie@Einführung in Geometrie und Topologie (M)}

% For later referencing
\label{mod_3101.dp_997}

\begin{courselist}
1026 & Einführung in Geometrie und Topologie (S.~\pageref{cour_7867.dp_997}) & 6 & W & 9 & S. Kühnlein, E. Leuzinger, W. Tuschmann\\
\end{courselist}

\begin{styleenv}
\begin{assessment}
Prüfung: schriftliche oder mündliche Prüfung\newline
Notenbildung: Note der Prüfung


\end{assessment}

\begin{conditions}Das Modul \emph{Proseminar Mathematik} [IN3MATHPS] muss geprüft werden.

 

Das Modul \emph{Riemannsche Geometrie}, \emph{Algebra}, \emph{Einführung in die Algebra und Zahlentheorie} oder \emph{Funktionstheorie} muss geprüft werden.

\end{conditions}

\begin{recommendations}Folgende Module sollten bereits belegt worden sein (Empfehlung):\newline
Lineare Algebra 1+2\newline
Analysis 1+2

\end{recommendations}
\end{styleenv}

\begin{learningoutcomes}
\begin{itemize}\item Einführung in exemplarische Gegenstände und Denkweisen der modernen Geometrie  \item Vorbereitung auf Seminare und weiterführende Vorlesungen im Bereich Geometrie  \end{itemize}
\end{learningoutcomes}

\begin{content}
\begin{itemize}\item Topologische und metrische Räume  \item Klassifikation von Flächen  \item Differentialgeometrie von Flächen  \item Optional: Raumformen  \end{itemize}
\end{content}



\end{module}

