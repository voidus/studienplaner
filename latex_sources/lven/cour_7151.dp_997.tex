% Lehrveranstaltungsbeschreibung
% Informationsgrad : extern
% Sprache: de
\begin{course}

\setdoclanguagegerman
\coursedegreeprogramme{Informatik}
\coursemodulename{Schlüsselqualifikationen (S.~\pageref{mod_2523.dp_997})[IN1HOCSQ]}
\courseID{PMP}
\coursename{Projektmanagement aus der Praxis}
\coursecoordination{K. Böhm, W. Schnober}

\documentdate{2010-10-08 12:00:58.977654}

\courselevel{4}
\coursecredits{1}
\courseterm{Sommersemester}
\coursehours{2}
\courseinstructionlanguage{de}

\coursehead

% For index (key word@display). Key word is used for sorting - no Umlauts please.
\index{Projektmanagement aus der Praxis@Projektmanagement aus der Praxis}

% For later referencing
\label{cour_7151.dp_997}


\begin{styleenv}
\begin{assessment}
Die Erfolgskontrolle erfolgt in Form einer “Erfolgskontrolle anderer Art” und besteht aus mehreren Teilaufgaben (§ 4 Abs. 2 Nr. 3 SPO). Dazu gehören Vorträge, Projektarbeiten, schriftliche Arbeiten und Seminararbeiten.

 

Die Veranstaltung wird mit “bestanden” oder “nicht bestanden” bewertet (§ 7 Abs. 3 SPO). Zum Bestehen der Veranstaltung müssen alle Teilaufgaben erfolgreich bestanden werden.


\end{assessment}

\begin{conditions}Es müssen weitere Vorlesungen/Seminare im Umfang von insgesamt mindestens 5 LP aus dem Gebiet der Informationssysteme (Info) am Lehrstuhl Prof. Böhm gehört worden sein bzw. im selben Semester belegt werden. Dazu zählt nicht die Veranstaltung \emph{Datenbanksysteme} [24516].

\end{conditions}

\begin{recommendations}Kenntnisse zu Grundlagen des Projektmanagements.

\end{recommendations}
\end{styleenv}

\begin{learningoutcomes}
Am Ende der LV sind die Teilnehmer in der Lage:

 \begin{itemize}\item Die Grundlagen des Projektmanagements zu kennen und in praktischen Anwendungsfällen anzuwenden.  \item Insbesondere kennen sie Projektphasen, Projektplanungs-Grundlagen, wesentliche Elemente der Planung wie Projekt Charter \& Scope Definitionen, Zielbeschreibungen, Aktivitätenplanung, Meilensteine, Projektstrukturpläne, Termin- und Kostenplanung, Risikomanagement, sowie wesentliche Elemente der Projektdurchführung, Krisenmanagement, Eskalationen und schließlich Projektabschlussaktivitäten.  \item Insbesondere lernen die Teilnehmer die objektiven Planungsgrundlagen als auch die subjektiven Faktoren, die in einem Projekt Relevanz haben, kennen und verstehen diese anzuwenden, u.a. Themen wie Kommunikation, Teamprozesse und Teambildung, Leadership, kreative Lösungsmethoden, Risikoabschätzungsmethoden.  \end{itemize}

Schlüsselfähigkeiten, die vermittelt werden, sind:

 \begin{itemize}\item Projektplanung  \item Projektsteuerung  \item Kommunikation  \item Führungsverhalten  \item Krisenmanagement  \item Erkennen und Behandeln schwieriger Situationen  \item Teambildung  \item Motivation (Eigen-/Fremd-)  \end{itemize}
\end{learningoutcomes}

\begin{content}
\begin{itemize}\item Projektrahmenbedinungen  \item  Projektziele / Kreative Methoden zur Projektzielfindung und Priorisierung  \item  Projektplanung  \item  Aktivitätenplanung  \item  Kosten-/Zeiten-/Ressourcenplanung  \item  Phasenmodelle  \item  Risikomanagement  \item  Projektsteuerung / Erfolgskontrolle / Monitoring  \item  Krisenmanagement  \item  Projektabschluss / Lessons Learned  \end{itemize}
\end{content}

\begin{media}Vorlesungsfolien, SW-Screenshots, diverse Präsentationstechniken (Kartentechnik u.ä.).

\end{media}



\begin{remarks}Die Unterlagen zur Lehrveranstaltung sind teilweise in Englisch.

 

Die Plätze sind begrenzt und die Anmeldung findet durch das Sekretariat Prof. Böhm statt.

 

Die Veranstaltung findet planmäßig alle drei Semester statt. Das nächste mal voraussichtlich im Sommersemester 2010.

\end{remarks}

\end{course}