% Lehrveranstaltungsbeschreibung
% Informationsgrad : extern
% Sprache: de
\begin{course}

\setdoclanguagegerman
\coursedegreeprogramme{Informatik}
\coursemodulename{Basispraktikum TI: Hardwarenaher Systementwurf (S.~\pageref{mod_4069.dp_997})[IN2INBPHS]}
\courseID{24309/24901}
\coursename{Basispraktikum TI: Hardwarenaher Systementwurf}
\coursecoordination{W. Karl}

\documentdate{2011-03-04 15:09:29.896638}

\courselevel{2}
\coursecredits{4}
\courseterm{Winter-/Sommersemester}
\coursehours{4}
\courseinstructionlanguage{de}

\coursehead

% For index (key word@display). Key word is used for sorting - no Umlauts please.
\index{Basispraktikum TI: Hardwarenaher Systementwurf@Basispraktikum TI: Hardwarenaher Systementwurf}

% For later referencing
\label{cour_8241.dp_997}


\begin{styleenv}
\begin{assessment}
Die Erfolgskontrolle wird in der Modulbeschreibung erläutert.


\end{assessment}

\begin{conditions}Keine.\end{conditions}

\begin{recommendations}Es wird der Besuch der LV \emph{Digitaltechnik und Entwurfsverfahren} empfohlen.

\end{recommendations}
\end{styleenv}

\begin{learningoutcomes}
Das Basispraktikum soll die Studierenden in die praktische Fähigkeit erwerben, mit Hilfe einer Hardware-Beschreibungssprache das Verhalten und die Struktur einer Schaltung zu beschreiben, und diese dann mit Hilfe von Hardware-Entwurfswerkzeugen zu implementieren und auf FPGA-Evaluierungsboards zu testen. \newline
Die Studenten sollen die Fähigkeit erwerben, in Teams zusammenzuarbeiten und die Aufgaben in projektorientierter Form zu lösen.


\end{learningoutcomes}

\begin{content}
Der Entwurf von Schaltungen und integrierten Schaltkreisen erfolgt heute durch hochsprachlichen Entwurf mit Hilfe von Hardware¬Beschreibungs¬sprachen. \newline
Im Rahmen dieses Basispraktikums werden in Form von Übungsaufgaben Schaltungen mit Hilfe von Hardware-Beschreibungssprachen entworfen und mit Hilfe von Entwurfswerkzeugen implementiert und getestet. \newline
\newline
Das Praktikum umfasst

 \begin{itemize}\item die schrittweise Einführung in die Hardware-Beschreibungssprache VHDL  \item die schrittweise Einführung in Hardware-Entwurfswerkzeuge   \item die Einführung in programmierbare Logik-Bausteine und  \item den Schaltungsentwurf, die Implementierung und den Test von einfachen Schaltungen   \end{itemize}
\end{content}

\begin{media}Versuchsbeschreibung, HW-Entwurfswerkzeuge, FPGA-Evaluierungsboards

\end{media}





\end{course}