% Modulbeschreibung 
% Informationsgrad : extern
% Sprache: de
\begin{module}

\setdoclanguagegerman
\moduledegreeprogramme{Informatik (B.Sc.)}
\modulesubject{EF Recht}
\moduleID{IN3INJUR2}
\modulename{Wirtschaftsprivatrecht}
\modulecoordination{P. Sester}

\documentdate{2011-07-29 10:36:20.552031}

\modulecredits{9}
\moduleduration{2}
\modulecycle{Jedes Semester}



\modulehead

% For index (key word@display). Key word is used for sorting - no Umlauts please.
\index{Wirtschaftsprivatrecht@Wirtschaftsprivatrecht (M)}

% For later referencing
\label{mod_2655.dp_997}

\begin{courselist}
24504 & BGB für Fortgeschrittene (S.~\pageref{cour_4387.dp_997}) & 2/0 & S & 3 & T. Dreier, P. Sester\\
24011 & Handels- und Gesellschaftsrecht (S.~\pageref{cour_4389.dp_997}) & 2/0 & W & 3 & P. Sester\\
24506/24017 & Privatrechtliche Übung (S.~\pageref{cour_4397.dp_997}) & 2/0 & W/S & 3 & P. Sester, T. Dreier\\
\end{courselist}

\begin{styleenv}
\begin{assessment}
Die Erfolgskontrolle des Moduls erfolgt in Form einer schriftlichen Prüfung über die belegten Vorlesungen (Erfolgskontrolle nach § 4(2), 1 SPO). Diese schriftliche Prüfung erfolgt im Rahmen der Privatrechlichen Übung.

 

Die Modulnote entspricht der Note der schriftlichen Prüfung.


\end{assessment}

\begin{conditions}Keine.\end{conditions}


\end{styleenv}

\begin{learningoutcomes}
Der/die Studierende

 \begin{itemize}\item besitzt vertiefte Kenntnisse des allgemeinen und des besonderen Schuldrechts sowie des Sachenrechts,  \item ist in der Lage, das Zusammenwirken der gesetzlichen Regelungen im BGB (betreffend die verschiedenen Vertragstypen und die dazugehörigen Haftungsfragen, Leistungsabwicklung, Leistungsstörungen, verschiedene Übereignungsarten sowie die dinglichen Sicherungsrechte) und im Handels- und Gesellschaftsrecht (hier insbesondere betreffend die Besonderheiten der Handelsgeschäfte, die handelsrechtliche Stellvertretung und das Kaufmannsrecht sowie die Organisationsformen, die das deutsche Gesellschaftsrecht für unternehmerische Aktivität zur Verfügung stellt) zu durchschauen,  \item erwirbt in der Privatrechtlichen Übung die Fähigkeit, juristische Problemfälle mit juristischen Mitteln methodisch sauber zu lösen.   \end{itemize}
\end{learningoutcomes}

\begin{content}
Das Modul baut auf dem Modul „Einführung in das Privatrecht“ auf. Der Studierende bekommt vertiefte Kenntnisse über besondere Vertragsarten des BGB sowie über komplexere gesellschaftsrechtliche Konstruktionen. Ferner wird den Studenten die Fähigkeit vermittelt, wie auch ein komplexerer juristischer Sachverhalt methodisch sauber zu lösen ist.


\end{content}



\end{module}

