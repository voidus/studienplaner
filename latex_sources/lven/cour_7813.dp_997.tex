% Lehrveranstaltungsbeschreibung
% Informationsgrad : extern
% Sprache: de
\begin{course}

\setdoclanguagegerman
\coursedegreeprogramme{Informatik}
\coursemodulename{Methodische Grundlagen des OR (S.~\pageref{mod_3833.dp_997})[IN3WWOR3], Anwendungen des Operations Research (S.~\pageref{mod_3831.dp_997})[IN3WWOR2], Supply Chain Management (S.~\pageref{mod_2721.dp_997})[IN3WWBWL14]}
\courseID{2550486}
\coursename{Standortplanung und strategisches Supply Chain Management}
\coursecoordination{S. Nickel}

\documentdate{2011-12-01 13:10:28.436427}

\courselevel{4}
\coursecredits{4,5}
\courseterm{Sommersemester}
\coursehours{2/1}
\courseinstructionlanguage{de}

\coursehead

% For index (key word@display). Key word is used for sorting - no Umlauts please.
\index{Standortplanung und strategisches Supply Chain Management@Standortplanung und strategisches Supply Chain Management}

% For later referencing
\label{cour_7813.dp_997}


\begin{styleenv}
\begin{assessment}
Die Erfolgskontrolle erfolgt in Form einer 120-minütigen schriftlichen Prüfung (nach §4(2), 1 SPO).

 

Die Prüfung wird jedes Semester angeboten.

 

Zulassungsvoraussetzung zur Klausur ist die erfolgreiche Teilnahme an den Online-Übungen.


\end{assessment}

\begin{conditions}Zulassungsvoraussetzung zur Klausur ist die erfolgreiche Teilnahme an den Online-Übungen.

\end{conditions}


\end{styleenv}

\begin{learningoutcomes}
Die Vorlesung vermittelt grundlegende quantitative Methoden der Standortplanung im Rahmen des strategischen Supply Chain Managements. Neben verschiedenen Möglichkeiten zur Standortbeurteilung werden die Studierenden mit den klassischen Standortplanungsmodellen (planare Modelle, Netzwerkmodelle und diskrete Modelle) sowie speziellen Standortplanungsmodellen für das Supply Chain Management (Einperiodenmodelle, Mehrperiodenmodelle) vertraut gemacht. Die parallel zur Vorlesung angebotenen Übungen bieten die Gelegenheit, die erlernten Verfahren praxisnah umzusetzen.


\end{learningoutcomes}

\begin{content}
Die Bestimmung eines optimalen Standortes in Bezug auf existierende Kunden ist spätestens seit der klassischen Arbeit von Weber „Über den Standort der Industrien“ aus dem Jahr 1909 eng mit der strategischen Logistikplanung verbunden. Strategische Entscheidungen, die sich auf die Platzierung von Anlagen wie Produktionsstätten, Vertriebszentren und Lager beziehen, sind von großer Bedeutung für die Rentabilität von Supply-Chains. Sorgfältig durchgeführte Standortplanungen erlauben einen effizienteren Materialfluss und führen zu verringerten Kosten und besserem Kundenservice.

 

Gegenstand der Vorlesung ist eine Einführung in die Begriffe der Standortplanung und die Vorstellung der wichtigsten quantitativen Standortplanungsmodelle. Darüber hinaus werden Modelle der Standortplanung im Supply Chain Management besprochen, wie sie auch teilweise bereits in kommerziellen SCM-Tools zur strategischen Planung Einzug gehalten haben.


\end{content}



\begin{literature}\textbf{Weiterführende Literatur:}

 \begin{itemize}\item Daskin: Network and Discrete Location: Models, Algorithms, and Applications, Wiley, 1995  \item Domschke, Drexl: Logistik: Standorte, 4. Auflage, Oldenbourg, 1996  \item Francis, McGinnis, White: Facility Layout and Location: An Analytical Approach, 2nd Edition, Prentice Hall, 1992  \item Love, Morris, Wesolowsky: Facilities Location: Models and Methods, North Holland, 1988  \item Thonemann: Operations Management - Konzepte, Methoden und Anwendungen, Pearson Studium, 2005  \end{itemize}\end{literature}

\begin{remarks}Das für drei Studienjahre im Voraus geplante Lehrangebot kann im Internet nachgelesen werden.

\end{remarks}

\end{course}