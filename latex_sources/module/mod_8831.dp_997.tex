% Modulbeschreibung 
% Informationsgrad : extern
% Sprache: de
\begin{module}

\setdoclanguagegerman
\moduledegreeprogramme{Informatik (B.Sc.)}
\modulesubject{}
\moduleID{IN3INSTW}
\modulename{Steuerungstechnik für Roboter und Werkzeugmaschinen}
\modulecoordination{H. Wörn}

\documentdate{2010-05-17 11:58:48.144359}

\modulecredits{3}
\moduleduration{1}
\modulecycle{Jedes 2. Semester, Sommersemester}



\modulehead

% For index (key word@display). Key word is used for sorting - no Umlauts please.
\index{Steuerungstechnik fuer Roboter und Werkzeugmaschinen@Steuerungstechnik für Roboter und Werkzeugmaschinen (M)}

% For later referencing
\label{mod_8831.dp_997}

\begin{courselist}
24700 & Steuerungstechnik für Roboter und Werkzeugmaschinen  (S.~\pageref{cour_5597.dp_997}) & 2 & S & 3 & H. Wörn\\
\end{courselist}

\begin{styleenv}
\begin{assessment}
Die Erfolgskontrolle erfolgt in Form einer mündlichen Prüfung im Umfang von i.d.R. 30 Minuten nach § 4 Abs. 2 Nr. 2 der SPO.

 

Die Modulnote ist die Note der mündlichen Prüfung.


\end{assessment}

\begin{conditions}Keine.\end{conditions}

\begin{recommendations}Es wird empfohlen voher das Modul \emph{Steuerungstechnik für Roboter} [IN3INSTR] zu absolvieren.

\end{recommendations}
\end{styleenv}

\begin{learningoutcomes}
\begin{itemize}\item Der Student soll die prinzipielle Sensordatenverarbeitung mit taktilen und visuellen Sensoren verstehen und aktuelle Roboterforschungsgebiete wie Mensch-Roboter-Kooperation, Medizinrobotik, Mikro- und Schwarmrobotik kennenlernen.  \item Der Student soll Bauformen und Komponenten von Fertigungsmaschinen verstehen.   \item Der Student soll die Funktionsweise und die Programmierung einer NC (Numerische Steuerung) verstehen und anwenden lernen.  \item Der Student soll die Funktionsweise und die Programmierung einer SPS (Speicherprogrammierbare Steuerung) verstehen, analysieren und anwenden lernen.  \item Der Student soll eine NC-Hardwarearchitektur und eine NC-Softwarearchitektur, die in einzelne Tasks mit Prioritäten gegliedert ist, analysieren und entwerfen können.  \item Der Student soll grundlegende Verfahren für die Bewegungsführung, für die Interpolation und für die Maschinenachsenregelung kennenlernen und anwenden können.  \end{itemize}
\end{learningoutcomes}

\begin{content}
Es werden Sensoren für Roboter und die prinzipielle Sensordatenverarbeitung bei sensorgestützten Robotern behandelt. Neue Roboterforschungsbereiche wie Mensch-Roboter-Kooperation, Medizinrobotik, Mikro- und Schwarmrobotik werden behandelt.

 

Es wird der Aufbau und die Struktur einer numerischen Steuerung (NC) mit den wesentlichen Funktionen einer NC, z.B. Bedien- und Steuerdaten Ein-/Ausgabe, Interpreter, Datenvorbereitung, Interpolation, Transformation, Regelung, Logikbearbeitung sowie der Informationsfluss innerhalb der NC behandelt. Darauf aufbauend wird eine modulare Softwarestruktur einer NC als Referenzmodell definiert. Als Steuerungshardware-Plattform werden Eingebettete Systeme, modulare Mehrprozessor-Systeme und PC-Systeme dargestellt. Die Gliederung der NC-Software in einzelne priorisierte Tasks mit Hilfe eines Echtzeitbetriebsystems wird behandelt. Ein Konzept für eine komponentenbasierte, wieder verwendbare Software wird vorgestellt. Der prinzipielle Hardware- und Software-Aufbau sowie die prinzipiellen Programmierverfahren einer Speicherprogrammierbaren Steuerung (SPS) werden erläutert. Die einzelnen Verfahren zur Programmieren von Maschinen z.B. Programmieren nach DIN 66025, Maschinelles Programmieren, Programmieren mit EXAPT, Simulationsgestütztes Programmieren, Werkstattorientiertes Programmieren werden mit Beispielen präsentiert. Die grundlegenden Verfahren für das Entwerfen einer Bewegungssteuerung z.B. Trajektorienberechnung, satzübergreifende Bewegungsführung, Geschwindigkeitsprofilerzeugung und Interpolation (Linear-, Zirkular- und Spli-neinterpolation) werden behandelt. Es werden Algorithmen zur Steuerung und Regelung von Elektromotoren sowie digitale Antriebsbussysteme vorgestellt.


\end{content}



\end{module}

