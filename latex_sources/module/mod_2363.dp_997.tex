% Modulbeschreibung 
% Informationsgrad : extern
% Sprache: de
\begin{module}

\setdoclanguagegerman
\moduledegreeprogramme{Informatik (B.Sc.)}
\modulesubject{}
\moduleID{IN3INEZS}
\modulename{Echtzeitsysteme}
\modulecoordination{H. Wörn}

\documentdate{2010-05-17 11:11:36.274348}

\modulecredits{6}
\moduleduration{1}
\modulecycle{Jedes 2. Semester, Sommersemester}



\modulehead

% For index (key word@display). Key word is used for sorting - no Umlauts please.
\index{Echtzeitsysteme@Echtzeitsysteme (M)}

% For later referencing
\label{mod_2363.dp_997}

\begin{courselist}
24576 & Echtzeitsysteme (S.~\pageref{cour_6213.dp_997}) & 3/1 & S & 6 & H. Wörn, T. Längle\\
\end{courselist}

\begin{styleenv}
\begin{assessment}
Die Erfolgskontrolle erfolgt in Form einer schriftlichen Prüfung im Umfang von 60 Minuten gemäß § 4 Abs. 2 Nr. 1 SPO. \newline
Die Modulnote ist die Note der schriftlichen Prüfung.


\end{assessment}

\begin{conditions}Keine.\end{conditions}


\end{styleenv}

\begin{learningoutcomes}
Der Student soll grundlegende Verfahren, Modellierungen und Architekturen von Echtzeitsystemen am Beispiel der Automatisierungstechnik mit Steuerungen und Regelungen verstehen und anwenden lernen. Er soll in der Lage sein, Echtzeitsysteme bezüglich Hard- und Software zu analysieren, zu strukturieren und zu entwerfen. Der Student soll weiter in die Grundkonzepte der Echtzeitsysteme, Robotersteuerung, Werkzeugmaschinesteuerung und speicherprogrammierbaren Steuerung eingeführt werden.


\end{learningoutcomes}

\begin{content}
Es werden die grundlegenden Prinzipien, Funktionsweisen und Architekturen von Echtzeitsystemen vermittelt. Einführend werden zunächst grundlegende Methoden für Modellierung und Entwurf von diskreten Steuerungen und zeitkontinuierlichen und zeitdiskreten Regelungen für die Automation von technischen Prozessen behandelt. Danach werden die grundlegenden Rechnerarchitekturen (Mikrorechner, Mikrokontroller Signalprozessoren, Parallelbusse) sowie Hardwareschnittstellen zwischen Echtzeitsystem und Prozess dargestellt. Echtzeitkommunikation am Beispiel Industrial Ethernet und Feldbusse werden eingeführt. Es werden weiterhin die grundlegenden Methoden der Echtzeitprogrammierung (synchrone und asynchrone Programmierung), der Echtzeitbetriebssysteme (Taskkonzept, Echtzeitscheduling, Synchronisation, Ressourcenverwaltung) sowie der Echtzeit-Middleware dargestellt. Abgeschlossen wird die Vorlesung durch Anwendungsbeispiele von Echtzeitsystemen aus der Fabrikautomation wie Speicherprogrammierbare Steuerung, Werkzeugmaschinensteuerung und Robotersteuerung.


\end{content}

\begin{remarks}Das Modul \emph{Echtzeitsysteme} ist ein Stammmodul.

\end{remarks}

\end{module}

