% Lehrveranstaltungsbeschreibung
% Informationsgrad : extern
% Sprache: de
\begin{course}

\setdoclanguagegerman
\coursedegreeprogramme{Informatik}
\coursemodulename{Proseminar Mathematik (S.~\pageref{mod_4165.dp_997})[IN3MATHPS]}
\courseID{ProsemMath}
\coursename{Proseminar Mathematik}
\coursecoordination{Dozenten der Fakultät für Mathematik}

\documentdate{2011-11-14 11:20:12.511919}

\courselevel{3}
\coursecredits{3}
\courseterm{Winter-/Sommersemester}
\coursehours{2}
\courseinstructionlanguage{de}

\coursehead

% For index (key word@display). Key word is used for sorting - no Umlauts please.
\index{Proseminar Mathematik@Proseminar Mathematik}

% For later referencing
\label{cour_8397.dp_997}


\begin{styleenv}
\begin{assessment}
Die Erfolgskontrolle erfolgt durch eine mündliche Präsentation eines Seminarthemas als Erfolgskontrolle anderer Art nach § 4 Abs. 2 Nr. 3 der SPO. Die Note entspricht dieser Präsentationsnote.


\end{assessment}

\begin{conditions}Keine.\end{conditions}


\end{styleenv}

\begin{learningoutcomes}
\begin{itemize}\item Die Studierenden erhalten eine erste Einführung in das wissenschaftliche Arbeiten auf einem speziellen Fachgebiet.  \item Die Bearbeitung der Proseminararbeit bereitet zudem auf die Abfassung der Bachelorarbeit vor.  \item Mit dem Besuch der Proseminarveranstaltungen werden neben Techniken des wissenschaftlichen Arbeitens auch Schlüsselqualifikationen integrativ vermittelt.  \end{itemize}
\end{learningoutcomes}

\begin{content}
Das Proseminarmodul behandelt in den angebotenen Proseminaren spezifische Themen, die teilweise in entsprechenden Vorlesungen angesprochen wurden und vertieft diese. In der Regel ist die Voraussetzung für das Bestehen des Moduls die Anfertigung einer schriftlichen Ausarbeitung von max. 15 Seiten sowie eine mündliche Präsentation von 20 - 45 Minuten. Dabei ist auf ein ausgewogenes Verhältnis zu achten.


\end{content}







\end{course}