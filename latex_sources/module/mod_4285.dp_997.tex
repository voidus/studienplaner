% Modulbeschreibung 
% Informationsgrad : extern
% Sprache: de
\begin{module}

\setdoclanguagegerman
\moduledegreeprogramme{Informatik (B.Sc.)}
\modulesubject{EF Informationsmanagement im Ingenieurwesen}
\moduleID{IN3MACHPPRF}
\modulename{Produkt-, Prozess- und Ressourcenintegration in der Fahrzeugentstehung}
\modulecoordination{Maier}

\documentdate{2011-07-20 10:04:54.622813}

\modulecredits{4}
\moduleduration{1}
\modulecycle{Jedes Semester}



\modulehead

% For index (key word@display). Key word is used for sorting - no Umlauts please.
\index{Produkt-, Prozess- und Ressourcenintegration in der Fahrzeugentstehung@Produkt-, Prozess- und Ressourcenintegration in der Fahrzeugentstehung (M)}

% For later referencing
\label{mod_4285.dp_997}

\begin{courselist}
2123364 & Produkt-, Prozess- und Ressourcenintegration in der Fahrzeugentstehung (S.~\pageref{cour_7513.dp_997}) & 2/1 & S & 4 & S. Mbang\\
\end{courselist}

\begin{styleenv}
\begin{assessment}
Die Erfolgskontrolle erfolgt in Form einer mündlichen Prüfung im Umfang von 30 Minuten (nach §4 (2), 2 SPO).

 

Die Note entspricht der Note der mündlichen Prüfung.


\end{assessment}

\begin{conditions}Die Module \textbf{\emph{Product Lifecycle Managemet}} [IN3MACHPLM] und \textbf{\emph{Technische Informationssysteme}} [IN3INMACHTI] müssen im Ergänzungsfach Informationsmanagement im Ingenieurwesen geprüft werden.

\end{conditions}


\end{styleenv}

\begin{learningoutcomes}
Der/ die Studierende

 \begin{itemize}\item hat einen Überblick zur Fahrzeugentstehung (Prozess- und Arbeitsabläufe, IT-Systeme) und zu den integrierten Produktmodellen in der Fahrzeugindustrie (Produkt-, Prozess- und Ressourcensichten),   \item ist in der Lage, neue CAx-Modellierungsmethoden (intelligente Feature-Technologie, Template- und Skelett-Methodik, funktionale Modellierung) anzuwenden,   \item versteht die Anforderungs- und prozessgerechte Fahrzeugentstehung (3D-Master Prinzip, Toleranzmodelle) sowie die Anwendung wissensbasierte Mechanismen in der Konstruktion und Produktionsplanung,  \item versteht den Einsatz virtueller Techniken und Methoden in der Fahrzeugentstehung anhand der Prinzipien der digitalen und virtuellen Fabrik.  \end{itemize}
\end{learningoutcomes}

\begin{content}
Themengebiete der Vorlesung:

 \begin{itemize}\item die gemeinsame Erarbeitung von Grundlagen basierend auf dem Stand der Technik in der Industrie und in der Forschung,   \item die praxisorientierte Ausarbeitung von Anforderungen und Konzepten zur Darstellung einer durchgängigen CAx-Prozesskette,   \item die Einführung in die Paradigmen der integrierten, prozessorientierten Produktgestaltung,   \item die Vermittlung praktischer, industrieller Kenntnisse in der durchgängigen Fahrzeugentstehung.   \end{itemize}

Durch die Kombination von Ingenieurwissen mit praktischen, realen Erkenntnissen aus der Industrie gibt die Vorlesung einen Einblick in konkrete industrielle Anwendungen, wie auch die Möglichkeit, die industriellen IT-Applikationen, IT-Prozesse und Arbeitsabläufe in der Automobilindustrie kennen zu lernen. Entsprechend ist eine begleitende, praktische Industrieprojektarbeit auf Basis eines durchgängigen Szenarios (von der Konstruktion über die Prüf- und Methodenplanung bis hin zur Betriebsmittelfertigung) vorgesehen. \newline
\newline
 Neben der eigentlichen Durchführung der Projektarbeit, in der die Studenten/Studentinnen ein oder mehrere interdisziplinäre Teams bilden, sollen auch die Arbeitsabläufe, die Kommunikation und die verteilte Entwicklung (Concurrent Engineering) eine zentrale Rolle spielen.


\end{content}



\end{module}

