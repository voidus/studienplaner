% Lehrveranstaltungsbeschreibung
% Informationsgrad : extern
% Sprache: de
\begin{course}

\setdoclanguagegerman
\coursedegreeprogramme{Informatik}
\coursemodulename{Bauökologie (S.~\pageref{mod_1639.dp_997})[IN3WWBWL16]}
\courseID{26404w}
\coursename{Bauökologie I}
\coursecoordination{T. Lützkendorf}

\documentdate{2011-06-27 15:17:05.719382}

\courselevel{3}
\coursecredits{4,5}
\courseterm{Wintersemester}
\coursehours{2/1}
\courseinstructionlanguage{de}

\coursehead

% For index (key word@display). Key word is used for sorting - no Umlauts please.
\index{Bauoekologie I@Bauökologie I}

% For later referencing
\label{cour_6851.dp_997}


\begin{styleenv}
\begin{assessment}
Die Erfolgskontrolle erfolgt in Form einer mündlichen Prüfung (20min.) (nach §4(2), 1 SPO).\newline
Die Prüfung wird in jedem Semester angeboten und kann zu jedem ordentlichen Prüfungstermin wiederholt werden.


\end{assessment}

\begin{conditions}Eine Kombination mit dem Modul \emph{Real Estate Management} [IN3WWBWL11] und mit einem ingenieurwissenschaftlichem Modul aus den Bereichen Bauphysik oder Baukonstruktion wird empfohlen.

\end{conditions}


\end{styleenv}

\begin{learningoutcomes}
Kenntnisse im Bereich des nachhaltigen Bauens auf den Ebenen Gesamtgebäude, Bauteile und Haustechniksysteme sowie Bauprodukte


\end{learningoutcomes}

\begin{content}
Am Beispiel von Niedrigenergiehäusern erfolgt eine Einführung in das kostengünstige, energiesparende, ressourcenschonende und gesundheitsgerechte Planen, Bauen und Bewirtschaften. Fragen der Umsetzung einer nachhaltigen Entwicklung im Baubereich werden auf den Ebenen Gesamtgebäude, Bauteile und Haustechniksysteme sowie Bauprodukte behandelt. Neben der Darstellung konstruktiver und technischer Zusammenhänge werden jeweils Grundlagen für eine Grobdimensionierung und Ansätze für eine ökonomisch-ökologische Bewertung vermittelt. Auf die Rolle der am Bau Beteiligten bei der Auswahl und Bewertung von Lösungen wird eingegangen. Themen sind u.a.: Integration ökonomischer und ökologischer Aspekte in die Planung, Energiekonzepte, Niedrigenergie- und Passivhäuser, aktive und passive Solarenergienutzung, Auswahl und Bewertung von Anschluss- und Detaillösungen, Auswahl und Bewertung von Dämm- und Wandbaustoffen, Gründächer, Sicherung von Gesundheit und Behaglichkeit, Regenwassernutzung, Haustechnik und Recycling.


\end{content}

\begin{media}Zur besseren Veranschaulichung der Lehrinhalte werden Videos und Simulationstools eingesetzt.

\end{media}

\begin{literature}\textbf{Weiterführende Literatur:}

 \begin{itemize}\item Umweltbundesamt (Hrsg.): „Leitfaden zum ökologisch orientierten Bauen“. C.F.Müller 1997  \item IBO (Hrsg.): „Ökologie der Dämmstoffe“. Springer 2000  \item Feist (Hrsg.): „Das Niedrigenergiehaus – Standard für energiebewusstes Bauen“. C.F.Müller 1998  \item Bundesarchitektenkammer (Hrsg.): „Energiegerechtes Bauen und Modernisieren“. Birkhäuser 1996  \item Schulze-Darup: „Bauökologie“. Bauverlag 1996  \end{itemize}\end{literature}



\end{course}