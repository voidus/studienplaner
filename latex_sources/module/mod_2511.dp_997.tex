% Modulbeschreibung 
% Informationsgrad : extern
% Sprache: de
\begin{module}

\setdoclanguagegerman
\moduledegreeprogramme{Informatik (B.Sc.)}
\modulesubject{}
\moduleID{IN3INKS}
\modulename{Kognitive Systeme}
\modulecoordination{R. Dillmann, A. Waibel}

\documentdate{2010-09-13 11:48:25.433017}

\modulecredits{6}
\moduleduration{1}
\modulecycle{Jedes 2. Semester, Sommersemester}



\modulehead

% For index (key word@display). Key word is used for sorting - no Umlauts please.
\index{Kognitive Systeme@Kognitive Systeme (M)}

% For later referencing
\label{mod_2511.dp_997}

\begin{courselist}
24572 & Kognitive Systeme (S.~\pageref{cour_7113.dp_997}) & 3/1 & S & 6 & R. Dillmann, A. Waibel, Christian Mohr, Markus Przybylski, Kai Welke\\
\end{courselist}

\begin{styleenv}
\begin{assessment}
Die Erfolgskontrolle erfolgt in Form einer schriftlichen Prüfung (Klausur) im Umfang von 60 Minuten nach § 4 Abs. 2 Nr. 1 der SPO.

 

Die Modulnote ist die Note der schriftlichen Prüfung (Klausur).\newline
\newline
Zusätzliche 6 Bonuspunkte zur Verbesserung der Note sind über die Abgabe der Übungsblätter erzielbar (keine Pflicht). Diese werden erst angerechnet, wenn die Klausur ohne die Bonuspunkte bestanden wurde.


\end{assessment}

\begin{conditions}Keine.\end{conditions}


\end{styleenv}

\begin{learningoutcomes}
\begin{itemize}\item Die relevanten Elemente des technischen kognitiven Systems können benannt und deren Aufgaben beschrieben werden.  \item Die Problemstellungen dieser verschiedenen Bereiche können erkannt und bearbeitet werden.  \item Weiterführende Verfahren können selbständig erschlossen und erfolgreich bearbeitet werden.  \item Variationen der Problemstellung können erfolgreich gelöst werden.  \item Die Lernziele sollen mit dem Besuch der zugehörigen Übung erreicht sein.  \end{itemize}
\end{learningoutcomes}

\begin{content}
Kognitive Systeme handeln aus der Erkenntnis heraus. Nach der Reizaufnahme durch Perzeptoren werden die Signale verarbeitet und aufgrund einer hinterlegten Wissensbasis gehandelt. In der Vorlesung werden die einzelnen Module eines kognitiven Systems vorgestellt. Hierzu gehören neben der Aufnahme und Verarbeitung von Umweltinformationen (z. B. Bilder, Sprache), die Repräsentation des Wissens sowie die Zuordnung einzelner Merkmale mit Hilfe von Klassifikatoren. Weitere Schwerpunkte der Vorlesung sind Lern- und Planungsmethoden und deren Umsetzung. In den Übungen werden die vorgestellten Methoden durch Aufgaben vertieft.


\end{content}

\begin{remarks}Das Modul \emph{Kognitive Systeme} ist ein Stammmodul.

\end{remarks}

\end{module}

