% Lehrveranstaltungsbeschreibung
% Informationsgrad : extern
% Sprache: de
\begin{course}

\setdoclanguagegerman
\coursedegreeprogramme{Informatik}
\coursemodulename{Softwaretechnik I (S.~\pageref{mod_2515.dp_997})[IN1INSWT1]}
\courseID{24518}
\coursename{Softwaretechnik I}
\coursecoordination{W. Tichy, Andreas Höfer}

\documentdate{2010-10-08 10:17:00.369162}

\courselevel{1}
\coursecredits{6}
\courseterm{Sommersemester}
\coursehours{3/1/2}
\courseinstructionlanguage{de}

\coursehead

% For index (key word@display). Key word is used for sorting - no Umlauts please.
\index{Softwaretechnik I@Softwaretechnik I}

% For later referencing
\label{cour_7119.dp_997}


\begin{styleenv}
\begin{assessment}
Die Erfolgskontrolle erfolgt in Form einer schriftlichen Prüfung im Umfang von 60 Minuten gemäß § 4 Abs. 2 Nr. 1 SPO.

 

Zusätzlich muss ein unbenoteter Übungsschein als Erfolgskontrolle anderer Art nach § 4 Abs. 2 Nr. 3 SPO erbracht werden.


\end{assessment}

\begin{conditions}Keine.\end{conditions}

\begin{recommendations}Das Modul \emph{Programmieren} sollte abgeschlossen sein.

\end{recommendations}
\end{styleenv}

\begin{learningoutcomes}
Der/Die Studierende soll

 \begin{itemize}\item Grundwissen über die Prinzipien, Methoden und Werkzeuge der Softwaretechnik erwerben.  \item komplexe Softwaresysteme ingenieurmäßig entwickeln und warten sollen.  \end{itemize}
\end{learningoutcomes}

\begin{content}
Inhalt der Vorlesung ist der gesamte Lebenszyklus von Software von der Projektplanung über die Systemanalyse, die Kostenschätzung, den Entwurf und die Implementierung, die Validation und Verifikation, bis hin zur Wartung von Software. Weiter werden UML, Entwurfsmuster, Software-Werkzeuge, Programmierumgebungen und Konfigurationskontrolle behandelt.


\end{content}

\begin{media}Folien (pdf), Übungsblätter

\end{media}

\begin{literature}\textbf{Weiterführende Literatur:}

 \begin{itemize}\item Objektorientierte Softwaretechnik : mit UML, Entwurfsmustern und Java / Bernd Brügge ; Allen H. Dutoit\newline
 München [u.a.] : Pearson Studium, 2004. - 747 S., ISBN 978-3-8273-7261-1  \item Lehrbuch der Software-Technik - Software Entwicklung / Helmut Balzert\newline
Spektrum-Akademischer Vlg; Auflage: 2., überarb. und erw. A. (Dezember 2000), ISBN-13: 978-3827404800  \item Software engineering / Ian Sommerville. - 7. ed.\newline
Boston ; Munich [u.a.] : Pearson, Addison-Wesley, 2004. - XXII, 759 S. \newline
(International computer science series), ISBN 0-321-21026-3  \item Design Patterns: Elements of Reusable Object-Oriented Software / Gamma, Erich and Helm, Richard and Johnson, Ralph and Vlissides, John, Addison-Wesley 2002\newline
ISBN 0-201-63361-2  \item C\# 3.0 design patterns : [Up-to-date for C\#3.0] / Judith Bishop\newline
Bejing ; Köln [u.a.] : O'Reilly, 2008. - XXI, 290 S. \newline
ISBN 0-596-52773-X, ISBN 978-0-596-52773-0  \end{itemize}\end{literature}



\end{course}