% Lehrveranstaltungsbeschreibung
% Informationsgrad : extern
% Sprache: de
\begin{course}

\setdoclanguagegerman
\coursedegreeprogramme{Informatik}
\coursemodulename{Web Engineering (S.~\pageref{mod_2579.dp_997})[IN3INWEBE]}
\courseID{24124}
\coursename{Web Engineering}
\coursecoordination{H. Hartenstein, M. Nußbaumer}

\documentdate{2010-07-23 11:56:56.926966}

\courselevel{4}
\coursecredits{4}
\courseterm{Wintersemester}
\coursehours{2/0}
\courseinstructionlanguage{de}

\coursehead

% For index (key word@display). Key word is used for sorting - no Umlauts please.
\index{Web Engineering@Web Engineering}

% For later referencing
\label{cour_5051.dp_997}


\begin{styleenv}
\begin{assessment}
Die Erfolgskontrolle wird in der Modulbeschreibung erläutert.


\end{assessment}

\begin{conditions}Keine.\end{conditions}

\begin{recommendations}Kenntnisse zu Grundlagen aus dem Stammmodul \emph{Softwaretechnik II }[IN3INSWT2]

\end{recommendations}
\end{styleenv}

\begin{learningoutcomes}
\begin{itemize}\item  Der Studierende soll die Grundbegriffe des Web Engineering erlernen und in aktuelle Methoden und Techniken eingeführt werden.  \item Studierende eignen sich Wissen über aktuelle Web-Technologien an und erlernen Grundkenntnisse zum eigenständigen Anwendungsentwurf und Managment von Web-Projekten im praxisnahen Umfeld.   \item Studierende erlernen praktische Methoden zur Analyse von Standards und Technologien im Web. Die Arbeit und der Umgang mit wissenschaftlichen Texten und Standard-Spezifikationen in englischer Fachsprache werden in besonderem Maße gefördert.  \item Die Studierenden können Probleme und Anforderungen im Bereich des Web Engineering analysieren, strukturieren und beschreiben.   \end{itemize}
\end{learningoutcomes}

\begin{content}
Die Lehrveranstaltung gibt eine Einführung in die Disziplin Web Engineering. Im Vordergrund stehen Vorgehensweisen und Methoden, die zu einer systematischen Konstruktion webbasierter Anwendungen und Systeme führen. Auf dedizierte Phasen und Aspekte der Lebenszyklen von Web-Anwendungen wird ebenfalls eingegangen. Dabei wird das Phänomen „Web” aus unterschiedlichen Perspektiven, wie der des Web Designers, Analysten, Architekten oder Ingenieurs, betrachtet und Methoden zum Umgang mit Anforderungen, Web Design, Architektur, Entwicklung und Management werden diskutiert. Es werden Verfahren zur systematischen Konstruktion von Web-Anwendungen und agilen Systemen vermittelt, die wichtige Bereiche, wie Anforderungsanalyse, Konzepterstellung, Entwurf, Entwicklung, Testen sowie Betrieb, Wartung und Evolution als integrale Bestandteile behandeln. Darüber hinaus demonstrieren Beispiele die Notwendigkeit einer agilen Ausrichtung von Teams, Prozessen und Technologien.


\end{content}

\begin{media}Folien

\end{media}

\begin{literature}Wird in der Vorlesung bekannt gegeben.

\end{literature}



\end{course}