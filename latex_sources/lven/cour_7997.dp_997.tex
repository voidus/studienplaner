% Lehrveranstaltungsbeschreibung
% Informationsgrad : extern
% Sprache: de
\begin{course}

\setdoclanguagegerman
\coursedegreeprogramme{Informatik}
\coursemodulename{Systemtheorie (S.~\pageref{mod_3937.dp_997})[IN3EITST]}
\courseID{23105}
\coursename{Messtechnik}
\coursecoordination{F. Puente León}

\documentdate{2011-02-17 13:23:19.492905}

\courselevel{3}
\coursecredits{5}
\courseterm{Sommersemester}
\coursehours{2/1}
\courseinstructionlanguage{de}

\coursehead

% For index (key word@display). Key word is used for sorting - no Umlauts please.
\index{Messtechnik@Messtechnik}

% For later referencing
\label{cour_7997.dp_997}


\begin{styleenv}
\begin{assessment}
Die Erfolgskontrolle erfolgt in Form einer schriftlichen Prüfung nach § 4 Abs. 2 Nr. 1 SPO im Umfang von i.d.R. 3 Stunden.

 

Die LV-Note ist die Note der Kausur.


\end{assessment}

\begin{conditions}Empfehlung: Kenntnisse über Integraltransformationen und Wahrscheinlichkeitsrechnung sind vorteilhaft.

\end{conditions}


\end{styleenv}

\begin{learningoutcomes}
Die Studentinnen und Studenten werden in die Lage versetzt, Probleme im Bereich der Messtechnik zu analysieren, formal systemtheoretisch zu beschreiben und zu lösen. Die Werkzeuge die Ihnen hierbei vorgestellt und am Ende beherrscht werden sollen sind Verfahren der Kurvenanpassung, verschiedene Grundverschaltungen von Messsystemen und stochastische Beschreibungen mittels Zufallsvariablen und stochastischen Prozessen sowie deren Korrelationsfunktionen.


\end{learningoutcomes}

\begin{content}
In dieser Vorlesung werden systemtechnische Grundlagen der Messtechnik vermittelt werden.\newline
Zunächst werden die Begriffe Messen und Messkennlinie eingeführt. Mögliche Ursachen für die stets auftretenden Messfehler werden vorgestellt und eine Klassifikation in systematische und zufällige Messfehler vorgenommen. Für beide Klassen von Fehlern werden im weiteren Verlauf der Vorlesung Wege aufgezeigt, diese zu vermindern.\newline
Da die Kennlinie realer Messsysteme i.A. nicht analytisch gegeben ist, sondern aus vorliegenden Messpunkten abgeleitet werden muss, werden grundlegende Verfahren der Kurvenanpassung vorgestellt. Hierbei werden sowohl Verfahren zur Approximation (Least-Squares-Schätzer) als auch zur Interpolation (Polynom-Interpolation nach Lagrange und Newton, Spline-Interpolation) behandelt.\newline
Ein weiterer Teil der Vorlesung beschäftigt sich mit dem stationären Verhalten von Messsystemen. Dazu wird zunächst die in den meisten Messsystemen verwendete ideale Kennlinie eingeführt und dadurch entstehende Kennlinienfehler betrachtet. Anschließend werden Konzepte zur Verringerung dieser Kennlinienfehler vorgeführt, zum einen unter spezifizierten Normalbedingungen zum anderen bei Abweichung davon.\newline
Um auch zufällige Messfehler betrachten zu können, werden kurz die wichtigsten Grundlagen der Wahrscheinlichkeitstheorie wiederholt. Als neues Mittel, um Aussagen über die i.A. unbekannten Wahrscheinlichkeitsdichten der betrachteten Größen zu erhalten, werden Stichproben eingeführt. Des Weiteren werden mit Parameter- und Anpassungstests statistische Testverfahren vorgestellt, mit denen erhaltene Vermutungen über die gesuchten Dichten be-/widerlegen lassen.\newline
Als weiteres mächtiges Werkzeug der Messtechnik wird die Korrelationsmesstechnik behandelt. Als hierzu nötige Grundlagen werden stochastische Prozesse knapp wiederholt und darauf aufbauend Anwendungen aus den Bereichen der Laufzeit- und Dopplermessung vorgestellt. Mithilfe des Leistungsdichtespektrums als Fourier-Transformierte der Korrelationsfunktion werden Möglichkeiten zur Systemidentifikation aufgezeigt und das Wiener-Filter als Optimalfilter zur Signalrekonstruktion vorgestellt.\newline
Da reale Messwerte heutzutage fast ausschließlich in Digitalrechnern verarbeitet werden, werden auch die Fehler, die bei der analog/digital Wandlung entstehen, sowohl im Zeit- als auch Amplitudenbereich näher beleuchtet. Hierbei werden sowohl Abtast- und Quantisierungstheorem sowie Verfahren um diese zu erfüllen (Anti-Aliasing Filter, Dithering), als auch einige der gängigsten A/D- und D/A-Umsetzungsprinzipien vorgestellt.\newline
Übungen\newline
Begleitend zum Vorlesungsstoff werden Übungsaufgaben und die zugehörigen Lösungen ausgegeben und in Hörsaalübungen besprochen. Weiterhin werden auf der Übungshomepage Weblearning-Aufgaben angeboten, bei denen die Studenten selbständig ihr Verständnis von Zusammenhängen zwischen Zeit- und Frequenzbereich sowie Zeitsignal und AKF bzw. LDS testen können.


\end{content}

\begin{media}Vorlesungsfolien

\end{media}

\begin{literature}U. Kiencke, R. Eger: Messtechnik, 7. überarbeitete Auflage; Springer, 2008.

\textbf{Weiterführende Literatur:}

G. Lebelt, F. Puente León: Übungsaufgaben zur Messtechnik und Sensorik; Shaker, 2008.

\end{literature}



\end{course}