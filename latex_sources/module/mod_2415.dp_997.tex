% Modulbeschreibung 
% Informationsgrad : extern
% Sprache: de
\begin{module}

\setdoclanguagegerman
\moduledegreeprogramme{Informatik (B.Sc.)}
\modulesubject{}
\moduleID{IN1MATHANA}
\modulename{Analysis}
\modulecoordination{R. Schnaubelt, Plum, Reichel, Weis}

\documentdate{2008-07-25 10:04:28}

\modulecredits{18}
\moduleduration{2}
\modulecycle{Jedes 2. Semester, Wintersemester}



\modulehead

% For index (key word@display). Key word is used for sorting - no Umlauts please.
\index{Analysis@Analysis (M)}

% For later referencing
\label{mod_2415.dp_997}

\begin{courselist}
01001 & Analysis 1 (S.~\pageref{cour_7027.dp_997}) & 4/2/2 & W & 9 & G. Herzog, M. Plum, W. Reichel, C. Schmoeger, R. Schnaubelt, L. Weis\\
01501 & Analysis 2 (S.~\pageref{cour_7029.dp_997}) & 4/2/2 & S & 9 & R. Schnaubelt, Plum, Reichel, Weis\\
\end{courselist}

\begin{styleenv}
\begin{assessment}
Die Erfolgskontrolle erfolgt in Form einer schriftlichen Gesamtprüfung am Ende des Moduls nach § 4 Abs. 2 Nr. 1 und einer Erfolgskontrolle anderer Art nach § 4 Abs. 2 Nr. 3 SPO (mindestens ein Übungsschein aus den Lehrveranstaltungen \emph{Analysis 1} [1001] oder \emph{Analysis 2 }[1501] ).

 

Die Modulnote ist die Note der schriftlichen Prüfung.

 

\textbf{Achtung:} Diese Prüfung oder die Prüfung zum Modul \emph{Höhere Mathematik} [IN1MATHHM] oder zum Modul \emph{Lineare Algebra} [IN1MATHLA] oder zum Modul \emph{Lineare Algebra und Analytische Geometrie} [IN1MATHLAAG] ist bis zum Ende des 2. Fachsemesters anzutreten und bis zum Ende des 3. Fachsemesters zu bestehen, da sie Bestandsteil der Orientierungsprüfung nach § 8 Abs. 1 SPO ist.


\end{assessment}

\begin{conditions}Keine.\end{conditions}


\end{styleenv}

\begin{learningoutcomes}
Die Studierenden sollen am Ende des Moduls

 \begin{itemize}\item den Übergang von der Schule zur Universität bewältigt haben,  \item mit logischem Denken und strengen Beweisen vertraut sein,  \item die Grundlagen der Differential- und Integralrechnung von Funktionen einer reellen Variablen und der Differentialrechnung von Funktionen in mehreren Variablen beherrschen.  \end{itemize}
\end{learningoutcomes}

\begin{content}
Vollständige Induktion, reelle und komplexe Zahlen, Konvergenz, Vollständigkeit, Zahlenreihen, Potenzreihen, elementare Funktionen. Stetigkeit reeller Funktionen, Satz vom Maximum, Zwischenwertsatz. Differentiation reeller Funktionen, Mittelwertsatz, Regel von L'Hospital, Monotonie, Extrema, Konvexität, Satz von Taylor, Newton Verfahren, Differentiation von Reihen. Integration reeller Funktionen: Riemannintegral, Hauptsatz der Differential- und Integralrechnung, Integrationsmethoden, numerische Integration, uneigentliches Integral. \newline
Konvergenz von Funktionenfolgen- und reihen. Normierte Vektorräume und topologische Grundbegriffe, Fixpunktsatz von Banach. Mehrdimensionale Differentiation (lineare Approximation, partielle Ableitungen, Satz von Schwarz), Satz von Taylor, Umkehrsatz, implizit definierte Funktionen, Extrema ohne/mit Nebenbedingungen. Kurvenintegral, Wegunabhängigkeit. Iterierte Riemannintegrale, Volumenberechnung. Einführung in gewöhnliche Differentialgleichungen: Trennung der Variablen, Satz von Picard und Lindelöf, Systeme linearer Differentialgleichungen und ihre Stabilität.


\end{content}

\begin{remarks}Moduldauer: 2 Semester

\end{remarks}

\end{module}

