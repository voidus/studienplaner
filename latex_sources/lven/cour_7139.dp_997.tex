% Lehrveranstaltungsbeschreibung
% Informationsgrad : extern
% Sprache: de
\begin{course}

\setdoclanguagegerman
\coursedegreeprogramme{Informatik}
\coursemodulename{Schlüsselqualifikationen (S.~\pageref{mod_2523.dp_997})[IN1HOCSQ]}
\courseID{PLV}
\coursename{Praxis des Lösungsvertriebs}
\coursecoordination{K. Böhm, Hellriegel}

\documentdate{2010-10-08 10:23:46.243852}

\courselevel{4}
\coursecredits{1}
\courseterm{Sommersemester}
\coursehours{2}
\courseinstructionlanguage{de}

\coursehead

% For index (key word@display). Key word is used for sorting - no Umlauts please.
\index{Praxis des Loesungsvertriebs@Praxis des Lösungsvertriebs}

% For later referencing
\label{cour_7139.dp_997}


\begin{styleenv}
\begin{assessment}
Die Erfolgskontrolle erfolgt in Form einer “Erfolgskontrolle anderer Art” und besteht aus mehreren Teilaufgaben (s. § 4 Abs. 2 Nr. 3 SPO). Dazu gehören Gruppenarbeit und Rollenspiel, wobei die Teilnehmer wiederkehrend Ausarbeitungen anfertigen und vortragen müssen und teilweise auch Rollen spielen, wie z.B. Account Manager, Vertriebsleiter und Projekt Manager.

 

Die Veranstaltung wird mit “bestanden” oder “nicht bestanden” bewertet (§ 7 Abs. 3 SPO). Zum Bestehen der Veranstaltung müssen alle Teilaufgaben erfolgreich bestanden werden.


\end{assessment}

\begin{conditions}Es müssen weitere Vorlesungen/Seminare im Umfang von insgesamt mindestens 5 LP aus dem Gebiet der Informationssysteme am Lehrstuhl Prof. Böhm gehört worden sein bzw. im selben Semester belegt werden. Dazu zählt nicht die Veranstaltung \emph{Datenbanksysteme} [24516].

\end{conditions}

\begin{recommendations}Absolvierte Praktika mit Kundenbezug, z.B. Kundenberatung und Kundenunterstützung sind hilfreich.

\end{recommendations}
\end{styleenv}

\begin{learningoutcomes}
Am Ende der Lehrveranstaltung sollen die Teilnehmer

 \begin{enumerate}\item Wissen und Verständnis für den Lösungs-Vertriebsprozess entwickelt haben,  \item Wissen und Verständnis für typische Rollen und Aufgaben erworben haben und  \item Praxis- und Anwendungsbezug durch die Bearbeitung einer ausführlichen Fallstudie und Rollenspiele gewonnen haben.  \end{enumerate}
\end{learningoutcomes}

\begin{content}
Eine der Schlüsselqualifikationen für alle kundennahen Aktivitäten in Lösungsgeschäften stellt nicht nur für Vertriebsmitarbeiter sondern auch für kundennah arbeitende Berater, Projektleiter und Entwickler das Verständnis und Grundfähigkeiten des Lösungsvertriebs dar.\newline
Nach einem kurzen Überblick über unterschiedliche Geschäftsarten und den daraus resultierenden Anforderungen an Marketing und Vertrieb im Allgemeinen wird speziell der Lösungsvertriebsprozess behandelt.\newline
\newline
Die Themenblöcke sind wie folgt gegliedert:

 \begin{enumerate}\item Den Markt verstehen: welche Informationen über Kunden- und Anbietermärkte sollten eingeholt werden und wo finde ich diese Informationen.  \item Den Kunden kennen: was über den Kunden und wen beim Kunden sollte die Anbieterseite kennen – bis hin zur Frage, mit welchen “Typen” hat man es zu tun.  \item Den Vertriebsprozess planen: Verkaufen ist ein Prozess mit Phasen, Meilensteinen und präzise beschreibbaren Zwischen-Ergebnissen.  \item Das Vertriebsteam gestalten: Lösungen werden von Teams bestehend aus unterschiedlich spezialisierten ‚Spielern‘ erarbeitet und verkauft – wie spielt man dieses Spiel?  \item Die Lösung positionieren: natürlich ist auch eine wettbewerbsfähige Lösung, technisch wie kommerziell, zu erarbeiten.  \item Den Vertrag schließen: worauf es ganz zum Schluss ankommt: die letzte Überzeugungsarbeit.  \end{enumerate}

Auf Basis einer aus der Realität stammenden Fallstudie haben die Studierenden die Gelegenheit in Gruppenarbeiten und Rollenspielen das Gehörte zu reflektieren und zu üben und so ersten Realitätsbezug herzustellen. Angereichert wird der Stoff durch viele Beispiele aus der Praxis.


\end{content}

\begin{media}Präsentation, Fallstudien- und Gruppenarbeitsmaterial.

\end{media}

\begin{literature}\textbf{Weiterführende Literatur:}

 

Reiner Czichos: Creaktives Account-Management.

\end{literature}

\begin{remarks}Die Plätze sind begrenzt und die Anmeldung findet durch das Sekretariat Prof. Böhm statt.

 

Die Veranstaltung findet planmäßig alle drei Semester statt. Das nächste mal voraussichtlich im Wintersemester 2010/2011.

\end{remarks}

\end{course}