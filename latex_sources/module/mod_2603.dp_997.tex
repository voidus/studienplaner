% Modulbeschreibung 
% Informationsgrad : extern
% Sprache: de
\begin{module}

\setdoclanguagegerman
\moduledegreeprogramme{Informatik (B.Sc.)}
\modulesubject{}
\moduleID{IN3INNAP}
\modulename{Netzsicherheit: Architekturen und Protokolle}
\modulecoordination{M. Zitterbart}

\documentdate{2010-06-07 10:03:13.304031}

\modulecredits{4}
\moduleduration{1}
\modulecycle{Jedes 2. Semester, Sommersemester}



\modulehead

% For index (key word@display). Key word is used for sorting - no Umlauts please.
\index{Netzsicherheit: Architekturen und Protokolle@Netzsicherheit: Architekturen und Protokolle (M)}

% For later referencing
\label{mod_2603.dp_997}

\begin{courselist}
24601 & Netzsicherheit: Architekturen und Protokolle (S.~\pageref{cour_5339.dp_997}) & 2/0 & S & 4 & M. Schöller\\
\end{courselist}

\begin{styleenv}
\begin{assessment}
Die Erfolgskontrolle erfolgt in Form einer mündlichen Prüfung im Umfang von i.d.R. 20 Minuten nach § 4 Abs. 2 Nr. 2 SPO.

 

Die Modulnote ist die Note der mündlichen Prüfung.


\end{assessment}

\begin{conditions}Inhalte der Vorlesung \emph{Einführung in Rechnernetze} [24519] (Teil des Pflichtmoduls \emph{Kommunikation und Datenhaltung} [IN3INKD]) und des Stammmoduls \emph{Telematik }[IN3INTM] werden vorausgesetzt.

 

Das Stammmodul Telematik muss belegt und geprüft werden.

\end{conditions}


\end{styleenv}

\begin{learningoutcomes}
Ziel der Vorlesung ist es, die Studenten mit Grundlagen des Entwurfs sicherer Kommunikationsprotokolle vertraut zu machen und Ihnen Kenntnisse bestehender Sicherheitsprotokolle, wie sie im Internet und in lokalen Netzen verwendet werden, zu vermitteln.


\end{learningoutcomes}

\begin{content}
Die Vorlesung „Netzsicherheit: Architekturen und Protokolle“ beginnt mit einem Überblick über die Herausforderungen, die sich beim Entwurf sicherer Kommunikationsprotokolle stellen. Im Anschluss wird zunächst das Kerberos-Verfahren betrachtet, das für Aufgaben der Authentisierung und Autorisierung herangezogen werden kann. Während hier noch auf asymmetrische Kryptographieverfahren verzichtet werden kann, gilt dies für zahlreiche andere Sicherheitsprotokolle nicht. Deshalb wird eine Einführung in die praktische Verwendung solcher Verfahren – Public Key Infrastructure und Privilege Management Infrastructure – gegeben, bevor konkrete Protokolle vorgestellt werden. Im Einzelnen handelt es sich dabei um X.509 und PGP, E-Mail-Sicherheit mit S/MIME, Sicherheit auf der Vermittlungsschicht (IPsec), auf der Transportschicht (SSL/TLS) und den Schutz von Infrastrukturen im Netz. Die Vorlesung schließt mit dem immer mehr an Bedeutung gewinnenden Thema des technischen Datenschutzes, Anonymität und Privatsphäre in Netzen.


\end{content}



\end{module}

