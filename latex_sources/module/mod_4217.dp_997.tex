% Modulbeschreibung 
% Informationsgrad : extern
% Sprache: de
\begin{module}

\setdoclanguagegerman
\moduledegreeprogramme{Informatik (B.Sc.)}
\modulesubject{}
\moduleID{IN3INKM}
\modulename{Kognitive Modellierung}
\modulecoordination{T. Schultz}

\documentdate{2012-01-20 12:29:41.254417}

\modulecredits{3}
\moduleduration{1}
\modulecycle{Jedes 2. Semester, Sommersemester}



\modulehead

% For index (key word@display). Key word is used for sorting - no Umlauts please.
\index{Kognitive Modellierung@Kognitive Modellierung (M)}

% For later referencing
\label{mod_4217.dp_997}

\begin{courselist}
24612 & Kognitive Modellierung (S.~\pageref{cour_8513.dp_997}) & 2 & S & 3 & T. Schultz, F. Putze\\
\end{courselist}

\begin{styleenv}
\begin{assessment}
Es wird 6 Wochen im Voraus angekündigt, ob die Erfolgskontrolle in Form einer schriftlichen Prüfung (Klausur) im Umfang von i.d.R. 1h nach § 4 Abs. 2 Nr. 1 SPO oder in Form einer mündlichen Prüfung im Umfang von i.d.R. 15 min. nach § 4 Abs. 2 Nr. 2 SPO stattfinden wird.

 

Die Modulnote entspricht dieser Note.

 

Terminvereinbarung bitte per E-Mail an: helga.scherer@kit.edu. Es wird empfohlen, sich frühzeitig um einen Prüfungstermin zu kümmern.


\end{assessment}

\begin{conditions}Keine.\end{conditions}

\begin{recommendations}Kenntnisse im Bereich der Kognitiven Systeme oder Biosignale sind hilfreich.

\end{recommendations}
\end{styleenv}

\begin{learningoutcomes}
Die Studierenden haben einen breiten Überblick über die Methoden zur Modellierung menschlicher Kognition und menschlichen Affekts im Kontext der Mensch-Maschine-Interaktion. Sie sind in der Lage, menschliches Verhalten anwendungsspezifisch zu modellieren, um z.B. realistische virtuelle Umgebungen zu simulieren oder eine natürliche Interaktion zwischen Benutzer und Maschine zu ermöglichen.


\end{learningoutcomes}

\begin{content}
Kognition und menschlichen Affekts im Kontext der Mensch-Maschine-Interaktion. Es werden Modelle thematisiert, die von Computersystemen genutzt werden können, um menschliches Verhalten zu beschreiben, zu erklären, und vorherzusagen. \newline
\newline
 Wichtige Inhalte der Lehrveranstaltung sind Modelle menschlichen Verhaltens, menschliches Lernen (Zusammenhang und Unterschiede zu maschinellen Lernverfahren), Repräsentation von Wissen, Emotionsmodelle, und kognitive Architekturen. Es wird die Relevanz kognitiver Modellierungen für zukünftige Computersysteme aufgezeigt und insbesondere auf die relevanten Fragestellungen der aktuellen Forschung im Bereich der Mensch-Maschine-Interaktion eingegangen


\end{content}



\end{module}

