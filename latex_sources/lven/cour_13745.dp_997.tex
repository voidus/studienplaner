% Lehrveranstaltungsbeschreibung
% Informationsgrad : extern
% Sprache: de
\begin{course}

\setdoclanguagegerman
\coursedegreeprogramme{Informatik}
\coursemodulename{Proseminar (S.~\pageref{mod_2385.dp_997})[IN2INPROSEM]}
\courseID{ProsemAT}
\coursename{Proseminar: Algorithmen-Theorie}
\coursecoordination{D. Wagner}

\documentdate{2012-01-05 12:40:18.099732}

\courselevel{2}
\coursecredits{3}
\courseterm{Winter-/Sommersemester}
\coursehours{2}
\courseinstructionlanguage{de}

\coursehead

% For index (key word@display). Key word is used for sorting - no Umlauts please.
\index{Proseminar: Algorithmen-Theorie@Proseminar: Algorithmen-Theorie}

% For later referencing
\label{cour_13745.dp_997}


\begin{styleenv}
\begin{assessment}
Die Erfolgskontrolle erfolgt durch Ausarbeiten einer schriftlichen Proseminararbeit sowie der Präsentation derselbigen als Erfolgskontrolle anderer Art nach § 4 Abs. 2 Nr. 3 der SPO. Die Gesamtnote setzt sich aus den benoteten und gewichteten Erfolgskontrollen (i.d.R. Seminararbeit 50 \%, Präsentation 50\%) zusammen.


\end{assessment}

\begin{conditions}Keine.\end{conditions}

\begin{recommendations}Je nach Thema sind Kenntnisse aus den Vorlesungen „Theoretische Grundlagen der Informatik“ bzw. „Algorithmen I“ erforderlich.

\end{recommendations}
\end{styleenv}

\begin{learningoutcomes}
Ziel der Lehrveranstaltung ist es, die Studierenden mit aktuellen Konzepten aus der theoretischen Informatik sowie der Algorithmik vertraut zu machen und damit die in den Vorlesungen „Theoretische Grundlagen der Informatik“ bzw. „Algorithmen I“ erworbenen Kenntnisse zu vertiefen. Die Studierenden sollen für die Problemstellungen in diesen Gebieten sensibilisiert werden und neue Lösungsansätze kennenlernen.

 

Die Studierenden erschließen sich im Rahmen des Seminars ein komplexes Thema in selbständiger Arbeit. Dazu gehört die Erarbeitung und Präsentation eines anschaulichen Vortrags sowie eine Zusammenfassung der erworbenen Kenntnisse im Rahmen einer Ausarbeitung.


\end{learningoutcomes}

\begin{content}
Das Proseminar vertieft im Anschluss an die Vorlesungen „Theoretische Grundlagen der Informatik“ sowie „Algorithmen I“ das in diesen Veranstaltungen erworbene Wissen um neue Konzepte und Lösungen anhand aktueller Publikationen aus den jeweiligen Bereichen.

 

Im Anschluss an die Vorlesung „Theoretische Grundlagen der Informatik“ werden unter anderem neue Ansätze zur Lösung des P-NP-Problems thematisiert.

 

Im Anschluss an die Vorlesung „Algorithmen I“ liegt der Fokus auf einer Vertiefung der algorithmischen Kenntnisse der Studierenden.


\end{content}

\begin{media}Tafel, Folien

\end{media}



\begin{remarks}\textcolor{red}{Dieses Seminar wird unregelmäßig angeboten.}

\end{remarks}

\end{course}