% Modulbeschreibung 
% Informationsgrad : extern
% Sprache: de
\begin{module}

\setdoclanguagegerman
\moduledegreeprogramme{Informatik (B.Sc.)}
\modulesubject{EF Informationsmanagement im Ingenieurwesen}
\moduleID{IN3MACHPLMF}
\modulename{Product Lifecycle Management in der Fertigungsindustrie}
\modulecoordination{Maier}

\documentdate{2011-07-20 10:05:45.059954}

\modulecredits{4}
\moduleduration{1}
\modulecycle{Jedes 2. Semester, Wintersemester}



\modulehead

% For index (key word@display). Key word is used for sorting - no Umlauts please.
\index{Product Lifecycle Management in der Fertigungsindustrie@Product Lifecycle Management in der Fertigungsindustrie (M)}

% For later referencing
\label{mod_4281.dp_997}

\begin{courselist}
2121366 & Product Lifecycle Management in der Fertigungsindustrie (S.~\pageref{cour_7499.dp_997}) & 2/0 & W & 4 & G. Meier\\
\end{courselist}

\begin{styleenv}
\begin{assessment}
Die Erfolgskontrolle erfolgt in Form einer mündlichen Prüfung im Umfang von 30 Minuten (nach § 4 (2), 2 SPO).

 

Die Note entspricht der Note der mündlichen Prüfung.


\end{assessment}

\begin{conditions}Die Module \textbf{\emph{Product Lifecycle Managemet}} [IN3MACHPLM] und \textbf{\emph{Technische Informationssysteme}} [IN3INMACHTI] müssen im Ergänzungsfach Informationsmanagement im Ingenieurwesen geprüft werden.

\end{conditions}


\end{styleenv}

\begin{learningoutcomes}
Der/ die Studierende

 \begin{itemize}\item versteht den technischen und organisatorischen Ablauf eines PLM-Projekts,  \item besitzt grundlegende Kenntnisse über die Einführung eines PLM-Systems in einem Unternehmen.   \end{itemize}
\end{learningoutcomes}

\begin{content}
Die Vorlesung stellt den PLM-Prozess allgemein und konkret am Beispiel der Heidelberger Druckmaschinen vor. Es werden der technische und organisatorische Ablauf eines PLM-Projekts sowie Themen wie Mitarbeitermotivation und Wirtschaftlichkeit vermittelt. Ein weiteres Thema ist die Einführung eines PLM-Systems als Projekt (Strategie, Herstellerauswahl, Barrieren gegen PLM, PLM und Psychologie).


\end{content}



\end{module}

