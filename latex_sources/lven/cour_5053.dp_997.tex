% Lehrveranstaltungsbeschreibung
% Informationsgrad : extern
% Sprache: de
\begin{course}

\setdoclanguagegerman
\coursedegreeprogramme{Informatik}
\coursemodulename{Vernetzte IT-Infrastrukturen (S.~\pageref{mod_4245.dp_997})[IN3INITIS]}
\courseID{VITI}
\coursename{Vernetzte IT-Infrastrukturen}
\coursecoordination{B. Neumair}

\documentdate{2011-08-29 15:25:37.040662}

\courselevel{4}
\coursecredits{5}
\courseterm{Wintersemester}
\coursehours{2/1}
\courseinstructionlanguage{de}

\coursehead

% For index (key word@display). Key word is used for sorting - no Umlauts please.
\index{Vernetzte IT-Infrastrukturen@Vernetzte IT-Infrastrukturen}

% For later referencing
\label{cour_5053.dp_997}


\begin{styleenv}
\begin{assessment}

\end{assessment}

\begin{conditions}Abhängigkeiten entsprechen der Modulbeschreibung.

\end{conditions}


\end{styleenv}

\begin{learningoutcomes}
Die Vorlesung behandelt die grundlegenden Modelle, Verfahren und Technologien, die heutzutage im Bereich der digitalen Telekommunikation zum Einsatz kommen. Fundament aller behandelten Themen ist dabei das sogenannte ISO/OSI-Basisreferenzmodell, ein allgemein akzeptiertes Schema zur schichtweisen Modellierung und Beschreibung von Kommunikationssystemen.


\end{learningoutcomes}

\begin{content}
Nach einer einleitenden Vorstellung verschiedener formaler Beschreibungsmethodiken sind auch die wesentlichen physikalischen Grundlagen im Bereich der Signalverarbeitung Bestandteil der Vorlesung. Anhand klassischer Netztechnologien wie Ethernet und Token Ring werden zudem verschiedene elementare Verfahren zur Realisierung des Medienzugriffs bzw. zur Gewährleitung einer gesicherten übertragung behandelt. Die Verknüpfung einzelner Rechner zu einem weltumspannenden Netzwerk und die dabei auftretenden Fragestellungen im Bereich der Wegewahl (Routing) werden anhand der im Internet im Einsatz befindlichen Protokolle ebenso vertieft wie die Bereitstellung eines zuverlässigen Datentransports zwischen den Teilnehmern. Darüber hinaus werden die Funktionsweise moderner Komponenten zur effizienten Netzkopplung sowie grundlegende Mechanismen im Bereich Netzsicherheit erläutert. Eine Beschreibung der Technik und der Dienste des Integrated Services Digital Network (ISDN) sowie die Vorstellung verschiedener anwendungsnaher Protokolle, wie z.B. des HyperText Transfer Protocols (HTTP), bilden den Abschluss der Vorlesung.


\end{content}

\begin{media}Folien.

\end{media}

\begin{literature}\begin{itemize}\item A.S. Tanenbaum, Computer Networks Prentice Hall, 4. Auflage, ISBN 0130661023, 2002.  \item Larry L. Peterson, Bruce S. Davie, Computer Networks - A Systems Approach, 3rd ed., Morgan Kaufmann Publishers, 2003.  \end{itemize}

\textbf{Weiterführende Literatur:}

 \begin{itemize}\item F. Halsall, Data Communications, Computer Networks and OSI, Addison-Wesley, 4. Auflage, ISBN 0-201-18244-0, 1997.  \item J.F. Kurose, K.W. Ross, Computer Networking - A Top-Down Approach featuring the Internet. Addison-Wesley, 2005.  \end{itemize}\end{literature}

\begin{remarks}\textcolor{red}{Diese LV wurde letztmalig im Wintersemester 2010/11 angeboten. Prüfungen sind möglich bis SS 2012 möglich.}

\end{remarks}

\end{course}