% Lehrveranstaltungsbeschreibung
% Informationsgrad : extern
% Sprache: de
\begin{course}

\setdoclanguagegerman
\coursedegreeprogramme{Informatik}
\coursemodulename{Praktikum Automation und Information (S.~\pageref{mod_3943.dp_997})[IN3EITPAI]}
\courseID{23169}
\coursename{Praktikum Automation und Information}
\coursecoordination{F. Puente, G.F. Trommer }

\documentdate{2011-12-06 14:35:36.318740}

\courselevel{3}
\coursecredits{6}
\courseterm{Sommersemester}
\coursehours{0/4}
\courseinstructionlanguage{de}

\coursehead

% For index (key word@display). Key word is used for sorting - no Umlauts please.
\index{Praktikum Automation und Information@Praktikum Automation und Information}

% For later referencing
\label{cour_8059.dp_997}


\begin{styleenv}
\begin{assessment}

\end{assessment}

\begin{conditions}Der erfolgreiche Besuch des Moduls “Systemtheorie” (IN3EITST) wird vorausgesetzt.

\end{conditions}


\end{styleenv}

\begin{learningoutcomes}
Im Praktikum \textbf{Automation und Information} werden einige grundlegende Verfahren der Automatisierungs- und Informationstechnik behandelt und von den Studierenden selbst erprobt. Das Spektrum umfasst neben Informationstechnischen Inhalten wie Datenerfassung, Messtechnik und Bildverarbeitung auch Automatisierungsaspekte wie die Identifikation, Regelung und Optimierung technischer Laboraufbauten.


\end{learningoutcomes}

\begin{content}
Die einzelnen Versuche und der Ablauf werden vor Beginn des Praktikums auf den Internetseiten des Instituts für Regelungs- und Steuerungssysteme (IRS) bekanntgegeben (http://www.irs.uni-karlsruhe.de/1430.php).


\end{content}

\begin{media}Versuchbeschreibungen

\end{media}



\begin{remarks}\textcolor{red}{Diese Lehrveranstaltung wird nicht mehr angeboten.}

\end{remarks}

\end{course}