% Lehrveranstaltungsbeschreibung
% Informationsgrad : extern
% Sprache: de
\begin{course}

\setdoclanguagegerman
\coursedegreeprogramme{Informatik}
\coursemodulename{Entwurf und Architekturen für Eingebettete Systeme (ES2) (S.~\pageref{mod_2591.dp_997})[IN3INES2]}
\courseID{24106}
\coursename{Entwurf und Architekturen für Eingebettete Systeme (ES2)}
\coursecoordination{J. Henkel}

\documentdate{2011-02-24 11:49:40.694048}

\courselevel{4}
\coursecredits{3}
\courseterm{Wintersemester}
\coursehours{2}
\courseinstructionlanguage{de}

\coursehead

% For index (key word@display). Key word is used for sorting - no Umlauts please.
\index{Entwurf und Architekturen fuer Eingebettete Systeme (ES2)@Entwurf und Architekturen für Eingebettete Systeme (ES2)}

% For later referencing
\label{cour_7229.dp_997}


\begin{styleenv}
\begin{assessment}
Die Erfolgskontrolle wird in der Modulbeschreibung näher erläutert.


\end{assessment}

\begin{conditions}Die Voraussetzungen werden in der Modulbeschreibung näher erläutert.

\end{conditions}


\end{styleenv}

\begin{learningoutcomes}
Erlernen von Methoden zur Beherrschung von Komplexität. \newline
Anwendung dieser Methoden auf den Entwurf eingebetteter Systeme.\newline
Beurteilung und Auswahl spezifischer Architekturen für Eingebettete Systeme.\newline
Zugang zu aktuellen Forschungsthemen erschließen.


\end{learningoutcomes}

\begin{content}
Heutzutage ist es möglich, mehrere Milliarden Transistoren auf einem einzigen Chip zu integrieren und damit komplette SoCs (Systems-On-Chip) zu realisieren. Der Trend, mehr und mehr Transistoren verwenden zu können, hält ungebremst an, so dass die Komplexität solcher Systeme ebenfalls immer weiter zulegen wird. Computer werden vermehrt ubiquitär sein, [... unverändert weiter ab hier]


\end{content}

\begin{media}Vorlesungsfolien

\end{media}





\end{course}