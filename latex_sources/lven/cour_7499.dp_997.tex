% Lehrveranstaltungsbeschreibung
% Informationsgrad : extern
% Sprache: de
\begin{course}

\setdoclanguagegerman
\coursedegreeprogramme{Informatik}
\coursemodulename{Product Lifecycle Management in der Fertigungsindustrie (S.~\pageref{mod_4281.dp_997})[IN3MACHPLMF]}
\courseID{2121366}
\coursename{Product Lifecycle Management in der Fertigungsindustrie}
\coursecoordination{G. Meier}

\documentdate{2012-01-17 12:10:54.253795}

\courselevel{4}
\coursecredits{4}
\courseterm{Wintersemester}
\coursehours{2/0}
\courseinstructionlanguage{de}

\coursehead

% For index (key word@display). Key word is used for sorting - no Umlauts please.
\index{Product Lifecycle Management in der Fertigungsindustrie@Product Lifecycle Management in der Fertigungsindustrie}

% For later referencing
\label{cour_7499.dp_997}


\begin{styleenv}
\begin{assessment}
Die Erfolgskontrolle erfolgt in Form einer mündlichen Prüfung im Umfang von 30 Minuten (nach § 4 (2), 2 SPO).

 

Die Note entspricht der Note der mündlichen Prüfung.


\end{assessment}

\begin{conditions}Keine.\end{conditions}

\begin{recommendations}Der vorherige Besuch der Veranstaltung \emph{Product Lifecycle Managemen}t [2121350] wird empfohlen.

\end{recommendations}
\end{styleenv}

\begin{learningoutcomes}
Der/ die Studierende

 \begin{itemize}\item versteht den technischen und organisatorischen Ablauf eines PLM-Projekts,  \item besitzt grundlegende Kenntnisse über die Einführung eines PLM-Systems in einem Unternehmen.  \end{itemize}
\end{learningoutcomes}

\begin{content}
Die Vorlesung stellt den PLM-Prozess allgemein und konkret am Beispiel der Heidelberger Druckmaschinen vor. Es werden der technische und organisatorische Ablauf eines PLM-Projekts sowie Themen wie Mitarbeitermotivation und Wirtschaftlichkeit vermittelt. Ein weiteres Thema ist die Einführung eines PLM-Systems als Projekt (Strategie, Herstellerauswahl, Barrieren gegen PLM, PLM und Psychologie).


\end{content}

\begin{media}Skript zur Veranstaltung, wird in der Vorlesung verteilt.

\end{media}





\end{course}