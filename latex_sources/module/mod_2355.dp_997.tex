% Modulbeschreibung 
% Informationsgrad : extern
% Sprache: de
\begin{module}

\setdoclanguagegerman
\moduledegreeprogramme{Informatik (B.Sc.)}
\modulesubject{Theoretische Informatik}
\moduleID{IN1INGI}
\modulename{Grundbegriffe der Informatik}
\modulecoordination{T. Schultz}

\documentdate{2010-09-17 10:36:43.615823}

\modulecredits{4}
\moduleduration{1}
\modulecycle{Jedes 2. Semester, Wintersemester}



\modulehead

% For index (key word@display). Key word is used for sorting - no Umlauts please.
\index{Grundbegriffe der Informatik@Grundbegriffe der Informatik (M)}

% For later referencing
\label{mod_2355.dp_997}

\begin{courselist}
24001 & Grundbegriffe der Informatik (S.~\pageref{cour_6187.dp_997}) & 2/1/2 & W & 4 & T. Schultz\\
\end{courselist}

\begin{styleenv}
\begin{assessment}
Für den erfolgreichen Abschluss dieses Moduls ist das Bestehen eines Übungsscheins (Erfolgskontrolle anderer Art nach § 4 Abs. 2 Nr. 3 SPO) sowie das Bestehen der Klausur (schriftliche Prüfung nach § 4 Abs. 2 Nr. 1 SPO) erforderlich. Der Umfang der Klausur beträgt zwei Stunden.

 

Die Modulnote ist die Note der Klausur.

 

Achtung: Dieses Modul ist Bestandteil der Orientierungsprüfung gemäß § 8 Abs. 1 SPO. Die Prüfung ist bis zum Ende des 2. Fachsemesters anzutreten und bis zum Ende des 3. Fachsemesters zu bestehen.


\end{assessment}

\begin{conditions}Keine.\end{conditions}


\end{styleenv}

\begin{learningoutcomes}
\begin{itemize}\item Die Studierenden kennen grundlegende Definitionsmethoden und sind in der Lage, entsprechende Definitionen zu lesen und zu verstehen.  \item Sie kennen den Unterschied zwischen Syntax und Semantik.  \item Die Studierenden kennen die grundlegenden Begriffe aus diskreter Mathematik und Informatik und sind in der Lage sie richtig zu benutzen, sowohl bei der Beschreibung von Problemen als auch bei Beweisen.  \end{itemize}
\end{learningoutcomes}

\begin{content}
\begin{itemize}\item Algorithmen informell, Grundlagen des Nachweises ihrer Korrektheit\newline
Berechnungskomplexität, „schwere“ Probleme\newline
O-Notation, Mastertheorem  \item Alphabete, Wörter, formale Sprachen\newline
endliche Akzeptoren, kontextfreie Grammatiken  \item induktive/rekursive Definitionen, vollständige und strukturelle Induktion\newline
Hüllenbildung  \item Relationen und Funktionen  \item Graphen  \end{itemize}
\end{content}



\end{module}

