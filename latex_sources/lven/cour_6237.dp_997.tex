% Lehrveranstaltungsbeschreibung
% Informationsgrad : extern
% Sprache: de
\begin{course}

\setdoclanguagegerman
\coursedegreeprogramme{Informatik}
\coursemodulename{Energiebewusste Systeme (S.~\pageref{mod_10583.dp_997})[IN3INEBS]}
\courseID{24181}
\coursename{Power Management Praktikum}
\coursecoordination{F. Bellosa,  Merkel}

\documentdate{2011-11-14 11:34:21.994789}

\courselevel{4}
\coursecredits{3}
\courseterm{Wintersemester}
\coursehours{2}
\courseinstructionlanguage{de}

\coursehead

% For index (key word@display). Key word is used for sorting - no Umlauts please.
\index{Power Management Praktikum@Power Management Praktikum}

% For later referencing
\label{cour_6237.dp_997}


\begin{styleenv}
\begin{assessment}
Die Erfolgskontrolle wird in der Modulbeschreibung erläutert.


\end{assessment}

\begin{conditions}Die LV kann nur erfolgreich besucht werden, wenn im gleichen Semester die Vorlesung “Power Management” [24127] besucht wird.

\end{conditions}


\end{styleenv}

\begin{learningoutcomes}
Der Student soll die in der Vorlesung Power Management erworbenen Kenntnisse an realen Systemen praktisch anwenden können. Der Student bekommt Einblicke in die Systemprogrammierung und ist in der Lage, selbst Erweiterungen an Betriebssystemen vorzunehmen und zu evaluieren. Der Student kann energiekritische Systeme instrumentieren und ausmessen.


\end{learningoutcomes}

\begin{content}
Themen:

 \begin{itemize}\item Temperaturverwaltung  \item Dynamisch Frequenzanpassung  \item Wahl von Ruhezuständen  \item Energie-gewahre Dateisysteme  \end{itemize}
\end{content}

\begin{media}Präsentationen, Betriebssystemquellen

\end{media}





\end{course}