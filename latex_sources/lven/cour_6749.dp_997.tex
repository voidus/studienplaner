% Lehrveranstaltungsbeschreibung
% Informationsgrad : extern
% Sprache: de
\begin{course}

\setdoclanguagegerman
\coursedegreeprogramme{Informatik}
\coursemodulename{Topics in Finance I (S.~\pageref{mod_1575.dp_997})[IN3WWBWL13]}
\courseID{2530232}
\coursename{Finanzintermediation}
\coursecoordination{M. Ruckes}

\documentdate{2012-01-09 19:07:34.709882}

\courselevel{3}
\coursecredits{4,5}
\courseterm{Wintersemester}
\coursehours{3}
\courseinstructionlanguage{de}

\coursehead

% For index (key word@display). Key word is used for sorting - no Umlauts please.
\index{Finanzintermediation@Finanzintermediation}

% For later referencing
\label{cour_6749.dp_997}


\begin{styleenv}
\begin{assessment}
Die Erfolgskontrolle erfolgt in Form einer schriftlichen Prüfung (60min.) (nach §4(2), 1 SPO).

 

Die Prüfung wird in jedem Semester angeboten und kann zu jedem ordentlichen Prüfungstermin wiederholt werden.


\end{assessment}

\begin{conditions}Keine.\end{conditions}


\end{styleenv}

\begin{learningoutcomes}
Die Studierenden werden in die theoretischen Grundlagen der Finanzintermediation eingeführt.


\end{learningoutcomes}

\begin{content}
\begin{itemize}\item Gründe für die Existenz von Finanzintermediären,  \item Analyse der vertraglichen Beziehungen zwischen Banken und Kreditnehmern,  \item Struktur des Bankenwettbewerbs,  \item Stabilität des Bankensystems,  \item Makroökonomische Rolle der Finanzintermediation.  \end{itemize}
\end{content}



\begin{literature}\textbf{Weiterführende Literatur:}

 \begin{itemize}\item Hartmann-Wendels/Pfingsten/Weber (2006): Bankbetriebslehre, 4. Auflage, Springer Verlag.  \item Freixas/Rochet (1997): Microeconomics of Banking, MIT Press.  \end{itemize}\end{literature}



\end{course}