% Lehrveranstaltungsbeschreibung
% Informationsgrad : extern
% Sprache: de
\begin{course}

\setdoclanguagegerman
\coursedegreeprogramme{Informatik}
\coursemodulename{Kommunikation und Datenhaltung (S.~\pageref{mod_2517.dp_997})[IN2INKD]}
\courseID{24519}
\coursename{Einführung in Rechnernetze}
\coursecoordination{M. Zitterbart}

\documentdate{2011-02-17 09:28:17.346768}

\courselevel{3}
\coursecredits{4}
\courseterm{Sommersemester}
\coursehours{2/1}
\courseinstructionlanguage{de}

\coursehead

% For index (key word@display). Key word is used for sorting - no Umlauts please.
\index{Einfuehrung in Rechnernetze@Einführung in Rechnernetze}

% For later referencing
\label{cour_8645.dp_997}


\begin{styleenv}
\begin{assessment}
Die Erfolgskontrolle wird in der Modulbeschreibung erläutert.


\end{assessment}

\begin{conditions}Keine.\end{conditions}

\begin{recommendations}Der Besuch von Vorlesungen zu Systemarchitektur und Softwaretechnik wird empfohlen, aber nicht vorausgesetzt.

\end{recommendations}
\end{styleenv}

\begin{learningoutcomes}
Der/die Studierende

 \begin{itemize}\item  kennt die Grundlagen der Datenübertragung sowie den Aufbau von Kommunikationssystemen,  \item ist mit der Zusammensetzung von Protokollen aus einzelnen Protokollmechanismen vertraut und konzipiert einfache Protokolle eigenständig,  \item  kennt und versteht das Zusammenspiel einzelner Kommunikationsschichten und Anwendungen.  \end{itemize}
\end{learningoutcomes}

\begin{content}
Das heutige Internet ist wohl das bekannteste und komplexeste Gebilde, das jemals von der Menschheit erschaffen wurde: Hunderte Millionen von vernetzten Computern und Verbindungsnetzwerke. Millionen von Benutzern, die sich zu den unterschiedlichsten Zeiten mittels der unterschiedlichsten Endgeräte mit dem Internet verbinden wie beispielsweise Handys, PDAs oder Laptops. In Anbetracht der enormen Ausmaße und der Vielseitigkeit des Internets stellt sich die Frage, inwieweit es möglich ist zu verstehen, wie die komplexen Strukturen dahinter funktionieren. Die Vorlesung versucht dabei den Einstieg in die Welt der Rechnernetze zu schaffen, indem sie sowohl theoretische als auch praktische Aspekte von Rechnernetzen vermittelt. Behandelt werden Grundlagen der Nachrichtentechnik, fundamentale Protokollmechanismen sowie die Schichtenarchitektur heutiger Rechnernetze. Hierbei werden systematisch sämtliche Schichten beginnend mit dem physikalischen Medium bis hin zur Anwendungsschicht besprochen.


\end{content}

\begin{media}Vorlesungsfolien.

\end{media}

\begin{literature}\begin{itemize}\item  J.F. Kurose, K.W. Ross: Computer Networking - A Top-Down Approach featuring the Internet. Addison-Wesley, 2007.   \item  W. Stallings: Data and Computer Communications. Prentice Hall, 2006.  \end{itemize}

\textbf{Weiterführende Literatur:}

 \begin{itemize}\item  F. Halsall: Computer Networking and the Internet. Addison-Wesley, 2005.  \item P. Lockemann, G. Krüger, H. Krumm: Telekommunikation und Datenhaltung. Hanser Verlag, 1993.   \item  S. Abeck, P.C. Lockemann, J. Schiller, J. Seitz: Verteilte Informationssysteme. dpunkt-Verlag, 2003  \end{itemize}\end{literature}

\begin{remarks}Diese Vorlesung ersetzt den Kommunikationsteil der Vorlesung \emph{Kommunikation und Datenhaltung}.

\end{remarks}

\end{course}