% Modulbeschreibung 
% Informationsgrad : extern
% Sprache: de
\begin{module}

\setdoclanguagegerman
\moduledegreeprogramme{Informatik (B.Sc.)}
\modulesubject{}
\moduleID{IN3INITSS}
\modulename{IT-Sicherheitsmanagement für vernetzte Systeme}
\modulecoordination{H. Hartenstein}

\documentdate{2012-01-24 13:10:42.449096}

\modulecredits{5}
\moduleduration{1}
\modulecycle{Jedes 2. Semester, Wintersemester}



\modulehead

% For index (key word@display). Key word is used for sorting - no Umlauts please.
\index{IT-Sicherheitsmanagement fuer vernetzte Systeme@IT-Sicherheitsmanagement für vernetzte Systeme (M)}

% For later referencing
\label{mod_15853.dp_997}

\begin{courselist}
24149 & IT-Sicherheitsmanagement für vernetzte Systeme (S.~\pageref{cour_5399.dp_997}) & 2/1 & W & 5 & H. Hartenstein\\
\end{courselist}

\begin{styleenv}
\begin{assessment}
Die Erfolgskontrolle erfolgt in Form einer mündlichen Prüfung im Umfang von i.d.R. 20 Minuten nach § 4 Abs. 2 Nr. 2 der SPO.

 

Die Modulnote ist die Note der mündlichen Prüfung.


\end{assessment}

\begin{conditions}Studierende die das Modul \emph{Netzwerk- und IT-Sicherheitsmanagement} [IN3INNITS] geprüft haben, dürfen dieses Modul nicht prüfen.

\end{conditions}

\begin{recommendations}Grundkenntnisse im Bereich Rechnernetze, entsprechend den Vorlesungen Einführung in Rechnernetze bzw. Telematik sind notwendig.

\end{recommendations}
\end{styleenv}

\begin{learningoutcomes}
Ziel der Vorlesung ist es, den Studenten die Grundlagen des IT-Sicherheitsmanagements für vernetzte Systeme zu vermitteln. Es sollen sowohl technische als auch zugrunde liegende Management-Aspekte verdeutlicht werden.


\end{learningoutcomes}

\begin{content}
Die Vorlesung dieses Moduls behandelt das Management moderner, verteilter IT-Systeme und - Dienste. Hierfür werden tragende Konzepte und Modelle in den Bereichen IT-Sicherheitsmanagement, Netzwerkmanagement, Identitätsmanagement und IT-Servicemanagement vorgestellt und diskutiert. Aufbauend werden konkrete technische Architekturen, Protokolle und Werkzeuge innerhalb der genannten Bereiche betrachtet.

 

Unter anderem werden die Konzepte von IT-Sicherheitsprozessen anhand des BSI Grundschutzes verdeutlicht, die Steuerung und Überwachung von hochverteilten Rechnernetzen erörtert und die öffentliche IP-Netzverwaltung betrachtet. Weitere Schwerpunkte bilden das Zugangs- und Identitätsmanagement sowie Firewalls, Intrusion Detection und Prevention. Die Themen werden ferner anhand zahlreicher Fallbeispiele aus dem operativen Betrieb des Steinbuch Centre for Computing (SCC) vertieft, wie zum Beispiel im Kontext des glasfasergebundenen Backbones KITnet. Anhand aktueller Forschungsaktivitäten aus den Bereichen Peer-to-Peer-Netze (z.B. BitTorrent) und soziale Netzwerke (z.B. Facebook) werden die vermittelten Managementansätze in einen globalen Kontext gesetzt.


\end{content}

\begin{remarks}Ersetzt das auslaufende Modul „Netzwerk- und IT-Sicherheitsmanagement“ ab WS 2012/13.

\end{remarks}

\end{module}

