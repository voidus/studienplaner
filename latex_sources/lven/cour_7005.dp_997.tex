% Lehrveranstaltungsbeschreibung
% Informationsgrad : extern
% Sprache: de
\begin{course}

\setdoclanguagegerman
\coursedegreeprogramme{Informatik}
\coursemodulename{Technische Informatik (S.~\pageref{mod_2409.dp_997})[IN1INTI]}
\courseID{24502}
\coursename{Rechnerorganisation}
\coursecoordination{T. Asfour, R. Dillmann, J. Henkel, W. Karl}

\documentdate{2010-08-20 10:36:22.532610}

\courselevel{1}
\coursecredits{6}
\courseterm{Sommersemester}
\coursehours{3/1/2}
\courseinstructionlanguage{de}

\coursehead

% For index (key word@display). Key word is used for sorting - no Umlauts please.
\index{Rechnerorganisation@Rechnerorganisation}

% For later referencing
\label{cour_7005.dp_997}


\begin{styleenv}
\begin{assessment}
Die Erfolgskontrolle wird in der Modulbeschreibung erläutert.


\end{assessment}

\begin{conditions}Die Lehrveranstaltung Rechnerorganisation (Technische Informatik I) kann nur mit der Lehrveranstaltung Digitaltechnik und Entwurfsverfahren (Technische Informatik II) geprüft werden.

\end{conditions}


\end{styleenv}

\begin{learningoutcomes}
Die Studierenden sollen in die Lage versetzt werden,

 \begin{itemize}\item grundlegendes Verständnis über den Aufbau, die Organisation und das Operationsprinzip von Rechnersystemen zu erwerben,  \item den Zusammenhang zwischen Hardware-Konzepten und den Auswirkungen auf die Software zu verstehen, um effiziente Programme erstellen zu können,   \item aus dem Verständnis über die Wechselwirkungen von Technologie, Rechnerkonzepten und Anwendungen die grundlegenden Prinzipien des Entwurfs nachvollziehen und anwenden zu können   \item einen Rechner aus Grundkomponenten aufbauen zu können.  \end{itemize}
\end{learningoutcomes}

\begin{content}
Der Inhalt der Lehrveranstaltung umfasst die Grundlagen des Aufbaus und der Organisation von Rechnern; die Befehlssatzarchitektur verbunden mit der Diskussion RISC – CISC; Pipelining des Maschinenbefehlszyklus, Pipeline-Hemmnisse und Methoden zur Auflösung von Pipeline-Konflikten; Speicherkomponenten, Speicherorganisation, Cache-Speicher; Ein-/Ausgabe-System und Schnittstellenbausteine; Interrupt-Verarbeitung; Bus-Systeme; Unterstützung von Betriebssystemfunktionen: virtuelle Speicherverwaltung, Schutzfunktionen.


\end{content}

\begin{media}Vorlesungsfolien, Aufgabenblätter

\end{media}

\begin{literature}\textbf{Weiterführende Literatur:}

 \begin{itemize}\item D. Patterson, J. Hennessy: Rechnerorganisation und -entwurf; Deutsche Auflage. Herausgegeben von Arndt Bode, Wolfgang Karl und Theo Ungerer, Spektrum Verlag, 2006  \item Th. Flick, H. Liebig: Mikroprozessortechnik; Springer-Lehrbuch, 5. Auflage 1998  \item Y.N. Patt \& S.J. Patel: Introduction to Computing Systems: From bits \& gates to C \& beyond; McGrawHill, August 2003  \end{itemize}\end{literature}



\end{course}