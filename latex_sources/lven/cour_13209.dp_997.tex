% Lehrveranstaltungsbeschreibung
% Informationsgrad : extern
% Sprache: de
\begin{course}

\setdoclanguagegerman
\coursedegreeprogramme{Informatik}
\coursemodulename{Virtual Reality Praktikum  (S.~\pageref{mod_13273.dp_997})[IN3MACHVRP]}
\courseID{2123375}
\coursename{Virtual Reality Praktikum }
\coursecoordination{J. Ovtcharova}

\documentdate{2012-01-17 15:15:09.996004}

\courselevel{4}
\coursecredits{4}
\courseterm{Sommersemester}
\coursehours{3}
\courseinstructionlanguage{de}

\coursehead

% For index (key word@display). Key word is used for sorting - no Umlauts please.
\index{Virtual Reality Praktikum @Virtual Reality Praktikum }

% For later referencing
\label{cour_13209.dp_997}


\begin{styleenv}
\begin{assessment}
Die Modulprüfung erfolgt als Erfolgskontrolle anderer Art (nach \$4(2) 3 SPO) und setzt sich zusammen aus: Präsentation der Projektarbeit (40\%), Individuelles Projektportfolio in der Anwendungsphase für die Arbeit im Team (30\%), Schriftliche Wissensabfrage (20\%) und soziale Kompetenz (10\%).

 
\end{assessment}

\begin{conditions}Die Lehrveranstaltung ist Wahlmöglichkeit im Modul\emph{ Virtual Engineering B }[WW4INGMB30]. Begrenzte Teilnehmeranzahl (Auswahlverfahren und Anmeldung siehe Homepage zur Lehrveranstaltung)\textbf{.}

\end{conditions}


\end{styleenv}

\begin{learningoutcomes}
Der/ die Studierende sind in der Lage die bestehende Infrastruktur (Hardware und Software) für Virtual Reality (VR) Anwendungen bedienen und benutzen zu können um:

 \begin{itemize}\item die Lösung einer komplexen Aufgabenstellung im Team zu konzipieren,  \item unter Berücksichtigung der Schnittstellen in kleineren Gruppen die Teilaufgaben innerhalb eines bestimmten Arbeitspaketes zu lösen und  \item diese anschließend in ein vollständiges Endprodukt zusammenzuführen.  \end{itemize}

Angestrebte Kompetenzen: \newline
 Methodisches Vorgehen mit praxisorientierten Ingenieuraufgaben, Teamfähigkeit, Arbeit in interdisziplinären Gruppen, Zeitmanagement


\end{learningoutcomes}

\begin{content}
Das Virtual Reality Praktikum besteht aus:

 

1. Einführung und Grundlagen in VR (Hardware, Software, Anwendungen)

 

2. Vorstellung und Nutzung von „3DVIA Virtools“ als Werkzeug und Entwicklungsumgebung

 

3. Anwendung des neu erworbenen Wissens zur Selbständigen Entwicklung eines Fahrsimulators in VR in kleinen Gruppen


\end{content}

\begin{media}Unterlagen zur Veranstaltung werden Praktikumsbegleitend zur Verfügung gestellt.

\end{media}



\begin{remarks}Die Teilnehmerzahl ist begrenzt.

\end{remarks}

\end{course}