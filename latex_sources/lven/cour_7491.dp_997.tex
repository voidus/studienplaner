% Lehrveranstaltungsbeschreibung
% Informationsgrad : extern
% Sprache: de
\begin{course}

\setdoclanguagegerman
\coursedegreeprogramme{Informatik}
\coursemodulename{Basispraktikum zum ICPC-Programmierwettbewerb (S.~\pageref{mod_15669.dp_997})[IN3INBP]}
\courseID{24876}
\coursename{Basispraktikum zum ICPC Programmierwettbewerb}
\coursecoordination{D. Wagner, W. Tichy, I. Rutter, Meder, Krug}

\documentdate{2012-01-27 10:53:42.545591}

\courselevel{2}
\coursecredits{4}
\courseterm{Sommersemester}
\coursehours{4}
\courseinstructionlanguage{de}

\coursehead

% For index (key word@display). Key word is used for sorting - no Umlauts please.
\index{Basispraktikum zum ICPC Programmierwettbewerb@Basispraktikum zum ICPC Programmierwettbewerb}

% For later referencing
\label{cour_7491.dp_997}


\begin{styleenv}
\begin{assessment}
Die Erfolgskontrolle wird in der Modulbeschreibung erläutert.


\end{assessment}

\begin{conditions}Keine.\end{conditions}


\end{styleenv}

\begin{learningoutcomes}
Der/Die Studierende soll

 \begin{itemize}\item vertiefte und erweiterte Kompetenzen in den Bereichen Problemanalyse, Softwareentwicklung und Teamarbeit erwerben.  \item die Fähigkeit, in einem vorgegebenen Zeitrahmen zu einer vorgegebenen Aufgabe eine Lösung selbständig erarbeiten und praktiksch umsetzen zu koennen, erwerben.  \end{itemize}
\end{learningoutcomes}

\begin{content}
Der \emph{ACM International Collegiate Programming Contest} (ICPC) ist ein jährlich stattfindender, weltweiter Programmierwettbewerb für Studierende. Der Wettbewerb findet in zwei Runden statt. Im Herbst jedes Jahres treten Teams aus jeweils drei Studierenden, die sich in den ersten vier Jahren ihres Studiums befinden müssen, in weltweit 32 Regional Contests gegeneinander an. Das Gewinnerteam jedes Regionalwettbewerbs hat im Frühjahr des Folgejahres die Möglichkeit, an den \emph{World Finals} teilzunehmen. \newline
\newline
Im Praktikum werden zu allen für den Wettbewerb relevanten Themengebieten die wichtigsten theoretisch Grundlagen vermittelt und an praktischen Übungsaufgaben erprobt. Höhepunkte des Praktikums sind Local Contests, in denen sich die Praktikumsteilnehmer unter Wettbewerbsbedingungen miteinander messen. \newline
\newline
Aus den Teilnehmern des Praktikums werden außerdem die Teams ausgewählt, die KIT beim \emph{ACM ICPC Regionalwettbewerb der Region Nordwesteuropa (NWERC)} im Herbst vertreten werden.


\end{content}



\begin{literature}\textbf{Weiterführende Literatur:}

 \begin{itemize}\item Cormen, Leiserson, Rivest, Stein: Introduction to Algorithms, MIT Press  \item Skiena, Revilla: Programming Challenges, Springer  \end{itemize}\end{literature}

\begin{remarks}\textcolor{red}{Der Umfang der Leistungspunkte wird ab dem SS 2012 auf 4 erhöht.}

\end{remarks}

\end{course}