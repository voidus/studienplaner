% Lehrveranstaltungsbeschreibung
% Informationsgrad : extern
% Sprache: de
\begin{course}

\setdoclanguagegerman
\coursedegreeprogramme{Informatik}
\coursemodulename{Biomedizinische Technik I (S.~\pageref{mod_4005.dp_997})[IN3EITBIOM]}
\courseID{23264}
\coursename{Bioelektrische Signale und Felder}
\coursecoordination{G. Seemann}

\documentdate{2010-07-15 12:01:42.935188}

\courselevel{3}
\coursecredits{3}
\courseterm{Sommersemester}
\coursehours{2}
\courseinstructionlanguage{de}

\coursehead

% For index (key word@display). Key word is used for sorting - no Umlauts please.
\index{Bioelektrische Signale und Felder@Bioelektrische Signale und Felder}

% For later referencing
\label{cour_8145.dp_997}


\begin{styleenv}
\begin{assessment}
Die Erfolgskontrolle erfolgt in Form einer mündlichen Prüfung im Umfang von i.d.R. 20 Minuten nach § 4 Abs. 2 Nr. 2 SPO.


\end{assessment}

\begin{conditions}Keine.\end{conditions}


\end{styleenv}

\begin{learningoutcomes}
Bioelektrizität und mathematische Modellierung der zugrundeliegenden Mechanismen


\end{learningoutcomes}

\begin{content}
\begin{itemize}\item  Zellmembranen und Ionenkanäle  \item  Zellenphysiologie  \item  Ausbreitung von Aktionspotentialen   \item  Numerische Feldberechnung im menschlichen Körper  \item  Messung bioelektrischer Signale  \item  Elektrokardiographie und Elektrographie, Elektromyographie und Neurographie  \item  Elektroenzephalogramm, Elektrokortigogramm und Evozierte Potentiale, Magnetoenzephalogramm und Magnetokardiogramm  \item  Abbildung bioelektrischer Quellen  \end{itemize}
\end{content}







\end{course}