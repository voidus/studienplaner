% Lehrveranstaltungsbeschreibung
% Informationsgrad : extern
% Sprache: de
\begin{course}

\setdoclanguagegerman
\coursedegreeprogramme{Informatik}
\coursemodulename{Höhere Mathematik (S.~\pageref{mod_1481.dp_997})[IN1MATHHM]}
\courseID{01330}
\coursename{Höhere Mathematik I (Analysis) für die Fachrichtung Informatik}
\coursecoordination{C. Schmoeger}

\documentdate{2008-04-14 09:44:19}

\courselevel{1}
\coursecredits{9}
\courseterm{Wintersemester}
\coursehours{4/2}
\courseinstructionlanguage{de}

\coursehead

% For index (key word@display). Key word is used for sorting - no Umlauts please.
\index{Hoehere Mathematik I (Analysis) fuer die Fachrichtung Informatik@Höhere Mathematik I (Analysis) für die Fachrichtung Informatik}

% For later referencing
\label{cour_6173.dp_997}


\begin{styleenv}
\begin{assessment}
Die Erfolgskontrolle wird in der Modulbeschreibung erläutert.


\end{assessment}

\begin{conditions}Keine.\end{conditions}


\end{styleenv}

\begin{learningoutcomes}
Die Studierenden sollen am Ende des Moduls

 \begin{itemize}\item den Übergang von Schule zu Universität bewältigt haben,  \item mit logischem Denken und strengen Beweisen vertraut sein,  \item die Methoden und grundlegenden Strukturen der (reellen) Analysis beherrschen.  \end{itemize}
\end{learningoutcomes}

\begin{content}
\begin{itemize}\item \textbf{Reelle Zahlen }(Körpereigenschaften, natürliche Zahlen, Induktion)  \item \textbf{Konvergenz in R} ( Folgen, Reihen, Potenzreihen, elementare Funktionen, q-adische Entwicklung reeller Zahlen)  \item \textbf{Funktionen }(Grenzwerte bei Funktionen, Stetigkeit, Funktionenfolgen und -reihen)  \item \textbf{Differentialrechnung} (Ableitungen, Mittelwertsätze, Regel v. de l'Hospital, Satz von Taylor)  \item \textbf{Integralrechnung} (Riemann- Integral, Hauptsätze, Substitution, part. Integration, uneigentliche Integrale)  \item \textbf{Fourierreihen}  \end{itemize}
\end{content}

\begin{media}Vorlesungspräsentation

\end{media}

\begin{literature}Wird in der Vorlesung bekannt gegeben.

\textbf{Weiterführende Literatur:}

Wird in der Vorlesung bekannt gegeben

\end{literature}



\end{course}