% Modulbeschreibung 
% Informationsgrad : extern
% Sprache: de
\begin{module}

\setdoclanguagegerman
\moduledegreeprogramme{Informatik (B.Sc.)}
\modulesubject{EF Betriebswirtschaftslehre}
\moduleID{IN3WWBWL17}
\modulename{Real Estate Management}
\modulecoordination{T. Lützkendorf}

\documentdate{2011-06-27 15:17:05.719382}

\modulecredits{9}
\moduleduration{2}
\modulecycle{Jedes 2. Semester, Wintersemester}



\modulehead

% For index (key word@display). Key word is used for sorting - no Umlauts please.
\index{Real Estate Management@Real Estate Management (M)}

% For later referencing
\label{mod_1641.dp_997}

\begin{courselist}
26400w & Real Estate Management I (S.~\pageref{cour_6843.dp_997}) & 2/2 & W & 4,5 & T. Lützkendorf\\
2585400/2586400 & Real Estate Management II (S.~\pageref{cour_6861.dp_997}) & 2/2 & S & 4,5 & T. Lützkendorf\\
\end{courselist}

\begin{styleenv}
\begin{assessment}
Die Modulprüfung erfolgt in Form von Teilprüfungen (nach §4 (2) SPO) über die einzelnen Lehrveranstaltungen des Moduls, mit denen in Summe die Mindestanforderung an Leistungspunkten erfüllt wird. Die Erfolgskontrolle wird bei jeder Lehrveranstaltung dieses Moduls beschrieben.

 

Die jeweiligen Prüfungen zu den Lehrveranstaltungen erfolgen i.d.R durch eine 60-minütige Klausur. Eine 20-minütige mündliche Prüfung wird i.d.R. nur nach der zweiten nicht erfolgreich absolvierten Prüfung zugelassen. Die jeweilige Teilprüfung (REM I bzw. REM II) erfolgt nur in dem Semester, in dem die entsprechende Vorlesung angeboten wird. Derzeit wird damit REM I nur im Wintersemester und REM II nur im Sommersemester geprüft. Die Prüfung wird in jedem Semester zweimal angeboten und kann zu jedem ordentlichen Prüfungstermin wiederholt werden.

 

Die Gesamtnote des Moduls wird aus den mit Leistungspunkten gewichtete Noten der Teilprüfungen gebildet und nach der ersten Nachkommastelle abgeschnitten. Innerhalb des Moduls kann optional eine Seminar- oder Studienarbeit aus dem Bereich “Real Estate Management” angefertigt werden, die mit einer Gewichtung von 20\% in die Modulnote eingerechnet werden kann.


\end{assessment}

\begin{conditions}Nur prüfbar in Kombination mit dem Modul \emph{Grundlagen der BWL}.

 \end{conditions}

\begin{recommendations}Es wird eine Kombination mit dem Modul \emph{Bauökologie} [IN3WWBWL16] empfohlen. Weiterhin empfehlenswert ist die Kombination mit Lehrveranstaltungen aus den Bereichen

 \begin{itemize}\item Finanzwirtschaft und Banken  \item Versicherungen  \item Bauingenieurwesen und Architektur (Bauphysik, Baukonstruktion, Facility Management)  \end{itemize}\end{recommendations}
\end{styleenv}

\begin{learningoutcomes}
Der/die Studierende

 \begin{itemize}\item besitzt einen Überblick über die verschiedenen Facetten und Zusammenhänge innerhalb der Immobilienwirtschaft, über die wesentlichen Entscheidungen im Lebenszyklus von Immobilien und über die Sichten und Interessen der am Bau Beteiligten,  \item kann die im bisherigen Studium erlernten Verfahren und Methoden der Betriebswirtschaftlehre auf Problemstellungen aus dem Bereich der Immobilienwirtschaft übertragen und anwenden.  \end{itemize}
\end{learningoutcomes}

\begin{content}
Die Bau-, Wohnungs- und Immobilienwirtschaft bietet den Absolventen des Studiengangs interessante Aufgaben sowie gute Arbeits- und Aufstiegschancen. Das Lehrangebot gibt einen Einblick in die volkswirtschaftliche Bedeutung der Branche, erörtert betriebswirtschaftliche Fragestellungen im Immobilien- und Wohnungsunternehmen und vermittelt die Grundlagen für das Treffen von Entscheidungen im Lebenszyklus von Gebäuden sowie beim Management von Gebäudebeständen. Innovative Betreiber- und Finanzierungsmodelle werden ebenso dargestellt wie aktuelle Entwicklungen bei der Betrachtung von Immobilien als Asset-Klasse. Das Lehrangebot eignet sich insbesondere auch für Studierende, die volkswirtschaftliche, betriebswirtschaftliche oder finanzierungstechnische Fragestellungen in der Bau- und Immobilienbranche bearbeiten möchten.


\end{content}



\end{module}

