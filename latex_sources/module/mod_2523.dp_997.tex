% Modulbeschreibung 
% Informationsgrad : extern
% Sprache: de
\begin{module}

\setdoclanguagegerman
\moduledegreeprogramme{Informatik (B.Sc.)}
\modulesubject{Schlüsselqualifikationen}
\moduleID{IN1HOCSQ}
\modulename{Schlüsselqualifikationen}
\modulecoordination{M. Stolle}

\documentdate{2011-10-07 11:38:12.335838}

\modulecredits{4}
\moduleduration{1}
\modulecycle{Jedes Semester}



\modulehead

% For index (key word@display). Key word is used for sorting - no Umlauts please.
\index{Schluesselqualifikationen@Schlüsselqualifikationen (M)}

% For later referencing
\label{mod_2523.dp_997}

\begin{courselist}
PLV & Praxis des Lösungsvertriebs (S.~\pageref{cour_7139.dp_997}) & 2 & S & 1 & K. Böhm, Hellriegel\\
PUB & Praxis der Unternehmensberatung (S.~\pageref{cour_7141.dp_997}) & 2 & W/S & 1 & K. Böhm, Dürr\\
PMP & Projektmanagement aus der Praxis (S.~\pageref{cour_7151.dp_997}) & 2 & S & 1 & K. Böhm, W. Schnober\\
SQHoC & Schlüsselqualifikationen HoC (S.~\pageref{cour_14497.dp_997}) & 2 &  & 4 & M. Stolle\\
\end{courselist}

\begin{styleenv}
\begin{assessment}
Die Erfolgskontrolle zu den Lehrveranstaltungen sind in der jeweiligen LV-Beschreibung erläutert und ggf. in den Lehrveranstaltungsbeschreibungen des House of Competence (HoC).

 

Die Gesamtnote des Moduls wird ggf. aus den mit LP gewichteten Noten der Teilprüfungen gebildet und nach der ersten Kommastelle abgeschnitten.


\end{assessment}

\begin{conditions}Keine.\end{conditions}


\end{styleenv}

\begin{learningoutcomes}
Lernziele lassen sich in in drei Hauptkategorien einteilen, die sich wechselseitig ergänzen:\newline
\newline
1. Orientierungswissen

 \begin{itemize}\item Die Studierenden werden sich der kulturellen Prägung ihrer Position bewusst und sind in der Lage, die Sichtweisen und Interessen anderer (über Fach-, Kultur- und Sprachgrenzen hinweg) zu berücksichtigen.  \item Sie erweitern ihre Fähigkeiten, sich an wissenschaftlichen oder öffentlichen Diskussionen sachgerecht und angemessen zu beteiligen.  \end{itemize}

2. Praxisorientierung

 \begin{itemize}\item Studierende erhalten Einsicht in die Routinen professionellen Handelns.  \item Sie entwickeln ihre Lernfähigkeit weiter.  \item Sie erweitern durch Ausbau ihrer Fremdsprachenkenntnisse ihre Handlungsfähigkeit.  \item Sie können grundlegende betriebswirtschaftliche und rechtliche Sacherverhalte mit ihrem Erfahrungsfeld verbinden.  \end{itemize}

3. Basiskompetenzen

 \begin{itemize}\item Die Studierenden können geplant und zielgerichtet sowie methodisch fundiert selbständig neues Wissen erwerben und dieses bei der Lösung von Aufgaben und Problemen einsetzen.  \item Sie können die eigene Arbeit auswerten.  \item Sie verfügen über effiziente Arbeitstechniken, können Prioritäten setzen, Entscheidungen treffen und Verantwortung übernehmen.  \end{itemize}
\end{learningoutcomes}

\begin{content}
Das House of Competence bietet mit dem Modul Schlüsselqalifikationen eine breite Auswahl aus fünf Wahlbereichen, in denen Veranstaltungen zur besseren Orientierung thematisch zusammengefasst werden. Die Inhalte werden in den Beschreibungen der Veranstaltungen auf den Internetseiten des HoC (http://www.hoc.kit.edu/) detailliert erläutert.\newline
\newline
Wahlbereiche des HoC:

 \begin{itemize}\item „Kultur – Politik – Wissenschaft – Technik“, 2-3 LP  \item „Kompetenz- und Kreativitätswerkstatt“, 2-3 LP  \item „Fremdsprachen“, 2-3 LP  \item „Persönliche Fitness \& Emotionale Kompetenz“, 2-3 LP  \item „Tutorenprogramme“, 3 LP  \item „Mikrobausteine“, 1 LP  \end{itemize}

Ferner können auch Lehrveranstaltungen der Fakultät für Informatik gewählt werden, die Inhalte werden in den jeweiligen Lehrveranstaltungsbeschreibungen erläutert.


\end{content}

\begin{remarks}Dieses Modul wurde im Umfang reduziert, weil das Pflichtmodul \emph{Teamarbeit in der Software-Entwicklung} [IN2INSWPS] mit 2 LP dem Fach Schlüsselqualifiationen zugeordnet wird. Studierende, die bereits das alte Modul abgeschlossen haben und das Modul Praxis der Software-Entwicklung [IN2INSWP] noch nicht bestanden haben, kontaktieren bitte das Service-Zentrum Studium und Lehre.

\end{remarks}

\end{module}

