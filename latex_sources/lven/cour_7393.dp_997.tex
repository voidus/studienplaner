% Lehrveranstaltungsbeschreibung
% Informationsgrad : extern
% Sprache: de
\begin{course}

\setdoclanguagegerman
\coursedegreeprogramme{Informatik}
\coursemodulename{Datenschutz und Privatheit in vernetzten Informationssystemen (S.~\pageref{mod_3067.dp_997})[IN3INDPI]}
\courseID{24605}
\coursename{Datenschutz und Privatheit in vernetzten Informationssystemen}
\coursecoordination{K. Böhm, Buchmann}

\documentdate{2010-05-21 11:23:35.855901}

\courselevel{4}
\coursecredits{3}
\courseterm{Sommersemester}
\coursehours{2}
\courseinstructionlanguage{de}

\coursehead

% For index (key word@display). Key word is used for sorting - no Umlauts please.
\index{Datenschutz und Privatheit in vernetzten Informationssystemen@Datenschutz und Privatheit in vernetzten Informationssystemen}

% For later referencing
\label{cour_7393.dp_997}


\begin{styleenv}
\begin{assessment}
Die Erfolgskontrolle wird in der Modulbeschreibung erläutert.


\end{assessment}

\begin{conditions}Grundkenntnisse zu Datenbanken, verteilten Informationssystemen, Systemarchitekturen und Kommunikationsinfrastrukturen, z.B. aus den Vorlesungen \emph{Datenbanksysteme }[24516] und \emph{Einführung in Rechnernetze} [24519].

\end{conditions}


\end{styleenv}

\begin{learningoutcomes}
Die Studenten sollen in die Ziele und Grundbegriffe der Informationellen Selbstbestimmung eingeführt werden. Sie sollen dazu die grundlegende Herausforderungen des Datenschutzes und ihre vielfältigen Auswirkungen auf Gesellschaft und Individuen benennen können. Weiterhin sollen die Studenten aktuelle Technologien zum Datenschutz beherrschen und anwenden können, z.B. Methoden des Spatial \& Temporal Cloaking. Die Studenten sollen damit in die Lage versetzt werden, die Risiken unbekannter Technologien für die Privatheit zu analysieren, geeignete Maßnahmen zum Umgang mit diesen Risiken vorzuschlagen und die Effektivität dieser Maßnahmen abzuschätzen.


\end{learningoutcomes}

\begin{content}
In diesem Modul soll vermittelt werden, welchen Einfluss aktuelle und derzeit in der Entwicklung befindliche Informationssysteme auf die Privatheit ausüben. Diesen Herausforderungen werden technische Maßnahmen zum Datenschutz gegenübergestellt, die derzeit in der Forschung diskutiert werden. Ein Exkurs zu den gesellschaftlichen Implikationen von Datenschutzproblen und Datenschutztechniken rundet das Modul ab.


\end{content}

\begin{media}Vorlesungsfolien

\end{media}

\begin{literature}In den Vorlesungsfolien wird auf ausgewählte aktuelle Forschungspapiere verwiesen.

\end{literature}

\begin{remarks}\textcolor{red}{Diese Lehrveranstaltung wird im Bachelor-Studiengang Informatik nicht mehr angeboten.}

\end{remarks}

\end{course}