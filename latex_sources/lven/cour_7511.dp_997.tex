% Lehrveranstaltungsbeschreibung
% Informationsgrad : extern
% Sprache: de
\begin{course}

\setdoclanguagegerman
\coursedegreeprogramme{Informatik}
\coursemodulename{Virtual Engineering für mechatronische Produkte (S.~\pageref{mod_4293.dp_997})[IN3MACHVEMP]}
\courseID{2121370}
\coursename{Virtual Engineering für mechatronische Produkte}
\coursecoordination{J. Ovtcharova, S. Rude}

\documentdate{2012-01-17 12:15:52.887420}

\courselevel{4}
\coursecredits{4}
\courseterm{Wintersemester}
\coursehours{3/0}
\courseinstructionlanguage{de}

\coursehead

% For index (key word@display). Key word is used for sorting - no Umlauts please.
\index{Virtual Engineering fuer mechatronische Produkte@Virtual Engineering für mechatronische Produkte}

% For later referencing
\label{cour_7511.dp_997}


\begin{styleenv}
\begin{assessment}
Die Modulprüfung erfolgt in Form einer mündlichen Gesamtprüfung (20 min.) (nach §4(2),1-3 SPO). Die Prüfung wird jedes Semester angeboten und kann zu jedem ordentlichen Prüfungstermin wiederholt werden. Die Gesamtnote des Moduls entspricht der Note der mündlichen Prüfung.


\end{assessment}

\begin{conditions}Keine.\end{conditions}

\begin{recommendations}Es werden Kenntnisse über CAx vorausgesetzt. Daher empfiehlt es sich, die Lehrveranstaltung Virtual Engineering I [2121352] im Vorfeld zu besuchen.

\end{recommendations}
\end{styleenv}

\begin{learningoutcomes}
Die Studierenden sind in der Lage die Vorgehensweise bei der Integration mechatronischer Komponenten in Produkte anzuwenden.

 

Die Studierende verstehen die besonderen Anforderungen bei funktional vernetzten Systemen.

 

Begreifen der praktischen Relevanz der erlernten Methoden anhand Anwendungsbeispielen aus der Automobilindustrie.


\end{learningoutcomes}

\begin{content}
Der Einzug mechatronischer Komponenten in alle Produkte verändert geometrieorientierte Konstruktionsabläufe in funktionsorientierte Abläufe. Damit verbunden ist die Anwendung von IT-Systemen neu auszurichten. Die Vorlesung behandelt hierzu:

 \begin{itemize}\item Herausforderungen an den Konstruktionsprozess aus der Sicht der Integration mechatronischer Komponenten in Produkte,  \item Unterstützung der Aufgabenklärung durch Anforderungsmanagement,  \item Lösungsfindung auf Basis funktional vernetzter Systeme,  \item Realisierung von Lösungen auf Basis von Elektronik (Sensoren, Aktuatoren, vernetzte Steuergeräte),  \item Beherrschung verteilter Software-Systeme durch Software-Engineering und  \item Herausforderungen an Test und Absicherung aus der Sicht zu erreichender Systemqualität.  \end{itemize}

Anwendungsfelder und Systembeispiele stammen aus der Automobilindustrie.


\end{content}

\begin{media}Skript zur Veranstaltung

\end{media}



\begin{remarks}Einwöchige Blockveranstaltung.

\end{remarks}

\end{course}