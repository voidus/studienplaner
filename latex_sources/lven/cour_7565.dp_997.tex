% Lehrveranstaltungsbeschreibung
% Informationsgrad : extern
% Sprache: de
\begin{course}

\setdoclanguagegerman
\coursedegreeprogramme{Informatik}
\coursemodulename{Lineare Algebra und Analytische Geometrie (S.~\pageref{mod_3027.dp_997})[IN1MATHLAAG]}
\courseID{01007}
\coursename{Lineare Algebra und Analytische Geometrie 1}
\coursecoordination{F. Herrlich, E. Leuzinger, C. Schmidt, W. Tuschmann}

\documentdate{2009-02-16 11:10:50}

\courselevel{1}
\coursecredits{9}
\courseterm{Wintersemester}
\coursehours{4/2/2}
\courseinstructionlanguage{de}

\coursehead

% For index (key word@display). Key word is used for sorting - no Umlauts please.
\index{Lineare Algebra und Analytische Geometrie 1@Lineare Algebra und Analytische Geometrie 1}

% For later referencing
\label{cour_7565.dp_997}


\begin{styleenv}
\begin{assessment}
Die Erfolgskontrolle wird in der Modulbeschreibung erläutert.


\end{assessment}

\begin{conditions}Keine.\end{conditions}


\end{styleenv}

\begin{learningoutcomes}
Die Studierenden sollen am Ende des Moduls

 \begin{itemize}\item den Übergang von der Schule zur Universität bewältigt haben,  \item mit logischem Denken und strengen Beweisen vertraut sein,   \item die Methoden und grundlegenden Strukturen der Linearen Algebra und Analytischen Geometrie beherrschen.  \end{itemize}
\end{learningoutcomes}

\begin{content}
\begin{itemize}\item Grundbegriffe (Mengen, Abbildungen, Relationen, Gruppen, Ringe, Körper, Matrizen, Polynome)   \item Lineare Gleichungssysteme (Gauß´sches Eliminationsverfahren, Lösungstheorie)   \item Vektorräume (Beispiele, Unterräume, Quotientenräume, Basis und Dimension)   \item Lineare Abbildungen (Kern, Bild, Rang, Homomorphiesatz, Vektorräume von Abbildungen, Dualraum, Darstellungsmatrizen, Basiswechsel, Endomorphismenalgebra, Automorphismengruppe)   \item Determinanten   \item Eigenwerttheorie (Eigenwerte, Eigenvektoren, charakteristisches Polynom, Normalformen)  \end{itemize}
\end{content}



\begin{literature}Wird in der Vorlesung bekannt gegeben.

\end{literature}



\end{course}