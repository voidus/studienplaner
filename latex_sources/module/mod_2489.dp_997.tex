% Modulbeschreibung 
% Informationsgrad : extern
% Sprache: de
\begin{module}

\setdoclanguagegerman
\moduledegreeprogramme{Informatik (B.Sc.)}
\modulesubject{}
\moduleID{IN3INRS}
\modulename{Rechnerstrukturen}
\modulecoordination{W. Karl}

\documentdate{2011-02-17 10:43:55.807093}

\modulecredits{6}
\moduleduration{1}
\modulecycle{Jedes 2. Semester, Sommersemester}



\modulehead

% For index (key word@display). Key word is used for sorting - no Umlauts please.
\index{Rechnerstrukturen@Rechnerstrukturen (M)}

% For later referencing
\label{mod_2489.dp_997}

\begin{courselist}
24570 & Rechnerstrukturen (S.~\pageref{cour_7099.dp_997}) & 3/1 & S & 6 & J. Henkel, W. Karl\\
\end{courselist}

\begin{styleenv}
\begin{assessment}
Die Erfolgskontrolle erfolgt in Form einer schriftlichen Prüfung im Umfang von 60 Minuten nach § 4 Abs. 2 Nr. 1 SPO.

 

Die Modulnote ist die Note der schriftlichen Prüfung.


\end{assessment}

\begin{conditions}Der in dem Modul \emph{Technische Informatik} [IN1INTI] vermittelte Inhalt wird vorausgesetzt.

\end{conditions}


\end{styleenv}

\begin{learningoutcomes}
Die Lehrveranstaltung soll die Studierenden in die Lage versetzen,

 \begin{itemize}\item grundlegendes Verständnis über den Aufbau, die Organisation und das Operationsprinzip von Rechnersystemen zu erwerben,  \item aus dem Verständnis über die Wechselwirkungen von Technologie, Rechnerkonzepten und Anwendungen die grundlegenden Prinzipien des Entwurfs nachvollziehen und anwenden zu können,  \item Verfahren und Methoden zur Bewertung und Vergleich von Rechensystemen anwenden zu können,  \item grundlegendes Verständnis über die verschiedenen Formen der Parallelverarbeitung in Rechnerstrukturen zu erwerben.   \end{itemize}

Insbesondere soll die Lehrveranstaltung die Voraussetzung liefern, vertiefende Veranstaltungen über eingebettete Systeme, moderne Mikroprozessorarchitekturen, Parallelrechner, Fehlertoleranz und Leistungsbewertung zu besuchen und aktuelle Forschungsthemen zu verstehen.


\end{learningoutcomes}

\begin{content}
Der Inhalt umfasst:

 \begin{itemize}\item Einführung in die Rechnerarchitektur  \item Grundprinzipien des Rechnerentwurfs: Kompromissfindung zwischen Zielsetzungen, Randbedingungen, Gestaltungsgrundsätzen und Anforderungen  \item Leistungsbewertung von Rechensystemen  \item Parallelismus auf Maschinenbefehlsebene: Superskalartechnik, spekulative Ausführung, Sprungvorhersage, VLIW-Prinzip, mehrfädige Befehlsausführung  \item Parallelrechnerkonzepte, speichergekoppelte Parallelrechner (symmetrische Multiprozessoren, Multiprozessoren mit verteiltem gemeinsamem Speicher), nachrichtenorientierte Parallelrechner, Multicore-Architekturen, parallele Programmiermodelle  \item Verbindungsnetze (Topologien, Routing)  \item Grundlagen der Vektorverarbeitung, SIMD, Multimedia-Verarbeitung  \item Energie-effizienter Entwurf  \item Grundlagen der Fehlertoleranz, Zuverlässigkeit, Verfügbarkeit, Sicherheit  \end{itemize}
\end{content}

\begin{remarks}Studiengänge Informatik: Das Modul \emph{Rechnerstrukturen} ist ein Stammmodul.

\end{remarks}

\end{module}

