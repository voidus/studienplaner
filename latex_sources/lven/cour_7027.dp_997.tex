% Lehrveranstaltungsbeschreibung
% Informationsgrad : extern
% Sprache: de
\begin{course}

\setdoclanguagegerman
\coursedegreeprogramme{Informatik}
\coursemodulename{Analysis (S.~\pageref{mod_2415.dp_997})[IN1MATHANA]}
\courseID{01001}
\coursename{Analysis 1}
\coursecoordination{G. Herzog, M. Plum, W. Reichel, C. Schmoeger, R. Schnaubelt, L. Weis}

\documentdate{2008-07-25 11:47:06}

\courselevel{1}
\coursecredits{9}
\courseterm{Wintersemester}
\coursehours{4/2/2}
\courseinstructionlanguage{de}

\coursehead

% For index (key word@display). Key word is used for sorting - no Umlauts please.
\index{Analysis 1@Analysis 1}

% For later referencing
\label{cour_7027.dp_997}


\begin{styleenv}
\begin{assessment}
Die Erfolgskontrolle wird in der Modulbeschreibung erläutert.


\end{assessment}

\begin{conditions}Keine.\end{conditions}


\end{styleenv}

\begin{learningoutcomes}
Die Studierenden sollen am Ende des Moduls

 \begin{itemize}\item den Übergang von der Schule zur Universität bewältigt haben,  \item mit logischem Denken und strengen Beweisen vertraut sein,  \item die Grundlagen der Differential- und Integralrechnung von Funktionen einer reellen Variablen und der Differentialrechnung von Funktionen in mehreren Variablen beherrschen.  \end{itemize}
\end{learningoutcomes}

\begin{content}
Vollständige Induktion, reelle und komplexe Zahlen, Konvergenz, Vollständigkeit, Zahlenreihen, Potenzreihen, elementare Funktionen. Stetigkeit reeller Funktionen, Satz vom Maximum, Zwischenwertsatz. Differentiation reeller Funktionen, Mittelwertsatz, Regel von L'Hospital, Monotonie, Extrema, Konvexität, Satz von Taylor, Newton Verfahren, Differentiation von Reihen. Integration reeller Funktionen: Riemannintegral, Hauptsatz der Differential- und Integralrechnung, Integrationsmethoden, numerische Integration, uneigentliches Integral. \newline
Konvergenz von Funktionenfolgen- und reihen.


\end{content}



\begin{literature}Wird in der Vorlesung bekannt gegeben.

\end{literature}



\end{course}