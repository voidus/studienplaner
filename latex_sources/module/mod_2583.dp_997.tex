% Modulbeschreibung 
% Informationsgrad : extern
% Sprache: de
\begin{module}

\setdoclanguagegerman
\moduledegreeprogramme{Informatik (B.Sc.)}
\modulesubject{}
\moduleID{IN3INES1}
\modulename{Optimierung und Synthese Eingebetteter Systeme (ES1)}
\modulecoordination{J. Henkel}

\documentdate{2011-02-24 11:39:38.170753}

\modulecredits{3}
\moduleduration{1}
\modulecycle{Jedes 2. Semester, Wintersemester}



\modulehead

% For index (key word@display). Key word is used for sorting - no Umlauts please.
\index{Optimierung und Synthese Eingebetteter Systeme (ES1)@Optimierung und Synthese Eingebetteter Systeme (ES1) (M)}

% For later referencing
\label{mod_2583.dp_997}

\begin{courselist}
24143 & Optimierung und Synthese Eingebetteter Systeme (ES1)   (S.~\pageref{cour_7227.dp_997}) & 2 & W & 3 & J. Henkel\\
\end{courselist}

\begin{styleenv}
\begin{assessment}
Die Erfolgskontrolle erfolgt in Form einer mündlichen Prüfung nach § 4 Abs. 2 Nr. 2 der SPO im Umfang von i.d.R. 20 Minuten.

 

Die Modulnote ist die Note der mündlichen Prüfung.


\end{assessment}

\begin{conditions}Der erfolgreiche Abschluss des Moduls \emph{Technische Informatik} [IN1INTI] wird vorausgesetzt.

\end{conditions}


\end{styleenv}

\begin{learningoutcomes}
Der Studierende wird in die Lage versetzt, eingebettete Systeme entwickeln zu können. Er kann eigene Hardware spezifizieren, synthetisieren und optimieren. Er erlernt die Hardwarebeschreibungssprache VHDL und Verilog.\newline
Er weiß, wie reaktive Systeme zu entwerfen sind. Er kennt die besonderen Randbedingungen des Entwurfs eingebetteter Systeme.


\end{learningoutcomes}

\begin{content}
Die kostengünstige und fehlerfreie Entwicklung dieser Eingebetteten Systeme stellt momentan eine nicht zu unterschätzende Herausforderung dar, welche einen immer stärkeren Einfluss auf die Wertschöpfung des Gesamtsystems nimmt. Besonders in Europa gewinnt der Entwurf Eingebetteter Systeme in vielen Wirtschaftszweigen, wie etwa dem Automobilbereich, eine immer gewichtigere wirtschaftliche Rolle, so dass sich bereits heute schon eine Reihe von namhaften Firmen mit der Entwicklung Eingebetteter Systeme befassen.\newline
Die Vorlesung befasst sich umfassend mit allen Aspekten der Entwicklung Eingebetteter Systeme auf Hardware-, Software- sowie Systemebene. Dazu gehören vielfältige Bereiche wie Modellierung, Optimierung, Synthese und Verifikation der Systeme.


\end{content}



\end{module}

