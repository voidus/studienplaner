% Modulbeschreibung 
% Informationsgrad : extern
% Sprache: de
\begin{module}

\setdoclanguagegerman
\moduledegreeprogramme{Informatik (B.Sc.)}
\modulesubject{Praktische Informatik}
\moduleID{IN1INPROG}
\modulename{Programmieren}
\modulecoordination{G. Snelting}

\documentdate{2011-08-01 14:21:11.738033}

\modulecredits{5}
\moduleduration{1}
\modulecycle{Jedes 2. Semester, Wintersemester}



\modulehead

% For index (key word@display). Key word is used for sorting - no Umlauts please.
\index{Programmieren@Programmieren (M)}

% For later referencing
\label{mod_2373.dp_997}

\begin{courselist}
24004 & Programmieren (S.~\pageref{cour_6247.dp_997}) & 2/0/2 & W & 5 & A. Pretschner\\
\end{courselist}

\begin{styleenv}
\begin{assessment}
Zum erfolgreichen Bestehen der Lehrveranstaltung sind zwei Erfolgskontrollen zu erbringen.

 \begin{itemize}\item Bestehen eines unbenoteten Übungsscheins (nach § 4 Abs. 2 Nr. 3 SPO). \textbf{Der Übungsschein ist zwingende Voraussetzung für die Teilnahme an der zweiten Erfolgskontrolle}.   \item Diese zweite Kontrolle besteht im Bestehen zweier Abschlussaufgaben (nach § 4 Abs. 2 Nr. 3 SPO), die zeitlich getrennt abgegeben werden. Sollte diese Erfolgskontrolle nicht bestanden sein, kann sie, d.h. erneute Abgabe \textbf{beider }Abschlussaufgaben, einmal wiederholt werden.  \end{itemize}

Die Gesamtnote setzt sich aus den Noten der zwei Abschlussaufgaben zusammen.

 

Achtung: Dieses Modul ist Bestandteil der Orientierungsprüfung gemäß § 8 Abs. 1 SPO. Die Prüfung ist bis zum Ende des 2. Fachsemesters anzutreten und bis zum Ende des 3. Fachsemesters zu bestehen.


\end{assessment}

\begin{conditions}Keine.\end{conditions}

\begin{recommendations}Vorkenntnisse in Java-Programmierung können hilfreich sein, werden aber nicht vorausgesetzt.

\end{recommendations}
\end{styleenv}

\begin{learningoutcomes}
Der/die Studierende soll

 \begin{itemize}\item grundlegender Strukturen der Programmiersprache Java kennen und anwenden, insbesondere Kontrollstrukturen, einfache Datenstrukturen, Umgang mit Objekten, und Implementierung elementarer Algorithmen.  \item grundlegende Kenntnisse in Programmiermethodik und die Fähigkeit zur autonomen Erstellung kleiner bis mittlerer, lauffähiger Java-Programme erwerben.  \end{itemize}
\end{learningoutcomes}

\begin{content}
\begin{itemize}\item Objekte und Klassen  \item Typen, Werte und Variablen  \item Methoden  \item Kontrollstrukturen  \item Rekursion  \item Referenzen, Listen  \item Vererbung   \item Ein/-Ausgabe  \item Exceptions  \item Programmiermethodik  \item Implementierung elementarer Algorithmen (z.B. Sortierverfahren) in Java  \end{itemize}
\end{content}



\end{module}

