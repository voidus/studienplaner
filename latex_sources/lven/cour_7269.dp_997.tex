% Lehrveranstaltungsbeschreibung
% Informationsgrad : extern
% Sprache: de
\begin{course}

\setdoclanguagegerman
\coursedegreeprogramme{Informatik}
\coursemodulename{Analysis 3 (S.~\pageref{mod_3293.dp_997})[IN3MATHAN02]}
\courseID{01005}
\coursename{Analysis 3}
\coursecoordination{G. Herzog, M. Plum, W. Reichel, C. Schmoeger, R. Schnaubelt, L. Weis}

\documentdate{2011-10-07 12:06:34.618641}

\courselevel{3}
\coursecredits{9}
\courseterm{Wintersemester}
\coursehours{4/2}
\courseinstructionlanguage{de}

\coursehead

% For index (key word@display). Key word is used for sorting - no Umlauts please.
\index{Analysis 3@Analysis 3}

% For later referencing
\label{cour_7269.dp_997}


\begin{styleenv}
\begin{assessment}
Die Erfolgskontrolle erfolgt in Form einer schriftlichen Klausur nach § 4 Abs. 2 Nr. 1 SPO.

 

Zusätzlich muss ein Übungsschein bestanden werden (Erfolgskontrolle anderer Art nach § 4 Abs. 2 Nr. 3 SPO). Dieser wird mit “bestanden” / “nicht bestanden” bewertet.

 

Die Note ist die Note der Klausur.


\end{assessment}

\begin{conditions}Empfehlung: \emph{Lineare Algebra} [IN1MATHANA] und \emph{Analysis} [IN1MATHLA] sind empfohlene Grundlagen.

Die Bedingungen werden in der Modulbeschreibung erläutert.

\end{conditions}

\begin{recommendations}Folgende Module sollten bereits belegt worden sein (Empfehlung):\newline
Analysis 1+2\newline
Lineare Algebra 1+2

\end{recommendations}
\end{styleenv}

\begin{learningoutcomes}
\begin{itemize}\item Einführung in die Konzepte der Lebesgueschen Maß- und Integrationstheorie  \item Vertrautheit mit Integrationstechniken  \end{itemize}
\end{learningoutcomes}

\begin{content}
\begin{itemize}\item Lebesgueintegral  \item Messbarkeit  \item Konvergenzsätze  \item Satz von Fubini  \item Transformationssatz  \item Divergenzsatz  \item Satz von Stokes  \item Fourierreihen  \item Beispiele von Rand- und Eigenwertproblemen gewöhnlicher Differentialgleichungen  \end{itemize}
\end{content}



\begin{literature}Wird in Vorlesung bekannt gegeben.

 

\textbf{Weiterführende Literatur:}

 

Wird in Vorlesung bekannt gegeben.

\end{literature}



\end{course}