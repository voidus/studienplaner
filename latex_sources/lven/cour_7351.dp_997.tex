% Lehrveranstaltungsbeschreibung
% Informationsgrad : extern
% Sprache: de
\begin{course}

\setdoclanguagegerman
\coursedegreeprogramme{Informatik}
\coursemodulename{Proseminar (S.~\pageref{mod_2385.dp_997})[IN2INPROSEM]}
\courseID{ProSemSWT}
\coursename{Proseminar Softwaretechnik}
\coursecoordination{R. Reussner, G. Snelting}

\documentdate{2011-11-14 11:20:20.097117}

\courselevel{3}
\coursecredits{3}
\courseterm{Winter-/Sommersemester}
\coursehours{2}
\courseinstructionlanguage{de}

\coursehead

% For index (key word@display). Key word is used for sorting - no Umlauts please.
\index{Proseminar Softwaretechnik@Proseminar Softwaretechnik}

% For later referencing
\label{cour_7351.dp_997}


\begin{styleenv}
\begin{assessment}
Die Erfolgskontrolle erfolgt durch Ausarbeiten einer schriftlichen Proseminararbeit sowie der Präsentation derselbigen als Erfolgskontrolle anderer Art nach § 4 Abs. 2 Nr. 3 der SPO. Die Gesamtnote setzt sich aus den benoteten und gewichteten Erfolgskontrollen (i.d.R. 80 \% Proseminararbeit, 20 \% Präsentation) zusammen.


\end{assessment}

\begin{conditions}Kenntnisse zu Grundlagen der Softwaretechnik aus entsprechenden Vorlesungen oder praktischen Erfahrungen werden vorausgesetzt.\newline
\newline
Die Fähigkeit zum Erstellen von Programmen geringer Komplexität (Programmieren im Kleinen) und Beherrschung einer objektorientierten Programmiersprache wie z.B. Java, C\# oder C++ werden vorausgesetzt.\newline
\newline
Kenntnisse der englischen Fachsprache werden vorausgesetzt.

\end{conditions}


\end{styleenv}

\begin{learningoutcomes}
Studierende können,

 \begin{itemize}\item eine Literaturrecherche ausgehend von einem vorgegebenen Thema durchführen, die relevante Literatur identifizieren, auffinden, bewerten und schließlich auswerten.  \item ihre Proseminararbeit (und später die Bachelor-/Masterarbeit) mit minimalem Einarbeitungsaufwand anfertigen und dabei Formatvorgaben berücksichtigen, wie sie von allen Verlagen bei der Veröffentlichung von Dokumenten vorgegeben werden.  \item Präsentationen im Rahmen eines wissenschaftlichen Kontextes ausarbeiten. Dazu werden Techniken vorgestellt, die es ermöglichen, die vorzustellenden Inhalte auditoriumsgerecht aufzuarbeiten und vorzutragen.  \item die Ergebnisse der Recherchen in schriftlicher Form derart präsentieren, wie es im Allgemeinen in wissenschaftlichen Publikationen der Fall ist.  \end{itemize}
\end{learningoutcomes}

\begin{content}
Das Proseminar behandelt aktuelle Forschungsthemen aus der Softwaretechnik.


\end{content}





\begin{remarks}Diese Lehrveranstaltung ist ein generischer Platzhalter, der von semesterspezifischen Lehrveranstaltungen ausgefüllt wird. Die semesterspezifischen Veranstaltungen können auf den Webseiten der Lehrstühle/ der Veranstaltungsleiter eingesehen oder per Email erfragt werden.

\end{remarks}

\end{course}