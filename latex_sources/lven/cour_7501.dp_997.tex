% Lehrveranstaltungsbeschreibung
% Informationsgrad : extern
% Sprache: de
\begin{course}

\setdoclanguagegerman
\coursedegreeprogramme{Informatik}
\coursemodulename{Rechnerintegrierte Planung neuer Produkte (S.~\pageref{mod_4287.dp_997})[IN3MACHRPP]}
\courseID{2122387}
\coursename{Rechnerintegrierte Planung neuer Produkte}
\coursecoordination{R. Kläger}

\documentdate{2012-01-17 12:12:38.463228}

\courselevel{4}
\coursecredits{4}
\courseterm{Sommersemester}
\coursehours{2/0}
\courseinstructionlanguage{de}

\coursehead

% For index (key word@display). Key word is used for sorting - no Umlauts please.
\index{Rechnerintegrierte Planung neuer Produkte@Rechnerintegrierte Planung neuer Produkte}

% For later referencing
\label{cour_7501.dp_997}


\begin{styleenv}
\begin{assessment}
Die Erfolgskontrolle erfolgt in Form einer mündlichen Prüfung im Umfang von 30 Minuten (nach§ 4(2), 2 SPO).

 

Die Note entspricht der Note der Prüfung.


\end{assessment}

\begin{conditions}Keine.\end{conditions}


\end{styleenv}

\begin{learningoutcomes}
Der/ die Studierende

 \begin{itemize}\item versteht die Standardabläufe im Produktplanungsbereich,  \item besitzt grundlegende Kenntnisse über Zusammenhänge, Vorgänge und Strukturelemente als Handlungsleitfaden bei der Planung neuer Produkte,  \item besitzt grundlegende Kenntnisse über die Grundlagen und Merkmale der Rapid Prototyping Verfahrenstechnologien,  \item versteht die simultane Unterstützung des Produktplanungsprozesses durch entwicklungsbegleitend einsetzbare Rapid Prototyping (RP)-Systeme.  \end{itemize}
\end{learningoutcomes}

\begin{content}
Die Steigerung der Kreativität und Innovationsstärke bei der Planung und Entwicklung neuer Produkte wird u.a. durch einen verstärkten Rechnereinsatz für alle Unternehmen zu einer der entscheidenden Einflussgrößen für die Wettbewerbsfähigkeit der Industrie im globalen Wettbewerb geworden ist. \newline
\newline
 Entsprechend verfolgt die Vorlesung folgende Ziele:

 \begin{itemize}\item Das Grundverständnis für Standardabläufe im Produktplanungsbereich erlangen, Kenntnis über Zusammenhänge, Vorgänge und Strukturelemente erwerben und als Handlungsleitfaden bei der Planung neuer Produkte benutzen lernen;  \item Kenntnis über die Anforderungen und Möglichkeiten der Rechnerunterstützung erhalten, um die richtigen Methoden und Werkzeuge für die effiziente und sinnvolle Unterstützung eines spezifischen Anwendungsfalles auszuwählen;  \item mit den Elementen und Methoden des rechnerunterstützten Ideenmanagements vertraut gemacht werden;  \item die Möglichkeiten der simultanen Unterstützung des Produktplanungsprozesses durch entwicklungsbegleitend einsetzbare Rapid Prototyping (RP)-Systeme kennen lernen;  \end{itemize}

Kenntnis über die Grundlagen und Merkmale dieser RP-Verfahrenstechnologien erwerben und - in Abhängigkeit des zu entwickelnden Produkts - anhand von Beispielen effizient und richtig zur Anwendung bringen können.


\end{content}

\begin{media}Skript zur Veranstaltung wird in der Vorlesung verteilt.

\end{media}





\end{course}