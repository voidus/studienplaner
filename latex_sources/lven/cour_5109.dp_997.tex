% Lehrveranstaltungsbeschreibung
% Informationsgrad : extern
% Sprache: de
\begin{course}

\setdoclanguagegerman
\coursedegreeprogramme{Informatik}
\coursemodulename{Risk and Insurance Management (S.~\pageref{mod_1553.dp_997})[IN3WWBWL6]}
\courseID{2530326}
\coursename{Enterprise Risk Management}
\coursecoordination{U. Werner}

\documentdate{2012-01-22 17:15:52.541680}

\courselevel{4}
\coursecredits{4,5}
\courseterm{Wintersemester}
\coursehours{3/0}
\courseinstructionlanguage{de}

\coursehead

% For index (key word@display). Key word is used for sorting - no Umlauts please.
\index{Enterprise Risk Management@Enterprise Risk Management}

% For later referencing
\label{cour_5109.dp_997}


\begin{styleenv}
\begin{assessment}
Die Erfolgskontrolle setzt sich zusammen aus einer mündlichen Prüfung (nach §4(2), 2 SPO) und Vorträgen und Ausarbeitungen im Rahmen der Veranstaltung (nach §4(2), 3 SPO).

 

Die Note setzt sich zu je 50\% aus den Vortragsleistungen (inkl. Ausarbeitungen) und der mündlichen Prüfung zusammen.


\end{assessment}

\begin{conditions}Keine.\end{conditions}


\end{styleenv}

\begin{learningoutcomes}
Unternehmerische Risiken identifizieren, analysieren und bewerten können sowie darauf aufbauend geeignete Strategien und Maßnahmenbündel entwerfen, die das unternehmensweite Chancen- und Gefahrenpotential optimieren, unter Berücksichtigung bereichsspezifischer Ziele, Risikotragfähigkeit und –akzeptanz.


\end{learningoutcomes}

\begin{content}
Diese Einführung in das Risikomanagement von (Industrie)Unternehmen soll ein umfassendes Verständnis für die Herausforderungen unternehmerischer Tätigkeit schaffen. Risiko wird dabei als Chance \emph{und} Gefährdung konzipiert; beides muss identifiziert, analysiert und vor dem Hintergrund der gesetzten Unternehmensziele sowie der wirtschaftlichen, rechtlichen oder ökologischen Rahmenbedingungen bewertet werden, bevor entschieden werden kann, welche risikopolitischen Maßnahmen optimal sind.

 

Nach Vermittlung konzeptioneller Grundlagen und einer kurzen Wiederholung der betriebswirtschaftlichen Entscheidungslehre werden Ziele, Strategien und Maßnahmen des Risikomanagements in Unternehmen vorgestellt. Schwerpunkte bilden die Schadenfinanzierung durch Versicherung, die Gestaltung der Risikomanagement-Kultur und die Organisation des Risikomanagements


\end{content}



\begin{literature}\begin{itemize}\item K. Hoffmann. Risk Management - Neue Wege der betrieblichen Risikopolitik. 1985.  \item R. Hölscher, R. Elfgen. Herausforderung Risikomanagement. Identifikation, Bewertung und Steuerung industrieller Risiken. Wiesbaden 2002.  \item W. Gleissner, F. Romeike. Risikomanagement - Umsetzung, Werkzeuge, Risikobewertung. Freiburg im Breisgau 2005.  \item H. Schierenbeck (Hrsg.). Risk Controlling in der Praxis. Zürich 2006.  \end{itemize}

\textbf{Weiterführende Literatur:}

 

Erweiterte Literaturangaben werden in der Vorlesung bekannt gegeben.

\end{literature}

\begin{remarks}Aus organisatorischen Gründen ist für die Teilnahme an der Veranstaltung eine Anmeldung erforderlich im Sekretariat des Lehrstuhls:thomas.mueller3@kit.edu.

\end{remarks}

\end{course}