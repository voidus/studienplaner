% Lehrveranstaltungsbeschreibung
% Informationsgrad : extern
% Sprache: de
\begin{course}

\setdoclanguagegerman
\coursedegreeprogramme{Informatik}
\coursemodulename{Algebra (S.~\pageref{mod_3107.dp_997})[IN3MATHAG05]}
\courseID{1031}
\coursename{Algebra}
\coursecoordination{F. Herrlich, S. Kühnlein, C. Schmidt, G. Weitze-Schmithüsen}

\documentdate{2011-10-06 17:47:50.825529}

\courselevel{}
\coursecredits{9}
\courseterm{Wintersemester}
\coursehours{4/2}
\courseinstructionlanguage{}

\coursehead

% For index (key word@display). Key word is used for sorting - no Umlauts please.
\index{Algebra@Algebra}

% For later referencing
\label{cour_7973.dp_997}


\begin{styleenv}
\begin{assessment}
Die Erfolgskontrolle wird in der Modulbeschreibung erläutert.


\end{assessment}

\begin{conditions}Keine.\end{conditions}

\begin{recommendations}Folgende Module sollten bereits belegt worden sein (Empfehlung):\newline
Lineare Algebra 1+2\newline
Analysis 1+2\newline
Einführung in Algebra und Zahlentheorie

\end{recommendations}
\end{styleenv}

\begin{learningoutcomes}
\begin{itemize}\item Konzepte und Methoden der Algebra  \item Vorbereitung auf Seminare und weiterführende Vorlesungen im Bereich Algebraische Geometrie und Zahlentheorie  \end{itemize}
\end{learningoutcomes}

\begin{content}
\begin{itemize}\item Körper: \newline
Körpererweiterungen, Galoistheorie, Einheitswurzeln und Kreisteilung  \item Bewertungen: \newline
Beträge, Bewertungsringe, Betragsfortsetzung, lokale Körper  \item Dedekindringe: \newline
ganze Ringerweiterungen, Normalisierung, noethersche Ringe  \end{itemize}
\end{content}







\end{course}