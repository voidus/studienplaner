% Lehrveranstaltungsbeschreibung
% Informationsgrad : extern
% Sprache: de
\begin{course}

\setdoclanguagegerman
\coursedegreeprogramme{Informatik}
\coursemodulename{Sicherheit (S.~\pageref{mod_2963.dp_997})[IN3INSICH]}
\courseID{24941}
\coursename{Sicherheit}
\coursecoordination{J. Müller-Quade}

\documentdate{2011-11-14 11:27:48.865137}

\courselevel{4}
\coursecredits{6}
\courseterm{Sommersemester}
\coursehours{3/1}
\courseinstructionlanguage{de}

\coursehead

% For index (key word@display). Key word is used for sorting - no Umlauts please.
\index{Sicherheit@Sicherheit}

% For later referencing
\label{cour_7373.dp_997}


\begin{styleenv}
\begin{assessment}
Die Erfolgskontrolle wird in der Modulbeschreibung erläutert.


\end{assessment}

\begin{conditions}Keine.\end{conditions}


\end{styleenv}

\begin{learningoutcomes}
Der /die Studierende

 \begin{itemize}\item kennt die theoretischen Grundlagen sowie grundlegende Sicherheitsmechanismen aus der Computersicherheit und der Kryptographie,  \item versteht die Mechanismen der Computersicherheit und kann sie erklären,  \item liest und versteht aktuelle wissenschaftliche Artikel,  \item beurteilt die Sicherheit gegebener Verfahren und erkennt Gefahren,  \item wendet Mechanismen der Computersicherheit in neuem Umfeld an.  \end{itemize}
\end{learningoutcomes}

\begin{content}
\begin{itemize}\item Theoretische und praktische Aspekte der Computersicherheit  \item Erarbeitung von Schutzzielen und Klassifikation von Bedrohungen  \item Vorstellung und Vergleich verschiedener formaler Access-Control-Modelle  \item Formale Beschreibung von Authentifikationssystemen, Vorstellung und Vergleich verschiedener Authentifikationsmethoden (Kennworte, Biometrie, Challenge-Response-Protokolle)  \item Analyse typischer Schwachstellen in Programmen und Web-Applikationen sowie Erarbeitung geeigneter Schutzmassnahmen/Vermeidungsstrategien  \item Einführung in Schlüsselmanagement und Public-Key-Infrastrukturen  \item Vorstellung und Vergleich gängiger Sicherheitszertifizierungen  \item Blockchiffren, Hashfunktionen, elektronische Signatur, Public-Key-Verschlüsselung bzw. digitale Signatur (RSA,ElGamal) sowie verschiedene Methoden des Schlüsselaustauschs (z.B. Diffie-Hellman)  \item Einführung in beweisbare Sicherheit mit einer Vorstellung der grundlegenden Sicherheitsbegriffe (wie IND-CCA)  \item Darstellung von Kombinationen kryptographischer Bausteine anhand aktuell eingesetzter Protokolle wie Secure Shell (SSH) und Transport Layer Security (TLS)  \end{itemize}
\end{content}







\end{course}