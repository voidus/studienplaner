% Lehrveranstaltungsbeschreibung
% Informationsgrad : extern
% Sprache: de
\begin{course}

\setdoclanguagegerman
\coursedegreeprogramme{Informatik}
\coursemodulename{Kurven und Flächen im CAD (S.~\pageref{mod_15887.dp_997})[IN3INKFC]}
\courseID{24626}
\coursename{Kurven und Flächen im CAD}
\coursecoordination{H. Prautzsch}

\documentdate{2012-01-27 09:12:10.099021}

\courselevel{3}
\coursecredits{9}
\courseterm{Winter-/Sommersemester}
\coursehours{4/2}
\courseinstructionlanguage{}

\coursehead

% For index (key word@display). Key word is used for sorting - no Umlauts please.
\index{Kurven und Flaechen im CAD@Kurven und Flächen im CAD}

% For later referencing
\label{cour_15889.dp_997}


\begin{styleenv}
\begin{assessment}
Die Erfolgskontrolle wird in der Modulbeschreibung erläutert.


\end{assessment}

\begin{conditions}Keine.\end{conditions}


\end{styleenv}

\begin{learningoutcomes}
Siehe Modulbeschreibung.


\end{learningoutcomes}

\begin{content}
Siehe Modulbeschreibung.


\end{content}

\begin{media}Tafel, Folien

\end{media}

\begin{literature}Prautzsch, Boehm, Paluszny. Bézier and B-spline techniques. Springer 2002

\end{literature}

\begin{remarks}Die Veranstaltung findet in Deutsch/Englisch statt.

\end{remarks}

\end{course}