% Lehrveranstaltungsbeschreibung
% Informationsgrad : extern
% Sprache: de
\begin{course}

\setdoclanguagegerman
\coursedegreeprogramme{Informatik}
\coursemodulename{Grundlagen der Nachrichtentechnik (S.~\pageref{mod_3965.dp_997})[IN3EITGNT]}
\courseID{23616}
\coursename{Communication Systems and Protocols }
\coursecoordination{Leuthold, Becker, Hübner}

\documentdate{2011-12-06 14:34:36.056060}

\courselevel{3}
\coursecredits{5}
\courseterm{Sommersemester}
\coursehours{2/1}
\courseinstructionlanguage{en}

\coursehead

% For index (key word@display). Key word is used for sorting - no Umlauts please.
\index{Communication Systems and Protocols @Communication Systems and Protocols }

% For later referencing
\label{cour_8099.dp_997}


\begin{styleenv}
\begin{assessment}
Die Erfolgskontrolle erfolgt in Form einer schriftlichen Prüfung im Umfang von i.d.R. 120 Minuten nach § 4 Abs. 2 Nr. 1 SPO.

 

Die Note der Lehrveranstaltung ist die Note der schriftlichen Prüfung.


\end{assessment}

\begin{conditions}Die Vorlesung baut auf Kenntnissen der Vorlesungen „Grundlagen der Digitaltechnik“ (Lehrveranstaltung Nr. 23615) auf.

\end{conditions}


\end{styleenv}

\begin{learningoutcomes}
Ziel dieser Vorlesung ist es, Begriffe und grundlegende Konzepte dieser Übertragungsmethoden einzuführen und gemeinsame Aspekte herauszuarbeiten. Beispielhaft wird auf einige typische und weit verbreitete Lösungen eingegangen.


\end{learningoutcomes}

\begin{content}
Diese Vorlesung für Elektrotechniker und Informationstechniker gibt einen Einblick in Theorie und Praxis des Datenaustausches innerhalb und zwischen Computern sowie dedizierten Kommunikationsgeräten. Die verschiedenen Ebenen der Datenkommunikation werden erläutert, wobei der Bogen von hochintegrierten Verbindungen unterschiedlicher Komponenten auf Mikrochips über rechnerinterne Systembusse bis hin zu Weitverkehrsnetzwerken gespannt wird.

 

Neben dem wichtigen Kriterium der Geschwindigkeit, bzw. der Übertragungsleistung eines Kommunikationssystems werden noch zusätzlich Sicherheitsaspekte oder die Kosten beim Systementwurf betrachtet. Es werden Beschreibungen aktueller Implementierungen behandelt, unter anderem serielle und parallele Schnittstellen, die Busse PCI, SCSI, FireWire, USB, IEC, CAN und AMBA.


\end{content}

\begin{media}Folien, Tafel

\end{media}

\begin{literature}Wird in der Vorlesung bekanntgegeben.

 

\textbf{Weiterführende Literatur:}

 

Wird in der Vorlesung bekanntgegeben.

\end{literature}

\begin{remarks}\textcolor{red}{Die Lehrveranstaltung wurde von \emph{\textbf{Kommunikationssyteme und Protokolle} }in \textbf{\emph{Communication Systems and Protocols}} umbenannt und in englisch gehalten.}

\end{remarks}

\end{course}