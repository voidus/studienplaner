% Modulbeschreibung 
% Informationsgrad : extern
% Sprache: de
\begin{module}

\setdoclanguagegerman
\moduledegreeprogramme{Informatik (B.Sc.)}
\modulesubject{EF Betriebswirtschaftslehre}
\moduleID{IN3WWBWL3}
\modulename{Essentials of Finance}
\modulecoordination{M. Uhrig-Homburg, M. Ruckes}

\documentdate{2011-12-23 18:37:06.105164}

\modulecredits{9}
\moduleduration{1}
\modulecycle{Jedes 2. Semester, Sommersemester}



\modulehead

% For index (key word@display). Key word is used for sorting - no Umlauts please.
\index{Essentials of Finance@Essentials of Finance (M)}

% For later referencing
\label{mod_1541.dp_997}

\begin{courselist}
2530575 & Investments (S.~\pageref{cour_6789.dp_997}) & 2/1 & S & 4,5 & M. Uhrig-Homburg\\
2530216 & Financial Management (S.~\pageref{cour_6835.dp_997}) & 2/1 & S & 4,5 & M. Ruckes\\
\end{courselist}

\begin{styleenv}
\begin{assessment}
Die Modulprüfung erfolgt von schriftlichen Teilprüfungen (nach §4(2), 1 SPO) über die einzelnen Lehrveranstaltungen des Moduls. Die Prüfungen werden in jedem Semester angeboten und können zu jedem ordentlichen Prüfungstermin wiederholt werden. Die Erfolgskontrolle wird bei jeder Lehrveranstaltung dieses Moduls beschrieben.

 

Die Gesamtnote des Moduls wird aus den mit LP gewichteten Noten der Teilprüfungen gebildet und nach der ersten Nachkommastelle abgeschnitten.


\end{assessment}

\begin{conditions}Nur prüfbar in Kombination mit dem Modul \emph{Grundlagen der BWL}.

 

Das Modul ist nur zusammen mit dem Pflichtmodul \emph{Grundlagen der BWL} [IN3WWBWL] prüfbar.

\end{conditions}


\end{styleenv}

\begin{learningoutcomes}
Der/die Studierende

 \begin{itemize}\item besitzt grundlegende Kenntnisse in moderner Finanzwirtschaft,  \item besitzt grundlegende Kenntnisse zur Fundierung von Investitionsentscheidungen auf Aktien-, Renten- und Derivatemärkten,  \item wendet konkrete Modelle zur Beurteilung von Investitionsentscheidungen auf Finanzmärkten sowie für Investitions- und Finanzierungsentscheidungen von Unternehmen an.  \end{itemize}
\end{learningoutcomes}

\begin{content}
Das Modul \emph{Essentials of Finance} beschäftigt sich mit den grundlegenden Fragestellungen der modernen Finanzwirtschaft. In den Lehrveranstaltungen werden die Grundfragen der Bewertung von Aktien diskutiert. Ein weiterer Schwerpunkt ist die Vermittlung der modernen Portfoliotheorie und analytischer Methoden der Investitionsrechnung und Unternehmensfinanzierung.


\end{content}



\end{module}

