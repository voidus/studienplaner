% Modulbeschreibung 
% Informationsgrad : extern
% Sprache: de
\begin{module}

\setdoclanguagegerman
\moduledegreeprogramme{Informatik (B.Sc.)}
\modulesubject{EF Informationsmanagement im Ingenieurwesen}
\moduleID{IN3MACHCADP}
\modulename{CAD-Praktika CATIA V5}
\modulecoordination{}

\documentdate{2011-07-25 11:21:15.980675}

\modulecredits{2}
\moduleduration{1}
\modulecycle{Jedes Semester}



\modulehead

% For index (key word@display). Key word is used for sorting - no Umlauts please.
\index{CAD-Praktika CATIA V5@CAD-Praktika CATIA V5 (M)}

% For later referencing
\label{mod_14453.dp_997}

\begin{courselist}
2123356 & CAD-Praktika CATIA V5 (S.~\pageref{cour_14459.dp_997}) & 3 &  & 2 & \\
\end{courselist}

\begin{styleenv}
\begin{assessment}

\end{assessment}

\begin{conditions}Die Module \textbf{\emph{Product Lifecycle Managemet}} [IN3MACHPLM] und \textbf{\emph{Technische Informationssysteme}} [IN3INMACHTI] müssen im Ergänzungsfach Informationsmanagement im Ingenieurwesen geprüft werden.

\end{conditions}


\end{styleenv}

\begin{learningoutcomes}

\end{learningoutcomes}

\begin{content}

\end{content}

\begin{remarks}Dieses Modul beinhaltet ein einwöchiges Blockpraktikum in der vorlesungsfreien Zeit.

\end{remarks}

\end{module}

