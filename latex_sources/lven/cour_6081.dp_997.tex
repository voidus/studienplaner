% Lehrveranstaltungsbeschreibung
% Informationsgrad : extern
% Sprache: de
\begin{course}

\setdoclanguagegerman
\coursedegreeprogramme{Informatik}
\coursemodulename{Grundlagen der BWL (S.~\pageref{mod_3033.dp_997})[IN3WWBWL]}
\courseID{2600024}
\coursename{Allgemeine Betriebswirtschaftslehre B}
\coursecoordination{M. Ruckes, W. Fichtner, M. Klarmann, Th. Lützkendorf, F. Schultmann}

\documentdate{2012-01-10 13:04:14.369760}

\courselevel{1}
\coursecredits{4}
\courseterm{Sommersemester}
\coursehours{2/0/2}
\courseinstructionlanguage{de}

\coursehead

% For index (key word@display). Key word is used for sorting - no Umlauts please.
\index{Allgemeine Betriebswirtschaftslehre B@Allgemeine Betriebswirtschaftslehre B}

% For later referencing
\label{cour_6081.dp_997}


\begin{styleenv}
\begin{assessment}
Die Erfolgskontrolle erfolgt in Form einer schriftlichen Prüfung (Klausur) im Umfang von 90 min nach § 4 Abs. 2 Nr. 1 SPO.

 

Die Prüfung wird in jedem Semester angeboten und kann zu jedem ordentlichen Prüfungstermin wiederholt werden.


\end{assessment}

\begin{conditions}Keine.\end{conditions}


\end{styleenv}

\begin{learningoutcomes}
Ziel der Vorlesung und der sie begleitenden Tutorien ist es, den Studierenden Grundkenntnisse und Basiswissen im Bereich des Marketing und der Produktionswirtschaft zu vermitteln. Die Entscheidungsfindung für die BWL-Module im Vertiefungsteil des Bachelorstudiums soll auf dieser Grundlage erleichtert werden.


\end{learningoutcomes}

\begin{content}
Die Lehrveranstaltung setzt sich zusammen aus den Teilgebieten:

 

\textbf{Marketing}

 

\textbf{Produktionswirtschaft:}

 

Dieses Teilgebiet vermittelt eine erste Einführung in sämtliche betriebliche Aufgaben, die mit der Erzeugung materieller und immaterieller Güter zusammenhängen. Neben dem verarbeitenden Gewerbe (Grundstoff- und Produktionsgütergewerbe, Investitionsgüter bzw. Verbrauchsgüter produzierendes Gewerbe, Nahrungs- und Genussmittelgewerbe, d.h. Produktionswirtschaft i.e.S.) werden die Bereiche Energiewirtschaft, Bau- und Immobilienwirtschaft sowie die Arbeitswissenschaften betrachtet.

 

Behandelte Themen im Einzelnen:

 \begin{itemize}\item Einführung in das Teilgebiet (systemtheoretische Einordnung, allgemeine Aufgaben, Querschnittsthemen)  \item Industrielle Produktion (Standortplanung, Transportplanung, Beschaffung, Anlagenwirtschaft, Produktionsmanagement)  \item Elektrizitätswirtschaft (Energiebedarf und Energieversorgung, Energiesystemplanung, Technological Foresight, Kostenstrukturen)  \item Bau- und Immobilienwirtschaft  \end{itemize}
\end{content}



\begin{literature}Ausführliche Literaturhinweise werden gegeben in den Materialen zur Vorlesung BWL B.

\end{literature}

\begin{remarks}Die Schlüsselqualifikation umfasst die aktive Beteiligung in den Tutorien durch Präsentation eigener Lösungen und Einbringung von Diskussionsbeiträgen.

 

Die Teilgebiete werden von den jeweiligen BWL-Fachvertretern präsentiert. Ergänzt wird die Vorlesung durch begleitende Tutorien.

\end{remarks}

\end{course}