% Modulbeschreibung 
% Informationsgrad : extern
% Sprache: de
\begin{module}

\setdoclanguagegerman
\moduledegreeprogramme{Informatik (B.Sc.)}
\modulesubject{EF Betriebswirtschaftslehre}
\moduleID{IN3WWBWL13}
\modulename{Topics in Finance I}
\modulecoordination{M. Uhrig-Homburg, M. Ruckes}

\documentdate{2012-01-10 12:53:14.159067}

\modulecredits{9}
\moduleduration{1}
\modulecycle{Jedes Semester}



\modulehead

% For index (key word@display). Key word is used for sorting - no Umlauts please.
\index{Topics in Finance I@Topics in Finance I (M)}

% For later referencing
\label{mod_1575.dp_997}

\begin{courselist}
2530210 & Interne Unternehmensrechnung (Rechnungswesen II) (S.~\pageref{cour_6801.dp_997}) & 2/1 & S & 4,5 & T. Lüdecke\\
2530232 & Finanzintermediation (S.~\pageref{cour_6749.dp_997}) & 3 & W & 4,5 & M. Ruckes\\
2530550 & Derivate (S.~\pageref{cour_4501.dp_997}) & 2/1 & S & 4,5 & M. Uhrig-Homburg\\
2530296 & Börsen (S.~\pageref{cour_6289.dp_997}) & 1 & S & 1,5 & J. Franke\\
2530299 & Geschäftspolitik der Kreditinstitute (S.~\pageref{cour_6423.dp_997}) & 2 & W & 3 & W. Müller\\
2530570 & Internationale Finanzierung (S.~\pageref{cour_6447.dp_997}) & 2 & S & 3 & M. Uhrig-Homburg, Walter\\
2540454 & eFinance: Informationswirtschaft für den Wertpapierhandel (S.~\pageref{cour_4851.dp_997}) & 2/1 & W & 4,5 & R. Riordan\\
2561129 & Spezielle Steuerlehre (S.~\pageref{cour_10165.dp_997}) & 3 & W & 4,5 & B. Wigger\\
\end{courselist}

\begin{styleenv}
\begin{assessment}
Die Modulprüfung erfolgt in Form von Teilprüfungen (nach §4(2) SPO) über die gewählten Lehrveranstaltungen des Moduls, mit denen in Summe die Mindestanforderung an LP erfüllt wird. Die Teilprüfungen werden zu Beginn der vorlesungsfreien Zeit des Semesters angeboten. Wiederholungsprüfungen sind zu jedem ordentlichen Prüfungstermin möglich. Die Erfolgskontrolle wird bei jeder Lehrveranstaltung dieses Moduls beschrieben.

 

Die Gesamtnote des Moduls wird aus den mit LP gewichteten Noten der Teilprüfungen gebildet und nach der ersten Nachkommastelle abgeschnitten.


\end{assessment}

\begin{conditions}Nur in Verbindung mit dem Modul \emph{Grundlagen der BWL} prüfbar.

 \end{conditions}


\end{styleenv}

\begin{learningoutcomes}
Der/die Studierende

 \begin{itemize}\item besitzt weiterführende Kenntnisse in moderner Finanzwirtschaft  \item wendet diese Kenntnisse in den Bereichen Finanz- und Rechnungswesen, Finanzmärkte und Banken in der beruflichen Praxis an.  \end{itemize}
\end{learningoutcomes}

\begin{content}
Das Modul \emph{Topics in Finance} I baut inhaltlich auf dem Modul \emph{Essentials of Finance} auf. In den Veranstaltungen werden weiterführende Fragestellungen aus den Bereichen Finanz- und Rechnungswesen, Finanzmärkte und Banken aus theoretischer und praktischer Sicht behandelt.


\end{content}



\end{module}

