% Modulbeschreibung 
% Informationsgrad : extern
% Sprache: de
\begin{module}

\setdoclanguagegerman
\moduledegreeprogramme{Informatik (B.Sc.)}
\modulesubject{}
\moduleID{IN3INAMB}
\modulename{Analyse und Modellierung menschlicher Bewegungsabläufe}
\modulecoordination{T. Schultz}

\documentdate{2012-01-13 09:40:05.593641}

\modulecredits{3}
\moduleduration{1}
\modulecycle{Jedes 2. Semester, Wintersemester}



\modulehead

% For index (key word@display). Key word is used for sorting - no Umlauts please.
\index{Analyse und Modellierung menschlicher Bewegungsablaeufe@Analyse und Modellierung menschlicher Bewegungsabläufe (M)}

% For later referencing
\label{mod_10169.dp_997}

\begin{courselist}
ammb & Analyse und Modellierung menschlicher Bewegungsabläufe (S.~\pageref{cour_7019.dp_997}) & 2 & W & 3 & T. Schultz\\
\end{courselist}

\begin{styleenv}
\begin{assessment}
Die Erfolgskontrolle erfolgt in Form einer mündlichen Prüfung im Umfang von i.d.R. 15 Minuten nach § 4 Abs. 2 Nr. 2 der SPO.

 

Die Modulnote ist die Note der mündlichen Prüfung.

 

Terminvereinbarung bitte per E-Mail an: helga.scherer@kit.edu. Es wird empfohlen, sich frühzeitig um einen Prüfungstermin zu kümmern. 


\end{assessment}

\begin{conditions}Keine.\end{conditions}


\end{styleenv}

\begin{learningoutcomes}
\begin{itemize}\item Es soll ein breiter Überblick über das behandelte Arbeitsgebiet vermittelt werden.  \item Die Studierenden sollen lernen, die fallspezifischen vorgestellten Methodiken auch auf verallgemeinerte oder modifizierte Szenarien zu übertragen.  \item Die Studierenden werden in die Grundlagen der Datenverarbeitung menschlicher Bewegungen eingeführt und erhalten dabei einen  \item Einblick in die Zusammenhänge und Abfolgen der verschiedenen Prozessschritte.  \item Die Studierenden lernen, Probleme im Bereich der Bewegungserfassung, der Erkennung und der Generierung zu analysieren, zu strukturieren und formal zu beschreiben.  \item Ein Ziel der Veranstaltung ist es, die Studierenden dazu anzuregen, die vorgestellten Methoden durch weiterführende Studien eigenständig umsetzen und auf andere Szenarien und Aufgaben zu übertragen  \end{itemize}
\end{learningoutcomes}

\begin{content}
Die Vorlesung bietet eine Einführung in die Grundlagen der Analyse und Modellierung menschlicher Bewegungsabläufe auf der Basis aufgezeichneter Bewegungssequenzen. Dabei werden Zielsetzungen der Bewegungsanalyse besprochen, die sich über sehr unterschiedliche Gebiete erstrecken. Im Hinblick auf die dargelegten Zielsetzungen werden die Grundlagen der jeweils notwendigen Datenverarbeitungsschritte erläutert. Diese umfassen im Wesentlichen die Methoden der Aufzeichnung und Verarbeitung von Bewegungssequenzen, sowie die Modellierung der Bewegung aus biomechanischer und kinematischer Sicht. Zur statistischen Modellierung und Erkennung von Bewegungen werden die Hidden-Markov-Modelle vorgestellt. Die Ausführungen werden anhand aktueller Forschungsarbeiten veranschaulicht.


\end{content}

\begin{remarks}\textcolor{red}{Das Modul wird nicht mehr angeboten, Prüfungen sind möglich bis WS 2012/13 möglich.}

\end{remarks}

\end{module}

