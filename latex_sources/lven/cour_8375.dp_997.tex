% Lehrveranstaltungsbeschreibung
% Informationsgrad : extern
% Sprache: de
\begin{course}

\setdoclanguagegerman
\coursedegreeprogramme{Informatik}
\coursemodulename{Makroökonomische Theorie (S.~\pageref{mod_2451.dp_997})[IN3WWVWL8]}
\courseID{25549}
\coursename{Konjunkturtheorie (Theory of Business Cycles)}
\coursecoordination{M. Hillebrand}

\documentdate{2012-01-09 16:31:33.855213}

\courselevel{3}
\coursecredits{4,5}
\courseterm{Wintersemester}
\coursehours{2/1}
\courseinstructionlanguage{en}

\coursehead

% For index (key word@display). Key word is used for sorting - no Umlauts please.
\index{Konjunkturtheorie (Theory of Business Cycles)@Konjunkturtheorie (Theory of Business Cycles)}

% For later referencing
\label{cour_8375.dp_997}


\begin{styleenv}
\begin{assessment}
Die Erfolgskontrolle erfolgt in Abhängigkeit der Teilnehmerzahl in Form einer schriftlichen (60min.) oder mündlichen (20min.) Prüfung (nach §4(2), 1 o. 2) zu Beginn der vorlesungsfreien Zeit des Semesters.

 

Die Prüfung wird in jedem Semester angeboten und kann zu jedem ordentlichen Prüfungstermin wiederholt werden.


\end{assessment}

\begin{conditions}Keine.\end{conditions}

\begin{recommendations}Grundlegende mikro- und makroökonomische Kenntnisse, wie sie beispielsweise in den Veranstaltungen \emph{Volkswirtschaftslehre I (Mikroökonomie)} [2600012] und \emph{Volkswirtschaftslehre II (Makroökonomie)} [2600014] vermittelt werden, werden vorausgesetzt.

 

Aufgrund der inhaltlichen Ausrichtung der Veranstaltung wird ein Interesse an quantitativ-mathematischer Modellierung vorausgesetzt.

\end{recommendations}
\end{styleenv}

\begin{learningoutcomes}
Der/die Studierende

 \begin{itemize}\item ist in der Lage, mit Hilfe eines analytischen Instrumentariums grundlegende Fragestellungen der Makroökonomie zu bearbeiten,  \item kann sich selbstständig ein fundiertes Urteil über ökonomische Fragestellungen bilden.  \end{itemize}
\end{learningoutcomes}

\begin{content}
Im Rahmen der Vorlesung werden Modelle zur Erklärung gesamtwirtschaftlicher Fluktuationen und möglicher Ungleichgewichtssituationen auf Güter–, Arbeits- und Finanzmärkten betrachtet. \newline
Die dabei erlernten Techniken werden speziell zur Analyse von geld- und fiskalpolitischen Maßnahmen im Hinblick auf makroökonomische Schlüsselvariablen wie Volkseinkommen (BIP), Beschäftigung und Inflation untersucht.


\end{content}



\begin{literature}\textbf{Weiterführende Literatur:}

 

David Romer, \emph{Advanced Macroeconomics}, 3rd edition, MaGraw-Hill (2006)

 

Lutz Arnold: Makroökonomik. Eine Einführung in die Theorie der Güter-, Arbeits- und Finanzmärkte (2003)

\end{literature}

\begin{remarks}Nach Absprache mit den Studierenden besteht die Möglichkeit, die Lehrveranstaltung in englischer Sprache zu halten.

\end{remarks}

\end{course}