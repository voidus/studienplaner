% Lehrveranstaltungsbeschreibung
% Informationsgrad : extern
% Sprache: de
\begin{course}

\setdoclanguagegerman
\coursedegreeprogramme{Informatik}
\coursemodulename{Wahrscheinlichkeitstheorie (S.~\pageref{mod_3639.dp_997})[IN3MATHST02]}
\courseID{1598}
\coursename{Wahrscheinlichkeitstheorie}
\coursecoordination{N. Bäuerle, N. Henze, B. Klar, G. Last}

\documentdate{2011-10-06 18:01:56.021160}

\courselevel{}
\coursecredits{6}
\courseterm{Sommersemester}
\coursehours{3/1}
\courseinstructionlanguage{}

\coursehead

% For index (key word@display). Key word is used for sorting - no Umlauts please.
\index{Wahrscheinlichkeitstheorie@Wahrscheinlichkeitstheorie}

% For later referencing
\label{cour_8039.dp_997}


\begin{styleenv}
\begin{assessment}
Die Erfolgskontrolle wird in der Modulbeschreibung erläutert.


\end{assessment}

\begin{conditions}Keine.\end{conditions}

\begin{recommendations}Folgende Module sollten bereits belegt worden sein (Empfehlung):\newline
Analysis 3\newline
Einführung in die Stochastik

\end{recommendations}
\end{styleenv}

\begin{learningoutcomes}
Die Studierenden sollen am Ende des Moduls:

 \begin{itemize}\item mit modernen wahrscheinlichkeitstheoretischen Methoden vertraut sein,  \item Grundlagen für die Stochastik, Statistik und die moderne Finanzmathematik erworben haben.  \end{itemize}
\end{learningoutcomes}

\begin{content}
\begin{itemize}\item Maß-Integral  \item Monotone und majorisierte Konvergenz  \item Lemma von Fatou  \item Nullmengen u. Maße mit Dichten  \item Satz von Radon-Nikodym  \item Produkt-$\sigma{}$-Algebra  \item Familien von unabhängigen Zufallsvariablen  \item Transformationssatz für Dichten  \item Schwache Konvergenz  \item Charakteristische Funktion  \item Zentraler Grenzwertsatz  \item Bedingte Erwartungswerte  \item Zeitdiskrete Martingale und Stoppzeiten  \end{itemize}
\end{content}







\end{course}