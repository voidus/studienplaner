% Modulbeschreibung 
% Informationsgrad : extern
% Sprache: de
\begin{module}

\setdoclanguagegerman
\moduledegreeprogramme{Informatik (B.Sc.)}
\modulesubject{EF E-Technik}
\moduleID{IN3EITDSP}
\modulename{Praktikum Digitale Signalverarbeitung}
\modulecoordination{Puente }

\documentdate{2012-01-17 15:44:28.352428}

\modulecredits{6}
\moduleduration{1}
\modulecycle{Jedes 2. Semester, Sommersemester}



\modulehead

% For index (key word@display). Key word is used for sorting - no Umlauts please.
\index{Praktikum Digitale Signalverarbeitung@Praktikum Digitale Signalverarbeitung (M)}

% For later referencing
\label{mod_15783.dp_997}

\begin{courselist}
23134 & Praktikum Digitale Signalverarbeitung (S.~\pageref{cour_15785.dp_997}) & 4 & S & 6 & F. Puente\\
\end{courselist}

\begin{styleenv}
\begin{assessment}
Die Erfolgskontrolle erfolgt in Form einer schriftlichen Prüfung nach § 4 Abs. 2 Nr. 1 SPO.

 

Die Modulnote ist die Note der schriftlichen Prüfung.


\end{assessment}

\begin{conditions}Keine.\end{conditions}

\begin{recommendations}Grundlagen Mathematik, Wahrscheinlichkeitstheorie, Grundlagen Signalverarbeitung

\end{recommendations}
\end{styleenv}

\begin{learningoutcomes}
Ziel ist die Anwendung zuvor erlernter theoretischer Grundlagen


\end{learningoutcomes}

\begin{content}
Dieses Praktikum richtet sich an Studenten der Elektro- und Informationstechnik in der Vertiefungsrichtung AI. Die erlernten theoretischen Grundlagen der digitalen Signalverarbeitung sollen im Rahmen dieses Praktikums anhand von derzeit acht Versuchen angewendet und das Verstandnis vertieft werden. Der erste Versuch dient als Einfuhrung in den Umgang mit den heutzutage unumganglichen Werkzeugen Matlab und LabVIEW und als Basis fur die weiterfuhrenden Versuche. Die weiteren Versuche beschaftigen sich mit den wesentlichen Inhalten der digitalen Signalverarbeitung.\newline
Als zweiter Versuch ist die Verwendung der Korrelationsmesstechnik zur Laufzeitmessung vorgesehen. Mittels zweier fest installierter optischer Sensoren werden Signale aufgenommen und mit Hilfe von Korrelationsfunktionen auf die Laufzeit von Schuttgut auf einem Forderband geschlossen.\newline
Ein weiterer Versuch dient der Untersuchung von Effekten, wie Aliasing, Leckeffekt und Quantisierungsrauschen, die im Zusammenhang mit der digitalen Messwerteerfassung auftreten.\newline
Eine bedeutende Stellung in der Signalverarbeitung kommt der Filterung zu. Diese kann sowohl analog als auch digital erfolgen. Beide Filtermethoden werden im Rahmen eines Versuchs betrachtet, wobei heutzutage die digitale Filterung, aufgrund der zahlreichen Vorteile im Vordergrund steht und somit auch Hauptbestandteil des Versuchs ist.\newline
Ein wichtiges Messverfahren ist die Doppler-Messtechnik. Diese soll im Rahmen dieses Versuchs zur Bestimmung der Stromungsgeschwindigkeit von roten Blutkorperchen angewendet werden. Da das aufgenommene Signal, bedingt durch die unterschiedlichen Geschwindigkeiten der einzelnen Blutkorperchen, ein komplettes Spektrum von Frequenzverschiebungen (Doppler-Spektrum) bildet, wird ein leistungsfahiger PC zur Auswertung in Echtzeit verwendet.\newline
Das Kalman-Filter ist ein machtiges Instrument der Signalverarbeitung und dient beispielsweise der Datenfusion mehrerer Sensoren. Eine mogliche Anwendung ist die Lokalisierung eines Fahrzeugs, wie sie in diesem Versuch durchgefuhrt werden soll. Als Sensoren dienen dabei Inkrementalgeber an den Radern, Beschleunigungssensoren fur die Langs- und Querbeschleunigung sowie ein Gierratensensor.


\end{content}



\end{module}

