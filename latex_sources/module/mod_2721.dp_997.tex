% Modulbeschreibung 
% Informationsgrad : extern
% Sprache: de
\begin{module}

\setdoclanguagegerman
\moduledegreeprogramme{Informatik (B.Sc.)}
\modulesubject{EF Betriebswirtschaftslehre}
\moduleID{IN3WWBWL14}
\modulename{Supply Chain Management}
\modulecoordination{S. Nickel}

\documentdate{2011-12-01 13:15:05.037626}

\modulecredits{9}
\moduleduration{1}
\modulecycle{Jedes Semester}



\modulehead

% For index (key word@display). Key word is used for sorting - no Umlauts please.
\index{Supply Chain Management@Supply Chain Management (M)}

% For later referencing
\label{mod_2721.dp_997}

\begin{courselist}
2590452 & Management of Business Networks (S.~\pageref{cour_4811.dp_997}) & 2/1 & W & 4,5 & C. Weinhardt, J. Kraemer\\
2540496 & Management of Business Networks (Introduction) (S.~\pageref{cour_8445.dp_997}) & 2 & W & 3 & C. Weinhardt, J. Kraemer\\
2550486 & Standortplanung und strategisches Supply Chain Management (S.~\pageref{cour_7813.dp_997}) & 2/1 & S & 4,5 & S. Nickel\\
2118078 & Logistik - Aufbau, Gestaltung und Steuerung von Logistiksystemen (S.~\pageref{cour_5631.dp_997}) & 3/1 & S & 6 & K. Furmans\\
2118090 & Quantitatives Risikomanagement von Logistiksystemen (S.~\pageref{cour_7063.dp_997}) & 3/1 & W & 6 &  A. Cardeneo\\
2550488 & Taktisches und operatives Supply Chain Management (S.~\pageref{cour_7815.dp_997}) & 2/1 & W & 4,5 & S. Nickel\\
\end{courselist}

\begin{styleenv}
\begin{assessment}
Die Erfolgskontrolle erfolgt in Form von Teilprüfungen (nach §4(2), 1-3 SPO) über die Lehrveranstaltungen des Moduls im Umfang von insgesamt 9 LP. Die Erfolgskontrolle wird bei jeder Lehrveranstaltung dieses Moduls beschrieben.

 

Die Gesamtnote des Moduls wird aus den mit LP gewichteten Noten der Teilprüfungen gebildet und nach der ersten Nachkommastelle abgeschnitten.


\end{assessment}

\begin{conditions}Nur prüfbar in Kombination mit dem Modul \emph{Grundlagen der BWL}.

 \end{conditions}

\begin{recommendations}Es wird empfohlen genau eine der beiden Lehrveranstaltungen

 \begin{itemize}\item \emph{Management of Business Networks}  \item \emph{Management of Business Networks (Introduction)}  \end{itemize}

zu belegen.

\end{recommendations}
\end{styleenv}

\begin{learningoutcomes}
 

Die Studierenden

 \begin{itemize}\item verstehen und bewerten aus strategischer und operativer Sicht die Steuerung von unternehmensübergreifenden Lieferketten,  \item analysieren die Koordinationsprobleme innerhalb der Lieferketten,  \item identifizieren und integrieren geeignete Informationssystemlandschaften zur Unterstützung der Lieferketten,  \item wenden theoretische Methoden aus dem Operations Research und dem Informationsmanagement an,  \item erarbeiten Lösungen in Teams.  \end{itemize}
\end{learningoutcomes}

\begin{content}
Das Modul “Supply Chain Management” vermittelt einen Überblick über die gegenseitigen Abhängigkeiten von unternehmensübergreifenden Lieferketten und Informationssystemen. Aus den Spezifika der Lieferketten und deren Informationsbedarf ergeben sich besondere Anforderungen an das betriebliche Informationsmanagement. In der Kernveranstaltung “Management of Business Networks” wird insbesondere auf die strategischen Aspekte des Managements von Lieferketten und der Informationsunterstützung abgezielt. Über den englischsprachigen Vorlesungsteil hinaus vermittelt der Kurs das Wissen anhand einer Fallstudie, die in enger Zusammenarbeit mit Professor Gregory Kersten an der Concordia University in Montreal, Kanada, ausgearbeitet wurde.Die Veranstaltung MBN Introduction behandelt nur den ersten Teil der regulären MBN und wird ohne die Bearbeitung der Fallstudie gewertet. In der vollständigen Version der Vorlesung hingegen wird weiterhin Wert auf die individuell betreute und interdisziplinäre Fallstudie gelegt.

 

Das Teilmodul wird durch ein Wahlfach abgerundet, welches geeignete Optimierungsmethoden für das Supply Chain Management bzw. moderne Logistikansätze adressiert.


\end{content}

\begin{remarks}Das geplante Vorlesungsangebot in den nächsten Semestern finden Sie auf den Webseiten der einzelnen Institute IISM, IFL und IOR.

\end{remarks}

\end{module}

