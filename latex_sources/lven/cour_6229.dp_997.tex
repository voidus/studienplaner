% Lehrveranstaltungsbeschreibung
% Informationsgrad : extern
% Sprache: de
\begin{course}

\setdoclanguagegerman
\coursedegreeprogramme{Informatik}
\coursemodulename{Energiebewusste Systeme (S.~\pageref{mod_10583.dp_997})[IN3INEBS]}
\courseID{24127}
\coursename{Power Management}
\coursecoordination{F. Bellosa}

\documentdate{2011-11-14 11:34:10.620192}

\courselevel{4}
\coursecredits{3}
\courseterm{Wintersemester}
\coursehours{2}
\courseinstructionlanguage{de}

\coursehead

% For index (key word@display). Key word is used for sorting - no Umlauts please.
\index{Power Management@Power Management}

% For later referencing
\label{cour_6229.dp_997}


\begin{styleenv}
\begin{assessment}
Die Erfolgskontrolle wird in der Modulbeschreibung erläutert.


\end{assessment}

\begin{conditions}Keine.\end{conditions}


\end{styleenv}

\begin{learningoutcomes}
Der Student soll Mechanismen und Strategien zur Verwaltung der Ressource Energie in Rechnersystemen kennen. Er soll zum einen Kenntnisse erwerben über die verschiedenen Möglichkeiten, welche die Hardware bietet um ihren Energieverbrauch zu beeinflussen, sowie über die Auswirkungen, die dies auf die Performance hat. Weiter soll er verstehen, welche Möglichkeiten das Betriebssystem besitzt, Informationen über Energiezustände und Energieverbrauch der Hardware zu erlangen und wie der Energieverbrauch dem jeweiligen Verursacher, z.B. einzelnen Anwendungen, zugeordnet werden kann.


\end{learningoutcomes}

\begin{content}
Inhalt:

 \begin{itemize}\item CPU Power Management  \item Thermal Management  \item Memory Power Management  \item I/O Power Management  \item Battery Power Management  \item Cluster Power Management  \end{itemize}
\end{content}

\begin{media}Vorlesungsfolien in englischer Sprache

\end{media}





\end{course}