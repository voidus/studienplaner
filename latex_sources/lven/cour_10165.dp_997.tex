% Lehrveranstaltungsbeschreibung
% Informationsgrad : extern
% Sprache: de
\begin{course}

\setdoclanguagegerman
\coursedegreeprogramme{Informatik}
\coursemodulename{Topics in Finance I (S.~\pageref{mod_1575.dp_997})[IN3WWBWL13]}
\courseID{2561129}
\coursename{Spezielle Steuerlehre}
\coursecoordination{B. Wigger}

\documentdate{2011-12-16 15:10:39.269993}

\courselevel{3}
\coursecredits{4,5}
\courseterm{Wintersemester}
\coursehours{3}
\courseinstructionlanguage{de}

\coursehead

% For index (key word@display). Key word is used for sorting - no Umlauts please.
\index{Spezielle Steuerlehre@Spezielle Steuerlehre}

% For later referencing
\label{cour_10165.dp_997}


\begin{styleenv}
\begin{assessment}
Die Erfolgskontrolle erfolgt in Form einer schriftlichen Prüfung (Klausur) im Umfang von 1h nach § 4, Abs. 2, 1 SPO. Die Note entspricht der Note der schriftlichen Prüfung.


\end{assessment}

\begin{conditions}Keine.\end{conditions}

\begin{recommendations}Es werden Kenntnisse über die Erhebung staatlicher Einnahmen vorausgesetzt. Daher empfiehlt es sich, die Lehrveranstaltungen “Öffentliche Einnahmen” im Vorfeld zu besuchen.

\end{recommendations}
\end{styleenv}

\begin{learningoutcomes}
Der/ die Studierende

 \begin{itemize}\item besitzt weiterführende Kenntnisse in der Ausgestaltung des deutschen Steuersystems.  \item ist in der Lage die Auswirkungen verschiedener Besteuerungsarten zu beurteilen.  \item versteht Umfang, Struktur und Formen des internationalen Steuerrechts.  \end{itemize}
\end{learningoutcomes}

\begin{content}
Die Vorlesung zur speziellen Steuerlehre betrachtet die Bedeutung und Auswirkungen der wichtigsten Steuerarten. Schwerpunkt bildet zunächst das deutsche Steuerrecht, darüber hinaus werden Aspekte des internationalen, insbesondere des europäischen Steuerrechts behandelt.\newline
Hierzu werden zunächst spezielle Steuerprobleme betrachtet, zum Beispiel von Unternehmenssteuern, Einkommensteuer und Konsumsteuer und anschließend die Vor- und Nachteile der einzelnen Steuerarten hinsichtlich ihrer Belastungswirkung (Inzidenz) sowie ihre Auswirkung im Wertschöpfungsprozess. Im Folgenden bildet die Differenzierung der Steuern nach ihrer Bedeutung für die Finanzierung der öffentlichen Haushalte den Schwerpunkt der Vorlesung. Abschließend werden vergleichend Steuersysteme im inner- und außereuropäischen Ausland behandelt. \newline
Als Besonderheit werden im Rahmen der Vorlesung auch Referenten aus der Praxis Gastvorlesungen halten.


\end{content}

\begin{media}Skript zur Veranstaltung.

\end{media}

\begin{literature}\textbf{Weiterführende Literatur:}

 \begin{itemize}\item  

Andel, N. (1998): \emph{Finanzwissenschaft}, 4. Aufl., Mohr Siebeck.

   \item  

Betsch, O., Groh, A.P. und Schmidt, K. (2000): \emph{Gründungs- und Wachstumsfinanzierung innovativer Unternehmen}, Oldenbourg.

   \item  

Cloer, A. und Lavrelashvili, N. (2008): \emph{Einführung in das Europäische Steuerrecht}, Schmidt Erich.

   \item  

Homburg, S.(2007) : \emph{Allgemeine Steuerlehre}, 5. Aufl., Vahlen.

   \item  

Kravitz, N. (Hrsg.) (2010) : \emph{Internationale Aspekte der Unternehmensbesteuerung, }Zeitschrift für Betriebswirtschaft, Special Issue 2/2010.

   \end{itemize}\begin{itemize}\item  

Scheffler, W. (2009) : \emph{Besteuerung von Unternehmen I – Ertrags- Substanz- und Verkehrssteuern}, 11. Aufl., Müller Jur..

   \item  

Scheffler, W. (2009): \emph{Besteuerung von Unternehmen II – Steuerbilanz}, 11. Aufl., Müller Jur..

   \item  

Wigger, B. U. (2006):\emph{ Grundzüge der Finanzwissenschaft}, 2. Aufl., Springer.

   \end{itemize}\end{literature}



\end{course}