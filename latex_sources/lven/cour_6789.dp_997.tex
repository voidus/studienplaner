% Lehrveranstaltungsbeschreibung
% Informationsgrad : extern
% Sprache: de
\begin{course}

\setdoclanguagegerman
\coursedegreeprogramme{Informatik}
\coursemodulename{Essentials of Finance (S.~\pageref{mod_1541.dp_997})[IN3WWBWL3]}
\courseID{2530575}
\coursename{Investments}
\coursecoordination{M. Uhrig-Homburg}

\documentdate{2011-12-23 18:40:16.126353}

\courselevel{3}
\coursecredits{4,5}
\courseterm{Sommersemester}
\coursehours{2/1}
\courseinstructionlanguage{de}

\coursehead

% For index (key word@display). Key word is used for sorting - no Umlauts please.
\index{Investments@Investments}

% For later referencing
\label{cour_6789.dp_997}


\begin{styleenv}
\begin{assessment}
Die Erfolgskontrolle erfolgt in Form einer schriftlichen Prüfung (75min.) (nach §4(2), 1 SPO).

 

Die Prüfung wird in jedem Semester angeboten und kann zu jedem ordentlichen Prüfungstermin wiederholt werden.

 

Bonuspunkte (maximal 4) können durch die Abgabe von Übungsaufgaben während der Vorlesungszeit erreicht werden.


\end{assessment}

\begin{conditions}Keine.\end{conditions}

\begin{recommendations}Kenntnisse aus der Veranstaltung Allgemeine BWL C [25026/25027] sind sehr hilfreich.

\end{recommendations}
\end{styleenv}

\begin{learningoutcomes}
Ziel der Vorlesung ist es, die Studierenden mit den Grundlagen von Investitionsentscheidungen auf Aktien- und Rentenmärkten vertraut zu machen. Die Studierenden werden in die Lage versetzt, konkrete Modelle zur Fundierung von Investitionsentscheidungen anzuwenden und die resultierenden Entscheidungen über geeignete Performancemaße zu beurteilen.


\end{learningoutcomes}

\begin{content}
Die Vorlesung beschäftigt sich mit Investitionsentscheidungen unter Unsicherheit, wobei der Schwerpunkt auf Investitionsentscheidungen auf Aktienmärkten liegt. Nach einer Diskussion der Grundfragen der Bewertung von Aktien steht dann die Portfoliotheorie im Mittelpunkt der Veranstaltung. Im Anschluss daran erfolgt die Analyse von Ertrag und Risiko im Gleichgewicht mit der Ableitung des Capital Asset Pricing Models und der Arbitrage Pricing Theory. Abschließend werden Finanzinvestitionen auf Rentenmärkten behandelt.


\end{content}



\begin{literature}\textbf{Weiterführende Literatur:}

 

Bodie/Kane/Marcus (2010): Essentials of Investments, 8. Aufl., McGraw-Hill Irwin, Boston

\end{literature}



\end{course}