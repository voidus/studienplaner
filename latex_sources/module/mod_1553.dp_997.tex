% Modulbeschreibung 
% Informationsgrad : extern
% Sprache: de
\begin{module}

\setdoclanguagegerman
\moduledegreeprogramme{Informatik (B.Sc.)}
\modulesubject{EF Betriebswirtschaftslehre}
\moduleID{IN3WWBWL6}
\modulename{Risk and Insurance Management}
\modulecoordination{U. Werner}

\documentdate{2012-01-22 16:37:24.403568}

\modulecredits{9}
\moduleduration{2}
\modulecycle{Jedes Semester}



\modulehead

% For index (key word@display). Key word is used for sorting - no Umlauts please.
\index{Risk and Insurance Management@Risk and Insurance Management (M)}

% For later referencing
\label{mod_1553.dp_997}

\begin{courselist}
2550055 & Principles of Insurance Management (S.~\pageref{cour_6993.dp_997}) & 3/0 & S & 4,5 & U. Werner\\
2530326 & Enterprise Risk Management (S.~\pageref{cour_5109.dp_997}) & 3/0 & W & 4,5 & U. Werner\\
\end{courselist}

\begin{styleenv}
\begin{assessment}
Die Modulprüfung erfolgt in Form von Teilprüfungen (nach §4(2), 1-3 SPO) über die Lehrveranstaltungen des Moduls. Die Lehrveranstaltungen werden durch Vorträge und entsprechende Ausarbeitungen im Rahmen der Vorlesungen geprüft. Zudem findet eine abschließende mündliche Prüfung statt.

 

Die Note der jeweiligen Teilprüfung setzt sich je zu 50\% aus den Vortragsleistungen (inkl. Ausarbeitungen) und zu 50\% aus der mündlichen Prüfung zusammen. Die Gesamtnote des Moduls wird aus den mit LP gewichteten Noten der Teilprüfungen gebildet und nach der ersten Nachkommastelle abgeschnitten.


\end{assessment}

\begin{conditions}Nur in Kombination mit dem Modul \emph{Grundlagen der BWL} prüfbar.

 

Das Modul ist nur zusammen mit dem Pflichtmodul \emph{Grundlagen der BWL} [IN3WWBWL] prüfbar.

\end{conditions}


\end{styleenv}

\begin{learningoutcomes}
Der/die Studierende

 \begin{itemize}\item kann unternehmerische Risiken identifizieren, analysieren und bewerten.  \item ist in der Lage, geeignete Strategien und Maßnahmenbündel für das operationale Risikomanagement zu entwerfen  \item kann die Funktion von Versicherungsschutz als risikopolitisches Mittel auf einzel- und gesamtwirtschaftlicher Ebene einschätzen,  \item kennt und versteht die rechtlichen Rahmenbedingungen und Techniken der Produktion von Versicherungsschutz sowie weiterer Leistungen von Versicherungsunternehmen (Risikoberatung, Schadenmanagement).  \end{itemize}
\end{learningoutcomes}

\begin{content}
Das Modul führt in die Funktion von Versicherungsschutz als risikopolitisches Mittel auf einzel- und gesamtwirtschaftlicher Ebene ein, sowie in die rechtlichen Rahmenbedingungen und die Technik der Produktion von Versicherungsschutz. Ferner werden Kenntnisse vermittelt, die der Identifikation, Analyse und Bewertung unternehmerischer Risiken dienen. Darauf aufbauend werden Strategien und Maßnahmenbündel im Management des unternehmensweiten Chancen- und Gefahrenpotentials diskutiert, unter Berücksichtigung bereichsspezifischer Ziele, Risikotragfähigkeit und –akzeptanz.


\end{content}



\end{module}

