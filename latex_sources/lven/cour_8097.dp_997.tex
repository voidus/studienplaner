% Lehrveranstaltungsbeschreibung
% Informationsgrad : extern
% Sprache: de
\begin{course}

\setdoclanguagegerman
\coursedegreeprogramme{Informatik}
\coursemodulename{Grundlagen der Nachrichtentechnik (S.~\pageref{mod_3965.dp_997})[IN3EITGNT]}
\courseID{23506}
\coursename{Nachrichtentechnik I}
\coursecoordination{F. Jondral}

\documentdate{2011-02-17 13:23:19.492905}

\courselevel{3}
\coursecredits{6}
\courseterm{Sommersemester}
\coursehours{3/1}
\courseinstructionlanguage{de}

\coursehead

% For index (key word@display). Key word is used for sorting - no Umlauts please.
\index{Nachrichtentechnik I@Nachrichtentechnik I}

% For later referencing
\label{cour_8097.dp_997}


\begin{styleenv}
\begin{assessment}
Die Erfolgskontrolle erfolgt in Form einer schriftlichen Prüfung im Umfang von i.d.R. 3 Stunden nach § 4 Abs. 2 Nr. 1 SPO.

 

Die Lehrveranstaltungsnote ist die Note der schriftlichen Prüfung.


\end{assessment}

\begin{conditions}Die Vorlesung baut auf Kenntnissen der Vorlesungen “Wahrscheinlichkeitstheorie” (1305) und “Signale und Systeme” (23109) auf. Kenntnisse der höheren Mathematik werden vorausgesetzt.

\end{conditions}


\end{styleenv}

\begin{learningoutcomes}
Der Studierende soll die grundlegenden Definitionen und Aussagen der Nachrichtentechnik verstehen und anwenden lernen. Hierzu werden die zugrundeliegenden Mechanismen und Prinzipien, sowie deren Anwendung in nachrichtentechnischen Systemen behandelt.


\end{learningoutcomes}

\begin{content}
1. Signale und Systeme im komplexen Basisband, 2. Grundbegriffe der Informationstheorie, 3. Übertragungskanäle, 4. Quellencodierung, 5. Kanalcodierung 1: Allgemeine Bemerkungen und Blockcodierung, 6. Kanalcodierung 2: Faltungscodierung, 7. Modulationsverfahren, 8. Grundzüge der Entscheidungstheorie, 9. Demodulation, 10. Realisierungsgrenzen beim Systementwurf, 11. Multiple Input Multiple Output, 12. Vielfachzugriff, 13. Synchronisation, 14. Kanalentzerrung, 15. Netzwerke, 16. Das Global System for Mobile Communication, 17. Mobilfunk der dritten Generation, 18. Digital Audio Broadcast


\end{content}

\begin{media}Tafel, Folien

\end{media}

\begin{literature}Wird in der Vorlesung bekanntgegeben.

\textbf{Weiterführende Literatur:}

Wird in der Vorlesung bekanntgegeben.

\end{literature}



\end{course}